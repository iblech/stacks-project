\input{preamble}

% OK, start here.
%
\begin{document}

\title{Topologies on Schemes}

\maketitle

\phantomsection
\label{section-phantom}

\tableofcontents



\section{Introduction}
\label{section-introduction}

\noindent
In this document we explain what the different topologies on the
category of schemes are. Some references are \cite{SGA1} and \cite{Ner}.
Before doing so we would like to point out that there are many
different choices of sites (as defined in
Sites, Definition \ref{sites-definition-site}) which give rise to
the same notion of sheaf on the underlying category. Hence
our choices may be slightly different from those in the references
but ultimately lead to the same cohomology groups, etc.

\section{The general procedure}
\label{section-procedure}

\noindent
In this section we explain a general procedure for producing the
sites we will be working with. Suppose we want to study sheaves
over schemes with respect to some topology $\tau$. In order to
get a site, as in Sites, Definition \ref{sites-definition-site},
of schemes with that topology we have to do some work. Namely,
we cannot simply say ``consider all schemes with the Zariski topology''
since that would give a ``big'' category. Instead, in each section of
this chapter we will proceed as follows:
\begin{enumerate}
\item We define a class $\text{Cov}_\tau$ of coverings of schemes
satisfying the axioms of Sites, Definition \ref{sites-definition-site}.
It will always be the case that a Zariski open covering of
a scheme is a covering for $\tau$.
\item We single out a notion of standard
$\tau$-covering within the category of affine schemes.
\item We define what is an ``absolute'' big $\tau$-site $\Sch_\tau$.
These are the sites one gets by appropriately choosing a set of schemes
and a set of coverings.
\item For any object $S$ of $\Sch_\tau$
we define the big $\tau$-site $(\Sch/S)_\tau$ and for suitable
$\tau$ the small\footnote{The words big and
small here do not relate to bigness/smallness of the corresponding
categories.} $\tau$-site $S_\tau$.
\item In addition there is a site $(\textit{Aff}/S)_\tau$ using the
notion of standard $\tau$-covering of affines whose category of sheaves
is equivalent to the category of sheaves on $(\Sch/S)_\tau$.
\end{enumerate}
The above is a little clumsy in that we do not end up with a canonical
choice for the big $\tau$-site of a scheme, or even the small
$\tau$-site of a scheme. If you are willing to ignore set theoretic
difficulties, then you can work with classes and end up with
canonical big and small sites...







\section{The Zariski topology}
\label{section-zariski}

\begin{definition}
\label{definition-zariski-covering}
Let $T$ be a scheme. A {\it Zariski covering of $T$} is a family
of morphisms $\{f_i : T_i \to T\}_{i \in I}$ of schemes
such that each $f_i$ is an open immersion and such
that $T = \bigcup f_i(T_i)$.
\end{definition}

\noindent
This defines a (proper) class of coverings.
Next, we show that this notion satisfies the conditions of
Sites, Definition \ref{sites-definition-site}.

\begin{lemma}
\label{lemma-zariski}
Let $T$ be a scheme.
\begin{enumerate}
\item If $T' \to T$ is an isomorphism then $\{T' \to T\}$
is a Zariski covering of $T$.
\item If $\{T_i \to T\}_{i\in I}$ is a Zariski covering and for each
$i$ we have a Zariski covering $\{T_{ij} \to T_i\}_{j\in J_i}$, then
$\{T_{ij} \to T\}_{i \in I, j\in J_i}$ is a Zariski covering.
\item If $\{T_i \to T\}_{i\in I}$ is a Zariski covering
and $T' \to T$ is a morphism of schemes then
$\{T' \times_T T_i \to T'\}_{i\in I}$ is a Zariski covering.
\end{enumerate}
\end{lemma}

\begin{proof}
Omitted.
\end{proof}

\begin{lemma}
\label{lemma-zariski-affine}
Let $T$ be an affine scheme. Let $\{T_i \to T\}_{i \in I}$ be a
Zariski covering of $T$. Then there exists a Zariski covering
$\{U_j \to T\}_{j = 1, \ldots, m}$ which is a refinement
of $\{T_i \to T\}_{i \in I}$ such that each $U_j$ is a standard
open of $T$, see
Schemes, Definition \ref{schemes-definition-standard-covering}.
Moreover, we may choose each $U_j$ to be an open of one of the $T_i$.
\end{lemma}

\begin{proof}
Follows as $T$ is quasi-compact and standard opens form a basis
for its topology. This is also proved in
Schemes, Lemma \ref{schemes-lemma-standard-open}.
\end{proof}

\noindent
Thus we define the corresponding standard coverings of affines as follows.

\begin{definition}
\label{definition-standard-Zariski}
Compare Schemes, Definition \ref{schemes-definition-standard-covering}.
Let $T$ be an affine scheme. A {\it standard Zariski covering}
of $T$ is a a Zariski covering $\{U_j \to T\}_{j = 1, \ldots, m}$
with each $U_j \to T$ inducing an isomorphism with a standard affine open
of $T$.
\end{definition}

\begin{definition}
\label{definition-big-zariski-site}
A {\it big Zariski site} is any site $\Sch_{Zar}$ as in
Sites, Definition \ref{sites-definition-site} constructed as follows:
\begin{enumerate}
\item Choose any set of schemes $S_0$, and any set of Zariski coverings
$\text{Cov}_0$ among these schemes.
\item As underlying category of $\Sch_{Zar}$
take any category $\Sch_\alpha$ constructed as in
Sets, Lemma \ref{sets-lemma-construct-category} starting with the set $S_0$.
\item As coverings of $\Sch_{Zar}$ choose any set of coverings as in
Sets, Lemma \ref{sets-lemma-coverings-site} starting with the
category $\Sch_\alpha$ and the class of Zariski coverings,
and the set $\text{Cov}_0$ chosen above.
\end{enumerate}
\end{definition}

\noindent
It is shown in Sites, Lemma \ref{sites-lemma-choice-set-coverings-immaterial}
that, after having chosen the category $\Sch_\alpha$, the
category of sheaves on $\Sch_\alpha$ does not depend on the
choice of coverings chosen in (3) above. In other words, the topos
$\Sh(\Sch_{Zar})$ only depends on the choice of
the category $\Sch_\alpha$. It is shown in
Sets, Lemma \ref{sets-lemma-what-is-in-it} that these categories
are closed under many constructions of algebraic geometry, e.g.,
fibre products and taking open and closed subschemes. We can also show
that the exact choice of $\Sch_\alpha$ does not matter
too much, see Section \ref{section-change-alpha}.

\medskip\noindent
Another approach would be to assume the existence of a
strongly inaccessible cardinal and to define $\Sch_{Zar}$
to be the category of schemes contained in a chosen universe with
set of coverings the Zariski coverings contained in that same
universe.

\medskip\noindent
Before we continue with the introduction of the big Zariski site of
a scheme $S$, let us point out that the topology on a big Zariski site
$\Sch_{Zar}$ is in some sense induced from the Zariski topology
on the category of all schemes.

\begin{lemma}
\label{lemma-zariski-induced}
Let $\Sch_{Zar}$ be a big Zariski site as in
Definition \ref{definition-big-zariski-site}.
Let $T \in \Ob(\Sch_{Zar})$.
Let $\{T_i \to T\}_{i \in I}$ be an arbitrary Zariski covering of $T$.
There exists a covering $\{U_j \to T\}_{j \in J}$ of $T$ in the site
$\Sch_{Zar}$ which is tautologically equivalent (see
Sites, Definition \ref{sites-definition-combinatorial-tautological})
to $\{T_i \to T\}_{i \in I}$.
\end{lemma}

\begin{proof}
Since each $T_i \to T$ is an open immersion, we see by
Sets, Lemma \ref{sets-lemma-what-is-in-it}
that each $T_i$ is isomorphic to an object $V_i$ of $\Sch_{Zar}$.
The covering $\{V_i \to T\}_{i \in I}$ is tautologically equivalent
to $\{T_i \to T\}_{i \in I}$ (using the identity map on $I$ both ways).
Moreover, $\{V_i \to T\}_{i \in I}$ is combinatorially equivalent to a
covering $\{U_j \to T\}_{j \in J}$ of $T$ in the site $\Sch_{Zar}$ by
Sets, Lemma \ref{sets-lemma-coverings-site}.
\end{proof}

\begin{definition}
\label{definition-big-small-Zariski}
Let $S$ be a scheme. Let $\Sch_{Zar}$ be a big Zariski
site containing $S$.
\begin{enumerate}
\item The {\it big Zariski site of $S$}, denoted
$(\Sch/S)_{Zar}$, is the site $\Sch_{Zar}/S$
introduced in Sites, Section \ref{sites-section-localize}.
\item The {\it small Zariski site of $S$}, which we denote
$S_{Zar}$, is the full subcategory of $(\Sch/S)_{Zar}$
whose objects are those $U/S$ such that $U \to S$ is an open immersion.
A covering of $S_{Zar}$ is any covering $\{U_i \to U\}$ of
$(\Sch/S)_{Zar}$ with $U \in \Ob(S_{Zar})$.
\item The {\it big affine Zariski site of $S$}, denoted
$(\textit{Aff}/S)_{Zar}$, is the full subcategory of
$(\Sch/S)_{Zar}$ whose objects are affine $U/S$.
A covering of $(\textit{Aff}/S)_{Zar}$ is any covering
$\{U_i \to U\}$ of $(\Sch/S)_{Zar}$ which is a
standard Zariski covering.
\end{enumerate}
\end{definition}

\noindent
It is not completely clear that the small Zariski site and
the big affine Zariski site are sites. We check this now.

\begin{lemma}
\label{lemma-verify-site-Zariski}
Let $S$ be a scheme. Let $\Sch_{Zar}$ be a big Zariski
site containing $S$.
Both $S_{Zar}$ and $(\textit{Aff}/S)_{Zar}$ are sites.
\end{lemma}

\begin{proof}
Let us show that $S_{Zar}$ is a site. It is a category with a
given set of families of morphisms with fixed target. Thus we
have to show properties (1), (2) and (3) of
Sites, Definition \ref{sites-definition-site}.
Since $(\Sch/S)_{Zar}$ is a site, it suffices to prove
that given any covering $\{U_i \to U\}$ of $(\Sch/S)_{Zar}$
with $U \in \Ob(S_{Zar})$ we also have $U_i \in \Ob(S_{Zar})$.
This follows from the definitions
as the composition of open immersions is an open immersion.

\medskip\noindent
Let us show that $(\textit{Aff}/S)_{Zar}$ is a site.
Reasoning as above, it suffices to show that the collection
of standard Zariski coverings of affines satisfies properties
(1), (2) and (3) of
Sites, Definition \ref{sites-definition-site}.
Let $R$ be a ring. Let $f_1, \ldots, f_n \in R$ generate the unit ideal.
For each $i \in \{1, \ldots, n\}$ let $g_{i1}, \ldots, g_{in_i} \in R_{f_i}$
be elements generating the unit ideal of $R_{f_i}$. Write
$g_{ij} = f_{ij}/f_i^{e_{ij}}$ which is possible. After replacing
$f_{ij}$ by $f_i f_{ij}$ if necessary, we have that
$D(f_{ij}) \subset D(f_i) \cong \Spec(R_{f_i})$ is
equal to $D(g_{ij}) \subset \Spec(R_{f_i})$. Hence we see that
the family of morphisms $\{D(g_{ij}) \to \Spec(R)\}$
is a standard Zariski covering. From these considerations
it follows that (2) holds for standard Zariski coverings.
We omit the verification of (1) and (3).
\end{proof}

\begin{lemma}
\label{lemma-fibre-products-Zariski}
Let $S$ be a scheme. Let $\Sch_{Zar}$ be a big Zariski
site containing $S$. The underlying categories of the sites
$\Sch_{Zar}$, $(\Sch/S)_{Zar}$,
$S_{Zar}$, and $(\textit{Aff}/S)_{Zar}$ have fibre products.
In each case the obvious functor into the category $\Sch$ of
all schemes commutes with taking fibre products. The categories
$(\Sch/S)_{Zar}$, and $S_{Zar}$ both have a final object,
namely $S/S$.
\end{lemma}

\begin{proof}
For $\Sch_{Zar}$ it is true by construction, see
Sets, Lemma \ref{sets-lemma-what-is-in-it}.
Suppose we have $U \to S$, $V \to U$, $W \to U$ morphisms
of schemes with $U, V, W \in \Ob(\Sch_{Zar})$.
The fibre product $V \times_U W$ in $\Sch_{Zar}$
is a fibre product in $\Sch$ and
is the fibre product of $V/S$ with $W/S$ over $U/S$ in
the category of all schemes over $S$, and hence also a
fibre product in $(\Sch/S)_{Zar}$.
This proves the result for $(\Sch/S)_{Zar}$.
If $U \to S$, $V \to U$ and $W \to U$ are open immersions then so is
$V \times_U W \to S$ and hence we get the result for $S_{Zar}$.
If $U, V, W$ are affine, so is $V \times_U W$ and hence the
result for $(\textit{Aff}/S)_{Zar}$.
\end{proof}

\noindent
Next, we check that the big affine site defines the same
topos as the big site.

\begin{lemma}
\label{lemma-affine-big-site-Zariski}
Let $S$ be a scheme. Let $\Sch_{Zar}$ be a big Zariski
site containing $S$.
The functor $(\textit{Aff}/S)_{Zar} \to (\Sch/S)_{Zar}$
is a special cocontinuous functor. Hence it induces an equivalence
of topoi from $\Sh((\textit{Aff}/S)_{Zar})$ to
$\Sh((\Sch/S)_{Zar})$.
\end{lemma}

\begin{proof}
The notion of a special cocontinuous functor is introduced in
Sites, Definition \ref{sites-definition-special-cocontinuous-functor}.
Thus we have to verify assumptions (1) -- (5) of
Sites, Lemma \ref{sites-lemma-equivalence}.
Denote the inclusion functor
$u : (\textit{Aff}/S)_{Zar} \to (\Sch/S)_{Zar}$.
Being cocontinuous just means that any Zariski covering of
$T/S$, $T$ affine, can be refined by a standard Zariski covering of $T$.
This is the content of
Lemma \ref{lemma-zariski-affine}.
Hence (1) holds. We see $u$ is continuous simply because a standard
Zariski covering is a Zariski covering. Hence (2) holds.
Parts (3) and (4) follow immediately from the fact that $u$ is
fully faithful. And finally condition (5) follows from the
fact that every scheme has an affine open covering.
\end{proof}

\noindent
Let us check that the notion of a sheaf on the small Zariski site
corresponds to notion of a sheaf on $S$.

\begin{lemma}
\label{lemma-Zariski-usual}
The category of sheaves on $S_{Zar}$ is equivalent to the
category of sheaves on the underlying topological space of $S$.
\end{lemma}

\begin{proof}
We will use repeatedly that for any object
$U/S$ of $S_{Zar}$ the morphism $U \to S$ is an isomorphism
onto an open subscheme.
Let $\mathcal{F}$ be a sheaf on $S$. Then we define a sheaf
on $S_{Zar}$ by the rule $\mathcal{F}'(U/S) = \mathcal{F}(\Im(U \to S))$.
For the converse, we choose for every open subscheme $U \subset S$ an object
$U'/S \in \Ob(S_{Zar})$ with $\Im(U' \to S) = U$
(here you have to use Sets, Lemma \ref{sets-lemma-what-is-in-it}).
Given a sheaf $\mathcal{G}$ on $S_{Zar}$ we define a sheaf on $S$ by setting
$\mathcal{G}(U) = \mathcal{G}(U'/S)$. To see that $\mathcal{G}'$ is
a sheaf we use that for any open covering $U = \bigcup_{i \in I} U_i$
the covering $\{U_i \to U\}_{i \in I}$
is combinatorially equivalent to a covering $\{U_j' \to U'\}_{j \in J}$
in $S_{Zar}$ by Sets, Lemma \ref{sets-lemma-coverings-site},
and we use Sites, Lemma \ref{sites-lemma-tautological-same-sheaf}.
Details omitted.
\end{proof}

\noindent
From now on we will not make any distinction between a sheaf on
$S_{Zar}$ or a sheaf on $S$. We will always use the procedures
of the proof of the lemma to go between the two notions.
Next, we establish some relationships between the topoi
associated to these sites.

\begin{lemma}
\label{lemma-put-in-T}
Let $\Sch_{Zar}$ be a big Zariski site.
Let $f : T \to S$ be a morphism in $\Sch_{Zar}$.
The functor $T_{Zar} \to (\Sch/S)_{Zar}$
is cocontinuous and induces a morphism of topoi
$$
i_f :
\Sh(T_{Zar})
\longrightarrow
\Sh((\Sch/S)_{Zar})
$$
For a sheaf $\mathcal{G}$ on $(\Sch/S)_{Zar}$
we have the formula $(i_f^{-1}\mathcal{G})(U/T) = \mathcal{G}(U/S)$.
The functor $i_f^{-1}$ also has a left adjoint $i_{f, !}$ which commutes
with fibre products and equalizers.
\end{lemma}

\begin{proof}
Denote the functor $u : T_{Zar} \to (\Sch/S)_{Zar}$.
In other words, given and open immersion $j : U \to T$ corresponding
to an object of $T_{Zar}$ we set $u(U \to T) = (f \circ j : U \to S)$.
This functor commutes with fibre products, see
Lemma \ref{lemma-fibre-products-Zariski}.
Moreover, $T_{Zar}$ has equalizers (as any two morphisms with the same
source and target are the same) and $u$ commutes with them.
It is clearly cocontinuous.
It is also continuous as $u$ transforms coverings to coverings and
commutes with fibre products. Hence the lemma follows from
Sites, Lemmas \ref{sites-lemma-when-shriek}
and \ref{sites-lemma-preserve-equalizers}.
\end{proof}

\begin{lemma}
\label{lemma-at-the-bottom}
Let $S$ be a scheme. Let $\Sch_{Zar}$ be a big Zariski
site containing $S$.
The inclusion functor $S_{Zar} \to (\Sch/S)_{Zar}$
satisfies the hypotheses of Sites, Lemma \ref{sites-lemma-bigger-site}
and hence induces a morphism of sites
$$
\pi_S : (\Sch/S)_{Zar} \longrightarrow S_{Zar}
$$
and a morphism of topoi
$$
i_S : \Sh(S_{Zar}) \longrightarrow \Sh((\Sch/S)_{Zar})
$$
such that $\pi_S \circ i_S = \text{id}$. Moreover, $i_S = i_{\text{id}_S}$
with $i_{\text{id}_S}$ as in Lemma \ref{lemma-put-in-T}. In particular the
functor $i_S^{-1} = \pi_{S, *}$ is described by the rule
$i_S^{-1}(\mathcal{G})(U/S) = \mathcal{G}(U/S)$.
\end{lemma}

\begin{proof}
In this case the functor $u : S_{Zar} \to (\Sch/S)_{Zar}$,
in addition to the properties seen in the proof of
Lemma \ref{lemma-put-in-T} above, also is fully faithful
and transforms the final object into the final object.
The lemma follows.
\end{proof}

\begin{definition}
\label{definition-restriction-small-zariski}
In the situation of
Lemma \ref{lemma-at-the-bottom}
the functor $i_S^{-1} = \pi_{S, *}$ is often
called the {\it restriction to the small Zariski site}, and for a sheaf
$\mathcal{F}$ on the big Zariski site we denote $\mathcal{F}|_{S_{Zar}}$
this restriction.
\end{definition}

\noindent
With this notation in place we have for a sheaf $\mathcal{F}$ on the
big site and a sheaf $\mathcal{G}$ on the big site that
\begin{align*}
\Mor_{\Sh(S_{Zar})}(\mathcal{F}|_{S_{Zar}}, \mathcal{G})
& =
\Mor_{\Sh((\Sch/S)_{Zar})}(\mathcal{F},
i_{S, *}\mathcal{G}) \\
\Mor_{\Sh(S_{Zar})}(\mathcal{G}, \mathcal{F}|_{S_{Zar}})
& =
\Mor_{\Sh((\Sch/S)_{Zar})}(\pi_S^{-1}\mathcal{G},
\mathcal{F})
\end{align*}
Moreover, we have $(i_{S, *}\mathcal{G})|_{S_{Zar}} = \mathcal{G}$
and we have $(\pi_S^{-1}\mathcal{G})|_{S_{Zar}} = \mathcal{G}$.

\begin{lemma}
\label{lemma-morphism-big}
Let $\Sch_{Zar}$ be a big Zariski site.
Let $f : T \to S$ be a morphism in $\Sch_{Zar}$.
The functor
$$
u : (\Sch/T)_{Zar} \longrightarrow (\Sch/S)_{Zar},
\quad
V/T \longmapsto V/S
$$
is cocontinuous, and has a continuous right adjoint
$$
v : (\Sch/S)_{Zar} \longrightarrow (\Sch/T)_{Zar},
\quad
(U \to S) \longmapsto (U \times_S T \to T).
$$
They induce the same morphism of topoi
$$
f_{big} :
\Sh((\Sch/T)_{Zar})
\longrightarrow
\Sh((\Sch/S)_{Zar})
$$
We have $f_{big}^{-1}(\mathcal{G})(U/T) = \mathcal{G}(U/S)$.
We have $f_{big, *}(\mathcal{F})(U/S) = \mathcal{F}(U \times_S T/T)$.
Also, $f_{big}^{-1}$ has a left adjoint $f_{big!}$ which commutes with
fibre products and equalizers.
\end{lemma}

\begin{proof}
The functor $u$ is cocontinuous, continuous, and commutes with fibre products
and equalizers (details omitted; compare with proof of
Lemma \ref{lemma-put-in-T}).
Hence
Sites, Lemmas \ref{sites-lemma-when-shriek} and
\ref{sites-lemma-preserve-equalizers}
apply and we deduce the formula
for $f_{big}^{-1}$ and the existence of $f_{big!}$. Moreover,
the functor $v$ is a right adjoint because given $U/T$ and $V/S$
we have $\Mor_S(u(U), V) = \Mor_T(U, V \times_S T)$
as desired. Thus we may apply
Sites, Lemmas \ref{sites-lemma-have-functor-other-way} and
\ref{sites-lemma-have-functor-other-way-morphism}
to get the formula for $f_{big, *}$.
\end{proof}

\begin{lemma}
\label{lemma-morphism-big-small}
Let $\Sch_{Zar}$ be a big Zariski site.
Let $f : T \to S$ be a morphism in $\Sch_{Zar}$.
\begin{enumerate}
\item We have $i_f = f_{big} \circ i_T$ with $i_f$ as in
Lemma \ref{lemma-put-in-T} and $i_T$ as in
Lemma \ref{lemma-at-the-bottom}.
\item The functor $S_{Zar} \to T_{Zar}$,
$(U \to S) \mapsto (U \times_S T \to T)$ is continuous and induces
a morphism of topoi
$$
f_{small} :
\Sh(T_{Zar})
\longrightarrow
\Sh(S_{Zar}).
$$
The functors $f_{small}^{-1}$ and $f_{small, *}$ agree with
the usual notions $f^{-1}$ and $f_*$ is we identify sheaves
on $T_{Zar}$, resp.\ $S_{Zar}$ with sheaves on $T$, resp.\ $S$
via Lemma \ref{lemma-Zariski-usual}.
\item We have a commutative diagram of morphisms of sites
$$
\xymatrix{
T_{Zar} \ar[d]_{f_{small}} &
(\Sch/T)_{Zar} \ar[d]^{f_{big}} \ar[l]^{\pi_T} \\
S_{Zar} &
(\Sch/S)_{Zar} \ar[l]_{\pi_S}
}
$$
so that $f_{small} \circ \pi_T = \pi_S \circ f_{big}$ as morphisms of topoi.
\item We have $f_{small} = \pi_S \circ f_{big} \circ i_T = \pi_S \circ i_f$.
\end{enumerate}
\end{lemma}

\begin{proof}
The equality $i_f = f_{big} \circ i_T$ follows from the
equality $i_f^{-1} = i_T^{-1} \circ f_{big}^{-1}$ which is
clear from the descriptions of these functors above.
Thus we see (1).

\medskip\noindent
Statement (2): See Sites, Example \ref{sites-example-continuous-map}.

\medskip\noindent
Part (3) follows because $\pi_S$ and $\pi_T$ are given by
the inclusion functors and $f_{small}$ and $f_{big}$ by the
base change functor $U \mapsto U \times_S T$.

\medskip\noindent
Statement (4) follows from (3) by precomposing with $i_T$.
\end{proof}

\noindent
In the situation of the lemma, using the terminology of
Definition \ref{definition-restriction-small-zariski}
we have: for $\mathcal{F}$ a sheaf on the big Zariski site of $T$
$$
(f_{big, *}\mathcal{F})|_{S_{Zar}} =
f_{small, *}(\mathcal{F}|_{T_{Zar}}),
$$
This equality is clear from the commutativity of the diagram of
sites of the lemma, since restriction to the small Zariski site of
$T$, resp.\ $S$ is given by $\pi_{T, *}$, resp.\ $\pi_{S, *}$. A similar
formula involving pullbacks and restrictions is false.

\begin{lemma}
\label{lemma-composition}
Given schemes $X$, $Y$, $Y$ in $(\Sch/S)_{Zar}$
and morphisms $f : X \to Y$, $g : Y \to Z$ we have
$g_{big} \circ f_{big} = (g \circ f)_{big}$ and
$g_{small} \circ f_{small} = (g \circ f)_{small}$.
\end{lemma}

\begin{proof}
This follows from the simple description of pushforward
and pullback for the functors on the big sites from
Lemma \ref{lemma-morphism-big}. For the functors
on the small sites this is
Sheaves, Lemma \ref{sheaves-lemma-pushforward-composition}
via the identification of Lemma \ref{lemma-Zariski-usual}.
\end{proof}

\noindent
We can think about a sheaf on the big Zariski site of $S$ as a collection
of ``usual'' sheaves on all schemes over $S$.

\begin{lemma}
\label{lemma-characterize-sheaf-big}
Let $S$ be a scheme contained in a big Zariski site $\Sch_{Zar}$.
A sheaf $\mathcal{F}$ on the big Zariski site $(\Sch/S)_{Zar}$
is given by the following data:
\begin{enumerate}
\item for every $T/S \in \Ob((\Sch/S)_{Zar})$ a sheaf
$\mathcal{F}_T$ on $T$,
\item for every $f : T' \to T$ in
$(\Sch/S)_{Zar}$ a map
$c_f : f^{-1}\mathcal{F}_T \to \mathcal{F}_{T'}$.
\end{enumerate}
These data are subject to the following conditions:
\begin{enumerate}
\item[(a)] given any $f : T' \to T$ and $g : T'' \to T'$ in
$(\Sch/S)_{Zar}$ the composition $c_g \circ g^{-1}c_f$
is equal to $c_{f \circ g}$, and
\item[(b)] if $f : T' \to T$ in $(\Sch/S)_{Zar}$ is an
open immersion then $c_f$ is an isomorphism.
\end{enumerate}
\end{lemma}

\begin{proof}
Given a sheaf $\mathcal{F}$ on $\Sh((\Sch/S)_{Zar})$
we set $\mathcal{F}_T = i_p^{-1}\mathcal{F}$ where $p : T \to S$
is the structure morphism. Note that
$\mathcal{F}_T(U) = \mathcal{F}(U'/S)$ for any open $U \subset T$,
and $U' \to T$ an open immersion in $(\Sch/T)_{Zar}$
with image $U$, see Lemmas \ref{lemma-Zariski-usual} and \ref{lemma-put-in-T}.
Hence given $f : T' \to T$ over $S$ and $U, U' \to T$ we get a canonical
map $\mathcal{F}_T(U) = \mathcal{F}(U'/S) \to \mathcal{F}(U'\times_T T'/S)
= \mathcal{F}_{T'}(f^{-1}(U))$ where the middle is the restriction map
of $\mathcal{F}$ with respect to the morphism
$U' \times_T T' \to U'$ over $S$. The collection of these maps are
compatible with restrictions, and hence define an $f$-map $c_f$
from $\mathcal{F}_T$ to $\mathcal{F}_{T'}$, see
Sheaves, Definition \ref{sheaves-definition-f-map} and the discussion
surrounding it. It is clear that $c_{f \circ g}$ is the composition of
$c_f$ and $c_g$, since composition of restriction maps of $\mathcal{F}$
gives restriction maps.

\medskip\noindent
Conversely, given a system $(\mathcal{F}_T, c_f)$ as in the lemma
we may define a presheaf $\mathcal{F}$ on $\Sh((\Sch/S)_{Zar})$
by simply setting $\mathcal{F}(T/S) = \mathcal{F}_T(T)$. As restriction
mapping, given $f : T' \to T$ we set for $s \in \mathcal{F}(T)$
the pullback $f^*(s)$ equal to $c_f(s)$ (where we think of $c_f$ as
an $f$-map again). The condition on the $c_f$ guarantees that
pullbacks satisfy the required functoriality property.
We omit the verification that this is a sheaf.
It is clear that the constructions so defined are mutually inverse.
\end{proof}























\section{The \'etale topology}
\label{section-etale}

\noindent
Let $S$ be a scheme. We would like to define the \'etale-topology on
the category of schemes over $S$. According to our general principle
we first introduce the notion of an \'etale covering.

\begin{definition}
\label{definition-etale-covering}
Let $T$ be a scheme. An {\it \'etale covering of $T$} is a family
of morphisms $\{f_i : T_i \to T\}_{i \in I}$ of schemes
such that each $f_i$ is \'etale and such that $T = \bigcup f_i(T_i)$.
\end{definition}

\begin{lemma}
\label{lemma-zariski-etale}
Any Zariski covering is an \'etale covering.
\end{lemma}

\begin{proof}
This is clear from the definitions and the fact that an open immersion
is an \'etale morphism, see
Morphisms, Lemma \ref{morphisms-lemma-open-immersion-etale}.
\end{proof}

\noindent
Next, we show that this notion satisfies the conditions of
Sites, Definition \ref{sites-definition-site}.

\begin{lemma}
\label{lemma-etale}
Let $T$ be a scheme.
\begin{enumerate}
\item If $T' \to T$ is an isomorphism then $\{T' \to T\}$
is an \'etale covering of $T$.
\item If $\{T_i \to T\}_{i\in I}$ is an \'etale covering and for each
$i$ we have an \'etale covering $\{T_{ij} \to T_i\}_{j\in J_i}$, then
$\{T_{ij} \to T\}_{i \in I, j\in J_i}$ is an \'etale covering.
\item If $\{T_i \to T\}_{i\in I}$ is an \'etale covering
and $T' \to T$ is a morphism of schemes then
$\{T' \times_T T_i \to T'\}_{i\in I}$ is an \'etale covering.
\end{enumerate}
\end{lemma}

\begin{proof}
Omitted.
\end{proof}

\begin{lemma}
\label{lemma-etale-affine}
Let $T$ be an affine scheme.
Let $\{T_i \to T\}_{i \in I}$ be an \'etale covering of $T$.
Then there exists an \'etale covering
$\{U_j \to T\}_{j = 1, \ldots, m}$ which is a refinement
of $\{T_i \to T\}_{i \in I}$ such that each $U_j$ is an affine
scheme. Moreover, we may choose each $U_j$ to be open affine
in one of the $T_i$.
\end{lemma}

\begin{proof}
Omitted.
\end{proof}

\noindent
Thus we define the corresponding standard coverings of affines as follows.

\begin{definition}
\label{definition-standard-etale}
Let $T$ be an affine scheme. A {\it standard \'etale covering}
of $T$ is a family $\{f_j : U_j \to T\}_{j = 1, \ldots, m}$
with each $U_j$ is affine and \'etale over $T$ and
$T = \bigcup f_j(U_j)$.
\end{definition}

\noindent
In the definition above we do {\bf not} assume the morphisms $f_j$ are
standard \'etale. The reason is that if we did then the standard \'etale
coverings would not define a site on $\textit{Aff}/S$, for example because of
Algebra, Lemma \ref{algebra-lemma-standard-etale} part (4).
On the other hand, an \'etale morphism of affines is automatically
standard smooth, see
Algebra, Lemma \ref{algebra-lemma-etale-standard-smooth}.
Hence a standard \'etale covering is a standard smooth
covering and a standard syntomic covering.

\begin{definition}
\label{definition-big-etale-site}
A {\it big \'etale site} is any site $\Sch_\etale$ as in
Sites, Definition \ref{sites-definition-site} constructed as follows:
\begin{enumerate}
\item Choose any set of schemes $S_0$, and any set of \'etale coverings
$\text{Cov}_0$ among these schemes.
\item As underlying category take any category $\Sch_\alpha$
constructed as in Sets, Lemma \ref{sets-lemma-construct-category}
starting with the set $S_0$.
\item Choose any set of coverings as in
Sets, Lemma \ref{sets-lemma-coverings-site} starting with the
category $\Sch_\alpha$ and the class of \'etale coverings,
and the set $\text{Cov}_0$ chosen above.
\end{enumerate}
\end{definition}

\noindent
See the remarks following Definition \ref{definition-big-zariski-site}
for motivation and explanation regarding the definition of big sites.

\medskip\noindent
Before we continue with the introduction of the big \'etale site of
a scheme $S$, let us point out that the topology on a big \'etale site
$\Sch_\etale$ is in some sense induced from the \'etale
topology on the category of all schemes.

\begin{lemma}
\label{lemma-etale-induced}
Let $\Sch_\etale$ be a big \'etale site as in
Definition \ref{definition-big-etale-site}.
Let $T \in \Ob(\Sch_\etale)$.
Let $\{T_i \to T\}_{i \in I}$ be an arbitrary \'etale covering of $T$.
\begin{enumerate}
\item There exists a covering $\{U_j \to T\}_{j \in J}$ of $T$ in the site
$\Sch_\etale$ which refines $\{T_i \to T\}_{i \in I}$.
\item If $\{T_i \to T\}_{i \in I}$ is a standard \'etale covering, then
it is tautologically equivalent to a covering in $\Sch_\etale$.
\item If $\{T_i \to T\}_{i \in I}$ is a Zariski covering, then
it is tautologically equivalent to a covering in $\Sch_\etale$.
\end{enumerate}
\end{lemma}

\begin{proof}
For each $i$ choose an affine open covering $T_i = \bigcup_{j \in J_i} T_{ij}$
such that each $T_{ij}$ maps into an affine open subscheme of $T$. By
Lemma \ref{lemma-etale}
the refinement $\{T_{ij} \to T\}_{i \in I, j \in J_i}$ is an \'etale covering
of $T$ as well. Hence we may assume each $T_i$ is affine, and maps into
an affine open $W_i$ of $T$. Applying
Sets, Lemma \ref{sets-lemma-what-is-in-it}
we see that $W_i$ is isomorphic to an object of $\Sch_{Zar}$.
But then $T_i$ as a finite type scheme over $W_i$
is isomorphic to an object $V_i$ of $\Sch_{Zar}$ by a second
application of
Sets, Lemma \ref{sets-lemma-what-is-in-it}.
The covering $\{V_i \to T\}_{i \in I}$ refines $\{T_i \to T\}_{i \in I}$
(because they are isomorphic).
Moreover, $\{V_i \to T\}_{i \in I}$ is combinatorially equivalent to a
covering $\{U_j \to T\}_{j \in J}$ of $T$ in the site
$\Sch_{Zar}$ by
Sets, Lemma \ref{sets-lemma-what-is-in-it}.
The covering $\{U_j \to T\}_{j \in J}$ is a refinement as in (1).
In the situation of (2), (3) each of the
schemes $T_i$ is isomorphic to an object of $\Sch_\etale$ by
Sets, Lemma \ref{sets-lemma-what-is-in-it},
and another application of
Sets, Lemma \ref{sets-lemma-coverings-site}
gives what we want.
\end{proof}

\begin{definition}
\label{definition-big-small-etale}
Let $S$ be a scheme. Let $\Sch_\etale$ be a big \'etale
site containing $S$.
\begin{enumerate}
\item The {\it big \'etale site of $S$}, denoted
$(\Sch/S)_\etale$, is the site
$\Sch_\etale/S$ introduced in
Sites, Section \ref{sites-section-localize}.
\item The {\it small \'etale site of $S$}, which we denote
$S_\etale$, is the full subcategory of
$(\Sch/S)_\etale$
whose objects are those $U/S$ such that $U \to S$ is \'etale.
A covering of $S_\etale$ is any covering $\{U_i \to U\}$ of
$(\Sch/S)_\etale$ with $U \in \Ob(S_\etale)$.
\item The {\it big affine \'etale site of $S$}, denoted
$(\textit{Aff}/S)_\etale$, is the full subcategory of
$(\Sch/S)_\etale$ whose objects are affine $U/S$.
A covering of $(\textit{Aff}/S)_\etale$ is any covering
$\{U_i \to U\}$ of $(\Sch/S)_\etale$ which is a
standard \'etale covering.
\end{enumerate}
\end{definition}

\noindent
It is not completely clear that
the big affine \'etale site or the small \'etale site are sites.
We check this now.

\begin{lemma}
\label{lemma-verify-site-etale}
Let $S$ be a scheme. Let $\Sch_\etale$ be a big \'etale
site containing $S$.
Both $S_\etale$ and $(\textit{Aff}/S)_\etale$ are sites.
\end{lemma}

\begin{proof}
Let us show that $S_\etale$ is a site. It is a category with a
given set of families of morphisms with fixed target. Thus we
have to show properties (1), (2) and (3) of
Sites, Definition \ref{sites-definition-site}.
Since $(\Sch/S)_\etale$ is a site, it suffices to prove
that given any covering $\{U_i \to U\}$ of $(\Sch/S)_\etale$
with $U \in \Ob(S_\etale)$ we also have
$U_i \in \Ob(S_\etale)$.
This follows from the definitions as the composition of \'etale morphisms
is an \'etale morphism.

\medskip\noindent
Let us show that $(\textit{Aff}/S)_\etale$ is a site.
Reasoning as above, it suffices to show that the collection
of standard \'etale coverings of affines satisfies properties
(1), (2) and (3) of
Sites, Definition \ref{sites-definition-site}.
This is clear since for example, given a standard \'etale
covering $\{T_i \to T\}_{i\in I}$ and for each
$i$ we have a standard \'etale covering $\{T_{ij} \to T_i\}_{j\in J_i}$, then
$\{T_{ij} \to T\}_{i \in I, j\in J_i}$ is a standard \'etale covering
because $\bigcup_{i\in I} J_i$ is finite and each $T_{ij}$ is affine.
\end{proof}

\begin{lemma}
\label{lemma-fibre-products-etale}
Let $S$ be a scheme. Let $\Sch_\etale$ be a big \'etale
site containing $S$. The underlying categories of the sites
$\Sch_\etale$, $(\Sch/S)_\etale$,
$S_\etale$, and $(\textit{Aff}/S)_\etale$ have fibre products.
In each case the obvious functor into the category $\Sch$ of
all schemes commutes with taking fibre products. The categories
$(\Sch/S)_\etale$, and $S_\etale$ both have a
final object, namely $S/S$.
\end{lemma}

\begin{proof}
For $\Sch_\etale$ it is true by construction, see
Sets, Lemma \ref{sets-lemma-what-is-in-it}.
Suppose we have $U \to S$, $V \to U$, $W \to U$ morphisms
of schemes with $U, V, W \in \Ob(\Sch_\etale)$.
The fibre product $V \times_U W$ in $\Sch_\etale$
is a fibre product in $\Sch$ and
is the fibre product of $V/S$ with $W/S$ over $U/S$ in
the category of all schemes over $S$, and hence also a
fibre product in $(\Sch/S)_\etale$.
This proves the result for $(\Sch/S)_\etale$.
If $U \to S$, $V \to U$ and $W \to U$ are \'etale then so is
$V \times_U W \to S$ and hence we get the result for $S_\etale$.
If $U, V, W$ are affine, so is $V \times_U W$ and hence the
result for $(\textit{Aff}/S)_\etale$.
\end{proof}

\noindent
Next, we check that the big affine site defines the same
topos as the big site.

\begin{lemma}
\label{lemma-affine-big-site-etale}
Let $S$ be a scheme. Let $\Sch_\etale$ be a big \'etale
site containing $S$.
The functor
$(\textit{Aff}/S)_\etale \to (\Sch/S)_\etale$
is special cocontinuous and induces an equivalence of topoi from
$\Sh((\textit{Aff}/S)_\etale)$ to
$\Sh((\Sch/S)_\etale)$.
\end{lemma}

\begin{proof}
The notion of a special cocontinuous functor is introduced in
Sites, Definition \ref{sites-definition-special-cocontinuous-functor}.
Thus we have to verify assumptions (1) -- (5) of
Sites, Lemma \ref{sites-lemma-equivalence}.
Denote the inclusion functor
$u : (\textit{Aff}/S)_\etale \to (\Sch/S)_\etale$.
Being cocontinuous just means that any \'etale covering of
$T/S$, $T$ affine, can be refined by a standard \'etale covering of $T$.
This is the content of
Lemma \ref{lemma-etale-affine}.
Hence (1) holds. We see $u$ is continuous simply because a standard
\'etale covering is a \'etale covering. Hence (2) holds.
Parts (3) and (4) follow immediately from the fact that $u$ is
fully faithful. And finally condition (5) follows from the
fact that every scheme has an affine open covering.
\end{proof}

\noindent
Next, we establish some relationships between the topoi
associated to these sites.

\begin{lemma}
\label{lemma-put-in-T-etale}
Let $\Sch_\etale$ be a big \'etale site.
Let $f : T \to S$ be a morphism in $\Sch_\etale$.
The functor $T_\etale \to (\Sch/S)_\etale$
is cocontinuous and induces a morphism of topoi
$$
i_f :
\Sh(T_\etale)
\longrightarrow
\Sh((\Sch/S)_\etale)
$$
For a sheaf $\mathcal{G}$ on $(\Sch/S)_\etale$
we have the formula $(i_f^{-1}\mathcal{G})(U/T) = \mathcal{G}(U/S)$.
The functor $i_f^{-1}$ also has a left adjoint $i_{f, !}$ which commutes
with fibre products and equalizers.
\end{lemma}

\begin{proof}
Denote the functor $u : T_\etale \to (\Sch/S)_\etale$.
In other words, given an \'etale morphism $j : U \to T$ corresponding
to an object of $T_\etale$ we set $u(U \to T) = (f \circ j : U \to S)$.
This functor commutes with fibre products, see
Lemma \ref{lemma-fibre-products-etale}.
Let $a, b : U \to V$ be two morphisms in $T_\etale$.
In this case the equalizer of $a$ and $b$ (in the category of schemes) is
$$
V \times_{\Delta_{V/T}, V \times_T V, (a, b)} U \times_T U
$$
which is a fibre product of schemes \'etale over $T$, hence \'etale
over $T$. Thus $T_\etale$ has equalizers and $u$ commutes with them.
It is clearly cocontinuous.
It is also continuous as $u$ transforms coverings to coverings and
commutes with fibre products. Hence the Lemma follows from
Sites, Lemmas \ref{sites-lemma-when-shriek}
and \ref{sites-lemma-preserve-equalizers}.
\end{proof}

\begin{lemma}
\label{lemma-at-the-bottom-etale}
Let $S$ be a scheme. Let $\Sch_\etale$ be a big \'etale
site containing $S$.
The inclusion functor $S_\etale \to (\Sch/S)_\etale$
satisfies the hypotheses of Sites, Lemma \ref{sites-lemma-bigger-site}
and hence induces a morphism of sites
$$
\pi_S : (\Sch/S)_\etale \longrightarrow S_\etale
$$
and a morphism of topoi
$$
i_S : \Sh(S_\etale) \longrightarrow \Sh((\Sch/S)_\etale)
$$
such that $\pi_S \circ i_S = \text{id}$. Moreover, $i_S = i_{\text{id}_S}$
with $i_{\text{id}_S}$ as in Lemma \ref{lemma-put-in-T-etale}.
In particular the functor $i_S^{-1} = \pi_{S, *}$ is described by the rule
$i_S^{-1}(\mathcal{G})(U/S) = \mathcal{G}(U/S)$.
\end{lemma}

\begin{proof}
In this case the functor
$u : S_\etale \to (\Sch/S)_\etale$,
in addition to the properties seen in the proof of
Lemma \ref{lemma-put-in-T-etale} above, also is fully faithful
and transforms the final object into the final object.
The lemma follows from Sites, Lemma \ref{sites-lemma-bigger-site}.
\end{proof}

\begin{definition}
\label{definition-restriction-small-etale}
In the situation of
Lemma \ref{lemma-at-the-bottom-etale}
the functor $i_S^{-1} = \pi_{S, *}$ is often
called the {\it restriction to the small \'etale site}, and for a sheaf
$\mathcal{F}$ on the big \'etale site we denote
$\mathcal{F}|_{S_\etale}$ this restriction.
\end{definition}

\noindent
With this notation in place we have for a sheaf $\mathcal{F}$ on the
big site and a sheaf $\mathcal{G}$ on the small site that
\begin{align*}
\Mor_{\Sh(S_\etale)}(
\mathcal{F}|_{S_\etale},
\mathcal{G})
& =
\Mor_{\Sh((\Sch/S)_\etale)}(
\mathcal{F},
i_{S, *}\mathcal{G}) \\
\Mor_{\Sh(S_\etale)}(
\mathcal{G},
\mathcal{F}|_{S_\etale})
& =
\Mor_{\Sh((\Sch/S)_\etale)}(
\pi_S^{-1}\mathcal{G},
\mathcal{F})
\end{align*}
Moreover, we have $(i_{S, *}\mathcal{G})|_{S_\etale} = \mathcal{G}$
and we have $(\pi_S^{-1}\mathcal{G})|_{S_\etale} = \mathcal{G}$.

\begin{lemma}
\label{lemma-morphism-big-etale}
Let $\Sch_\etale$ be a big \'etale site.
Let $f : T \to S$ be a morphism in $\Sch_\etale$.
The functor
$$
u :
(\Sch/T)_\etale
\longrightarrow
(\Sch/S)_\etale,
\quad
V/T \longmapsto V/S
$$
is cocontinuous, and has a continuous right adjoint
$$
v :
(\Sch/S)_\etale
\longrightarrow
(\Sch/T)_\etale,
\quad
(U \to S) \longmapsto (U \times_S T \to T).
$$
They induce the same morphism of topoi
$$
f_{big} :
\Sh((\Sch/T)_\etale)
\longrightarrow
\Sh((\Sch/S)_\etale)
$$
We have $f_{big}^{-1}(\mathcal{G})(U/T) = \mathcal{G}(U/S)$.
We have $f_{big, *}(\mathcal{F})(U/S) = \mathcal{F}(U \times_S T/T)$.
Also, $f_{big}^{-1}$ has a left adjoint $f_{big!}$ which commutes with
fibre products and equalizers.
\end{lemma}

\begin{proof}
The functor $u$ is cocontinuous, continuous and commutes with fibre products
and equalizers (details omitted; compare with the proof of
Lemma \ref{lemma-put-in-T-etale}).
Hence
Sites, Lemmas \ref{sites-lemma-when-shriek} and
\ref{sites-lemma-preserve-equalizers}
apply and we deduce the formula
for $f_{big}^{-1}$ and the existence of $f_{big!}$. Moreover,
the functor $v$ is a right adjoint because given $U/T$ and $V/S$
we have $\Mor_S(u(U), V) = \Mor_T(U, V \times_S T)$
as desired. Thus we may apply
Sites, Lemmas \ref{sites-lemma-have-functor-other-way} and
\ref{sites-lemma-have-functor-other-way-morphism} to get the
formula for $f_{big, *}$.
\end{proof}

\begin{lemma}
\label{lemma-morphism-big-small-etale}
Let $\Sch_\etale$ be a big \'etale site.
Let $f : T \to S$ be a morphism in $\Sch_\etale$.
\begin{enumerate}
\item We have $i_f = f_{big} \circ i_T$ with $i_f$ as in
Lemma \ref{lemma-put-in-T-etale} and $i_T$ as in
Lemma \ref{lemma-at-the-bottom-etale}.
\item The functor $S_\etale \to T_\etale$,
$(U \to S) \mapsto (U \times_S T \to T)$ is continuous and induces
a morphism of topoi
$$
f_{small} :
\Sh(T_\etale)
\longrightarrow
\Sh(S_\etale).
$$
We have $f_{small, *}(\mathcal{F})(U/S) = \mathcal{F}(U \times_S T/T)$.
\item We have a commutative diagram of morphisms of sites
$$
\xymatrix{
T_\etale \ar[d]_{f_{small}} &
(\Sch/T)_\etale \ar[d]^{f_{big}} \ar[l]^{\pi_T}\\
S_\etale &
(\Sch/S)_\etale \ar[l]_{\pi_S}
}
$$
so that $f_{small} \circ \pi_T = \pi_S \circ f_{big}$ as morphisms of topoi.
\item We have $f_{small} = \pi_S \circ f_{big} \circ i_T = \pi_S \circ i_f$.
\end{enumerate}
\end{lemma}

\begin{proof}
The equality $i_f = f_{big} \circ i_T$ follows from the
equality $i_f^{-1} = i_T^{-1} \circ f_{big}^{-1}$ which is
clear from the descriptions of these functors above.
Thus we see (1).

\medskip\noindent
The functor
$u :
S_\etale
\to
T_\etale$, $u(U \to S) = (U \times_S T \to T)$
transforms coverings into coverings and commutes with fibre products,
see Lemma \ref{lemma-etale} (3) and \ref{lemma-fibre-products-etale}.
Moreover, both $S_\etale$, $T_\etale$ have final objects,
namely $S/S$ and $T/T$ and $u(S/S) = T/T$. Hence by
Sites, Proposition \ref{sites-proposition-get-morphism}
the functor $u$ corresponds to a morphism of sites
$T_\etale \to S_\etale$. This in turn gives rise to the
morphism of topoi, see
Sites, Lemma \ref{sites-lemma-morphism-sites-topoi}. The description
of the pushforward is clear from these references.

\medskip\noindent
Part (3) follows because $\pi_S$ and $\pi_T$ are given by the
inclusion functors and $f_{small}$ and $f_{big}$ by the
base change functors $U \mapsto U \times_S T$.

\medskip\noindent
Statement (4) follows from (3) by precomposing with $i_T$.
\end{proof}

\noindent
In the situation of the lemma, using the terminology of
Definition \ref{definition-restriction-small-etale}
we have: for $\mathcal{F}$ a sheaf on the big \'etale site of $T$
$$
(f_{big, *}\mathcal{F})|_{S_\etale} =
f_{small, *}(\mathcal{F}|_{T_\etale}),
$$
This equality is clear from the commutativity of the diagram of
sites of the lemma, since restriction to the small \'etale site of
$T$, resp.\ $S$ is given by $\pi_{T, *}$, resp.\ $\pi_{S, *}$. A similar
formula involving pullbacks and restrictions is false.

\begin{lemma}
\label{lemma-composition-etale}
Given schemes $X$, $Y$, $Y$ in $\Sch_\etale$
and morphisms $f : X \to Y$, $g : Y \to Z$ we have
$g_{big} \circ f_{big} = (g \circ f)_{big}$ and
$g_{small} \circ f_{small} = (g \circ f)_{small}$.
\end{lemma}

\begin{proof}
This follows from the simple description of pushforward
and pullback for the functors on the big sites from
Lemma \ref{lemma-morphism-big-etale}. For the functors
on the small sites this follows from the description of
the pushforward functors in Lemma \ref{lemma-morphism-big-small-etale}.
\end{proof}

\noindent
We can think about a sheaf on the big \'etale site of $S$ as a collection
of ``usual'' sheaves on all schemes over $S$.

\begin{lemma}
\label{lemma-characterize-sheaf-big-etale}
Let $S$ be a scheme contained in a big \'etale site
$\Sch_\etale$.
A sheaf $\mathcal{F}$ on the big \'etale site
$(\Sch/S)_\etale$ is given by the following data:
\begin{enumerate}
\item for every $T/S \in \Ob((\Sch/S)_\etale)$ a sheaf
$\mathcal{F}_T$ on $T_\etale$,
\item for every $f : T' \to T$ in
$(\Sch/S)_\etale$ a map
$c_f : f_{small}^{-1}\mathcal{F}_T \to \mathcal{F}_{T'}$.
\end{enumerate}
These data are subject to the following conditions:
\begin{enumerate}
\item[(a)] given any $f : T' \to T$ and $g : T'' \to T'$ in
$(\Sch/S)_\etale$ the composition
$c_g \circ g_{small}^{-1}c_f$ is equal to $c_{f \circ g}$, and
\item[(b)] if $f : T' \to T$ in $(\Sch/S)_\etale$
is \'etale then $c_f$ is an isomorphism.
\end{enumerate}
\end{lemma}

\begin{proof}
Given a sheaf $\mathcal{F}$ on $\Sh((\Sch/S)_\etale)$
we set $\mathcal{F}_T = i_p^{-1}\mathcal{F}$ where $p : T \to S$
is the structure morphism. Note that
$\mathcal{F}_T(U) = \mathcal{F}(U/S)$ for any $U \to T$
in $T_\etale$ see Lemma \ref{lemma-put-in-T-etale}.
Hence given $f : T' \to T$ over $S$ and $U \to T$ we get a canonical
map $\mathcal{F}_T(U) = \mathcal{F}(U/S) \to \mathcal{F}(U \times_T T'/S)
= \mathcal{F}_{T'}(U \times_T T')$ where the middle is the restriction map
of $\mathcal{F}$ with respect to the morphism
$U \times_T T' \to U$ over $S$. The collection of these maps are
compatible with restrictions, and hence define a map
$c'_f : \mathcal{F}_T \to f_{small, *}\mathcal{F}_{T'}$ where
$u : T_\etale \to T'_\etale$ is the base change functor
associated to $f$. By adjunction of $f_{small, *}$ (see
Sites, Section \ref{sites-section-continuous-functors}) with
$f_{small}^{-1}$ this is the same as a map
$c_f : f_{small}^{-1}\mathcal{F}_T \to \mathcal{F}_{T'}$.
It is clear that $c'_{f \circ g}$ is the composition of
$c'_f$ and $f_{small, *}c'_g$, since composition of restriction maps
of $\mathcal{F}$ gives restriction maps, and this gives the desired
relationship among $c_f$, $c_g$ and $c_{f \circ g}$.

\medskip\noindent
Conversely, given a system $(\mathcal{F}_T, c_f)$ as in the lemma
we may define a presheaf $\mathcal{F}$ on
$\Sh((\Sch/S)_\etale)$
by simply setting $\mathcal{F}(T/S) = \mathcal{F}_T(T)$. As restriction
mapping, given $f : T' \to T$ we set for $s \in \mathcal{F}(T)$
the pullback $f^*(s)$ equal to $c_f(s)$ where we think of $c_f$ as
a map $\mathcal{F}_T \to f_{small, *}\mathcal{F}_{T'}$ again.
The condition on the $c_f$ guarantees that
pullbacks satisfy the required functoriality property.
We omit the verification that this is a sheaf.
It is clear that the constructions so defined are mutually inverse.
\end{proof}























\section{The smooth topology}
\label{section-smooth}

\noindent
In this section we define the smooth topology.
This is a bit pointless as it will turn out later (see
More on Morphisms, Section \ref{more-morphisms-section-etale-over-smooth})
that this topology defines the same topos as the
\'etale topology. But still it makes sense and it is used
occasionally.

\begin{definition}
\label{definition-smooth-covering}
Let $T$ be a scheme. An {\it smooth covering of $T$} is a family
of morphisms $\{f_i : T_i \to T\}_{i \in I}$ of schemes
such that each $f_i$ is smooth and such
that $T = \bigcup f_i(T_i)$.
\end{definition}

\begin{lemma}
\label{lemma-zariski-etale-smooth}
Any \'etale covering is a smooth covering, and a fortiori,
any Zariski covering is a smooth covering.
\end{lemma}

\begin{proof}
This is clear from the definitions, the fact that an \'etale morphism is
smooth see
Morphisms, Definition \ref{morphisms-definition-etale}
and Lemma \ref{lemma-zariski-etale}.
\end{proof}

\noindent
Next, we show that this notion satisfies the conditions of
Sites, Definition \ref{sites-definition-site}.

\begin{lemma}
\label{lemma-smooth}
Let $T$ be a scheme.
\begin{enumerate}
\item If $T' \to T$ is an isomorphism then $\{T' \to T\}$
is an smooth covering of $T$.
\item If $\{T_i \to T\}_{i\in I}$ is a smooth covering and for each
$i$ we have a smooth covering $\{T_{ij} \to T_i\}_{j\in J_i}$, then
$\{T_{ij} \to T\}_{i \in I, j\in J_i}$ is a smooth covering.
\item If $\{T_i \to T\}_{i\in I}$ is a smooth covering
and $T' \to T$ is a morphism of schemes then
$\{T' \times_T T_i \to T'\}_{i\in I}$ is a smooth covering.
\end{enumerate}
\end{lemma}

\begin{proof}
Omitted.
\end{proof}

\begin{lemma}
\label{lemma-smooth-affine}
Let $T$ be an affine scheme.
Let $\{T_i \to T\}_{i \in I}$ be a smooth covering of $T$.
Then there exists a smooth covering
$\{U_j \to T\}_{j = 1, \ldots, m}$ which is a refinement
of $\{T_i \to T\}_{i \in I}$ such that each $U_j$ is an affine
scheme, and such that each morphism $U_j \to T$ is standard
smooth, see Morphisms, Definition \ref{morphisms-definition-smooth}.
Moreover, we may choose each $U_j$ to be open affine in one of the $T_i$.
\end{lemma}

\begin{proof}
Omitted, but see Algebra, Lemma \ref{algebra-lemma-smooth-syntomic}.
\end{proof}

\noindent
Thus we define the corresponding standard coverings of affines as follows.

\begin{definition}
\label{definition-standard-smooth}
Let $T$ be an affine scheme. A {\it standard smooth covering}
of $T$ is a family $\{f_j : U_j \to T\}_{j = 1, \ldots, m}$
with each $U_j$ is affine, $U_j \to T$ standard smooth
and $T = \bigcup f_j(U_j)$.
\end{definition}

\begin{definition}
\label{definition-big-smooth-site}
A {\it big smooth site} is any site $\Sch_{smooth}$ as in
Sites, Definition \ref{sites-definition-site} constructed as follows:
\begin{enumerate}
\item Choose any set of schemes $S_0$, and any set of smooth coverings
$\text{Cov}_0$ among these schemes.
\item As underlying category take any category $\Sch_\alpha$
constructed as in Sets, Lemma \ref{sets-lemma-construct-category}
starting with the set $S_0$.
\item Choose any set of coverings as in
Sets, Lemma \ref{sets-lemma-coverings-site} starting with the
category $\Sch_\alpha$ and the class of smooth coverings,
and the set $\text{Cov}_0$ chosen above.
\end{enumerate}
\end{definition}

\noindent
See the remarks following Definition \ref{definition-big-zariski-site}
for motivation and explanation regarding the definition of big sites.

\medskip\noindent
Before we continue with the introduction of the big smooth site of
a scheme $S$, let us point out that the topology on a big smooth site
$\Sch_{smooth}$ is in some sense induced from the smooth topology
on the category of all schemes.

\begin{lemma}
\label{lemma-smooth-induced}
Let $\Sch_{smooth}$ be a big smooth site as in
Definition \ref{definition-big-smooth-site}.
Let $T \in \Ob(\Sch_{smooth})$.
Let $\{T_i \to T\}_{i \in I}$ be an arbitrary smooth covering of $T$.
\begin{enumerate}
\item There exists a covering $\{U_j \to T\}_{j \in J}$ of $T$ in the site
$\Sch_{smooth}$ which refines $\{T_i \to T\}_{i \in I}$.
\item If $\{T_i \to T\}_{i \in I}$ is a standard smooth covering, then
it is tautologically equivalent to a covering of $\Sch_{smooth}$.
\item If $\{T_i \to T\}_{i \in I}$ is a Zariski covering, then
it is tautologically equivalent to a covering of $\Sch_{smooth}$.
\end{enumerate}
\end{lemma}

\begin{proof}
For each $i$ choose an affine open covering $T_i = \bigcup_{j \in J_i} T_{ij}$
such that each $T_{ij}$ maps into an affine open subscheme of $T$. By
Lemma \ref{lemma-smooth}
the refinement $\{T_{ij} \to T\}_{i \in I, j \in J_i}$ is an smooth covering
of $T$ as well. Hence we may assume each $T_i$ is affine, and maps into
an affine open $W_i$ of $T$. Applying
Sets, Lemma \ref{sets-lemma-what-is-in-it}
we see that $W_i$ is isomorphic to an object of $\Sch_{Zar}$.
But then $T_i$ as a finite type scheme over $W_i$
is isomorphic to an object $V_i$ of $\Sch_{Zar}$ by a second
application of
Sets, Lemma \ref{sets-lemma-what-is-in-it}.
The covering $\{V_i \to T\}_{i \in I}$ refines $\{T_i \to T\}_{i \in I}$
(because they are isomorphic).
Moreover, $\{V_i \to T\}_{i \in I}$ is combinatorially equivalent to a
covering $\{U_j \to T\}_{j \in J}$ of $T$ in the site
$\Sch_{Zar}$ by
Sets, Lemma \ref{sets-lemma-what-is-in-it}.
The covering $\{U_j \to T\}_{j \in J}$ is a refinement as in (1).
In the situation of (2), (3) each of the
schemes $T_i$ is isomorphic to an object of $\Sch_{smooth}$ by
Sets, Lemma \ref{sets-lemma-what-is-in-it},
and another application of
Sets, Lemma \ref{sets-lemma-coverings-site}
gives what we want.
\end{proof}

\begin{definition}
\label{definition-big-small-smooth}
Let $S$ be a scheme. Let $\Sch_{smooth}$ be a big smooth
site containing $S$.
\begin{enumerate}
\item The {\it big smooth site of $S$}, denoted
$(\Sch/S)_{smooth}$, is the site $\Sch_{smooth}/S$
introduced in Sites, Section \ref{sites-section-localize}.
\item The {\it big affine smooth site of $S$}, denoted
$(\textit{Aff}/S)_{smooth}$, is the full subcategory of
$(\Sch/S)_{smooth}$ whose objects are affine $U/S$.
A covering of $(\textit{Aff}/S)_{smooth}$ is any covering
$\{U_i \to U\}$ of $(\Sch/S)_{smooth}$ which is a
standard smooth covering.
\end{enumerate}
\end{definition}

\noindent
Next, we check that the big affine site defines the same
topos as the big site.

\begin{lemma}
\label{lemma-affine-big-site-smooth}
Let $S$ be a scheme. Let $\Sch_\etale$ be a big smooth
site containing $S$.
The functor
$(\textit{Aff}/S)_{smooth} \to (\Sch/S)_{smooth}$
is special cocontinuous and induces an equivalence of topoi from
$\Sh((\textit{Aff}/S)_{smooth})$ to
$\Sh((\Sch/S)_{smooth})$.
\end{lemma}

\begin{proof}
The notion of a special cocontinuous functor is introduced in
Sites, Definition \ref{sites-definition-special-cocontinuous-functor}.
Thus we have to verify assumptions (1) -- (5) of
Sites, Lemma \ref{sites-lemma-equivalence}.
Denote the inclusion functor
$u : (\textit{Aff}/S)_{smooth} \to (\Sch/S)_{smooth}$.
Being cocontinuous just means that any smooth covering of
$T/S$, $T$ affine, can be refined by a standard smooth covering of $T$.
This is the content of
Lemma \ref{lemma-smooth-affine}.
Hence (1) holds. We see $u$ is continuous simply because a standard
smooth covering is a smooth covering. Hence (2) holds.
Parts (3) and (4) follow immediately from the fact that $u$ is
fully faithful. And finally condition (5) follows from the
fact that every scheme has an affine open covering.
\end{proof}

\noindent
To be continued...

\begin{lemma}
\label{lemma-morphism-big-smooth}
Let $\Sch_{smooth}$ be a big smooth site.
Let $f : T \to S$ be a morphism in $\Sch_{smooth}$.
The functor
$$
u : (\Sch/T)_{smooth} \longrightarrow (\Sch/S)_{smooth},
\quad
V/T \longmapsto V/S
$$
is cocontinuous, and has a continuous right adjoint
$$
v : (\Sch/S)_{smooth} \longrightarrow (\Sch/T)_{smooth},
\quad
(U \to S) \longmapsto (U \times_S T \to T).
$$
They induce the same morphism of topoi
$$
f_{big} :
\Sh((\Sch/T)_{smooth})
\longrightarrow
\Sh((\Sch/S)_{smooth})
$$
We have $f_{big}^{-1}(\mathcal{G})(U/T) = \mathcal{G}(U/S)$.
We have $f_{big, *}(\mathcal{F})(U/S) = \mathcal{F}(U \times_S T/T)$.
Also, $f_{big}^{-1}$ has a left adjoint $f_{big!}$ which commutes with
fibre products and equalizers.
\end{lemma}

\begin{proof}
The functor $u$ is cocontinuous, continuous, and commutes with fibre products
and equalizers. Hence
Sites, Lemmas \ref{sites-lemma-when-shriek} and
\ref{sites-lemma-preserve-equalizers}
apply and we deduce the formula
for $f_{big}^{-1}$ and the existence of $f_{big!}$. Moreover,
the functor $v$ is a right adjoint because given $U/T$ and $V/S$
we have $\Mor_S(u(U), V) = \Mor_T(U, V \times_S T)$
as desired. Thus we may apply
Sites, Lemmas \ref{sites-lemma-have-functor-other-way} and
\ref{sites-lemma-have-functor-other-way-morphism} to get the
formula for $f_{big, *}$.
\end{proof}











\section{The syntomic topology}
\label{section-syntomic}

\noindent
In this section we define the syntomic topology.
This topology is quite interesting in that it often
has the same cohomology groups as the fppf topology
but is technically easier to deal with.

\begin{definition}
\label{definition-syntomic-covering}
Let $T$ be a scheme. An {\it syntomic covering of $T$} is a family
of morphisms $\{f_i : T_i \to T\}_{i \in I}$ of schemes
such that each $f_i$ is syntomic and such
that $T = \bigcup f_i(T_i)$.
\end{definition}

\begin{lemma}
\label{lemma-zariski-etale-smooth-syntomic}
Any smooth covering is a syntomic covering, and a fortiori,
any \'etale or Zariski covering is a syntomic covering.
\end{lemma}

\begin{proof}
This is clear from the definitions and the fact that a smooth
morphism is syntomic, see
Morphisms, Lemma \ref{morphisms-lemma-smooth-syntomic}
and Lemma \ref{lemma-zariski-etale-smooth}.
\end{proof}

\noindent
Next, we show that this notion satisfies the conditions of
Sites, Definition \ref{sites-definition-site}.

\begin{lemma}
\label{lemma-syntomic}
Let $T$ be a scheme.
\begin{enumerate}
\item If $T' \to T$ is an isomorphism then $\{T' \to T\}$
is an syntomic covering of $T$.
\item If $\{T_i \to T\}_{i\in I}$ is a syntomic covering and for each
$i$ we have a syntomic covering $\{T_{ij} \to T_i\}_{j\in J_i}$, then
$\{T_{ij} \to T\}_{i \in I, j\in J_i}$ is a syntomic covering.
\item If $\{T_i \to T\}_{i\in I}$ is a syntomic covering
and $T' \to T$ is a morphism of schemes then
$\{T' \times_T T_i \to T'\}_{i\in I}$ is a syntomic covering.
\end{enumerate}
\end{lemma}

\begin{proof}
Omitted.
\end{proof}

\begin{lemma}
\label{lemma-syntomic-affine}
Let $T$ be an affine scheme.
Let $\{T_i \to T\}_{i \in I}$ be a syntomic covering of $T$.
Then there exists a syntomic covering
$\{U_j \to T\}_{j = 1, \ldots, m}$ which is a refinement
of $\{T_i \to T\}_{i \in I}$ such that each $U_j$ is an affine
scheme, and such that each morphism $U_j \to T$ is standard
syntomic, see Morphisms, Definition \ref{morphisms-definition-syntomic}.
Moreover, we may choose each $U_j$ to be open affine in one of the $T_i$.
\end{lemma}

\begin{proof}
Omitted, but see Algebra, Lemma \ref{algebra-lemma-syntomic}.
\end{proof}

\noindent
Thus we define the corresponding standard coverings of affines as follows.

\begin{definition}
\label{definition-standard-syntomic}
Let $T$ be an affine scheme. A {\it standard syntomic covering} of $T$ is
a family $\{f_j : U_j \to T\}_{j = 1, \ldots, m}$ with each $U_j$ is
affine, $U_j \to T$ standard syntomic and $T = \bigcup f_j(U_j)$.
\end{definition}

\begin{definition}
\label{definition-big-syntomic-site}
A {\it big syntomic site} is any site $\Sch_{syntomic}$ as in
Sites, Definition \ref{sites-definition-site} constructed as follows:
\begin{enumerate}
\item Choose any set of schemes $S_0$, and any set of syntomic coverings
$\text{Cov}_0$ among these schemes.
\item As underlying category take any category $\Sch_\alpha$
constructed as in Sets, Lemma \ref{sets-lemma-construct-category}
starting with the set $S_0$.
\item Choose any set of coverings as in
Sets, Lemma \ref{sets-lemma-coverings-site} starting with the
category $\Sch_\alpha$ and the class of syntomic coverings,
and the set $\text{Cov}_0$ chosen above.
\end{enumerate}
\end{definition}

\noindent
See the remarks following Definition \ref{definition-big-zariski-site}
for motivation and explanation regarding the definition of big sites.

\medskip\noindent
Before we continue with the introduction of the big syntomic site of
a scheme $S$, let us point out that the topology on a big syntomic site
$\Sch_{syntomic}$ is in some sense induced from the syntomic topology
on the category of all schemes.

\begin{lemma}
\label{lemma-syntomic-induced}
Let $\Sch_{syntomic}$ be a big syntomic site as in
Definition \ref{definition-big-syntomic-site}.
Let $T \in \Ob(\Sch_{syntomic})$.
Let $\{T_i \to T\}_{i \in I}$ be an arbitrary syntomic covering of $T$.
\begin{enumerate}
\item There exists a covering $\{U_j \to T\}_{j \in J}$ of $T$ in the site
$\Sch_{syntomic}$ which refines $\{T_i \to T\}_{i \in I}$.
\item If $\{T_i \to T\}_{i \in I}$ is a standard syntomic covering, then
it is tautologically equivalent to a covering in $\Sch_{syntomic}$.
\item If $\{T_i \to T\}_{i \in I}$ is a Zariski covering, then
it is tautologically equivalent to a covering in $\Sch_{syntomic}$.
\end{enumerate}
\end{lemma}

\begin{proof}
For each $i$ choose an affine open covering $T_i = \bigcup_{j \in J_i} T_{ij}$
such that each $T_{ij}$ maps into an affine open subscheme of $T$. By
Lemma \ref{lemma-syntomic}
the refinement $\{T_{ij} \to T\}_{i \in I, j \in J_i}$ is an syntomic covering
of $T$ as well. Hence we may assume each $T_i$ is affine, and maps into
an affine open $W_i$ of $T$. Applying
Sets, Lemma \ref{sets-lemma-what-is-in-it}
we see that $W_i$ is isomorphic to an object of $\Sch_{Zar}$.
But then $T_i$ as a finite type scheme over $W_i$
is isomorphic to an object $V_i$ of $\Sch_{Zar}$ by a second
application of
Sets, Lemma \ref{sets-lemma-what-is-in-it}.
The covering $\{V_i \to T\}_{i \in I}$ refines $\{T_i \to T\}_{i \in I}$
(because they are isomorphic).
Moreover, $\{V_i \to T\}_{i \in I}$ is combinatorially equivalent to a
covering $\{U_j \to T\}_{j \in J}$ of $T$ in the site
$\Sch_{Zar}$ by
Sets, Lemma \ref{sets-lemma-what-is-in-it}.
The covering $\{U_j \to T\}_{j \in J}$ is a covering as in (1).
In the situation of (2), (3) each of the
schemes $T_i$ is isomorphic to an object of $\Sch_{Zar}$ by
Sets, Lemma \ref{sets-lemma-what-is-in-it},
and another application of
Sets, Lemma \ref{sets-lemma-coverings-site}
gives what we want.
\end{proof}

\begin{definition}
\label{definition-big-small-syntomic}
Let $S$ be a scheme. Let $\Sch_{syntomic}$ be a big syntomic
site containing $S$.
\begin{enumerate}
\item The {\it big syntomic site of $S$}, denoted
$(\Sch/S)_{syntomic}$, is the site $\Sch_{syntomic}/S$
introduced in Sites, Section \ref{sites-section-localize}.
\item The {\it big affine syntomic site of $S$}, denoted
$(\textit{Aff}/S)_{syntomic}$, is the full subcategory of
$(\Sch/S)_{syntomic}$ whose objects are affine $U/S$.
A covering of $(\textit{Aff}/S)_{syntomic}$ is any covering
$\{U_i \to U\}$ of $(\Sch/S)_{syntomic}$ which is a
standard syntomic covering.
\end{enumerate}
\end{definition}

\noindent
Next, we check that the big affine site defines the same
topos as the big site.

\begin{lemma}
\label{lemma-affine-big-site-syntomic}
Let $S$ be a scheme. Let $\Sch_{syntomic}$ be a big syntomic
site containing $S$.
The functor
$(\textit{Aff}/S)_{syntomic} \to (\Sch/S)_{syntomic}$
is special cocontinuous and induces an equivalence of topoi from
$\Sh((\textit{Aff}/S)_{syntomic})$ to
$\Sh((\Sch/S)_{syntomic})$.
\end{lemma}

\begin{proof}
The notion of a special cocontinuous functor is introduced in
Sites, Definition \ref{sites-definition-special-cocontinuous-functor}.
Thus we have to verify assumptions (1) -- (5) of
Sites, Lemma \ref{sites-lemma-equivalence}.
Denote the inclusion functor
$u : (\textit{Aff}/S)_{syntomic} \to (\Sch/S)_{syntomic}$.
Being cocontinuous just means that any syntomic covering of
$T/S$, $T$ affine, can be refined by a standard syntomic covering of $T$.
This is the content of
Lemma \ref{lemma-syntomic-affine}.
Hence (1) holds. We see $u$ is continuous simply because a standard
syntomic covering is a syntomic covering. Hence (2) holds.
Parts (3) and (4) follow immediately from the fact that $u$ is
fully faithful. And finally condition (5) follows from the
fact that every scheme has an affine open covering.
\end{proof}

\noindent
To be continued...

\begin{lemma}
\label{lemma-morphism-big-syntomic}
Let $\Sch_{syntomic}$ be a big syntomic site.
Let $f : T \to S$ be a morphism in $\Sch_{syntomic}$.
The functor
$$
u : (\Sch/T)_{syntomic} \longrightarrow (\Sch/S)_{syntomic},
\quad
V/T \longmapsto V/S
$$
is cocontinuous, and has a continuous right adjoint
$$
v : (\Sch/S)_{syntomic} \longrightarrow (\Sch/T)_{syntomic},
\quad
(U \to S) \longmapsto (U \times_S T \to T).
$$
They induce the same morphism of topoi
$$
f_{big} :
\Sh((\Sch/T)_{syntomic})
\longrightarrow
\Sh((\Sch/S)_{syntomic})
$$
We have $f_{big}^{-1}(\mathcal{G})(U/T) = \mathcal{G}(U/S)$.
We have $f_{big, *}(\mathcal{F})(U/S) = \mathcal{F}(U \times_S T/T)$.
Also, $f_{big}^{-1}$ has a left adjoint $f_{big!}$ which commutes with
fibre products and equalizers.
\end{lemma}

\begin{proof}
The functor $u$ is cocontinuous, continuous, and commutes with fibre products
and equalizers. Hence
Sites, Lemmas \ref{sites-lemma-when-shriek} and
\ref{sites-lemma-preserve-equalizers}
apply and we deduce the formula
for $f_{big}^{-1}$ and the existence of $f_{big!}$. Moreover,
the functor $v$ is a right adjoint because given $U/T$ and $V/S$
we have $\Mor_S(u(U), V) = \Mor_T(U, V \times_S T)$
as desired. Thus we may apply
Sites, Lemmas \ref{sites-lemma-have-functor-other-way} and
\ref{sites-lemma-have-functor-other-way-morphism} to get the
formula for $f_{big, *}$.
\end{proof}













\section{The fppf topology}
\label{section-fppf}

\noindent
Let $S$ be a scheme. We would like to define the fppf-topology\footnote{
The letters fppf stand for ``fid\`element plat de pr\'esentation finie''.} on
the category of schemes over $S$. According to our general principle
we first introduce the notion of an fppf-covering.

\begin{definition}
\label{definition-fppf-covering}
Let $T$ be a scheme. An {\it fppf covering of $T$} is a family
of morphisms $\{f_i : T_i \to T\}_{i \in I}$ of schemes
such that each $f_i$ is flat, locally of finite presentation and such
that $T = \bigcup f_i(T_i)$.
\end{definition}

\begin{lemma}
\label{lemma-zariski-etale-smooth-syntomic-fppf}
Any syntomic covering is an fppf covering, and a fortiori,
any smooth, \'etale, or Zariski covering is an fppf covering.
\end{lemma}

\begin{proof}
This is clear from the definitions, the fact that a syntomic morphism
is flat and locally of finite presentation, see
Morphisms, Lemmas
\ref{morphisms-lemma-syntomic-locally-finite-presentation} and
\ref{morphisms-lemma-syntomic-flat},
and
Lemma \ref{lemma-zariski-etale-smooth-syntomic}.
\end{proof}

\noindent
Next, we show that this notion satisfies the conditions of
Sites, Definition \ref{sites-definition-site}.

\begin{lemma}
\label{lemma-fppf}
Let $T$ be a scheme.
\begin{enumerate}
\item If $T' \to T$ is an isomorphism then $\{T' \to T\}$
is an fppf covering of $T$.
\item If $\{T_i \to T\}_{i\in I}$ is an fppf covering and for each
$i$ we have an fppf covering $\{T_{ij} \to T_i\}_{j\in J_i}$, then
$\{T_{ij} \to T\}_{i \in I, j\in J_i}$ is an fppf covering.
\item If $\{T_i \to T\}_{i\in I}$ is an fppf covering
and $T' \to T$ is a morphism of schemes then
$\{T' \times_T T_i \to T'\}_{i\in I}$ is an fppf covering.
\end{enumerate}
\end{lemma}

\begin{proof}
The first assertion is clear.
The second follows as the composition of flat morphisms is flat
(see Morphisms, Lemma \ref{morphisms-lemma-composition-flat})
and the composition of morphisms of finite presentation is
of finite presentation
(see Morphisms, Lemma \ref{morphisms-lemma-composition-finite-presentation}).
The third follows as the base change of a flat morphism is flat
(see Morphisms, Lemma \ref{morphisms-lemma-base-change-flat})
and the base change of a morphism of finite presentation is
of finite presentation
(see Morphisms, Lemma \ref{morphisms-lemma-base-change-finite-presentation}).
Moreover, the base change of a surjective family of morphisms is surjective
(proof omitted).
\end{proof}

\begin{lemma}
\label{lemma-fppf-affine}
Let $T$ be an affine scheme.
Let $\{T_i \to T\}_{i \in I}$ be an fppf covering of $T$.
Then there exists an fppf covering
$\{U_j \to T\}_{j = 1, \ldots, m}$ which is a refinement
of $\{T_i \to T\}_{i \in I}$ such that each $U_j$ is an affine
scheme. Moreover, we may choose each $U_j$ to be open affine
in one of the $T_i$.
\end{lemma}

\begin{proof}
This follows directly from the definitions using that a
morphism which is flat and locally of finite presentation is open,
see Morphisms, Lemma \ref{morphisms-lemma-fppf-open}.
\end{proof}

\noindent
Thus we define the corresponding standard coverings of affines as follows.

\begin{definition}
\label{definition-standard-fppf}
Let $T$ be an affine scheme. A {\it standard fppf covering}
of $T$ is a family $\{f_j : U_j \to T\}_{j = 1, \ldots, m}$
with each $U_j$ is affine, flat and of finite presentation over $T$
and $T = \bigcup f_j(U_j)$.
\end{definition}

\begin{definition}
\label{definition-big-fppf-site}
A {\it big fppf site} is any site $\Sch_{fppf}$ as in
Sites, Definition \ref{sites-definition-site} constructed as follows:
\begin{enumerate}
\item Choose any set of schemes $S_0$, and any set of fppf coverings
$\text{Cov}_0$ among these schemes.
\item As underlying category take any category $\Sch_\alpha$
constructed as in Sets, Lemma \ref{sets-lemma-construct-category}
starting with the set $S_0$.
\item Choose any set of coverings as in
Sets, Lemma \ref{sets-lemma-coverings-site} starting with the
category $\Sch_\alpha$ and the class of fppf coverings,
and the set $\text{Cov}_0$ chosen above.
\end{enumerate}
\end{definition}

\noindent
See the remarks following Definition \ref{definition-big-zariski-site}
for motivation and explanation regarding the definition of big sites.

\medskip\noindent
Before we continue with the introduction of the big fppf site of
a scheme $S$, let us point out that the topology on a big fppf site
$\Sch_{fppf}$ is in some sense induced from the fppf topology
on the category of all schemes.

\begin{lemma}
\label{lemma-fppf-induced}
Let $\Sch_{fppf}$ be a big fppf site as in
Definition \ref{definition-big-fppf-site}.
Let $T \in \Ob(\Sch_{fppf})$.
Let $\{T_i \to T\}_{i \in I}$ be an arbitrary fppf covering of $T$.
\begin{enumerate}
\item There exists a covering $\{U_j \to T\}_{j \in J}$ of $T$ in the site
$\Sch_{fppf}$ which refines $\{T_i \to T\}_{i \in I}$.
\item If $\{T_i \to T\}_{i \in I}$ is a standard fppf covering, then
it is tautologically equivalent to a covering of $\Sch_{fppf}$.
\item If $\{T_i \to T\}_{i \in I}$ is a Zariski covering, then
it is tautologically equivalent to a covering of $\Sch_{fppf}$.
\end{enumerate}
\end{lemma}

\begin{proof}
For each $i$ choose an affine open covering $T_i = \bigcup_{j \in J_i} T_{ij}$
such that each $T_{ij}$ maps into an affine open subscheme of $T$. By
Lemma \ref{lemma-fppf}
the refinement $\{T_{ij} \to T\}_{i \in I, j \in J_i}$ is an fppf covering
of $T$ as well. Hence we may assume each $T_i$ is affine, and maps into
an affine open $W_i$ of $T$. Applying
Sets, Lemma \ref{sets-lemma-what-is-in-it}
we see that $W_i$ is isomorphic to an object of $\Sch_{Zar}$.
But then $T_i$ as a finite type scheme over $W_i$
is isomorphic to an object $V_i$ of $\Sch_{Zar}$ by a second
application of
Sets, Lemma \ref{sets-lemma-what-is-in-it}.
The covering $\{V_i \to T\}_{i \in I}$ refines $\{T_i \to T\}_{i \in I}$
(because they are isomorphic).
Moreover, $\{V_i \to T\}_{i \in I}$ is combinatorially equivalent to a
covering $\{U_j \to T\}_{j \in J}$ of $T$ in the site
$\Sch_{Zar}$ by
Sets, Lemma \ref{sets-lemma-what-is-in-it}.
The covering $\{U_j \to T\}_{j \in J}$ is a refinement as in (1).
In the situation of (2), (3) each of the
schemes $T_i$ is isomorphic to an object of $\Sch_{fppf}$ by
Sets, Lemma \ref{sets-lemma-what-is-in-it},
and another application of
Sets, Lemma \ref{sets-lemma-coverings-site}
gives what we want.
\end{proof}

\begin{definition}
\label{definition-big-small-fppf}
Let $S$ be a scheme. Let $\Sch_{fppf}$ be a big fppf
site containing $S$.
\begin{enumerate}
\item The {\it big fppf site of $S$}, denoted
$(\Sch/S)_{fppf}$, is the site $\Sch_{fppf}/S$
introduced in Sites, Section \ref{sites-section-localize}.
\item The {\it big affine fppf site of $S$}, denoted
$(\textit{Aff}/S)_{fppf}$, is the full subcategory of
$(\Sch/S)_{fppf}$ whose objects are affine $U/S$.
A covering of $(\textit{Aff}/S)_{fppf}$ is any covering
$\{U_i \to U\}$ of $(\Sch/S)_{fppf}$ which is a
standard fppf covering.
\end{enumerate}
\end{definition}

\noindent
It is not completely clear that
the big affine fppf site is a site. We check this now.

\begin{lemma}
\label{lemma-verify-site-fppf}
Let $S$ be a scheme. Let $\Sch_{fppf}$ be a big fppf
site containing $S$. Then $(\textit{Aff}/S)_{fppf}$ is a site.
\end{lemma}

\begin{proof}
Let us show that $(\textit{Aff}/S)_{fppf}$ is a site.
Reasoning as in the proof of Lemma \ref{lemma-verify-site-etale}
it suffices to show that the collection
of standard fppf coverings of affines satisfies properties
(1), (2) and (3) of
Sites, Definition \ref{sites-definition-site}.
This is clear since for example, given a standard fppf
covering $\{T_i \to T\}_{i\in I}$ and for each
$i$ we have a standard fppf covering $\{T_{ij} \to T_i\}_{j\in J_i}$, then
$\{T_{ij} \to T\}_{i \in I, j\in J_i}$ is a standard fppf covering
because $\bigcup_{i\in I} J_i$ is finite and each $T_{ij}$ is affine.
\end{proof}

\begin{lemma}
\label{lemma-fibre-products-fppf}
Let $S$ be a scheme. Let $\Sch_{fppf}$ be a big fppf
site containing $S$. The underlying categories of the sites
$\Sch_{fppf}$, $(\Sch/S)_{fppf}$,
and $(\textit{Aff}/S)_{fppf}$ have fibre products.
In each case the obvious functor into the category $\Sch$ of
all schemes commutes with taking fibre products. The category
$(\Sch/S)_{fppf}$ has a final object, namely $S/S$.
\end{lemma}

\begin{proof}
For $\Sch_{fppf}$ it is true by construction, see
Sets, Lemma \ref{sets-lemma-what-is-in-it}.
Suppose we have $U \to S$, $V \to U$, $W \to U$ morphisms
of schemes with $U, V, W \in \Ob(\Sch_{fppf})$.
The fibre product $V \times_U W$ in $\Sch_{fppf}$
is a fibre product in $\Sch$ and
is the fibre product of $V/S$ with $W/S$ over $U/S$ in
the category of all schemes over $S$, and hence also a
fibre product in $(\Sch/S)_{fppf}$.
This proves the result for $(\Sch/S)_{fppf}$.
If $U, V, W$ are affine, so is $V \times_U W$ and hence the
result for $(\textit{Aff}/S)_{fppf}$.
\end{proof}

\noindent
Next, we check that the big affine site defines the same
topos as the big site.

\begin{lemma}
\label{lemma-affine-big-site-fppf}
Let $S$ be a scheme. Let $\Sch_{fppf}$ be a big fppf
site containing $S$.
The functor $(\textit{Aff}/S)_{fppf} \to (\Sch/S)_{fppf}$
is cocontinuous and induces an equivalence of topoi from
$\Sh((\textit{Aff}/S)_{fppf})$ to
$\Sh((\Sch/S)_{fppf})$.
\end{lemma}

\begin{proof}
The notion of a special cocontinuous functor is introduced in
Sites, Definition \ref{sites-definition-special-cocontinuous-functor}.
Thus we have to verify assumptions (1) -- (5) of
Sites, Lemma \ref{sites-lemma-equivalence}.
Denote the inclusion functor
$u : (\textit{Aff}/S)_{fppf} \to (\Sch/S)_{fppf}$.
Being cocontinuous just means that any fppf covering of
$T/S$, $T$ affine, can be refined by a standard fppf covering of $T$.
This is the content of
Lemma \ref{lemma-fppf-affine}.
Hence (1) holds. We see $u$ is continuous simply because a standard
fppf covering is a fppf covering. Hence (2) holds.
Parts (3) and (4) follow immediately from the fact that $u$ is
fully faithful. And finally condition (5) follows from the
fact that every scheme has an affine open covering.
\end{proof}

\noindent
Next, we establish some relationships between the topoi
associated to these sites.

\begin{lemma}
\label{lemma-morphism-big-fppf}
Let $\Sch_{fppf}$ be a big fppf site.
Let $f : T \to S$ be a morphism in $\Sch_{fppf}$.
The functor
$$
u : (\Sch/T)_{fppf} \longrightarrow (\Sch/S)_{fppf},
\quad
V/T \longmapsto V/S
$$
is cocontinuous, and has a continuous right adjoint
$$
v : (\Sch/S)_{fppf} \longrightarrow (\Sch/T)_{fppf},
\quad
(U \to S) \longmapsto (U \times_S T \to T).
$$
They induce the same morphism of topoi
$$
f_{big} :
\Sh((\Sch/T)_{fppf})
\longrightarrow
\Sh((\Sch/S)_{fppf})
$$
We have $f_{big}^{-1}(\mathcal{G})(U/T) = \mathcal{G}(U/S)$.
We have $f_{big, *}(\mathcal{F})(U/S) = \mathcal{F}(U \times_S T/T)$.
Also, $f_{big}^{-1}$ has a left adjoint $f_{big!}$ which commutes with
fibre products and equalizers.
\end{lemma}

\begin{proof}
The functor $u$ is cocontinuous, continuous, and commutes with fibre products
and equalizers. Hence
Sites, Lemmas \ref{sites-lemma-when-shriek} and
\ref{sites-lemma-preserve-equalizers}
apply and we deduce the formula
for $f_{big}^{-1}$ and the existence of $f_{big!}$. Moreover,
the functor $v$ is a right adjoint because given $U/T$ and $V/S$
we have $\Mor_S(u(U), V) = \Mor_T(U, V \times_S T)$
as desired. Thus we may apply
Sites, Lemmas \ref{sites-lemma-have-functor-other-way} and
\ref{sites-lemma-have-functor-other-way-morphism} to get the
formula for $f_{big, *}$.
\end{proof}

\begin{lemma}
\label{lemma-composition-fppf}
Given schemes $X$, $Y$, $Y$ in $(\Sch/S)_{fppf}$
and morphisms $f : X \to Y$, $g : Y \to Z$ we have
$g_{big} \circ f_{big} = (g \circ f)_{big}$.
\end{lemma}

\begin{proof}
This follows from the simple description of pushforward
and pullback for the functors on the big sites from
Lemma \ref{lemma-morphism-big-fppf}.
\end{proof}




































































































\section{The fpqc topology}
\label{section-fpqc}

\begin{definition}
\label{definition-fpqc-covering}
Let $T$ be a scheme. An {\it fpqc covering of $T$} is a family
of morphisms $\{f_i : T_i \to T\}_{i \in I}$ of schemes
such that each $f_i$ is flat and such that for every affine open
$U \subset T$ there exists $n \geq 0$, a map
$a : \{1, \ldots, n\} \to I$ and affine opens
$V_j \subset T_{a(j)}$, $j = 1, \ldots, n$
with $\bigcup_{j = 1}^n f_{a(j)}(V_j) = U$.
\end{definition}

\noindent
To be sure this condition implies that $T = \bigcup f_i(T_i)$.
It is slightly harder to recognize an fpqc covering, hence we provide
some lemmas to do so.

\begin{lemma}
\label{lemma-recognize-fpqc-covering}
Let $T$ be a scheme. Let $\{f_i : T_i \to T\}_{i \in I}$ be a family of
morphisms of schemes with target $T$. The following are equivalent
\begin{enumerate}
\item $\{f_i : T_i \to T\}_{i \in I}$ is an fpqc covering,
\item each $f_i$ is flat and for every affine open $U \subset T$
there exist quasi-compact opens
$U_i \subset T_i$ which are almost all empty,
such that $U = \bigcup f_i(U_i)$,
\item each $f_i$ is flat and there exists an affine open covering
$T = \bigcup_{\alpha \in A} U_\alpha$ and for each $\alpha \in A$
there exist $i_{\alpha, 1}, \ldots, i_{\alpha, n(\alpha)} \in I$
and quasi-compact opens $U_{\alpha, j} \subset T_{i_{\alpha, j}}$ such that
$U_\alpha =
\bigcup_{j = 1, \ldots, n(\alpha)} f_{i_{\alpha, j}}(U_{\alpha, j})$.
\end{enumerate}
If $T$ is quasi-separated, these are also equivalent to
\begin{enumerate}
\item[(4)] each $f_i$ is flat, and for every $t \in T$ there exist
$i_1, \ldots, i_n \in I$ and quasi-compact opens $U_j \subset T_{i_j}$
such that $\bigcup_{j = 1, \ldots, n} f_{i_j}(U_j)$ is a
(not necessarily open) neighbourhood of $t$ in $T$.
\end{enumerate}
\end{lemma}

\begin{proof}
We omit the proof of the equivalence of (1), (2), and (3).
From now on assume $T$ is quasi-separated.
We prove (4) implies (2). Let $U \subset T$ be an affine open.
To prove (2) it suffices to show that for every $t \in U$ there exist
finitely many quasi-compact opens $U_j \subset T_{i_j}$ such that
$f_{i_j}(U_j) \subset U$ and such that $\bigcup f_{i_j}(U_j)$
is a neighbourhood of $t$ in $U$. By assumption there do exist
finitely many quasi-compact opens $U'_j \subset T_{i_j}$ such that
such that $\bigcup f_{i_j}(U'_j)$ is a neighbourhood of $t$ in $T$.
Since $T$ is quasi-separated we see that $U_j = U'_j \cap f_j^{-1}(U)$
is quasi-compact open as desired. Since it is clear that (2) implies
(4) the proof is finished.
\end{proof}

\begin{lemma}
\label{lemma-disjoint-union-is-fpqc-covering}
Let $T$ be a scheme. Let $\{f_i : T_i \to T\}_{i \in I}$ be a family of
morphisms of schemes with target $T$. The following are equivalent
\begin{enumerate}
\item $\{f_i : T_i \to T\}_{i \in I}$ is an fpqc covering, and
\item setting $T' = \coprod_{i \in I} T_i$, and $f = \coprod_{i \in I} f_i$
the family $\{f : T' \to T\}$ is an fpqc covering.
\end{enumerate}
\end{lemma}

\begin{proof}
Suppose that $U \subset T$ is an affine open. If (1) holds, then we find
$i_1, \ldots, i_n \in I$ and affine opens $U_j \subset T_{i_j}$ such that
$U = \bigcup_{j = 1, \ldots, n} f_{i_j}(U_j)$. Then
$U_1 \amalg \ldots \amalg U_n \subset T'$ is a quasi-compact open surjecting
onto $U$. Thus $\{f : T' \to T\}$ is an fpqc covering by
Lemma \ref{lemma-recognize-fpqc-covering}.
Conversely, if (2) holds then there exists a quasi-compact open
$U' \subset T'$ with $U = f(U')$. Then $U_j = U' \cap T_j$ is quasi-compact
open in $T_j$ and empty for almost all $j$. By
Lemma \ref{lemma-recognize-fpqc-covering} we see that (1) holds.
\end{proof}

\begin{lemma}
\label{lemma-family-flat-dominated-covering}
Let $T$ be a scheme. Let $\{f_i : T_i \to T\}_{i \in I}$ be a family of
morphisms of schemes with target $T$. Assume that
\begin{enumerate}
\item each $f_i$ is flat, and
\item the family $\{f_i : T_i \to T\}_{i \in I}$ can be refined by a
fpqc covering of $T$.
\end{enumerate}
Then $\{f_i : T_i \to T\}_{i \in I}$ is a fpqc covering of $T$.
\end{lemma}

\begin{proof}
Let $\{g_j : X_j \to T\}_{j \in J}$ be an fpqc covering refining
$\{f_i : T_i \to T\}$. Suppose that $U \subset T$ is affine open.
Choose $j_1, \ldots, j_m \in J$ and $V_k \subset X_{j_k}$ affine
open such that $U = \bigcup g_{j_k}(V_k)$. For each $j$ pick $i_j \in I$
and a morphism $h_j : X_j \to T_{i_j}$ such that $g_j = f_{i_j} \circ h_j$.
Since $h_{j_k}(V_k)$ is quasi-compact we can find a quasi-compact
open $h_{j_k}(V_k) \subset U_k \subset f_{i_{j_k}}^{-1}(U)$.
Then $U = \bigcup f_{i_{j_k}}(U_k)$. We conclude that
$\{f_i : T_i \to T\}_{i \in I}$ is an fpqc covering by
Lemma \ref{lemma-recognize-fpqc-covering}.
\end{proof}

\begin{lemma}
\label{lemma-family-flat-fpqc-local-covering}
Let $T$ be a scheme. Let $\{f_i : T_i \to T\}_{i \in I}$ be a family of
morphisms of schemes with target $T$. Assume that
\begin{enumerate}
\item each $f_i$ is flat, and
\item there exists an fpqc covering
$\{g_j : S_j \to T\}_{j \in J}$ such that each
$\{S_j \times_T T_i \to S_j\}_{i \in I}$ is an fpqc covering.
\end{enumerate}
Then $\{f_i : T_i \to T\}_{i \in I}$ is a fpqc covering of $T$.
\end{lemma}

\begin{proof}
We will use Lemma \ref{lemma-recognize-fpqc-covering} without further
mention. Let $U \subset T$ be an affine open. By (2) we can find
quasi-compact opens $V_j \subset S_j$ for $j \in J$, almost all empty, such that
$U = \bigcup g_j(V_j)$. Then for each $j$ we can choose quasi-compact
opens $W_{ij} \subset S_j \times_T T_i$ for $i \in I$, almost all empty,
with $V_j = \bigcup_i \text{pr}_1(W_{ij})$. Thus
$\{S_j \times_T T_i \to T\}$ is an fpqc covering.
Since this covering refines $\{f_i : T_i \to T\}$ we conclude by
Lemma \ref{lemma-family-flat-dominated-covering}.
\end{proof}

\begin{lemma}
\label{lemma-zariski-etale-smooth-syntomic-fppf-fpqc}
Any fppf covering is an fpqc covering, and a fortiori,
any syntomic, smooth, \'etale or Zariski covering is an fpqc covering.
\end{lemma}

\begin{proof}
We will show that an fppf covering is an fpqc covering, and then the
rest follows from
Lemma \ref{lemma-zariski-etale-smooth-syntomic-fppf}.
Let $\{f_i : U_i \to U\}_{i \in I}$ be an fppf covering.
By definition this means that the $f_i$ are flat which checks the first
condition of Definition \ref{definition-fpqc-covering}. To check the
second, let $V \subset U$ be an affine open subset.
Write $f_i^{-1}(V) = \bigcup_{j \in J_i} V_{ij}$
for some affine opens $V_{ij} \subset U_i$. Since each $f_i$ is open
(Morphisms, Lemma \ref{morphisms-lemma-fppf-open}), we see that
$V = \bigcup_{i\in I} \bigcup_{j \in J_i} f_i(V_{ij})$
is an open covering of $V$.
Since $V$ is quasi-compact, this covering has a finite
refinement. This finishes the proof.
\end{proof}

\noindent
The fpqc\footnote{The letters fpqc stand for
``fid\`element plat quasi-compacte''.}
topology cannot be treated in the same way as the fppf
topology\footnote{A more precise statement would be that the analogue of
Lemma \ref{lemma-fppf-induced} for the fpqc topology does not hold.}.
Namely, suppose that $R$ is a nonzero ring. We will see in
Lemma \ref{lemma-no-set-of-fpqc-covers-is-initial}
that there does not exist a set $A$ of fpqc-coverings of $\Spec(R)$
such that every fpqc-covering can be refined by an element of $A$.
If $R = k$ is a field, then the reason for this unboundedness is that
there does not exist a field extension of $k$ such that every field extension
of $k$ is contained in it.

\medskip\noindent
If you ignore set theoretic difficulties, then you run into presheaves
which do not have a sheafification, see
\cite[Theorem 5.5]{Waterhouse-fpqc-sheafification}.
A mildly interesting option is to consider only those faithfully flat ring
extensions $R \to R'$ where the cardinality of $R'$ is suitably bounded.
(And if you consider all schemes in a fixed universe as in SGA4 then you
are bounding the cardinality by a strongly inaccessible cardinal.)
However, it is not so clear what happens if you change the cardinal
to a bigger one.

\medskip\noindent
For these reasons we do not introduce fpqc sites and we will not consider
cohomology with respect to the fpqc-topology.

\medskip\noindent
On the other hand, given a contravariant functor
$F : \Sch^{opp} \to \textit{Sets}$
it does make sense to ask whether $F$ satisfies the sheaf property
for the fpqc topology, see below.
Moreover, we can wonder about descent of object
in the fpqc topology, etc. Simply put, for certain results the correct
generality is to work with fpqc coverings.

\begin{lemma}
\label{lemma-fpqc}
Let $T$ be a scheme.
\begin{enumerate}
\item If $T' \to T$ is an isomorphism then $\{T' \to T\}$
is an fpqc covering of $T$.
\item If $\{T_i \to T\}_{i\in I}$ is an fpqc covering and for each
$i$ we have an fpqc covering $\{T_{ij} \to T_i\}_{j\in J_i}$, then
$\{T_{ij} \to T\}_{i \in I, j\in J_i}$ is an fpqc covering.
\item If $\{T_i \to T\}_{i\in I}$ is an fpqc covering
and $T' \to T$ is a morphism of schemes then
$\{T' \times_T T_i \to T'\}_{i\in I}$ is an fpqc covering.
\end{enumerate}
\end{lemma}

\begin{proof}
Part (1) is immediate. Recall that the composition of flat morphisms
is flat and that the base change of a flat morphism is flat
(Morphisms, Lemmas \ref{morphisms-lemma-base-change-flat} and
\ref{morphisms-lemma-composition-flat}).
Thus we can apply Lemma \ref{lemma-recognize-fpqc-covering}
in each case to check that our families of morphisms are fpqc coverings.

\medskip\noindent
Proof of (2). Assume $\{T_i \to T\}_{i\in I}$ is an fpqc covering and for each
$i$ we have an fpqc covering $\{f_{ij} : T_{ij} \to T_i\}_{j\in J_i}$.
Let $U \subset T$ be an affine open. We can find
quasi-compact opens $U_i \subset T_i$ for $i \in I$, almost all empty,
such that $U = \bigcup f_i(U_i)$. Then for each $i$ we can choose
quasi-compact opens $W_{ij} \subset T_{ij}$ for $j \in J_i$, almost all empty,
with $U_i = \bigcup_j f_{ij}(U_{ij})$. Thus
$\{T_{ij} \to T\}$ is an fpqc covering.

\medskip\noindent
Proof of (3). Assume $\{T_i \to T\}_{i\in I}$ is an fpqc covering
and $T' \to T$ is a morphism of schemes. Let $U' \subset T'$ be an affine
open which maps into the affine open $U \subset T$. Choose
quasi-compact opens $U_i \subset T_i$, almost all empty,
such that $U = \bigcup f_i(U_i)$. Then $U' \times_U U_i$ is
a quasi-compact open of $T' \times_T T_i$ and
$U' = \bigcup \text{pr}_1(U' \times_U U_i)$. Since $T'$
can be covered by such affine opens $U' \subset T'$ we see
that $\{T' \times_T T_i \to T'\}_{i\in I}$ is an fpqc covering by
Lemma \ref{lemma-recognize-fpqc-covering}
\end{proof}

\begin{lemma}
\label{lemma-fpqc-affine}
Let $T$ be an affine scheme.
Let $\{T_i \to T\}_{i \in I}$ be an fpqc covering of $T$.
Then there exists an fpqc covering
$\{U_j \to T\}_{j = 1, \ldots, n}$ which is a refinement
of $\{T_i \to T\}_{i \in I}$ such that each $U_j$ is an affine
scheme. Moreover, we may choose each $U_j$ to be open affine
in one of the $T_i$.
\end{lemma}

\begin{proof}
This follows directly from the definition.
\end{proof}

\begin{definition}
\label{definition-standard-fpqc}
Let $T$ be an affine scheme. A {\it standard fpqc covering}
of $T$ is a family $\{f_j : U_j \to T\}_{j = 1, \ldots, n}$
with each $U_j$ is affine, flat over $T$ and $T = \bigcup f_j(U_j)$.
\end{definition}

\noindent
Since we do not introduce the affine site we have to show directly
that the collection of all standard fpqc coverings satisfies the
axioms.

\begin{lemma}
\label{lemma-fpqc-affine-axioms}
Let $T$ be an affine scheme.
\begin{enumerate}
\item If $T' \to T$ is an isomorphism then $\{T' \to T\}$
is a standard fpqc covering of $T$.
\item If $\{T_i \to T\}_{i\in I}$ is a standard fpqc covering and for each
$i$ we have a standard fpqc covering $\{T_{ij} \to T_i\}_{j\in J_i}$, then
$\{T_{ij} \to T\}_{i \in I, j\in J_i}$ is a standard fpqc covering.
\item If $\{T_i \to T\}_{i\in I}$ is a standard fpqc covering
and $T' \to T$ is a morphism of affine schemes then
$\{T' \times_T T_i \to T'\}_{i\in I}$ is a standard fpqc covering.
\end{enumerate}
\end{lemma}

\begin{proof}
This follows formally from the fact that compositions and base changes
of flat morphisms are flat
(Morphisms, Lemmas \ref{morphisms-lemma-base-change-flat} and
\ref{morphisms-lemma-composition-flat})
and that fibre products of affine schemes are affine
(Schemes, Lemma \ref{schemes-lemma-fibre-product-affines}).
\end{proof}

\begin{lemma}
\label{lemma-fpqc-covering-affines-mapping-in}
Let $T$ be a scheme. Let $\{f_i : T_i \to T\}_{i \in I}$ be a family of
morphisms of schemes with target $T$. Assume that
\begin{enumerate}
\item each $f_i$ is flat, and
\item every affine scheme
$Z$ and morphism $h : Z \to T$ there exists a standard fpqc covering
$\{Z_j \to Z\}_{j = 1, \ldots, n}$ which refines the family
$\{T_i \times_T Z \to Z\}_{i \in I}$.
\end{enumerate}
Then $\{f_i : T_i \to T\}_{i \in I}$ is a fpqc covering of $T$.
\end{lemma}

\begin{proof}
Let $T = \bigcup U_\alpha$ be an affine open covering.
For each $\alpha$ the pullback family $\{T_i \times_T U_\alpha \to U_\alpha\}$
can be refined by a standard fpqc covering, hence is an
fpqc covering by Lemma
\ref{lemma-family-flat-dominated-covering}.
As $\{U_\alpha \to T\}$ is an fpqc covering we conclude that
$\{T_i \to T\}$ is an fpqc covering by
Lemma \ref{lemma-family-flat-fpqc-local-covering}.
\end{proof}

\begin{definition}
\label{definition-sheaf-property-fpqc}
Let $F$ be a contravariant functor on the category
of schemes with values in sets.
\begin{enumerate}
\item Let $\{U_i \to T\}_{i \in I}$ be a family of morphisms
of schemes with fixed target.
We say that $F$ {\it satisfies the sheaf property for the given family}
if for any collection of elements $\xi_i \in F(U_i)$ such that
$\xi_i|_{U_i \times_T U_j} = \xi_j|_{U_i \times_T U_j}$
there exists a unique element
$\xi \in F(T)$ such that $\xi_i = \xi|_{U_i}$ in $F(U_i)$.
\item We say that $F$ {\it satisfies the sheaf property for the
fpqc topology} if it satisfies the sheaf property for any
fpqc covering.
\end{enumerate}
\end{definition}

\noindent
We try to avoid using the terminology ``$F$ is a sheaf'' in this
situation since we are not defining a category of fpqc sheaves
as we explained above.

\begin{lemma}
\label{lemma-sheaf-property-fpqc}
Let $F$ be a contravariant functor on the category
of schemes with values in sets. Then $F$ satisfies
the sheaf property for the fpqc topology if and only
if it satisfies
\begin{enumerate}
\item the sheaf property for every Zariski covering, and
\item the sheaf property for any standard fpqc covering.
\end{enumerate}
Moreover, in the presence of (1) property (2) is equivalent to
property
\begin{enumerate}
\item[(2')] the sheaf property for $\{V \to U\}$
with $V$, $U$ affine and $V \to U$ faithfully flat.
\end{enumerate}
\end{lemma}

\begin{proof}
Assume (1) and (2) hold.
Let $\{f_i : T_i \to T\}_{i \in I}$ be an fpqc covering. Let $s_i \in F(T_i)$
be a family of elements such that $s_i$ and $s_j$ map to the same element
of $F(T_i \times_T T_j)$. Let $W \subset T$ be the maximal open subset
such that there exists a unique $s \in F(W)$ with
$s|_{f_i^{-1}(W)} = s_i|_{f_i^{-1}(W)}$ for all $i$.
Such a maximal open exists because $F$ satisfies the
sheaf property for Zariski coverings; in fact $W$ is the
union of all opens with this property. Let $t \in T$.
We will show $t \in W$. To do this we pick an affine open
$t \in U \subset T$ and we will show there is a unique
$s \in F(U)$ with
$s|_{f_i^{-1}(U)} = s_i|_{f_i^{-1}(U)}$ for all $i$.

\medskip\noindent
By Lemma \ref{lemma-fpqc-affine} we can find a standard fpqc covering
$\{U_j \to U\}_{j = 1, \ldots, n}$ refining $\{U \times_T T_i \to U\}$,
say by morphisms $h_j : U_j \to T_{i_j}$. By (2) we obtain a unique element
$s \in F(U)$ such that $s|_{U_j} = F(h_j)(s_{i_j})$. Note that for any
scheme $V \to U$ over $U$ there is a unique section $s_V \in F(V)$
which restricts to $F(h_j \circ \text{pr}_2)(s_{i_j})$ on
$V \times_U U_j$ for $j = 1, \ldots, n$. Namely, this is true if $V$
is affine by (2) as $\{V \times_U U_j \to V\}$ is a standard fpqc covering
and in general this follows from (1) and the affine case by choosing an
affine open covering of $V$. In particular, $s_V = s|_V$.
Now, taking $V = U \times_T T_i$ and using that
$s_{i_j}|_{T_{i_j} \times_T T_i} = s_i|_{T_{i_j} \times_T T_i}$
we conclude that $s|_{U \times_T T_i} = s_V = s_i|_{U \times_T T_i}$
which is what we had to show.

\medskip\noindent
Proof of the equivalence of (2) and (2') in the presence of (1).
Suppose $\{T_i \to T\}$ is a standard fpqc covering, then
$\coprod T_i \to T$ is a faithfully flat morphism of affine schemes.
In the presence of (1) we have $F(\coprod T_i) = \prod F(T_i)$
and similarly
$F((\coprod T_i) \times_T (\coprod T_i)) = \prod F(T_i \times_T T_{i'})$.
Thus the sheaf condition for $\{T_i \to T\}$ and $\{\coprod T_i \to T\}$
is the same.
\end{proof}

\noindent
The following lemma is here just to point out set theoretical difficulties
do indeed arise and should be ignored by most readers.

\begin{lemma}
\label{lemma-no-set-of-fpqc-covers-is-initial}
Let $R$ be a nonzero ring. There does not exist a set $A$ of
fpqc-coverings of $\Spec(R)$ such that every fpqc-covering can
be refined by an element of $A$.
\end{lemma}

\begin{proof}
Let us first explain this when $R = k$ is a field. For any set $I$ consider
the purely transcendental field extension
$k \subset k_I = k(\{t_i\}_{i \in I})$. Since $k \to k_I$ is faithfully flat
we see that $\{\Spec(k_I) \to \Spec(k)\}$ is an fpqc covering.
Let $A$ be a set and for each $\alpha \in A$ let
$\mathcal{U}_\alpha = \{S_{\alpha, j} \to \Spec(k)\}_{j \in J_\alpha}$ be an
fpqc covering. If $\mathcal{U}_\alpha$ refines $\{\Spec(k_I) \to \Spec(k)\}$
then the morphisms $S_{\alpha, j} \to \Spec(k)$ factor through
$\Spec(k_I)$. Since $\mathcal{U}_\alpha$ is a covering,
at least some $S_{\alpha, j}$ is nonempty. Pick a point
point $s \in S_{\alpha, j}$. Since we have the factorization
$S_{\alpha, j} \to \Spec(k_I) \to \Spec(k)$
we obtain a homomorphism of fields $k_I \to \kappa(s)$.
In particular, we see that the cardinality of $\kappa(s)$
is at least the cardinality of $I$. Thus if we take $I$ to be a set
of cardinality bigger than the cardinalities of the residue fields
of all the schemes $S_{\alpha, j}$, then such a factorization does
not exist and the lemma holds for $R = k$.

\medskip\noindent
General case. Since $R$ is nonzero it has a maximal prime ideal
$\mathfrak m$ with residue field $\kappa$. Let $I$ be a set and
consider $R_I = S_I^{-1} R[\{t_i\}_{i \in I}]$
where $S_I \subset R[\{t_i\}_{i \in I}]$ is the multiplicative
subset of $f \in R[\{t_i\}_{i \in I}]$ such that $f$ maps to
a nonzero element of $R/\mathfrak p[\{t_i\}_{i \in I}$ for
all primes $\mathfrak p$ of $R$. Then $R_I$ is a faithfully
flat $R$-algebra and $\{\Spec(R_I) \to \Spec(R)\}$ is an
fpqc covering. We leave it as an exercise to the reader to show that
$R_I \otimes_R \kappa \cong \kappa(\{t_i\}_{i \in I}) = \kappa_I$
with notation as above (hint: use that $R \to \kappa$ is surjective
and that any $f \in R[\{t_i\}_{i \in I}]$ one of whose monomials occurs
with coefficient $1$ is an element of $S_I$). Let $A$ be a set and
for each $\alpha \in A$ let
$\mathcal{U}_\alpha = \{S_{\alpha, j} \to \Spec(R)\}_{j \in J_\alpha}$ be an
fpqc covering. If $\mathcal{U}_\alpha$ refines $\{\Spec(R_I) \to \Spec(R)\}$,
then by base change we conclude that
$\{S_{\alpha, j} \times_{\Spec(R)} \Spec(\kappa) \to \Spec(\kappa)\}$
refines $\{\Spec(\kappa_I) \to \Spec(\kappa)\}$.
Hence by the result of the previous paragraph, there exists an $I$
such that this is not the case and the lemma is proved.
\end{proof}











\section{Change of topologies}
\label{section-change-topologies}

\noindent
Let $f : X \to Y$ be a morphism of schemes over a base scheme $S$.
In this case we have the following morphisms of sites
(with suitable choices of sites as in Remark \ref{remark-choice-sites}
below):
\begin{enumerate}
\item $(\Sch/X)_{fppf} \longrightarrow (\Sch/Y)_{fppf}$,
\item $(\Sch/X)_{fppf} \longrightarrow (\Sch/Y)_{syntomic}$,
\item $(\Sch/X)_{fppf} \longrightarrow (\Sch/Y)_{smooth}$,
\item $(\Sch/X)_{fppf} \longrightarrow
(\Sch/Y)_\etale$,
\item $(\Sch/X)_{fppf} \longrightarrow (\Sch/Y)_{Zar}$,
\item $(\Sch/X)_{syntomic} \longrightarrow (\Sch/Y)_{syntomic}$,
\item $(\Sch/X)_{syntomic} \longrightarrow (\Sch/Y)_{smooth}$,
\item $(\Sch/X)_{syntomic} \longrightarrow
(\Sch/Y)_\etale$,
\item $(\Sch/X)_{syntomic} \longrightarrow (\Sch/Y)_{Zar}$,
\item $(\Sch/X)_{smooth} \longrightarrow (\Sch/Y)_{smooth}$,
\item $(\Sch/X)_{smooth} \longrightarrow
(\Sch/Y)_\etale$,
\item $(\Sch/X)_{smooth} \longrightarrow (\Sch/Y)_{Zar}$,
\item $(\Sch/X)_\etale \longrightarrow
(\Sch/Y)_\etale$,
\item $(\Sch/X)_\etale \longrightarrow (\Sch/Y)_{Zar}$,
\item $(\Sch/X)_{Zar} \longrightarrow (\Sch/Y)_{Zar}$,
\item $(\Sch/X)_{fppf} \longrightarrow Y_\etale$,
\item $(\Sch/X)_{syntomic} \longrightarrow Y_\etale$,
\item $(\Sch/X)_{smooth} \longrightarrow Y_\etale$,
\item $(\Sch/X)_\etale \longrightarrow Y_\etale$,
\item $(\Sch/X)_{fppf} \longrightarrow Y_{Zar}$,
\item $(\Sch/X)_{syntomic} \longrightarrow Y_{Zar}$,
\item $(\Sch/X)_{smooth} \longrightarrow Y_{Zar}$,
\item $(\Sch/X)_\etale \longrightarrow Y_{Zar}$,
\item $(\Sch/X)_{Zar} \longrightarrow Y_{Zar}$,
\item $X_\etale \longrightarrow Y_\etale$,
\item $X_\etale \longrightarrow Y_{Zar}$,
\item $X_{Zar} \longrightarrow Y_{Zar}$,
\end{enumerate}
In each case the underlying continuous functor
$\Sch/Y \to \Sch/X$, or
$Y_\tau \to \Sch/X$
is the functor $Y'/Y \mapsto X \times_Y Y'/X$. Namely, in the sections
above we have seen the morphisms
$f_{big} : (\Sch/X)_\tau \to (\Sch/Y)_\tau$
and
$f_{small} : X_\tau \to Y_\tau$
for $\tau$ as above.
We also have seen the morphisms of sites
$\pi_Y : (\Sch/Y)_\tau \to Y_\tau$ for
$\tau \in \{\etale, Zariski\}$.
On the other hand, it is clear that the identity functor
$(\Sch/X)_\tau \to (\Sch/X)_{\tau'}$ defines
a morphism of sites when $\tau$ is a stronger topology than
$\tau'$. Hence composing these gives the list of possible morphisms
above.

\medskip\noindent
Because of the simple description of the underlying functor it
is clear that given morphisms of schemes $X \to Y \to Z$ the
composition of two of the morphisms of sites above, e.g.,
$$
(\Sch/X)_{\tau_0} \longrightarrow
(\Sch/Y)_{\tau_1} \longrightarrow
(\Sch/Z)_{\tau_2}
$$
is the corresponding morphism of sites associated to the morphism
of schemes $X \to Z$.

\begin{remark}
\label{remark-choice-sites}
Take any category $\Sch_\alpha$ constructed as in
Sets, Lemma \ref{sets-lemma-construct-category}
starting with the set of schemes $\{X, Y, S\}$. Choose any set of
coverings $\text{Cov}_{fppf}$ on $\Sch_\alpha$ as in
Sets, Lemma \ref{sets-lemma-coverings-site}
starting with the category $\Sch_\alpha$ and the class of fppf
coverings. Let $\Sch_{fppf}$ denote the big fppf site so
obtained. Next, for $\tau \in \{Zariski, \etale, smooth, syntomic\}$
let $\Sch_\tau$ have the same underlying category as
$\Sch_{fppf}$ with coverings
$\text{Cov}_\tau \subset \text{Cov}_{fppf}$ simply the subset of
$\tau$-coverings. It is straightforward to check that this gives rise
to a big site $\Sch_\tau$.
\end{remark}









\section{Change of big sites}
\label{section-change-alpha}

\noindent
In this section we explain what happens on changing the big
Zariski/fppf/\'etale sites.

\medskip\noindent
Let $\tau, \tau' \in \{Zariski, \etale, smooth, syntomic, fppf\}$.
Given two big sites $\Sch_\tau$ and $\Sch'_{\tau'}$
we say that
{\it $\Sch_\tau$ is contained in $\Sch'_{\tau'}$} if
$\Ob(\Sch_\tau) \subset \Ob(\Sch'_{\tau'})$
and
$\text{Cov}(\Sch_\tau) \subset \text{Cov}(\Sch'_{\tau'})$.
In this case $\tau$ is stronger than $\tau'$, for example, no fppf
site can be contained in an \'etale site.

\begin{lemma}
\label{lemma-contained-in}
Any set of big Zariski sites is contained in a common big Zariski site.
The same is true, mutatis mutandis, for big fppf and big \'etale sites.
\end{lemma}

\begin{proof}
This is true because the union of a set of sets is a set, and the
constructions in Sets, Lemmas \ref{sets-lemma-construct-category} and
\ref{sets-lemma-coverings-site}
allow one to start with any initially given set of schemes
and coverings.
\end{proof}

\begin{lemma}
\label{lemma-change-alpha}
Let $\tau \in \{Zariski, \etale, smooth, syntomic, fppf\}$.
Suppose given big sites $\Sch_\tau$ and $\Sch'_\tau$.
Assume that $\Sch_\tau$ is contained in $\Sch'_\tau$.
The inclusion functor $\Sch_\tau \to \Sch'_\tau$ satisfies
the assumptions of Sites, Lemma \ref{sites-lemma-bigger-site}.
There are morphisms of topoi
\begin{eqnarray*}
g : \Sh(\Sch_\tau) &
\longrightarrow &
\Sh(\Sch'_\tau) \\
f : \Sh(\Sch'_\tau) &
\longrightarrow &
\Sh(\Sch_\tau)
\end{eqnarray*}
such that $f \circ g \cong \text{id}$. For any object $S$
of $\Sch_\tau$ the inclusion functor
$(\Sch/S)_\tau \to (\Sch'/S)_\tau$ satisfies
the assumptions of Sites, Lemma \ref{sites-lemma-bigger-site}
also. Hence similarly we obtain morphisms
\begin{eqnarray*}
g : \Sh((\Sch/S)_\tau) &
\longrightarrow &
\Sh((\Sch'/S)_\tau) \\
f : \Sh((\Sch'/S)_\tau) &
\longrightarrow &
\Sh((\Sch/S)_\tau)
\end{eqnarray*}
with $f \circ g \cong \text{id}$.
\end{lemma}

\begin{proof}
Assumptions (b), (c), and (e) of
Sites, Lemma \ref{sites-lemma-bigger-site}
are immediate for the functors
$\Sch_\tau \to \Sch'_\tau$ and
$(\Sch/S)_\tau \to (\Sch'/S)_\tau$. Property (a) holds by
Lemma \ref{lemma-zariski-induced},
\ref{lemma-etale-induced},
\ref{lemma-smooth-induced},
\ref{lemma-syntomic-induced}, or
\ref{lemma-fppf-induced}.
Property (d) holds because
fibre products in the categories $\Sch_\tau$, $\Sch'_\tau$
exist and are compatible with fibre products in the category of schemes.
\end{proof}

\noindent
Discussion:
The functor $g^{-1} = f_*$ is simply the restriction functor which associates
to a sheaf $\mathcal{G}$ on $\Sch'_\tau$ the restriction
$\mathcal{G}|_{\Sch_\tau}$. Hence this lemma simply says that given
any sheaf of sets $\mathcal{F}$ on $\Sch_\tau$ there exists a
canonical sheaf $\mathcal{F}'$ on $\Sch'_\tau$ such that
$\mathcal{F}|_{\Sch'_\tau} = \mathcal{F}'$. In fact the sheaf
$\mathcal{F}'$ has the following description: it is the sheafification
of the presheaf
$$
\Sch'_\tau \longrightarrow \textit{Sets}, \quad
V \longmapsto \colim_{V \to U} \mathcal{F}(U)
$$
where $U$ is an object of $\Sch_\tau$. This is true because
$\mathcal{F}' = f^{-1}\mathcal{F} = (u_p\mathcal{F})^\#$ according to
Sites, Lemmas \ref{sites-lemma-when-shriek} and \ref{sites-lemma-bigger-site}.


\input{chapters}


\bibliography{my}
\bibliographystyle{amsalpha}

\end{document}
