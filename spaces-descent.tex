\input{preamble}

% OK, start here.
%
\begin{document}

\title{Descent and Algebraic Spaces}

\maketitle

\phantomsection
\label{section-phantom}

\tableofcontents

\section{Introduction}
\label{section-introduction}

\noindent
In the chapter on topologies on algebraic spaces (see
Topologies on Spaces, Section \ref{spaces-topologies-section-introduction})
we introduced \'etale, fppf, smooth, syntomic and fpqc coverings of
algebraic spaces.
In this chapter we discuss what kind of structures over algebraic spaces
can be descended through such coverings.
See for example \cite{Gr-I}, \cite{Gr-II}, \cite{Gr-III},
\cite{Gr-IV}, \cite{Gr-V}, and \cite{Gr-VI}.



\section{Conventions}
\label{section-conventions}

\noindent
The standing assumption is that all schemes are contained in
a big fppf site $\Sch_{fppf}$. And all rings $A$ considered
have the property that $\Spec(A)$ is (isomorphic) to an
object of this big site.

\medskip\noindent
Let $S$ be a scheme and let $X$ be an algebraic space over $S$.
In this chapter and the following we will write $X \times_S X$
for the product of $X$ with itself (in the category of algebraic
spaces over $S$), instead of $X \times X$.






\section{Descent data for quasi-coherent sheaves}
\label{section-equivalence}

\noindent
This section is the analogue of
Descent, Section \ref{descent-section-equivalence}
for algebraic spaces.
It makes sense to read that section first.

\begin{definition}
\label{definition-descent-datum-quasi-coherent}
Let $S$ be a scheme. Let $\{f_i : X_i \to X\}_{i \in I}$ be a family
of morphisms of algebraic spaces over $S$ with fixed target $X$.
\begin{enumerate}
\item A {\it descent datum $(\mathcal{F}_i, \varphi_{ij})$
for quasi-coherent sheaves} with respect to the given family
is given by a quasi-coherent sheaf $\mathcal{F}_i$ on $X_i$ for
each $i \in I$, an isomorphism of quasi-coherent
$\mathcal{O}_{X_i \times_X X_j}$-modules
$\varphi_{ij} : \text{pr}_0^*\mathcal{F}_i \to \text{pr}_1^*\mathcal{F}_j$
for each pair $(i, j) \in I^2$
such that for every triple of indices $(i, j, k) \in I^3$ the
diagram
$$
\xymatrix{
\text{pr}_0^*\mathcal{F}_i \ar[rd]_{\text{pr}_{01}^*\varphi_{ij}}
\ar[rr]_{\text{pr}_{02}^*\varphi_{ik}} & &
\text{pr}_2^*\mathcal{F}_k \\
& \text{pr}_1^*\mathcal{F}_j \ar[ru]_{\text{pr}_{12}^*\varphi_{jk}} &
}
$$
of $\mathcal{O}_{X_i \times_X X_j \times_X X_k}$-modules
commutes. This is called the {\it cocycle condition}.
\item A {\it morphism $\psi : (\mathcal{F}_i, \varphi_{ij}) \to
(\mathcal{F}'_i, \varphi'_{ij})$ of descent data} is given
by a family $\psi = (\psi_i)_{i\in I}$ of morphisms of
$\mathcal{O}_{X_i}$-modules $\psi_i : \mathcal{F}_i \to \mathcal{F}'_i$
such that all the diagrams
$$
\xymatrix{
\text{pr}_0^*\mathcal{F}_i \ar[r]_{\varphi_{ij}} \ar[d]_{\text{pr}_0^*\psi_i}
& \text{pr}_1^*\mathcal{F}_j \ar[d]^{\text{pr}_1^*\psi_j} \\
\text{pr}_0^*\mathcal{F}'_i \ar[r]^{\varphi'_{ij}} &
\text{pr}_1^*\mathcal{F}'_j \\
}
$$
commute.
\end{enumerate}
\end{definition}

\begin{lemma}
\label{lemma-map-families}
Let $S$ be a scheme.
Let $\mathcal{U} = \{U_i \to U\}_{i \in I}$ and
$\mathcal{V} = \{V_j \to V\}_{j \in J}$
be families of morphisms of algebraic spaces over $S$ with fixed targets.
Let $(g, \alpha : I \to J, (g_i)) : \mathcal{U} \to \mathcal{V}$
be a morphism of families of maps with fixed target, see
Sites, Definition \ref{sites-definition-morphism-coverings}.
Let $(\mathcal{F}_j, \varphi_{jj'})$ be a descent
datum for quasi-coherent sheaves with respect to the
family $\{V_j \to V\}_{j \in J}$. Then
\begin{enumerate}
\item The system
$$
\left(g_i^*\mathcal{F}_{\alpha(i)},
(g_i \times g_{i'})^*\varphi_{\alpha(i)\alpha(i')}\right)
$$
is a descent datum with respect to the family $\{U_i \to U\}_{i \in I}$.
\item This construction is functorial in the descent datum
$(\mathcal{F}_j, \varphi_{jj'})$.
\item Given a second morphism
$(g', \alpha' : I \to J, (g'_i))$
of families of maps with fixed target
with $g = g'$ there exists a functorial isomorphism of descent data
$$
(g_i^*\mathcal{F}_{\alpha(i)},
(g_i \times g_{i'})^*\varphi_{\alpha(i)\alpha(i')})
\cong
((g'_i)^*\mathcal{F}_{\alpha'(i)},
(g'_i \times g'_{i'})^*\varphi_{\alpha'(i)\alpha'(i')}).
$$
\end{enumerate}
\end{lemma}

\begin{proof}
Omitted. Hint: The maps
$g_i^*\mathcal{F}_{\alpha(i)} \to (g'_i)^*\mathcal{F}_{\alpha'(i)}$
which give the isomorphism of descent data in part (3)
are the pullbacks of the maps $\varphi_{\alpha(i)\alpha'(i)}$ by the
morphisms $(g_i, g'_i) : U_i \to V_{\alpha(i)} \times_V V_{\alpha'(i)}$.
\end{proof}

\noindent
Let $g : U \to V$ be a morphism of algebraic spaces.
The lemma above tells us that there is a well defined pullback functor
between the categories of descent data relative to families of
maps with target $V$ and $U$ provided there is a morphism between those
families of maps which ``lives over $g$''.

\begin{definition}
\label{definition-descent-datum-effective-quasi-coherent}
Let $S$ be a scheme.
Let $\{U_i \to U\}_{i \in I}$ be a family of morphisms of algebraic
spaces over $S$ with fixed target.
\begin{enumerate}
\item Let $\mathcal{F}$ be a quasi-coherent $\mathcal{O}_U$-module.
We call the unique descent on $\mathcal{F}$ datum with respect to the covering
$\{U \to U\}$ the {\it trivial descent datum}.
\item The pullback of the trivial descent datum to
$\{U_i \to U\}$ is called the {\it canonical descent datum}.
Notation: $(\mathcal{F}|_{U_i}, can)$.
\item A descent datum $(\mathcal{F}_i, \varphi_{ij})$
for quasi-coherent sheaves with respect to the given family
is said to be {\it effective} if there exists a quasi-coherent
sheaf $\mathcal{F}$ on $U$ such that $(\mathcal{F}_i, \varphi_{ij})$
is isomorphic to $(\mathcal{F}|_{U_i}, can)$.
\end{enumerate}
\end{definition}

\begin{lemma}
\label{lemma-zariski-descent-effective}
Let $S$ be a scheme. Let $U$ be an algebraic space over $S$.
Let $\{U_i \to U\}$ be a Zariski covering of $U$, see
Topologies on Spaces,
Definition \ref{spaces-topologies-definition-zariski-covering}.
Any descent datum on quasi-coherent sheaves
for the family $\mathcal{U} = \{U_i \to U\}$ is
effective. Moreover, the functor from the category of
quasi-coherent $\mathcal{O}_U$-modules to the category
of descent data with respect to $\{U_i \to U\}$ is fully faithful.
\end{lemma}

\begin{proof}
Omitted.
\end{proof}












\section{Fpqc descent of quasi-coherent sheaves}
\label{section-fpqc-descent-quasi-coherent}

\noindent
The main application of flat descent for modules is
the corresponding descent statement for quasi-coherent
sheaves with respect to fpqc-coverings.

\begin{proposition}
\label{proposition-fpqc-descent-quasi-coherent}
Let $S$ be a scheme.
Let $\{X_i \to X\}$ be an fpqc covering of algebraic spaces over $S$, see
Topologies on Spaces,
Definition \ref{spaces-topologies-definition-fpqc-covering}.
Any descent datum on quasi-coherent sheaves
for $\{X_i \to X\}$ is effective.
Moreover, the functor from the category of
quasi-coherent $\mathcal{O}_X$-modules to the category
of descent data with respect to $\{X_i \to X\}$ is fully faithful.
\end{proposition}

\begin{proof}
This is more or less a formal consequence of
the corresponding result for schemes, see
Descent, Proposition \ref{descent-proposition-fpqc-descent-quasi-coherent}.
Here is a strategy for a proof:
\begin{enumerate}
\item The fact that $\{X_i \to X\}$ is a refinement of the trivial
covering $\{X \to X\}$ gives, via
Lemma \ref{lemma-map-families},
a functor $\QCoh(\mathcal{O}_X) \to DD(\{X_i \to X\})$ from the
category of quasi-coherent $\mathcal{O}_X$-modules to the category of
descent data for the given family.
\item In order to prove the proposition we will construct a
quasi-inverse functor
$back : DD(\{X_i \to X\}) \to \QCoh(\mathcal{O}_X)$.
\item Applying again
Lemma \ref{lemma-map-families}
we see that there is a functor
$DD(\{X_i \to X\}) \to DD(\{T_j \to X\})$
if $\{T_j \to X\}$ is a refinement of the given family.
Hence in order to construct the functor $back$ we may assume that
each $X_i$ is a scheme, see
Topologies on Spaces,
Lemma \ref{spaces-topologies-lemma-refine-fpqc-schemes}.
This reduces us to the case where all the $X_i$ are schemes.
\item A quasi-coherent sheaf on $X$ is by definition a quasi-coherent
$\mathcal{O}_X$-module on $X_\etale$. Now for any
$U \in \Ob(X_\etale)$ we get an fppf covering
$\{U_i \times_X X_i \to U\}$ by schemes and a morphism
$g : \{U_i \times_X X_i \to U\} \to \{X_i \to X\}$ of coverings
lying over $U \to X$. Given a descent datum
$\xi = (\mathcal{F}_i, \varphi_{ij})$ we obtain a quasi-coherent
$\mathcal{O}_U$-module $\mathcal{F}_{\xi, U}$ corresponding
to the pullback $g^*\xi$ of
Lemma \ref{lemma-map-families}
to the covering of $U$ and using effectivity for fppf covering of schemes, see
Descent, Proposition \ref{descent-proposition-fpqc-descent-quasi-coherent}.
\item Check that $\xi \mapsto \mathcal{F}_{\xi, U}$ is functorial in $\xi$.
Omitted.
\item Check that $\xi \mapsto \mathcal{F}_{\xi, U}$ is compatible
with morphisms $U \to U'$ of the site $X_\etale$, so that
the system of sheaves $\mathcal{F}_{\xi, U}$ corresponds to a quasi-coherent
$\mathcal{F}_\xi$ on $X_\etale$, see
Properties of Spaces,
Lemma \ref{spaces-properties-lemma-characterize-quasi-coherent-small-etale}.
Details omitted.
\item Check that $back : \xi \mapsto \mathcal{F}_\xi$ is quasi-inverse
to the functor constructed in (1). Omitted.
\end{enumerate}
This finishes the proof.
\end{proof}





\section{Descent of finiteness properties of modules}
\label{section-descent-finiteness}

\noindent
This section is the analogue for the case of algebraic spaces of
Descent, Section \ref{descent-section-descent-finiteness}.
The goal is to show that one can check a quasi-coherent module
has a certain finiteness conditions by checking on the members of
a covering. We will repeatedly use the following proof scheme.
Suppose that $X$ is an algebraic space, and that $\{X_i \to X\}$
is a fppf (resp.\ fpqc) covering. Let $U \to X$ be a surjective
\'etale morphism such that $U$ is a scheme. Then there exists an
fppf (resp.\ fpqc) covering $\{Y_j \to X\}$ such that
\begin{enumerate}
\item $\{Y_j \to X\}$ is a refinement of $\{X_i \to X\}$,
\item each $Y_j$ is a scheme, and
\item each morphism $Y_j \to X$ factors though $U$, and
\item $\{Y_j \to U\}$ is an fppf (resp.\ fpqc) covering of $U$.
\end{enumerate}
Namely, first refine $\{X_i \to X\}$ by an fppf (resp.\ fpqc)
covering such that each $X_i$ is a scheme, see
Topologies on Spaces, Lemma \ref{spaces-topologies-lemma-refine-fppf-schemes},
resp.\ Lemma \ref{spaces-topologies-lemma-refine-fpqc-schemes}.
Then set $Y_i = U \times_X X_i$. A quasi-coherent
$\mathcal{O}_X$-module $\mathcal{F}$ is of finite type, of
finite presentation, etc if and only if the quasi-coherent
$\mathcal{O}_U$-module $\mathcal{F}|_U$ is of finite type, of
finite presentation, etc. Hence we can use the existence of the
refinement $\{Y_j \to X\}$ to reduce the proof of the following
lemmas to the case of schemes. We will indicate this by saying
that ``{\it the result follows from the case of schemes by
\'etale localization}''.

\begin{lemma}
\label{lemma-finite-type-descends}
Let $X$ be an algebraic space over a scheme $S$.
Let $\mathcal{F}$ be a quasi-coherent $\mathcal{O}_X$-module.
Let $\{f_i : X_i \to X\}_{i \in I}$ be an fpqc covering such that
each $f_i^*\mathcal{F}$ is a finite type $\mathcal{O}_{X_i}$-module.
Then $\mathcal{F}$ is a finite type $\mathcal{O}_X$-module.
\end{lemma}

\begin{proof}
This follows from the case of schemes, see
Descent, Lemma \ref{descent-lemma-finite-type-descends},
by \'etale localization.
\end{proof}

\begin{lemma}
\label{lemma-finite-presentation-descends}
Let $X$ be an algebraic space over a scheme $S$.
Let $\mathcal{F}$ be a quasi-coherent $\mathcal{O}_X$-module.
Let $\{f_i : X_i \to X\}_{i \in I}$ be an fpqc covering such that
each $f_i^*\mathcal{F}$ is an $\mathcal{O}_{X_i}$-module of finite
presentation. Then $\mathcal{F}$ is an $\mathcal{O}_X$-module
of finite presentation.
\end{lemma}

\begin{proof}
This follows from the case of schemes, see
Descent, Lemma \ref{descent-lemma-finite-presentation-descends},
by \'etale localization.
\end{proof}

\begin{lemma}
\label{lemma-flat-descends}
Let $X$ be an algebraic space over a scheme $S$.
Let $\mathcal{F}$ be a quasi-coherent $\mathcal{O}_X$-module.
Let $\{f_i : X_i \to X\}_{i \in I}$ be an fpqc covering such that
each $f_i^*\mathcal{F}$ is a flat $\mathcal{O}_{X_i}$-module.
Then $\mathcal{F}$ is a flat $\mathcal{O}_X$-module.
\end{lemma}

\begin{proof}
This follows from the case of schemes, see
Descent, Lemma \ref{descent-lemma-flat-descends},
by \'etale localization.
\end{proof}

\begin{lemma}
\label{lemma-finite-locally-free-descends}
Let $X$ be an algebraic space over a scheme $S$.
Let $\mathcal{F}$ be a quasi-coherent $\mathcal{O}_X$-module.
Let $\{f_i : X_i \to X\}_{i \in I}$ be an fpqc covering such that
each $f_i^*\mathcal{F}$ is a finite locally free $\mathcal{O}_{X_i}$-module.
Then $\mathcal{F}$ is a finite locally free $\mathcal{O}_X$-module.
\end{lemma}

\begin{proof}
This follows from the case of schemes, see
Descent, Lemma \ref{descent-lemma-finite-locally-free-descends},
by \'etale localization.
\end{proof}

\noindent
The definition of a locally projective quasi-coherent sheaf can be found in
Properties of Spaces, Section
\ref{spaces-properties-section-locally-projective}.
It is also proved there that this notion is preserved under pullback.

\begin{lemma}
\label{lemma-locally-projective-descends}
Let $X$ be an algebraic space over a scheme $S$.
Let $\mathcal{F}$ be a quasi-coherent $\mathcal{O}_X$-module.
Let $\{f_i : X_i \to X\}_{i \in I}$ be an fpqc covering such that
each $f_i^*\mathcal{F}$ is a locally projective $\mathcal{O}_{X_i}$-module.
Then $\mathcal{F}$ is a locally projective $\mathcal{O}_X$-module.
\end{lemma}

\begin{proof}
This follows from the case of schemes, see
Descent, Lemma \ref{descent-lemma-locally-projective-descends},
by \'etale localization.
\end{proof}

\noindent
We also add here two results which are related to the results above, but
are of a slightly different nature.

\begin{lemma}
\label{lemma-finite-over-finite-module}
Let $S$ be a scheme.
Let $f : X \to Y$ be a morphism of algebraic spaces over $S$.
Let $\mathcal{F}$ be a quasi-coherent $\mathcal{O}_X$-module.
Assume $f$ is a finite morphism.
Then $\mathcal{F}$ is an $\mathcal{O}_X$-module of finite type
if and only if $f_*\mathcal{F}$ is an $\mathcal{O}_Y$-module of finite
type.
\end{lemma}

\begin{proof}
As $f$ is finite it is representable. Choose a scheme $V$ and a surjective
\'etale morphism $V \to Y$. Then $U = V \times_Y X$ is a scheme with
a surjective \'etale morphism towards $X$ and a finite morphism
$\psi : U \to V$ (the base change of $f$). Since
$\psi_*(\mathcal{F}|_U) = f_*\mathcal{F}|_V$
the result of the lemma follows immediately from the schemes version which
is
Descent, Lemma \ref{descent-lemma-finite-over-finite-module}.
\end{proof}

\begin{lemma}
\label{lemma-finite-finitely-presented-module}
Let $S$ be a scheme.
Let $f : X \to Y$ be a morphism of algebraic spaces over $S$.
Let $\mathcal{F}$ be a quasi-coherent $\mathcal{O}_X$-module.
Assume $f$ is finite and of finite presentation.
Then $\mathcal{F}$ is an $\mathcal{O}_X$-module of finite presentation
if and only if $f_*\mathcal{F}$ is an $\mathcal{O}_Y$-module of finite
presentation.
\end{lemma}

\begin{proof}
As $f$ is finite it is representable. Choose a scheme $V$ and a surjective
\'etale morphism $V \to Y$. Then $U = V \times_Y X$ is a scheme with
a surjective \'etale morphism towards $X$ and a finite morphism
$\psi : U \to V$ (the base change of $f$). Since
$\psi_*(\mathcal{F}|_U) = f_*\mathcal{F}|_V$
the result of the lemma follows immediately from the schemes version which
is
Descent, Lemma \ref{descent-lemma-finite-finitely-presented-module}.
\end{proof}






\section{Fpqc coverings}
\label{section-fpqc}

\noindent
This section is the analogue of
Descent, Section \ref{descent-section-fpqc-universal-effective-epimorphisms}.
At the moment we do not know if all of the material for
fpqc coverings of schemes holds also for algebraic spaces.

\begin{lemma}
\label{lemma-open-fpqc-covering}
Let $S$ be a scheme.
Let $\{f_i : T_i \to T\}_{i \in I}$ be an fpqc covering
of algebraic spaces over $S$.
Suppose that for each $i$ we have an open subspace $W_i \subset T_i$
such that for all $i, j \in I$ we have
$\text{pr}_0^{-1}(W_i) = \text{pr}_1^{-1}(W_j)$ as open
subspaces of $T_i \times_T T_j$. Then there exists a unique open subspace
$W \subset T$ such that $W_i = f_i^{-1}(W)$ for each $i$.
\end{lemma}

\begin{proof}
By
Topologies on Spaces, Lemma \ref{spaces-topologies-lemma-refine-fpqc-schemes}
we may assume each $T_i$ is a scheme.
Choose a scheme $U$ and a surjective \'etale morphism $U \to T$.
Then $\{T_i \times_T U \to U\}$ is an fpqc covering of $U$
and $T_i \times_T U$ is a scheme for each $i$. Hence we
see that the collection of opens $W_i \times_T U$ comes from a unique
open subscheme $W' \subset U$ by
Descent, Lemma \ref{descent-lemma-open-fpqc-covering}.
As $U \to X$ is open we can define $W \subset X$ the Zariski
open which is the image of $W'$, see
Properties of Spaces, Section \ref{spaces-properties-section-points}.
We omit the verification that this works, i.e., that
$W_i$ is the inverse image of $W$ for each $i$.
\end{proof}

\begin{lemma}
\label{lemma-fpqc-universal-effective-epimorphisms}
Let $S$ be a scheme. Let $\{T_i \to T\}$ be an fpqc covering of algebraic
spaces over $S$, see Topologies on Spaces, Definition
\ref{spaces-topologies-definition-fpqc-covering}.
Then given an algebraic space $B$ over $S$ the sequence
$$
\xymatrix{
\Mor_S(T, B) \ar[r] &
\prod\nolimits_i \Mor_S(T_i, B) \ar@<1ex>[r] \ar@<-1ex>[r] &
\prod\nolimits_{i, j} \Mor_S(T_i \times_T T_j, B)
}
$$
is an equalizer diagram.
In other words, every representable functor on the category of
algebraic spaces over $S$ satisfies the sheaf condition for
fpqc coverings.
\end{lemma}

\begin{proof}
We know this is true if $\{T_i \to T\}$ is an fpqc covering of
schemes, see Properties of Spaces, Proposition
\ref{spaces-properties-proposition-sheaf-fpqc}.
This is the key fact and we encourage the reader to skip the rest
of the proof which is formal. Choose a scheme $U$ and a surjective
\'etale morphism
$U \to T$. Let $U_i$ be a scheme and let $U_i \to T_i \times_T U$
be a surjective \'etale morphism. Then $\{U_i \to U\}$ is an
fpqc covering. This follows from
Topologies on Spaces, Lemmas \ref{spaces-topologies-lemma-fpqc} and
\ref{spaces-topologies-lemma-recognize-fpqc-covering}.
By the above we have the result for $\{U_i \to U\}$.

\medskip\noindent
What this means is the following: Suppose that $b_i : T_i \to B$
is a family of morphisms with
$b_i \circ \text{pr}_0 = b_j \circ \text{pr}_1$ as morphisms
$T_i \times_T T_j \to B$. Then we let $a_i : U_i \to B$ be the
composition of $U_i \to T_i$ with $b_i$. By what was said above
we find a unique morphism $a : U \to B$ such that
$a_i$ is the composition of $a$ with $U_i \to U$.
The uniqueness guarantees that $a \circ \text{pr}_0 = a \circ \text{pr}_1$
as morphisms $U \times_T U \to B$. Then since $T = U/(U \times_T U)$
as a sheaf, we find that $a$ comes from a unique morphism $b : T \to B$.
Chasing diagrams we find that $b$ is the morphism we are looking for.
\end{proof}










\section{Descent of finiteness and smoothness properties of morphisms}
\label{section-descent-finiteness-morphisms}

\noindent
The following type of lemma is occasionally useful.

\begin{lemma}
\label{lemma-curiosity}
Let $S$ be a scheme. Let $X \to Y \to Z$ be morphism of algebraic spaces.
Let $P$ be one of the following properties of morphisms of algebraic spaces
over $S$:
flat, locally finite type, locally finite presentation.
Assume that $X \to Z$ has $P$ and that
$X \to Y$ is a surjection of sheaves on $(\Sch/S)_{fppf}$.
Then $Y \to Z$ is $P$.
\end{lemma}

\begin{proof}
Choose a scheme $W$ and a surjective \'etale morphism $W \to Z$.
Choose a scheme $V$ and a surjective \'etale morphism $V \to W \times_Z Y$.
Choose a scheme $U$ and a surjective \'etale morphism $U \to V \times_Y X$.
By assumption we can find an fppf covering $\{V_i \to V\}$ and
lifts $V_i \to X$ of the morphism $V_i \to Y$. Since $U \to X$ is surjective
\'etale we see that over the members of the fppf covering
$\{V_i \times_X U \to V\}$ we have lifts into $U$. Hence $U \to V$ induces
a surjection of sheaves on $(\Sch/S)_{fppf}$.
By our definition of what it means to have property $P$ for a
morphism of algebraic spaces (see
Morphisms of Spaces,
Definition \ref{spaces-morphisms-definition-flat},
Definition \ref{spaces-morphisms-definition-locally-finite-type}, and
Definition \ref{spaces-morphisms-definition-locally-finite-presentation})
we see that $U \to W$ has $P$ and we have to show $V \to W$ has $P$.
Thus we reduce the question to the case of morphisms of schemes
which is treated in
Descent, Lemma \ref{descent-lemma-curiosity}.
\end{proof}

\noindent
A more standard case of the above lemma is the following.
(The version with ``flat'' follows from
Morphisms of Spaces, Lemma \ref{spaces-morphisms-lemma-flat-permanence}.)

\begin{lemma}
\label{lemma-flat-finitely-presented-permanence}
Let $S$ be a scheme. Let
$$
\xymatrix{
X \ar[rr]_f \ar[rd]_p & &
Y \ar[dl]^q \\
& B
}
$$
be a commutative diagram of morphisms of algebraic spaces over $S$.
Assume that $f$ is surjective, flat, and locally of finite presentation
and assume that $p$ is locally of finite presentation (resp.\ locally
of finite type). Then $q$ is locally of finite presentation
(resp.\ locally of finite type).
\end{lemma}

\begin{proof}
Since $\{X \to Y\}$ is an fppf covering, it induces a surjection of
fppf sheaves (Topologies on Spaces, Lemma
\ref{spaces-topologies-lemma-fppf-covering-surjective}) and the
lemma is a special case of Lemma \ref{lemma-curiosity}.
On the other hand, an easier argument is to deduce it from
the analogue for schemes. Namely, the problem is \'etale local
on $B$ and $Y$ (Morphisms of Spaces, Lemmas
\ref{spaces-morphisms-lemma-finite-type-local} and
\ref{spaces-morphisms-lemma-finite-presentation-local}).
Hence we may assume that $B$ and $Y$ are affine
schemes. Since $|X| \to |Y|$ is open
(Morphisms of Spaces, Lemma \ref{spaces-morphisms-lemma-fppf-open}),
we can choose an affine
scheme $U$ and an \'etale morphism $U \to X$ such that the
composition $U \to Y$ is surjective. In this case the result
follows from Descent, Lemma
\ref{descent-lemma-flat-finitely-presented-permanence}.
\end{proof}

\begin{lemma}
\label{lemma-syntomic-smooth-etale-permanence}
Let $S$ be a scheme. Let
$$
\xymatrix{
X \ar[rr]_f \ar[rd]_p & &
Y \ar[dl]^q \\
& B
}
$$
be a commutative diagram of morphisms of algebraic spaces over $S$.
Assume that
\begin{enumerate}
\item $f$ is surjective, and syntomic (resp.\ smooth, resp.\ \'etale),
\item $p$ is syntomic (resp.\ smooth, resp.\ \'etale).
\end{enumerate}
Then $q$ is syntomic (resp.\ smooth, resp.\ \'etale).
\end{lemma}

\begin{proof}
We deduce this from the analogue for schemes.
Namely, the problem is \'etale local on $B$ and $Y$
(Morphisms of Spaces, Lemmas
\ref{spaces-morphisms-lemma-syntomic-local},
\ref{spaces-morphisms-lemma-smooth-local}, and
\ref{spaces-morphisms-lemma-etale-local}).
Hence we may assume that $B$ and $Y$ are affine
schemes. Since $|X| \to |Y|$ is open
(Morphisms of Spaces, Lemma \ref{spaces-morphisms-lemma-fppf-open}),
we can choose an affine
scheme $U$ and an \'etale morphism $U \to X$ such that the
composition $U \to Y$ is surjective. In this case the result
follows from Descent, Lemma
\ref{descent-lemma-syntomic-smooth-etale-permanence}.
\end{proof}

\noindent
Actually we can strengthen this result as follows.

\begin{lemma}
\label{lemma-smooth-permanence}
Let $S$ be a scheme. Let
$$
\xymatrix{
X \ar[rr]_f \ar[rd]_p & &
Y \ar[dl]^q \\
& B
}
$$
be a commutative diagram of morphisms of algebraic spaces over $S$. Assume that
\begin{enumerate}
\item $f$ is surjective, flat, and locally of finite presentation,
\item $p$ is smooth (resp.\ \'etale).
\end{enumerate}
Then $q$ is smooth (resp.\ \'etale).
\end{lemma}

\begin{proof}
We deduce this from the analogue for schemes.
Namely, the problem is \'etale local on $B$ and $Y$
(Morphisms of Spaces, Lemmas
\ref{spaces-morphisms-lemma-smooth-local} and
\ref{spaces-morphisms-lemma-etale-local}).
Hence we may assume that $B$ and $Y$ are affine
schemes. Since $|X| \to |Y|$ is open
(Morphisms of Spaces, Lemma \ref{spaces-morphisms-lemma-fppf-open}),
we can choose an affine
scheme $U$ and an \'etale morphism $U \to X$ such that the
composition $U \to Y$ is surjective. In this case the result
follows from Descent, Lemma
\ref{descent-lemma-smooth-permanence}.
\end{proof}

\begin{lemma}
\label{lemma-syntomic-permanence}
Let $S$ be a scheme. Let
$$
\xymatrix{
X \ar[rr]_f \ar[rd]_p & &
Y \ar[dl]^q \\
& B
}
$$
be a commutative diagram of morphisms of algebraic spaces over $S$. Assume that
\begin{enumerate}
\item $f$ is surjective, flat, and locally of finite presentation,
\item $p$ is syntomic.
\end{enumerate}
Then both $q$ and $f$ are syntomic.
\end{lemma}

\begin{proof}
We deduce this from the analogue for schemes.
Namely, the problem is \'etale local on $B$ and $Y$
(Morphisms of Spaces, Lemma
\ref{spaces-morphisms-lemma-syntomic-local}).
Hence we may assume that $B$ and $Y$ are affine
schemes. Since $|X| \to |Y|$ is open
(Morphisms of Spaces, Lemma \ref{spaces-morphisms-lemma-fppf-open}),
we can choose an affine
scheme $U$ and an \'etale morphism $U \to X$ such that the
composition $U \to Y$ is surjective. In this case the result
follows from Descent, Lemma
\ref{descent-lemma-syntomic-permanence}.
\end{proof}










\section{Descending properties of spaces}
\label{section-descending-properties-spaces}

\noindent
In this section we put some results of the following kind.

\begin{lemma}
\label{lemma-descend-unibranch}
Let $S$ be a scheme.
Let $f : X \to Y$ be a morphism of algebraic spaces over $S$.
Let $x \in |X|$.
If $f$ is flat at $x$ and $X$ is geometrically unibranch at $x$, then $Y$ is
geometrically unibranch at $f(x)$.
\end{lemma}

\begin{proof}
Consider the map of \'etale local rings
$\mathcal{O}_{Y, f(\overline{x})} \to \mathcal{O}_{X, \overline{x}}$.
By
Morphisms of Spaces, Lemma
\ref{spaces-morphisms-lemma-flat-at-point-etale-local-rings}
this is flat. Hence if $\mathcal{O}_{X, \overline{x}}$ has a unique minimal
prime, so does $\mathcal{O}_{Y, f(\overline{x})}$ (by going down, see
Algebra, Lemma \ref{algebra-lemma-flat-going-down}).
\end{proof}

\begin{lemma}
\label{lemma-descend-reduced}
\begin{slogan}
A flat and surjective morphism of algebraic spaces with a reduced source
has a reduced target.
\end{slogan}
Let $S$ be a scheme.
Let $f : X \to Y$ be a morphism of algebraic spaces over $S$.
If $f$ is flat and surjective and $X$ is reduced, then $Y$ is reduced.
\end{lemma}

\begin{proof}
Choose a scheme $V$ and a surjective \'etale morphism $V \to Y$.
Choose a scheme $U$ and a surjective \'etale morphism
$U \to X \times_Y V$. As $f$ is surjective and flat, the morphism of
schemes $U \to V$ is surjective and flat. In this way we reduce the
problem to the case of schemes (as reducedness of $X$ and $Y$ is defined
in terms of reducedness of $U$ and $V$, see
Properties of Spaces,
Section \ref{spaces-properties-section-types-properties}).
The case of schemes is
Descent, Lemma \ref{descent-lemma-descend-reduced}.
\end{proof}

\begin{lemma}
\label{lemma-descend-locally-Noetherian}
Let $f : X \to Y$ be a morphism of algebraic spaces.
If $f$ is locally of finite presentation, flat, and surjective and
$X$ is locally Noetherian, then $Y$ is locally Noetherian.
\end{lemma}

\begin{proof}
Choose a scheme $V$ and a surjective \'etale morphism $V \to Y$.
Choose a scheme $U$ and a surjective \'etale morphism
$U \to X \times_Y V$. As $f$ is surjective, flat, and locally of
finite presentation the morphism of schemes $U \to V$ is surjective, flat, and
locally of finite presentation. In this way we reduce the
problem to the case of schemes (as being locally Noetherian for $X$ and $Y$
is defined in terms of being locally Noetherian of $U$ and $V$, see
Properties of Spaces,
Section \ref{spaces-properties-section-types-properties}).
In the case of schemes the result follows from
Descent, Lemma \ref{descent-lemma-Noetherian-local-fppf}.
\end{proof}

\begin{lemma}
\label{lemma-descend-regular}
Let $f : X \to Y$ be a morphism of algebraic spaces.
If $f$ is locally of finite presentation, flat, and surjective and
$X$ is regular, then $Y$ is regular.
\end{lemma}

\begin{proof}
By
Lemma \ref{lemma-descend-locally-Noetherian}
we know that $Y$ is locally Noetherian.
Choose a scheme $V$ and a surjective \'etale morphism $V \to Y$.
It suffices to prove that the local rings of $V$ are all regular local
rings, see
Properties, Lemma \ref{properties-lemma-characterize-regular}.
Choose a scheme $U$ and a surjective \'etale morphism
$U \to X \times_Y V$. As $f$ is surjective and flat the morphism of schemes
$U \to V$ is surjective and flat. By assumption $U$ is a regular scheme
in particular all of its local rings are regular (by the lemma above).
Hence the lemma follows from
Algebra, Lemma \ref{algebra-lemma-flat-under-regular}.
\end{proof}



\section{Descending properties of morphisms}
\label{section-descending-properties-morphisms}

\noindent
In this section we introduce the notion of when a property of morphisms of
algebraic spaces is local on the target in a topology. Please compare with
Descent, Section \ref{descent-section-descending-properties-morphisms}.

\begin{definition}
\label{definition-property-morphisms-local}
Let $S$ be a scheme.
Let $\mathcal{P}$ be a property of morphisms of algebraic spaces over $S$.
Let $\tau \in \{fpqc, fppf, syntomic, smooth, \etale\}$.
We say $\mathcal{P}$ is {\it $\tau$ local on the base}, or
{\it $\tau$ local on the target}, or
{\it local on the base for the $\tau$-topology} if for any
$\tau$-covering $\{Y_i \to Y\}_{i \in I}$ of algebraic spaces
and any morphism of algebraic spaces $f : X \to Y$ we
have
$$
f \text{ has }\mathcal{P}
\Leftrightarrow
\text{each }Y_i \times_Y X \to Y_i\text{ has }\mathcal{P}.
$$
\end{definition}

\noindent
To be sure, since isomorphisms are always coverings
we see (or require) that property $\mathcal{P}$ holds for $X \to Y$
if and only if it holds for any arrow $X' \to Y'$ isomorphic to $X \to Y$.
If a property is $\tau$-local on the target then it is preserved
by base changes by morphisms which occur in $\tau$-coverings. Here
is a formal statement.

\begin{lemma}
\label{lemma-pullback-property-local-target}
Let $S$ be a scheme.
Let $\tau \in \{fpqc, fppf, syntomic, smooth, \etale\}$.
Let $\mathcal{P}$ be a property of morphisms of algebraic spaces over $S$
which is $\tau$ local on the target. Let $f : X \to Y$ have property
$\mathcal{P}$. For any morphism $Y' \to Y$ which is
flat, resp.\ flat and locally of finite presentation, resp.\ syntomic,
resp.\ \'etale, the base change
$f' : Y' \times_Y X \to Y'$ of $f$ has property $\mathcal{P}$.
\end{lemma}

\begin{proof}
This is true because we can fit $Y' \to Y$ into a family of
morphisms which forms a $\tau$-covering.
\end{proof}

\noindent
A simple often used consequence of the above is that if
$f : X \to Y$ has property $\mathcal{P}$ which is $\tau$-local
on the target and $f(X) \subset V$
for some open subspace $V \subset Y$, then also the induced
morphism $X \to V$ has $\mathcal{P}$. Proof: The base change
$f$ by $V \to Y$ gives $X \to V$.

\begin{lemma}
\label{lemma-largest-open-of-the-base}
Let $S$ be a scheme.
Let $\tau \in \{fppf, syntomic, smooth, \etale\}$.
Let $\mathcal{P}$ be a property of morphisms of algebraic spaces over $S$
which is $\tau$ local on the target. For any morphism of algebraic spaces
$f : X \to Y$ over $S$ there exists a largest open subspace
$W(f) \subset Y$ such that the restriction $X_{W(f)} \to W(f)$ has
$\mathcal{P}$. Moreover,
\begin{enumerate}
\item if $g : Y' \to Y$ is a morphism of algebraic spaces which is
flat and locally of finite presentation, syntomic, smooth, or \'etale
and the base change $f' : X_{Y'} \to Y'$ has $\mathcal{P}$, then
$g$ factors through $W(f)$,
\item if $g : Y' \to Y$ is flat and locally of finite presentation,
syntomic, smooth, or \'etale, then $W(f') = g^{-1}(W(f))$, and
\item if $\{g_i : Y_i \to Y\}$ is a $\tau$-covering, then
$g_i^{-1}(W(f)) = W(f_i)$, where $f_i$ is the base change of $f$
by $Y_i \to Y$.
\end{enumerate}
\end{lemma}

\begin{proof}
Consider the union $W_{set} \subset |Y|$ of the images
$g(|Y'|) \subset |Y|$ of morphisms $g : Y' \to Y$ with the properties:
\begin{enumerate}
\item $g$ is flat and locally of finite presentation, syntomic,
smooth, or \'etale, and
\item the base change $Y' \times_{g, Y} X \to Y'$ has property
$\mathcal{P}$.
\end{enumerate}
Since such a morphism $g$ is open (see
Morphisms of Spaces, Lemma \ref{spaces-morphisms-lemma-fppf-open})
we see that $W_{set}$ is an open subset of $|Y|$. Denote $W \subset Y$
the open subspace whose underlying set of points is $W_{set}$, see
Properties of Spaces, Lemma \ref{spaces-properties-lemma-open-subspaces}.
Since $\mathcal{P}$ is local in the $\tau$ topology the restriction
$X_W \to W$ has property $\mathcal{P}$ because we are given a covering
$\{Y' \to W\}$ of $W$ such that the pullbacks have $\mathcal{P}$.
This proves the existence and proves that $W(f)$ has property (1).
To see property (2) note that $W(f') \supset g^{-1}(W(f))$ because
$\mathcal{P}$ is stable under base change by flat and locally of finite
presentation, syntomic, smooth, or \'etale morphisms, see
Lemma \ref{lemma-pullback-property-local-target}.
On the other hand, if $Y'' \subset Y'$ is an open such that
$X_{Y''} \to Y''$ has property $\mathcal{P}$, then $Y'' \to Y$ factors
through $W$ by construction, i.e., $Y'' \subset g^{-1}(W(f))$. This
proves (2). Assertion (3) follows from (2) because each morphism
$Y_i \to Y$ is flat and locally of finite presentation, syntomic,
smooth, or \'etale by our definition of a $\tau$-covering.
\end{proof}

\begin{lemma}
\label{lemma-descending-properties-morphisms}
Let $S$ be a scheme. Let $\mathcal{P}$ be a property of morphisms of
algebraic spaces over $S$. Assume
\begin{enumerate}
\item if $X_i \to Y_i$, $i = 1, 2$ have property $\mathcal{P}$ so
does $X_1 \amalg X_2 \to Y_1 \amalg Y_2$,
\item a morphism of algebraic spaces $f : X \to Y$ has property
$\mathcal{P}$ if and only if for every affine scheme $Z$ and
morphism $Z \to Y$ the base change $Z \times_Y X \to Z$ of $f$
has property $\mathcal{P}$, and
\item for any surjective flat morphism of affine schemes
$Z' \to Z$ over $S$ and a morphism $f : X \to Z$ from an algebraic space
to $Z$ we have
$$
f' : Z' \times_Z X \to Z'\text{ has }\mathcal{P}
\Rightarrow
f\text{ has }\mathcal{P}.
$$
\end{enumerate}
Then $\mathcal{P}$ is fpqc local on the base.
\end{lemma}

\begin{proof}
If $\mathcal{P}$ has property (2), then it is automatically
stable under any base change. Hence the direct implication in
Definition \ref{definition-property-morphisms-local}.

\medskip\noindent
Let $\{Y_i \to Y\}_{i \in I}$ be an fpqc covering of algebraic spaces over $S$.
Let $f : X \to Y$ be a morphism of algebraic spaces over $S$.
Assume each base change $f_i : Y_i \times_Y X \to Y_i$ has property
$\mathcal{P}$. Our goal is to show that $f$ has $\mathcal{P}$.
Let $Z$ be an affine scheme, and let $Z \to Y$ be a morphism.
By (2) it suffices to show that the morphism of algebraic spaces
$Z \times_Y X \to Z$ has $\mathcal{P}$.
Since $\{Y_i \to Y\}_{i \in I}$ is an fpqc covering we know there
exists a standard fpqc covering $\{Z_j \to Z\}_{j = 1, \ldots , n}$
and morphisms $Z_j \to Y_{i_j}$ over $Y$ for suitable indices $i_j \in I$.
Since $f_{i_j}$ has $\mathcal{P}$ we see that
$$
Z_j \times_Y X
=
Z_j \times_{Y_{i_j}} (Y_{i_j} \times_Y X)
\longrightarrow
Z_j
$$
has $\mathcal{P}$ as a base change of $f_{i_j}$ (see first remark of the
proof). Set $Z' = \coprod_{j = 1, \ldots, n} Z_j$, so that $Z' \to Z$ is
a flat and surjective morphism of affine schemes over $S$. By (1)
we conclude that $Z' \times_Y X \to Z'$ has property $\mathcal{P}$.
Since this is the base change of the morphism $Z \times_Y X \to Z$
by the morphism $Z' \to Z$ we conclude that
$Z \times_Y X \to Z$ has property $\mathcal{P}$ as desired.
\end{proof}


\section{Descending properties of morphisms in the fpqc topology}
\label{section-descending-properties-morphisms-fpqc}


\noindent
In this section we find a large number of properties
of morphisms of algebraic spaces which are local on the base
in the fpqc topology. Please compare with
Descent, Section \ref{descent-section-descending-properties-morphisms-fpqc}
for the case of morphisms of schemes.

\begin{lemma}
\label{lemma-descending-property-quasi-compact}
Let $S$ be a scheme.
The property $\mathcal{P}(f) =$``$f$ is quasi-compact''
is fpqc local on the base on algebraic spaces over $S$.
\end{lemma}

\begin{proof}
We will use
Lemma \ref{lemma-descending-properties-morphisms}
to prove this. Assumptions (1) and (2) of that lemma follow from
Morphisms of Spaces,
Lemma \ref{spaces-morphisms-lemma-quasi-compact-local}.
Let $Z' \to Z$ be a surjective flat morphism of affine schemes over $S$.
Let $f : X \to Z$ be a morphism of algebraic spaces, and assume
that the base change $f' : Z' \times_Z X \to Z'$ is quasi-compact. We have
to show that $f$ is quasi-compact. To see this, using
Morphisms of Spaces,
Lemma \ref{spaces-morphisms-lemma-quasi-compact-local}
again, it is enough to show that for every affine scheme $Y$ and
morphism $Y \to Z$ the fibre product $Y \times_Z X$ is quasi-compact.
Here is a picture:
\begin{equation}
\label{equation-cube}
\vcenter{
\xymatrix{
Y \times_Z Z' \times_Z X \ar[dd] \ar[rr] \ar[rd] & &
Z' \times_Z X \ar'[d][dd]^{f'} \ar[rd] \\
& Y \times_Z X \ar[dd] \ar[rr] & & X \ar[dd]^f \\
Y \times_Z Z' \ar'[r][rr] \ar[rd] & & Z' \ar[rd] \\
& Y \ar[rr] & & Z
}
}
\end{equation}
Note that all squares are cartesian and the bottom square consists
of affine schemes. The assumption that $f'$ is quasi-compact combined with
the fact that $Y \times_Z Z'$ is affine implies that
$Y \times_Z Z' \times_Z X$ is quasi-compact. Since
$$
Y \times_Z Z' \times_Z X \longrightarrow Y \times_Z X
$$
is surjective as a base change of $Z' \to Z$
we conclude that $Y \times_Z X$ is quasi-compact, see
Morphisms of Spaces,
Lemma \ref{spaces-morphisms-lemma-surjection-from-quasi-compact}.
This finishes the proof.
\end{proof}

\begin{lemma}
\label{lemma-descending-property-quasi-separated}
Let $S$ be a scheme.
The property $\mathcal{P}(f) =$``$f$ is quasi-separated''
is fpqc local on the base on algebraic spaces over $S$.
\end{lemma}

\begin{proof}
A base change of a quasi-separated morphism is quasi-separated, see
Morphisms of Spaces,
Lemma \ref{spaces-morphisms-lemma-base-change-separated}.
Hence the direct implication in
Definition \ref{definition-property-morphisms-local}.

\medskip\noindent
Let $\{Y_i \to Y\}_{i \in I}$ be an fpqc covering of algebraic spaces over $S$.
Let $f : X \to Y$ be a morphism of algebraic spaces over $S$.
Assume each base change $X_i := Y_i \times_Y X \to Y_i$ is quasi-separated.
This means that each of the morphisms
$$
\Delta_i :
X_i
\longrightarrow
X_i \times_{Y_i} X_i = Y_i \times_Y (X \times_Y X)
$$
is quasi-compact. The base change of a fpqc covering is an fpqc covering, see
Topologies on Spaces, Lemma \ref{spaces-topologies-lemma-fpqc}
hence $\{Y_i \times_Y (X \times_Y X) \to X \times_Y X\}$
is an fpqc covering of algebraic spaces. Moreover, each
$\Delta_i$ is the base change of the morphism
$\Delta : X \to X \times_Y X$. Hence it follows from
Lemma \ref{lemma-descending-property-quasi-compact}
that $\Delta$ is quasi-compact, i.e., $f$ is quasi-separated.
\end{proof}

\begin{lemma}
\label{lemma-descending-property-universally-closed}
Let $S$ be a scheme.
The property $\mathcal{P}(f) =$``$f$ is universally closed''
is fpqc local on the base on algebraic spaces over $S$.
\end{lemma}

\begin{proof}
We will use
Lemma \ref{lemma-descending-properties-morphisms}
to prove this. Assumptions (1) and (2) of that lemma follow from
Morphisms of Spaces,
Lemma \ref{spaces-morphisms-lemma-universally-closed-local}.
Let $Z' \to Z$ be a surjective flat morphism of affine schemes over $S$.
Let $f : X \to Z$ be a morphism of algebraic spaces, and assume
that the base change $f' : Z' \times_Z X \to Z'$ is universally closed.
We have to show that $f$ is universally closed. To see this, using
Morphisms of Spaces,
Lemma \ref{spaces-morphisms-lemma-universally-closed-local}
again, it is enough to show that for every affine scheme $Y$ and
morphism $Y \to Z$ the map $|Y \times_Z X| \to |Y|$ is closed.
Consider the cube (\ref{equation-cube}).
The assumption that $f'$ is universally closed implies that
$|Y \times_Z Z' \times_Z X| \to |Y \times_Z Z'|$ is closed.
As $Y \times_Z Z' \to Y$ is quasi-compact, surjective, and flat
as a base change of $Z' \to Z$
we see the map $|Y \times_Z Z'| \to |Y|$ is submersive, see
Morphisms, Lemma \ref{morphisms-lemma-fpqc-quotient-topology}.
Moreover the map
$$
|Y \times_Z Z' \times_Z X|
\longrightarrow
|Y \times_Z Z'| \times_{|Y|} |Y \times_Z X|
$$
is surjective, see
Properties of Spaces, Lemma \ref{spaces-properties-lemma-points-cartesian}.
It follows by elementary topology that $|Y \times_Z X| \to |Y|$ is closed.
\end{proof}

\begin{lemma}
\label{lemma-descending-property-universally-open}
Let $S$ be a scheme.
The property $\mathcal{P}(f) =$``$f$ is universally open''
is fpqc local on the base on algebraic spaces over $S$.
\end{lemma}

\begin{proof}
The proof is the same as the proof of
Lemma \ref{lemma-descending-property-universally-closed}.
\end{proof}

\begin{lemma}
\label{lemma-descending-property-universally-submersive}
The property $\mathcal{P}(f) =$``$f$ is universally submersive''
is fpqc local on the base.
\end{lemma}

\begin{proof}
The proof is the same as the proof of
Lemma \ref{lemma-descending-property-universally-closed}.
\end{proof}

\begin{lemma}
\label{lemma-descending-property-surjective}
The property $\mathcal{P}(f) =$``$f$ is surjective''
is fpqc local on the base.
\end{lemma}

\begin{proof}
Omitted. (Hint: Use
Properties of Spaces, Lemma \ref{spaces-properties-lemma-points-cartesian}.)
\end{proof}

\begin{lemma}
\label{lemma-descending-property-universally-injective}
The property $\mathcal{P}(f) =$``$f$ is universally injective''
is fpqc local on the base.
\end{lemma}

\begin{proof}
We will use
Lemma \ref{lemma-descending-properties-morphisms}
to prove this. Assumptions (1) and (2) of that lemma follow from
Morphisms of Spaces,
Lemma \ref{spaces-morphisms-lemma-universally-closed-local}.
Let $Z' \to Z$ be a flat surjective morphism of affine schemes
over $S$ and let $f : X \to Z$ be a morphism from an algebraic space to $Z$.
Assume that the base change $f' : X' \to Z'$ is universally injective.
Let $K$ be a field, and let $a, b : \Spec(K) \to X$
be two morphisms such that $f \circ a = f \circ b$.
As $Z' \to Z$ is surjective there exists a field
extension $K \subset K'$ and a morphism
$\Spec(K') \to Z'$
such that the following solid diagram commutes
$$
\xymatrix{
\Spec(K') \ar[rrd] \ar@{-->}[rd]_{a', b'} \ar[dd] \\
 &
X' \ar[r] \ar[d] &
Z' \ar[d] \\
\Spec(K) \ar[r]^{a, b} &
X \ar[r] &
Z
}
$$
As the square is cartesian we get the two dotted arrows $a'$, $b'$ making the
diagram commute. Since $X' \to Z'$ is universally injective we get $a' = b'$.
This forces $a = b$ as $\{\Spec(K') \to \Spec(K)\}$
is an fpqc covering, see
Properties of Spaces, Proposition
\ref{spaces-properties-proposition-sheaf-fpqc}.
Hence $f$ is universally injective as desired.
\end{proof}

\begin{lemma}
\label{lemma-descending-property-universal-homeomorphism}
The property $\mathcal{P}(f) =$``$f$ is a universal homeomorphism''
is fpqc local on the base.
\end{lemma}

\begin{proof}
This can be proved in exactly the same manner as
Lemma \ref{lemma-descending-property-universally-closed}.
Alternatively, one can use that a map of topological spaces is a
homeomorphism if and only if it is injective, surjective, and open.
Thus a universal homeomorphism is the same thing as a
surjective, universally injective, and universally open morphism.
See Morphisms of Spaces, Lemma
\ref{spaces-morphisms-lemma-base-change-surjective} and
Morphisms of Spaces, Definitions
\ref{spaces-morphisms-definition-universally-injective},
\ref{spaces-morphisms-definition-surjective},
\ref{spaces-morphisms-definition-open},
\ref{spaces-morphisms-definition-universal-homeomorphism}.
Thus the lemma follows from
Lemmas \ref{lemma-descending-property-surjective},
\ref{lemma-descending-property-universally-injective}, and
\ref{lemma-descending-property-universally-open}.
\end{proof}

\begin{lemma}
\label{lemma-descending-property-locally-finite-type}
The property $\mathcal{P}(f) =$``$f$ is locally of finite type''
is fpqc local on the base.
\end{lemma}

\begin{proof}
We will use
Lemma \ref{lemma-descending-properties-morphisms}
to prove this. Assumptions (1) and (2) of that lemma follow from
Morphisms of Spaces,
Lemma \ref{spaces-morphisms-lemma-finite-type-local}.
Let $Z' \to Z$ be a surjective flat morphism of affine schemes over $S$.
Let $f : X \to Z$ be a morphism of algebraic spaces, and assume
that the base change $f' : Z' \times_Z X \to Z'$ is locally of finite type.
We have to show that $f$ is locally of finite type. Let $U$ be a scheme
and let $U \to X$ be surjective and \'etale. By
Morphisms of Spaces,
Lemma \ref{spaces-morphisms-lemma-finite-type-local}
again, it is enough to show that $U \to Z$ is locally of finite type.
Since $f'$ is locally of finite type, and since $Z' \times_Z U$ is a
scheme \'etale over $Z' \times_Z X$ we conclude (by the same lemma again) that
$Z' \times_Z U \to Z'$ is locally of finite type.
As $\{Z' \to Z\}$ is an fpqc covering we conclude that
$U \to Z$ is locally of finite type by
Descent, Lemma \ref{descent-lemma-descending-property-locally-finite-type}
as desired.
\end{proof}

\begin{lemma}
\label{lemma-descending-property-locally-finite-presentation}
The property $\mathcal{P}(f) =$``$f$ is locally of finite presentation''
is fpqc local on the base.
\end{lemma}

\begin{proof}
We will use
Lemma \ref{lemma-descending-properties-morphisms}
to prove this. Assumptions (1) and (2) of that lemma follow from
Morphisms of Spaces,
Lemma \ref{spaces-morphisms-lemma-finite-presentation-local}.
Let $Z' \to Z$ be a surjective flat morphism of affine schemes over $S$.
Let $f : X \to Z$ be a morphism of algebraic spaces, and assume
that the base change $f' : Z' \times_Z X \to Z'$ is locally of
finite presentation.
We have to show that $f$ is locally of finite presentation. Let $U$ be a scheme
and let $U \to X$ be surjective and \'etale. By
Morphisms of Spaces,
Lemma \ref{spaces-morphisms-lemma-finite-presentation-local}
again, it is enough to show that $U \to Z$ is locally of finite presentation.
Since $f'$ is locally of finite presentation, and since $Z' \times_Z U$ is a
scheme \'etale over $Z' \times_Z X$ we conclude (by the same lemma again) that
$Z' \times_Z U \to Z'$ is locally of finite presentation.
As $\{Z' \to Z\}$ is an fpqc covering we conclude that
$U \to Z$ is locally of finite presentation by
Descent,
Lemma \ref{descent-lemma-descending-property-locally-finite-presentation}
as desired.
\end{proof}

\begin{lemma}
\label{lemma-descending-property-finite-type}
The property $\mathcal{P}(f) =$``$f$ is of finite type''
is fpqc local on the base.
\end{lemma}

\begin{proof}
Combine Lemmas \ref{lemma-descending-property-quasi-compact}
and \ref{lemma-descending-property-locally-finite-type}.
\end{proof}

\begin{lemma}
\label{lemma-descending-property-finite-presentation}
The property $\mathcal{P}(f) =$``$f$ is of finite presentation''
is fpqc local on the base.
\end{lemma}

\begin{proof}
Combine Lemmas \ref{lemma-descending-property-quasi-compact},
\ref{lemma-descending-property-quasi-separated} and
\ref{lemma-descending-property-locally-finite-presentation}.
\end{proof}

\begin{lemma}
\label{lemma-descending-property-flat}
The property $\mathcal{P}(f) =$``$f$ is flat''
is fpqc local on the base.
\end{lemma}

\begin{proof}
We will use
Lemma \ref{lemma-descending-properties-morphisms}
to prove this. Assumptions (1) and (2) of that lemma follow from
Morphisms of Spaces,
Lemma \ref{spaces-morphisms-lemma-flat-local}.
Let $Z' \to Z$ be a surjective flat morphism of affine schemes over $S$.
Let $f : X \to Z$ be a morphism of algebraic spaces, and assume
that the base change $f' : Z' \times_Z X \to Z'$ is flat.
We have to show that $f$ is flat. Let $U$ be a scheme
and let $U \to X$ be surjective and \'etale. By
Morphisms of Spaces,
Lemma \ref{spaces-morphisms-lemma-flat-local}
again, it is enough to show that $U \to Z$ is flat.
Since $f'$ is flat, and since $Z' \times_Z U$ is a
scheme \'etale over $Z' \times_Z X$ we conclude (by the same lemma again) that
$Z' \times_Z U \to Z'$ is flat.
As $\{Z' \to Z\}$ is an fpqc covering we conclude that
$U \to Z$ is flat by
Descent, Lemma \ref{descent-lemma-descending-property-flat}
as desired.
\end{proof}

\begin{lemma}
\label{lemma-descending-property-open-immersion}
The property $\mathcal{P}(f) =$``$f$ is an open immersion''
is fpqc local on the base.
\end{lemma}

\begin{proof}
We will use
Lemma \ref{lemma-descending-properties-morphisms}
to prove this. Assumptions (1) and (2) of that lemma follow from
Morphisms of Spaces,
Lemma \ref{spaces-morphisms-lemma-closed-immersion-local}.
Consider a cartesian diagram
$$
\xymatrix{
X' \ar[r] \ar[d] & X \ar[d] \\
Z' \ar[r] & Z
}
$$
of algebraic spaces over $S$
where $Z' \to Z$ is a surjective flat morphism of affine schemes,
and $X' \to Z'$ is an open immersion. We have to show that $X \to Z$
is an open immersion. Note that $|X'| \subset |Z'|$ corresponds to an
open subscheme $U' \subset Z'$ (isomorphic to $X'$)
with the property that $\text{pr}_0^{-1}(U') = \text{pr}_1^{-1}(U')$
as open subschemes of $Z' \times_Z Z'$. Hence there exists an open
subscheme $U \subset Z$ such that $X' = (Z' \to Z)^{-1}(U)$, see
Descent, Lemma \ref{descent-lemma-open-fpqc-covering}.
By Properties of Spaces,
Proposition \ref{spaces-properties-proposition-sheaf-fpqc}
we see that $X$ satisfies the sheaf condition for the fpqc topology.
Now we have the fpqc covering $\mathcal{U} = \{U' \to U\}$
and the element $U' \to X' \to X \in \check{H}^0(\mathcal{U}, X)$.
By the sheaf condition we obtain a morphism $U \to X$ such that
$$
\xymatrix{
U' \ar[r] \ar[d]^{\cong} \ar@/_3ex/[dd] & U \ar[d] \ar@/^3ex/[dd] \\
X' \ar[r] \ar[d] & X \ar[d] \\
Z' \ar[r] & Z
}
$$
is commutative. On the other hand, we know that for any scheme $T$ over $S$
and $T$-valued point $T \to X$ the composition $T \to X \to Z$ is a
morphism such that $Z' \times_Z T \to Z'$ factors through $U'$. Clearly
this means that $T \to Z$ factors through $U$. In other words the map
of sheaves $U \to X$ is bijective and we win.
\end{proof}

\begin{lemma}
\label{lemma-descending-property-isomorphism}
The property $\mathcal{P}(f) =$``$f$ is an isomorphism''
is fpqc local on the base.
\end{lemma}

\begin{proof}
Combine Lemmas \ref{lemma-descending-property-surjective}
and \ref{lemma-descending-property-open-immersion}.
\end{proof}

\begin{lemma}
\label{lemma-descending-property-affine}
The property $\mathcal{P}(f) =$``$f$ is affine''
is fpqc local on the base.
\end{lemma}

\begin{proof}
We will use
Lemma \ref{lemma-descending-properties-morphisms}
to prove this. Assumptions (1) and (2) of that lemma follow from
Morphisms of Spaces,
Lemma \ref{spaces-morphisms-lemma-affine-local}.
Let $Z' \to Z$ be a surjective flat morphism of affine schemes over $S$.
Let $f : X \to Z$ be a morphism of algebraic spaces, and assume
that the base change $f' : Z' \times_Z X \to Z'$ is affine.
Let $X'$ be a scheme representing $Z' \times_Z X$.
We obtain a canonical isomorphism
$$
\varphi : X' \times_Z Z' \longrightarrow Z' \times_Z X'
$$
since both schemes represent the algebraic space $Z' \times_Z Z' \times_Z X$.
This is a descent datum for $X'/Z'/Z$, see
Descent, Definition \ref{descent-definition-descent-datum}
(verification omitted, compare with
Descent, Lemma \ref{descent-lemma-descent-data-sheaves}).
Since $X' \to Z'$ is affine this descent datum is effective, see
Descent, Lemma \ref{descent-lemma-affine}.
Thus there exists a scheme $Y \to Z$ over $Z$ and an
isomorphism $\psi : Z' \times_Z Y \to X'$ compatible with descent data.
Of course $Y \to Z$ is affine (by construction or by
Descent, Lemma \ref{descent-lemma-descending-property-affine}).
Note that $\mathcal{Y} = \{Z' \times_Z Y \to Y\}$ is a
fpqc covering, and interpreting $\psi$ as an element of
$X(Z' \times_Z Y)$ we see that $\psi \in \check{H}^0(\mathcal{Y}, X)$.
By the sheaf condition for $X$ with respect to this covering (see
Properties of Spaces, Proposition
\ref{spaces-properties-proposition-sheaf-fpqc})
we obtain a morphism $Y \to X$.
By construction the base change of this to $Z'$ is an isomorphism, hence
an isomorphism by
Lemma \ref{lemma-descending-property-isomorphism}.
This proves that $X$ is representable by an affine scheme and we win.
\end{proof}

\begin{lemma}
\label{lemma-descending-property-closed-immersion}
The property $\mathcal{P}(f) =$``$f$ is a closed immersion''
is fpqc local on the base.
\end{lemma}

\begin{proof}
We will use
Lemma \ref{lemma-descending-properties-morphisms}
to prove this. Assumptions (1) and (2) of that lemma follow from
Morphisms of Spaces,
Lemma \ref{spaces-morphisms-lemma-closed-immersion-local}.
Consider a cartesian diagram
$$
\xymatrix{
X' \ar[r] \ar[d] & X \ar[d] \\
Z' \ar[r] & Z
}
$$
of algebraic spaces over $S$
where $Z' \to Z$ is a surjective flat morphism of affine schemes,
and $X' \to Z'$ is a closed immersion. We have to show that $X \to Z$
is a closed immersion. The morphism $X' \to Z'$ is affine. Hence by
Lemma \ref{lemma-descending-property-affine}
we see that $X$ is a scheme and $X \to Z$ is affine.
It follows from
Descent, Lemma \ref{descent-lemma-descending-property-closed-immersion}
that $X \to Z$ is a closed immersion as desired.
\end{proof}

\begin{lemma}
\label{lemma-descending-property-separated}
The property $\mathcal{P}(f) =$``$f$ is separated''
is fpqc local on the base.
\end{lemma}

\begin{proof}
A base change of a separated morphism is separated, see
Morphisms of Spaces,
Lemma \ref{spaces-morphisms-lemma-base-change-separated}.
Hence the direct implication in
Definition \ref{definition-property-morphisms-local}.

\medskip\noindent
Let $\{Y_i \to Y\}_{i \in I}$ be an fpqc covering of algebraic spaces over $S$.
Let $f : X \to Y$ be a morphism of algebraic spaces over $S$.
Assume each base change $X_i := Y_i \times_Y X \to Y_i$ is separated.
This means that each of the morphisms
$$
\Delta_i :
X_i
\longrightarrow
X_i \times_{Y_i} X_i = Y_i \times_Y (X \times_Y X)
$$
is a closed immersion. The base change of a fpqc covering is an
fpqc covering, see
Topologies on Spaces, Lemma \ref{spaces-topologies-lemma-fpqc}
hence $\{Y_i \times_Y (X \times_Y X) \to X \times_Y X\}$
is an fpqc covering of algebraic spaces. Moreover, each
$\Delta_i$ is the base change of the morphism
$\Delta : X \to X \times_Y X$. Hence it follows from
Lemma \ref{lemma-descending-property-closed-immersion}
that $\Delta$ is a closed immersion, i.e., $f$ is separated.
\end{proof}

\begin{lemma}
\label{lemma-descending-property-proper}
The property $\mathcal{P}(f) =$``$f$ is proper''
is fpqc local on the base.
\end{lemma}

\begin{proof}
The lemma follows by combining
Lemmas \ref{lemma-descending-property-universally-closed},
\ref{lemma-descending-property-separated}
and \ref{lemma-descending-property-finite-type}.
\end{proof}

\begin{lemma}
\label{lemma-descending-property-quasi-affine}
The property $\mathcal{P}(f) =$``$f$ is quasi-affine''
is fpqc local on the base.
\end{lemma}

\begin{proof}
We will use
Lemma \ref{lemma-descending-properties-morphisms}
to prove this. Assumptions (1) and (2) of that lemma follow from
Morphisms of Spaces,
Lemma \ref{spaces-morphisms-lemma-quasi-affine-local}.
Let $Z' \to Z$ be a surjective flat morphism of affine schemes over $S$.
Let $f : X \to Z$ be a morphism of algebraic spaces, and assume
that the base change $f' : Z' \times_Z X \to Z'$ is quasi-affine.
Let $X'$ be a scheme representing $Z' \times_Z X$.
We obtain a canonical isomorphism
$$
\varphi : X' \times_Z Z' \longrightarrow Z' \times_Z X'
$$
since both schemes represent the algebraic space $Z' \times_Z Z' \times_Z X$.
This is a descent datum for $X'/Z'/Z$, see
Descent, Definition \ref{descent-definition-descent-datum}
(verification omitted, compare with
Descent, Lemma \ref{descent-lemma-descent-data-sheaves}).
Since $X' \to Z'$ is quasi-affine this descent datum is effective, see
Descent, Lemma \ref{descent-lemma-quasi-affine}.
Thus there exists a scheme $Y \to Z$ over $Z$ and an
isomorphism $\psi : Z' \times_Z Y \to X'$ compatible with descent data.
Of course $Y \to Z$ is quasi-affine (by construction or by
Descent, Lemma \ref{descent-lemma-descending-property-quasi-affine}).
Note that $\mathcal{Y} = \{Z' \times_Z Y \to Y\}$ is a
fpqc covering, and interpreting $\psi$ as an element of
$X(Z' \times_Z Y)$ we see that $\psi \in \check{H}^0(\mathcal{Y}, X)$.
By the sheaf condition for $X$ (see
Properties of Spaces, Proposition
\ref{spaces-properties-proposition-sheaf-fpqc})
we obtain a morphism $Y \to X$.
By construction the base change of this to $Z'$ is an isomorphism, hence
an isomorphism by
Lemma \ref{lemma-descending-property-isomorphism}.
This proves that $X$ is representable by a quasi-affine scheme and we win.
\end{proof}

\begin{lemma}
\label{lemma-descending-property-quasi-compact-immersion}
The property $\mathcal{P}(f) =$``$f$ is a quasi-compact immersion''
is fpqc local on the base.
\end{lemma}

\begin{proof}
We will use
Lemma \ref{lemma-descending-properties-morphisms}
to prove this. Assumptions (1) and (2) of that lemma follow from
Morphisms of Spaces,
Lemmas \ref{spaces-morphisms-lemma-closed-immersion-local} and
\ref{spaces-morphisms-lemma-quasi-compact-local}.
Consider a cartesian diagram
$$
\xymatrix{
X' \ar[r] \ar[d] & X \ar[d] \\
Z' \ar[r] & Z
}
$$
of algebraic spaces over $S$
where $Z' \to Z$ is a surjective flat morphism of affine schemes,
and $X' \to Z'$ is a quasi-compact immersion. We have to show that $X \to Z$
is a closed immersion. The morphism $X' \to Z'$ is quasi-affine. Hence by
Lemma \ref{lemma-descending-property-quasi-affine}
we see that $X$ is a scheme and $X \to Z$ is quasi-affine.
It follows from
Descent, Lemma \ref{descent-lemma-descending-property-quasi-compact-immersion}
that $X \to Z$ is a quasi-compact immersion as desired.
\end{proof}

\begin{lemma}
\label{lemma-descending-property-integral}
The property $\mathcal{P}(f) =$``$f$ is integral''
is fpqc local on the base.
\end{lemma}

\begin{proof}
An integral morphism is the same thing as an affine,
universally closed morphism. See
Morphisms of Spaces,
Lemma \ref{spaces-morphisms-lemma-integral-universally-closed}.
Hence the lemma follows on combining
Lemmas \ref{lemma-descending-property-universally-closed}
and \ref{lemma-descending-property-affine}.
\end{proof}

\begin{lemma}
\label{lemma-descending-property-finite}
The property $\mathcal{P}(f) =$``$f$ is finite''
is fpqc local on the base.
\end{lemma}

\begin{proof}
An finite morphism is the same thing as an integral,
morphism which is locally of finite type. See
Morphisms of Spaces, Lemma \ref{spaces-morphisms-lemma-finite-integral}.
Hence the lemma follows on combining
Lemmas \ref{lemma-descending-property-locally-finite-type}
and \ref{lemma-descending-property-integral}.
\end{proof}

\begin{lemma}
\label{lemma-descending-property-quasi-finite}
The properties
$\mathcal{P}(f) =$``$f$ is locally quasi-finite''
and
$\mathcal{P}(f) =$``$f$ is quasi-finite''
are fpqc local on the base.
\end{lemma}

\begin{proof}
We have already seen that ``quasi-compact'' is fpqc local on the base, see
Lemma \ref{lemma-descending-property-quasi-compact}. Hence it is enough
to prove the lemma for ``locally quasi-finite''. We will use
Lemma \ref{lemma-descending-properties-morphisms}
to prove this. Assumptions (1) and (2) of that lemma follow from
Morphisms of Spaces,
Lemma \ref{spaces-morphisms-lemma-quasi-finite-local}.
Let $Z' \to Z$ be a surjective flat morphism of affine schemes over $S$.
Let $f : X \to Z$ be a morphism of algebraic spaces, and assume
that the base change $f' : Z' \times_Z X \to Z'$ is locally quasi-finite.
We have to show that $f$ is locally quasi-finite. Let $U$ be a scheme
and let $U \to X$ be surjective and \'etale. By
Morphisms of Spaces,
Lemma \ref{spaces-morphisms-lemma-quasi-finite-local}
again, it is enough to show that $U \to Z$ is locally quasi-finite.
Since $f'$ is locally quasi-finite, and since $Z' \times_Z U$ is a
scheme \'etale over $Z' \times_Z X$ we conclude (by the same lemma again) that
$Z' \times_Z U \to Z'$ is locally quasi-finite.
As $\{Z' \to Z\}$ is an fpqc covering we conclude that
$U \to Z$ is locally quasi-finite by
Descent, Lemma \ref{descent-lemma-descending-property-quasi-finite}
as desired.
\end{proof}

\begin{lemma}
\label{lemma-descending-property-syntomic}
The property $\mathcal{P}(f) =$``$f$ is syntomic''
is fpqc local on the base.
\end{lemma}

\begin{proof}
We will use
Lemma \ref{lemma-descending-properties-morphisms}
to prove this. Assumptions (1) and (2) of that lemma follow from
Morphisms of Spaces,
Lemma \ref{spaces-morphisms-lemma-syntomic-local}.
Let $Z' \to Z$ be a surjective flat morphism of affine schemes over $S$.
Let $f : X \to Z$ be a morphism of algebraic spaces, and assume
that the base change $f' : Z' \times_Z X \to Z'$ is syntomic.
We have to show that $f$ is syntomic. Let $U$ be a scheme
and let $U \to X$ be surjective and \'etale. By
Morphisms of Spaces,
Lemma \ref{spaces-morphisms-lemma-syntomic-local}
again, it is enough to show that $U \to Z$ is syntomic.
Since $f'$ is syntomic, and since $Z' \times_Z U$ is a
scheme \'etale over $Z' \times_Z X$ we conclude (by the same lemma again) that
$Z' \times_Z U \to Z'$ is syntomic.
As $\{Z' \to Z\}$ is an fpqc covering we conclude that
$U \to Z$ is syntomic by
Descent, Lemma \ref{descent-lemma-descending-property-syntomic}
as desired.
\end{proof}

\begin{lemma}
\label{lemma-descending-property-smooth}
The property $\mathcal{P}(f) =$``$f$ is smooth''
is fpqc local on the base.
\end{lemma}

\begin{proof}
We will use
Lemma \ref{lemma-descending-properties-morphisms}
to prove this. Assumptions (1) and (2) of that lemma follow from
Morphisms of Spaces,
Lemma \ref{spaces-morphisms-lemma-smooth-local}.
Let $Z' \to Z$ be a surjective flat morphism of affine schemes over $S$.
Let $f : X \to Z$ be a morphism of algebraic spaces, and assume
that the base change $f' : Z' \times_Z X \to Z'$ is smooth.
We have to show that $f$ is smooth. Let $U$ be a scheme
and let $U \to X$ be surjective and \'etale. By
Morphisms of Spaces,
Lemma \ref{spaces-morphisms-lemma-smooth-local}
again, it is enough to show that $U \to Z$ is smooth.
Since $f'$ is smooth, and since $Z' \times_Z U$ is a
scheme \'etale over $Z' \times_Z X$ we conclude (by the same lemma again) that
$Z' \times_Z U \to Z'$ is smooth.
As $\{Z' \to Z\}$ is an fpqc covering we conclude that
$U \to Z$ is smooth by
Descent, Lemma \ref{descent-lemma-descending-property-smooth}
as desired.
\end{proof}

\begin{lemma}
\label{lemma-descending-property-unramified}
The property $\mathcal{P}(f) =$``$f$ is unramified''
is fpqc local on the base.
\end{lemma}

\begin{proof}
We will use
Lemma \ref{lemma-descending-properties-morphisms}
to prove this. Assumptions (1) and (2) of that lemma follow from
Morphisms of Spaces,
Lemma \ref{spaces-morphisms-lemma-unramified-local}.
Let $Z' \to Z$ be a surjective flat morphism of affine schemes over $S$.
Let $f : X \to Z$ be a morphism of algebraic spaces, and assume
that the base change $f' : Z' \times_Z X \to Z'$ is unramified.
We have to show that $f$ is unramified. Let $U$ be a scheme
and let $U \to X$ be surjective and \'etale. By
Morphisms of Spaces,
Lemma \ref{spaces-morphisms-lemma-unramified-local}
again, it is enough to show that $U \to Z$ is unramified.
Since $f'$ is unramified, and since $Z' \times_Z U$ is a
scheme \'etale over $Z' \times_Z X$ we conclude (by the same lemma again) that
$Z' \times_Z U \to Z'$ is unramified.
As $\{Z' \to Z\}$ is an fpqc covering we conclude that
$U \to Z$ is unramified by
Descent, Lemma \ref{descent-lemma-descending-property-unramified}
as desired.
\end{proof}

\begin{lemma}
\label{lemma-descending-property-etale}
The property $\mathcal{P}(f) =$``$f$ is \'etale''
is fpqc local on the base.
\end{lemma}

\begin{proof}
We will use
Lemma \ref{lemma-descending-properties-morphisms}
to prove this. Assumptions (1) and (2) of that lemma follow from
Morphisms of Spaces,
Lemma \ref{spaces-morphisms-lemma-etale-local}.
Let $Z' \to Z$ be a surjective flat morphism of affine schemes over $S$.
Let $f : X \to Z$ be a morphism of algebraic spaces, and assume
that the base change $f' : Z' \times_Z X \to Z'$ is \'etale.
We have to show that $f$ is \'etale. Let $U$ be a scheme
and let $U \to X$ be surjective and \'etale. By
Morphisms of Spaces,
Lemma \ref{spaces-morphisms-lemma-etale-local}
again, it is enough to show that $U \to Z$ is \'etale.
Since $f'$ is \'etale, and since $Z' \times_Z U$ is a
scheme \'etale over $Z' \times_Z X$ we conclude (by the same lemma again) that
$Z' \times_Z U \to Z'$ is \'etale.
As $\{Z' \to Z\}$ is an fpqc covering we conclude that
$U \to Z$ is \'etale by
Descent, Lemma \ref{descent-lemma-descending-property-etale}
as desired.
\end{proof}

\begin{lemma}
\label{lemma-descending-property-finite-locally-free}
The property $\mathcal{P}(f) =$``$f$ is finite locally free''
is fpqc local on the base.
\end{lemma}

\begin{proof}
Being finite locally free is equivalent to being
finite, flat and locally of finite presentation
(Morphisms of Spaces, Lemma \ref{spaces-morphisms-lemma-finite-flat}).
Hence this follows from Lemmas
\ref{lemma-descending-property-finite},
\ref{lemma-descending-property-flat}, and
\ref{lemma-descending-property-locally-finite-presentation}.
\end{proof}

\begin{lemma}
\label{lemma-descending-property-monomorphism}
The property $\mathcal{P}(f) =$``$f$ is a monomorphism''
is fpqc local on the base.
\end{lemma}

\begin{proof}
Let $f : X \to Y$ be a morphism of algebraic spaces.
Let $\{Y_i \to Y\}$ be an fpqc covering, and assume
each of the base changes $f_i : X_i \to Y_i$ of $f$ is
a monomorphism. We have to show that $f$ is a monomorphism.

\medskip\noindent
First proof. Note that $f$ is a monomorphism if and only if
$\Delta : X \to X \times_Y X$ is an isomorphism. By applying this to
$f_i$ we see that each of the morphisms
$$
\Delta_i :
X_i
\longrightarrow
X_i \times_{Y_i} X_i = Y_i \times_Y (X \times_Y X)
$$
is an isomorphism. The base change of an fpqc covering is an fpqc covering, see
Topologies on Spaces, Lemma \ref{spaces-topologies-lemma-fpqc}
hence $\{Y_i \times_Y (X \times_Y X) \to X \times_Y X\}$
is an fpqc covering of algebraic spaces. Moreover, each
$\Delta_i$ is the base change of the morphism
$\Delta : X \to X \times_Y X$. Hence it follows from
Lemma \ref{lemma-descending-property-isomorphism}
that $\Delta$ is an isomorphism, i.e., $f$ is a monomorphism.

\medskip\noindent
Second proof.
Let $V$ be a scheme, and let $V \to Y$ be a surjective \'etale morphism.
If we can show that $V \times_Y X \to V$ is a monomorphism, then it
follows that $X \to Y$ is a monomorphism. Namely, given any
cartesian diagram of sheaves
$$
\vcenter{
\xymatrix{
\mathcal{F} \ar[r]_a \ar[d]_b & \mathcal{G} \ar[d]^c \\
\mathcal{H} \ar[r]^d & \mathcal{I}
}
}
\quad
\quad
\mathcal{F} = \mathcal{H} \times_\mathcal{I} \mathcal{G}
$$
if $c$ is a surjection of sheaves, and $a$ is injective, then also
$d$ is injective. This reduces the problem to the case where $Y$ is
a scheme. Moreover, in this case we may assume that the algebraic spaces
$Y_i$ are schemes also, since we can always refine the covering to place
ourselves in this situation, see
Topologies on Spaces, Lemma \ref{spaces-topologies-lemma-refine-fpqc-schemes}.

\medskip\noindent
Assume $\{Y_i \to Y\}$ is an fpqc covering of schemes.
Let $a, b : T \to X$ be two morphisms
such that $f \circ a = f \circ b$. We have to show that $a = b$.
Since $f_i$ is a monomorphism we see that $a_i = b_i$, where
$a_i, b_i : Y_i \times_Y T \to X_i$ are
the base changes. In particular the compositions
$Y_i \times_Y T \to T \to X$ are equal.
Since $\{Y_i \times_Y T \to T\}$ is an fpqc covering we
deduce that $a = b$ from Properties of Spaces, Proposition
\ref{spaces-properties-proposition-sheaf-fpqc}.
\end{proof}




\section{Descending properties of morphisms in the fppf topology}
\label{section-descending-properties-morphisms-fppf}

\noindent
In this section we find some properties of morphisms of algebraic spaces
for which we could not (yet) show they are local on the base in
the fpqc topology which, however, are local on the base
in the fppf topology.

\begin{lemma}
\label{lemma-descending-fppf-property-immersion}
The property $\mathcal{P}(f) =$``$f$ is an immersion''
is fppf local on the base.
\end{lemma}

\begin{proof}
Let $f : X \to Y$ be a morphism of algebraic spaces.
Let $\{Y_i \to Y\}_{i \in I}$ be an fppf covering of $Y$.
Let $f_i : X_i \to Y_i$ be the base change of $f$.

\medskip\noindent
If $f$ is an immersion, then each $f_i$ is an immersion by
Spaces, Lemma \ref{spaces-lemma-base-change-immersions}.
This proves the direct implication in
Definition \ref{definition-property-morphisms-local}.

\medskip\noindent
Conversely, assume each $f_i$ is an immersion. By
Morphisms of Spaces,
Lemma \ref{spaces-morphisms-lemma-immersions-monomorphisms}
this implies each $f_i$ is separated. By
Morphisms of Spaces,
Lemma \ref{spaces-morphisms-lemma-immersion-quasi-finite}
this implies each $f_i$ is locally quasi-finite.
Hence we see that $f$ is locally quasi-finite and separated, by applying
Lemmas \ref{lemma-descending-property-separated}
and \ref{lemma-descending-property-quasi-finite}.
By
Morphisms of Spaces, Lemma
\ref{spaces-morphisms-lemma-locally-quasi-finite-separated-representable}
this implies that $f$ is representable!

\medskip\noindent
By
Morphisms of Spaces, Lemma \ref{spaces-morphisms-lemma-closed-immersion-local}
it suffices to show that for every scheme $Z$ and morphism $Z \to Y$
the base change $Z \times_Y X \to Z$ is an immersion. By
Topologies on Spaces, Lemma \ref{spaces-topologies-lemma-refine-fppf-schemes}
we can find an fppf covering $\{Z_i \to Z\}$ by schemes which refines
the pullback of the covering $\{Y_i \to Y\}$ to $Z$.
Hence we see that $Z \times_Y X \to Z$ (which is a morphism of schemes
according to the result of the preceding paragraph) becomes an immersion
after pulling back to the members of an fppf (by schemes) of $Z$.
Hence $Z \times_Y X \to Z$ is an immersion by the result for schemes, see
Descent, Lemma \ref{descent-lemma-descending-fppf-property-immersion}.
\end{proof}

\begin{lemma}
\label{lemma-descending-fppf-property-locally-separated}
The property $\mathcal{P}(f) =$``$f$ is locally separated''
is fppf local on the base.
\end{lemma}

\begin{proof}
A base change of a locally separated morphism is locally separated, see
Morphisms of Spaces,
Lemma \ref{spaces-morphisms-lemma-base-change-separated}.
Hence the direct implication in
Definition \ref{definition-property-morphisms-local}.

\medskip\noindent
Let $\{Y_i \to Y\}_{i \in I}$ be an fppf covering of algebraic spaces over $S$.
Let $f : X \to Y$ be a morphism of algebraic spaces over $S$.
Assume each base change $X_i := Y_i \times_Y X \to Y_i$ is locally separated.
This means that each of the morphisms
$$
\Delta_i :
X_i
\longrightarrow
X_i \times_{Y_i} X_i = Y_i \times_Y (X \times_Y X)
$$
is an immersion. The base change of a fppf covering is an
fppf covering, see
Topologies on Spaces, Lemma \ref{spaces-topologies-lemma-fppf}
hence $\{Y_i \times_Y (X \times_Y X) \to X \times_Y X\}$
is an fppf covering of algebraic spaces. Moreover, each
$\Delta_i$ is the base change of the morphism
$\Delta : X \to X \times_Y X$. Hence it follows from
Lemma \ref{lemma-descending-fppf-property-immersion}
that $\Delta$ is a immersion, i.e., $f$ is locally separated.
\end{proof}





\section{Application of descent of properties of morphisms}
\label{section-application-descending-properties-morphisms}

\noindent
This section is the analogue of Descent, Section
\ref{descent-section-application-descending-properties-morphisms}.

\begin{lemma}
\label{lemma-descending-property-ample}
Let $S$ be a scheme.
Let $f : X \to Y$ be a morphism of algebraic spaces over $S$.
Let $\mathcal{L}$ be an invertible $\mathcal{O}_X$-module.
Let $\{g_i : Y_i \to Y\}_{i \in I}$ be an fpqc covering.
Let $f_i : X_i \to Y_i$ be the base change of $f$ and let $\mathcal{L}_i$
be the pullback of $\mathcal{L}$ to $X_i$.
The following are equivalent
\begin{enumerate}
\item $\mathcal{L}$ is ample on $X/Y$, and
\item $\mathcal{L}_i$ is ample on $X_i/Y_i$
for every $i \in I$.
\end{enumerate}
\end{lemma}

\begin{proof}
The implication (1) $\Rightarrow$ (2) follows from
Divisors on Spaces, Lemma \ref{spaces-divisors-lemma-ample-base-change}.
Assume (2). To check $\mathcal{L}$ is ample on $X/Y$ we may
work \'etale localy on $Y$, see
Divisors on Spaces, Lemma \ref{spaces-divisors-lemma-relatively-ample-local}.
Thus we may assume that $Y$ is a scheme and then we may
in turn assume each $Y_i$ is a scheme too, see
Topologies on Spaces, Lemma \ref{spaces-topologies-lemma-refine-fpqc-schemes}.
In other words, we may assume that
$\{Y_i \to Y\}$ is an fpqc covering of schemes.

\medskip\noindent
By Divisors on Spaces, Lemma
\ref{spaces-divisors-lemma-relatively-ample-properties}
we see that $X_i \to Y_i$ is representable (i.e., $X_i$ is a scheme),
quasi-compact, and separated. Hence $f$ is quasi-compact and separated by
Lemmas \ref{lemma-descending-property-quasi-compact} and
\ref{lemma-descending-property-separated}.
This means that
$\mathcal{A} = \bigoplus_{d \geq 0} f_*\mathcal{L}^{\otimes d}$
is a quasi-coherent graded $\mathcal{O}_Y$-algebra
(Morphisms of Spaces, Lemma \ref{spaces-morphisms-lemma-pushforward}).
Moreover, the formation of $\mathcal{A}$ commutes with flat
base change by
Cohomology of Spaces, Lemma
\ref{spaces-cohomology-lemma-flat-base-change-cohomology}.
In particular, if we set
$\mathcal{A}_i = \bigoplus_{d \geq 0} f_{i, *}\mathcal{L}_i^{\otimes d}$
then we have $\mathcal{A}_i = g_i^*\mathcal{A}$.
It follows that the natural maps
$\psi_d : f^*\mathcal{A}_d \to \mathcal{L}^{\otimes d}$
of $\mathcal{O}_X$
pullback to give the natural maps
$\psi_{i, d} : f_i^*(\mathcal{A}_i)_d \to \mathcal{L}_i^{\otimes d}$
of $\mathcal{O}_{X_i}$-modules. Since $\mathcal{L}_i$ is ample on $X_i/Y_i$
we see that for any point $x_i \in X_i$, there exists a $d \geq 1$
such that $f_i^*(\mathcal{A}_i)_d \to \mathcal{L}_i^{\otimes d}$
is surjective on stalks at $x_i$. This follows either directly
from the definition of a relatively ample module or from
Morphisms, Lemma \ref{morphisms-lemma-characterize-relatively-ample}.
If $x \in |X|$, then we can choose an $i$ and an $x_i \in X_i$
mapping to $x$. Since
$\mathcal{O}_{X, \overline{x}} \to \mathcal{O}_{X_i, \overline{x}_i}$
is flat hence faithfully flat, we conclude that for every $x \in |X|$
there exists a $d \geq 1$ such that
$f^*\mathcal{A}_d \to \mathcal{L}^{\otimes d}$
is surjective on stalks at $x$.
This implies that the open subset $U(\psi) \subset X$ of
Divisors on Spaces, Lemma
\ref{spaces-divisors-lemma-invertible-map-into-relative-proj}
corresponding to the map
$\psi : f^*\mathcal{A} \to \bigoplus_{d \geq 0} \mathcal{L}^{\otimes d}$
of graded $\mathcal{O}_X$-algebras
is equal to $X$. Consider the corresponding morphism
$$
r_{\mathcal{L}, \psi} : X \longrightarrow \underline{\text{Proj}}_Y(\mathcal{A})
$$
It is clear from the above that the base change of
$r_{\mathcal{L}, \psi}$ to $Y_i$ is the morphism
$r_{\mathcal{L}_i, \psi_i}$ which is an open immersion by
Morphisms, Lemma \ref{morphisms-lemma-characterize-relatively-ample}.
Hence $r_{\mathcal{L}, \psi}$ is an open immersion
by Lemma \ref{lemma-descending-property-open-immersion}.
Hence $X$ is a scheme
and we conclude $\mathcal{L}$ is ample on $X/Y$ by
Morphisms, Lemma \ref{morphisms-lemma-characterize-relatively-ample}.
\end{proof}

\begin{lemma}
\label{lemma-ample-in-neighbourhood}
Let $S$ be a scheme.
Let $f : X \to Y$ be a proper morphism of algebraic spaces over $S$.
Let $\mathcal{L}$ be an invertible $\mathcal{O}_X$-module.
There exists an open subspace $V \subset Y$ characterized by
the following property:
A morphism $Y' \to Y$ of algebraic spaces factors
through $V$ if and only if the pullback $\mathcal{L}'$
of $\mathcal{L}$ to $X' = Y' \times_Y X$ is ample on $X'/Y'$
(as in Divisors on Spaces, Definition
\ref{spaces-divisors-definition-relatively-ample}).
\end{lemma}

\begin{proof}
Suppose that the lemma holds whenever $Y$ is a scheme.
Let $U$ be a scheme and let $U \to Y$ be a surjective \'etale morphism.
Let $R = U \times_Y U$ with projections $t, s : R \to U$.
Denote $X_U = U \times_Y X$ and $\mathcal{L}_U$ the pullback.
Then we get an open subscheme $V' \subset U$ as in the lemma
for $(X_U \to U, \mathcal{L}_U)$. By the functorial characterization
we see that $s^{-1}(V') = t^{-1}(V')$.
Thus there is an open subspace $V \subset Y$ such that
$V'$ is the inverse image of $V$ in $U$.
In particular $V' \to V$ is surjective \'etale and we
conclude that $\mathcal{L}_V$ is ample on $X_V/V$
(Divisors on Spaces, Lemma \ref{spaces-divisors-lemma-relatively-ample-local}).
Now, if $Y' \to Y$ is a morphism such that
$\mathcal{L}'$ is ample on $X'/Y'$, then
$U \times_Y Y' \to Y'$ must factor through $V'$
and we conclude that $Y' \to Y$ factors through $V$.
Hence $V \subset Y$ is as in the statement of the lemma.
In this way we reduce to the case dealt with in the next
paragraph.

\medskip\noindent
Assume $Y$ is a scheme. Since the question is local on $Y$
we may assume $Y$ is an affine scheme. We will show the
following:
\begin{enumerate}
\item[(A)] If $\Spec(k) \to Y$ is a morphism such that
$\mathcal{L}_k$ is ample on $X_k/k$, then there is an
open neighbourhood $V \subset Y$ of the image of $\Spec(k) \to Y$
such that $\mathcal{L}_V$ is ample on $X_V/V$.
\end{enumerate}
It is clear that (A) implies the truth of the lemma.

\medskip\noindent
Let $X \to Y$, $\mathcal{L}$, $\Spec(k) \to Y$ be as in (A).
By Lemma \ref{lemma-descending-property-ample}
we may assume that $k = \kappa(y)$ is the residue field of a point $y$
of $Y$.

\medskip\noindent
As $Y$ is affine we can find a directed set $I$ and an
inverse system of morphisms $X_i \to Y_i$ of algebraic spaces
with $Y_i$ of finite presentation over $\mathbf{Z}$, with affine
transition morphisms $X_i \to X_{i'}$ and $Y_i \to Y_{i'}$,
with $X_i \to Y_i$ proper and of finite presentation, and such that
$X \to Y = \lim (X_i \to Y_i)$. See Limits of Spaces, Lemma
\ref{spaces-limits-lemma-proper-limit-of-proper-finite-presentation-noetherian}.
After shrinking $I$ we may assume $Y_i$ is an (affine) scheme for all $i$,
see Limits of Spaces, Lemma \ref{spaces-limits-lemma-limit-is-affine}.
After shrinking $I$ we can assume we have a compatible system of
invertible $\mathcal{O}_{X_i}$-modules $\mathcal{L}_i$
pulling back to $\mathcal{L}$, see
Limits of Spaces, Lemma \ref{spaces-limits-lemma-descend-invertible-modules}.
Let $y_i \in Y_i$ be the image of $y$.
Then $\kappa(y) = \colim \kappa(y_i)$.
Hence $X_y = \lim X_{i, y_i}$ and after shrinking $I$ we
may assume $X_{i, y_i}$ is a scheme for all $i$, see
Limits of Spaces, Lemma \ref{spaces-limits-lemma-limit-is-scheme}.
Hence for some $i$ we have $\mathcal{L}_{i, y_i}$
is ample on $X_{i, y_i}$ by
Limits, Lemma \ref{limits-lemma-limit-ample}.
By Divisors on Spaces, Lemma \ref{spaces-divisors-lemma-ample-in-neighbourhood}
we find an open neigbourhood
$V_i \subset Y_i$ of $y_i$ such that
$\mathcal{L}_i$ restricted to $f_i^{-1}(V_i)$
is ample relative to $V_i$.
Letting $V \subset Y$ be the inverse image of
$V_i$ finishes the proof (hints: use
Morphisms, Lemma \ref{morphisms-lemma-ample-base-change} and
the fact that $X \to Y \times_{Y_i} X_i$ is affine
and the fact that the pullback of an
ample invertible sheaf by an affine morphism is ample by
Morphisms, Lemma \ref{morphisms-lemma-pullback-ample-tensor-relatively-ample}).
\end{proof}













\section{Properties of morphisms local on the source}
\label{section-properties-morphisms-local-source}

\noindent
In this section we define what it means for a property of morphisms of
algebraic spaces to be local on the source. Please compare with
Descent, Section \ref{descent-section-properties-morphisms-local-source}.

\begin{definition}
\label{definition-property-morphisms-local-source}
Let $S$ be a scheme.
Let $\mathcal{P}$ be a property of morphisms of algebraic spaces over $S$.
Let $\tau \in \{fpqc, \linebreak[0] fppf, \linebreak[0] syntomic, \linebreak[0]
smooth, \linebreak[0] \etale\}$. We say $\mathcal{P}$ is
{\it $\tau$ local on the source}, or
{\it local on the source for the $\tau$-topology} if for
any morphism $f : X \to Y$ of algebraic spaces over $S$, and any
$\tau$-covering $\{X_i \to X\}_{i \in I}$ of algebraic spaces we have
$$
f \text{ has }\mathcal{P}
\Leftrightarrow
\text{each }X_i \to Y\text{ has }\mathcal{P}.
$$
\end{definition}

\noindent
To be sure, since isomorphisms are always coverings
we see (or require) that property $\mathcal{P}$ holds for $X \to Y$
if and only if it holds for any arrow $X' \to Y'$ isomorphic to $X \to Y$.
If a property is $\tau$-local on the source then it is preserved by
precomposing with morphisms which occur in $\tau$-coverings. Here
is a formal statement.

\begin{lemma}
\label{lemma-precompose-property-local-source}
Let $S$ be a scheme.
Let $\tau \in \{fpqc, \linebreak[0] fppf, \linebreak[0] syntomic, \linebreak[0]
smooth, \linebreak[0] \etale\}$.
Let $\mathcal{P}$ be a property of morphisms of algebraic spaces over $S$
which is $\tau$ local on the source. Let $f : X \to Y$ have property
$\mathcal{P}$. For any morphism $a : X' \to X$ which is
flat, resp.\ flat and locally of finite presentation, resp.\ syntomic,
resp.\ smooth, resp.\ \'etale, the composition $f \circ a : X' \to Y$ has
property $\mathcal{P}$.
\end{lemma}

\begin{proof}
This is true because we can fit $X' \to X$ into a family of
morphisms which forms a $\tau$-covering.
\end{proof}

\begin{lemma}
\label{lemma-transfer-from-schemes}
Let $S$ be a scheme.
Let $\tau \in \{fpqc, \linebreak[0] fppf, \linebreak[0] syntomic, \linebreak[0]
smooth, \linebreak[0] \etale\}$.
Suppose that $\mathcal{P}$ is a property of morphisms of schemes over $S$
which is \'etale local on the source-and-target. Denote $\mathcal{P}_{spaces}$
the corresponding property of morphisms of algebraic spaces over $S$, see
Morphisms of Spaces, Definition \ref{spaces-morphisms-definition-P}.
If $\mathcal{P}$ is local on the source for the $\tau$-topology, then
$\mathcal{P}_{spaces}$ is local on the source for the $\tau$-topology.
\end{lemma}

\begin{proof}
Let $f : X \to Y$ be a morphism of algebraic spaces over $S$.
Let $\{X_i \to X\}_{i \in I}$ be a $\tau$-covering of algebraic spaces.
Choose a scheme $V$ and a surjective \'etale morphism $V \to Y$.
Choose a scheme $U$ and a surjective \'etale morphism $U \to X \times_Y V$.
For each $i$ choose a scheme $U_i$ and a surjective \'etale morphism
$U_i \to X_i \times_X U$.

\medskip\noindent
Note that $\{X_i \times_X U \to U\}_{i \in I}$ is a $\tau$-covering.
Note that each $\{U_i \to X_i \times_X U\}$ is an \'etale covering,
hence a $\tau$-covering. Hence $\{U_i \to U\}_{i \in I}$ is a
$\tau$-covering of algebraic spaces over $S$. But since $U$ and each $U_i$
is a scheme we see that $\{U_i \to U\}_{i \in I}$ is a
$\tau$-covering of schemes over $S$.

\medskip\noindent
Now we have
\begin{align*}
f \text{ has }\mathcal{P}_{spaces}
& \Leftrightarrow
U \to V \text{ has }\mathcal{P} \\
& \Leftrightarrow
\text{each }U_i \to V \text{ has }\mathcal{P} \\
& \Leftrightarrow
\text{each }X_i \to Y\text{ has }\mathcal{P}_{spaces}.
\end{align*}
the first and last equivalence by the definition of
$\mathcal{P}_{spaces}$ the middle equivalence because we assumed
$\mathcal{P}$ is local on the source in the $\tau$-topology.
\end{proof}






\section{Properties of morphisms local in the fpqc topology on the source}
\label{section-fpqc-local-source}

\noindent
Here are some properties of morphisms that are fpqc local on the source.

\begin{lemma}
\label{lemma-flat-fpqc-local-source}
The property $\mathcal{P}(f)=$``$f$ is flat'' is fpqc local on the source.
\end{lemma}

\begin{proof}
Follows from
Lemma \ref{lemma-transfer-from-schemes}
using
Morphisms of Spaces, Definition \ref{spaces-morphisms-definition-flat}
and
Descent, Lemma \ref{descent-lemma-flat-fpqc-local-source}.
\end{proof}










\section{Properties of morphisms local in the fppf topology on the source}
\label{section-fppf-local-source}

\noindent
Here are some properties of morphisms that are fppf local on the source.

\begin{lemma}
\label{lemma-locally-finite-presentation-fppf-local-source}
The property $\mathcal{P}(f)=$``$f$ is locally of finite presentation''
is fppf local on the source.
\end{lemma}

\begin{proof}
Follows from
Lemma \ref{lemma-transfer-from-schemes}
using
Morphisms of Spaces,
Definition \ref{spaces-morphisms-definition-locally-finite-presentation}
and
Descent,
Lemma \ref{descent-lemma-locally-finite-presentation-fppf-local-source}.
\end{proof}

\begin{lemma}
\label{lemma-locally-finite-type-fppf-local-source}
The property $\mathcal{P}(f)=$``$f$ is locally of finite type''
is fppf local on the source.
\end{lemma}

\begin{proof}
Follows from
Lemma \ref{lemma-transfer-from-schemes}
using
Morphisms of Spaces,
Definition \ref{spaces-morphisms-definition-locally-finite-type}
and
Descent, Lemma \ref{descent-lemma-locally-finite-type-fppf-local-source}.
\end{proof}

\begin{lemma}
\label{lemma-open-fppf-local-source}
The property $\mathcal{P}(f)=$``$f$ is open''
is fppf local on the source.
\end{lemma}

\begin{proof}
Follows from
Lemma \ref{lemma-transfer-from-schemes}
using
Morphisms of Spaces, Definition \ref{spaces-morphisms-definition-open}
and
Descent, Lemma \ref{descent-lemma-open-fppf-local-source}.
\end{proof}

\begin{lemma}
\label{lemma-universally-open-fppf-local-source}
The property $\mathcal{P}(f)=$``$f$ is universally open''
is fppf local on the source.
\end{lemma}

\begin{proof}
Follows from
Lemma \ref{lemma-transfer-from-schemes}
using
Morphisms of Spaces, Definition \ref{spaces-morphisms-definition-open}
and
Descent, Lemma \ref{descent-lemma-universally-open-fppf-local-source}.
\end{proof}



\section{Properties of morphisms local in the syntomic topology on the source}
\label{section-syntomic-local-source}

\noindent
Here are some properties of morphisms that are syntomic local on the source.

\begin{lemma}
\label{lemma-syntomic-syntomic-local-source}
The property $\mathcal{P}(f)=$``$f$ is syntomic''
is syntomic local on the source.
\end{lemma}

\begin{proof}
Follows from
Lemma \ref{lemma-transfer-from-schemes}
using
Morphisms of Spaces, Definition \ref{spaces-morphisms-definition-syntomic}
and
Descent, Lemma \ref{descent-lemma-syntomic-syntomic-local-source}.
\end{proof}




\section{Properties of morphisms local in the smooth topology on the source}
\label{section-smooth-local-source}

\noindent
Here are some properties of morphisms that are smooth local on the source.

\begin{lemma}
\label{lemma-smooth-smooth-local-source}
The property $\mathcal{P}(f)=$``$f$ is smooth''
is smooth local on the source.
\end{lemma}

\begin{proof}
Follows from
Lemma \ref{lemma-transfer-from-schemes}
using
Morphisms of Spaces, Definition \ref{spaces-morphisms-definition-smooth}
and
Descent, Lemma \ref{descent-lemma-smooth-smooth-local-source}.
\end{proof}



\section{Properties of morphisms local in the \'etale topology on the source}
\label{section-etale-local-source}

\noindent
Here are some properties of morphisms that are \'etale local on the source.

\begin{lemma}
\label{lemma-etale-etale-local-source}
The property $\mathcal{P}(f)=$``$f$ is \'etale''
is \'etale local on the source.
\end{lemma}

\begin{proof}
Follows from
Lemma \ref{lemma-transfer-from-schemes}
using
Morphisms of Spaces,
Definition \ref{spaces-morphisms-definition-etale}
and
Descent, Lemma \ref{descent-lemma-etale-etale-local-source}.
\end{proof}

\begin{lemma}
\label{lemma-locally-quasi-finite-etale-local-source}
The property $\mathcal{P}(f)=$``$f$ is locally quasi-finite''
is \'etale local on the source.
\end{lemma}

\begin{proof}
Follows from
Lemma \ref{lemma-transfer-from-schemes}
using
Morphisms of Spaces,
Definition \ref{spaces-morphisms-definition-locally-quasi-finite}
and
Descent, Lemma \ref{descent-lemma-locally-quasi-finite-etale-local-source}.
\end{proof}

\begin{lemma}
\label{lemma-unramified-etale-local-source}
The property $\mathcal{P}(f)=$``$f$ is unramified''
is \'etale local on the source.
\end{lemma}

\begin{proof}
Follows from
Lemma \ref{lemma-transfer-from-schemes}
using
Morphisms of Spaces, Definition \ref{spaces-morphisms-definition-unramified}
and
Descent, Lemma \ref{descent-lemma-unramified-etale-local-source}.
\end{proof}




\section{Properties of morphisms smooth local on source-and-target}
\label{section-properties-local-source-target}

\noindent
Let $\mathcal{P}$ be a property of morphisms of algebraic spaces. There is an
intuitive meaning to the phrase ``$\mathcal{P}$ is smooth local on the
source and target''. However, it turns out that this notion is not
the same as asking $\mathcal{P}$ to be both smooth
local on the source and smooth local on the target.
We have discussed a similar phenomenon (for the \'etale topology and
the category of schemes) in great detail in
Descent, Section \ref{descent-section-properties-etale-local-source-target}
(for a quick overview take a look at
Descent, Remark \ref{descent-remark-compare-definitions}).
However, there is an important difference between the case of the smooth
and the \'etale topology. To see this difference we encourage the reader
to ponder the difference between
Descent, Lemma \ref{descent-lemma-local-source-target-implies}
and
Lemma \ref{lemma-local-source-target-implies}
as well as the difference between
Descent, Lemma \ref{descent-lemma-local-source-target-characterize}
and
Lemma \ref{lemma-local-source-target-characterize}.
Namely, in the \'etale setting the choice of the \'etale ``covering'' of the
target is immaterial, whereas in the smooth setting it is not.

\begin{definition}
\label{definition-local-source-target}
Let $S$ be a scheme.
Let $\mathcal{P}$ be a property of morphisms of algebraic spaces over $S$.
We say $\mathcal{P}$ is {\it smooth local on source-and-target} if
\begin{enumerate}
\item (stable under precomposing with smooth maps)
if $f : X \to Y$ is smooth and $g : Y \to Z$ has $\mathcal{P}$,
then $g \circ f$ has $\mathcal{P}$,
\item (stable under smooth base change)
if $f : X \to Y$ has $\mathcal{P}$ and $Y' \to Y$ is smooth, then
the base change $f' : Y' \times_Y X \to Y'$ has $\mathcal{P}$, and
\item (locality) given a morphism $f : X \to Y$ the following are
equivalent
\begin{enumerate}
\item $f$ has $\mathcal{P}$,
\item for every $x \in |X|$ there exists a commutative diagram
$$
\xymatrix{
U \ar[d]_a \ar[r]_h & V \ar[d]^b \\
X \ar[r]^f & Y
}
$$
with smooth vertical arrows and $u \in |U|$ with $a(u) = x$ such that
$h$ has $\mathcal{P}$.
\end{enumerate}
\end{enumerate}
\end{definition}

\noindent
The above serves as our definition. In the lemmas below we will show that
this is equivalent to $\mathcal{P}$ being smooth local on the target,
smooth local on the source, and stable under post-composing by smooth morphisms.

\begin{lemma}
\label{lemma-local-source-target-implies}
Let $S$ be a scheme.
Let $\mathcal{P}$ be a property of morphisms of algebraic spaces over $S$
which is smooth local on source-and-target. Then
\begin{enumerate}
\item $\mathcal{P}$ is smooth local on the source,
\item $\mathcal{P}$ is smooth local on the target,
\item $\mathcal{P}$ is stable under postcomposing with smooth morphisms:
if $f : X \to Y$ has $\mathcal{P}$ and $g : Y \to Z$ is smooth, then
$g \circ f$ has $\mathcal{P}$.
\end{enumerate}
\end{lemma}

\begin{proof}
We write everything out completely.

\medskip\noindent
Proof of (1). Let $f : X \to Y$ be a morphism of algebraic spaces over $S$.
Let $\{X_i \to X\}_{i \in I}$ be a smooth covering of $X$. If each composition
$h_i : X_i \to Y$ has $\mathcal{P}$, then for each $|x| \in X$ we can find
an $i \in I$ and a point $x_i \in |X_i|$ mapping to $x$. Then
$(X_i, x_i) \to (X, x)$ is a smooth morphism of pairs, and
$\text{id}_Y : Y \to Y$ is a smooth morphism, and $h_i$ is as in part (3) of
Definition \ref{definition-local-source-target}.
Thus we see that $f$ has $\mathcal{P}$.
Conversely, if $f$ has $\mathcal{P}$ then each $X_i \to Y$ has
$\mathcal{P}$ by
Definition \ref{definition-local-source-target} part (1).

\medskip\noindent
Proof of (2). Let $f : X \to Y$ be a morphism of algebraic spaces over $S$.
Let $\{Y_i \to Y\}_{i \in I}$ be a smooth covering of $Y$.
Write $X_i = Y_i \times_Y X$ and $h_i : X_i \to Y_i$ for the base change
of $f$.  If each  $h_i : X_i \to Y_i$ has $\mathcal{P}$, then for each
$x \in |X|$ we pick an $i \in I$ and a point $x_i \in |X_i|$ mapping to $x$.
Then $(X_i, x_i) \to (X, x)$ is a smooth morphism of pairs, $Y_i \to Y$ is
smooth, and $h_i$ is as in part (3) of
Definition \ref{definition-local-source-target}.
Thus we see that $f$ has $\mathcal{P}$.
Conversely, if $f$ has $\mathcal{P}$, then each $X_i \to Y_i$ has
$\mathcal{P}$ by
Definition \ref{definition-local-source-target} part (2).

\medskip\noindent
Proof of (3). Assume $f : X \to Y$ has $\mathcal{P}$ and $g : Y \to Z$ is
smooth. For every $x \in |X|$ we can think of $(X, x) \to (X, x)$ as a
smooth morphism of pairs, $Y \to Z$ is a smooth morphism, and $h = f$ is as
in part (3) of
Definition \ref{definition-local-source-target}.
Thus we see that $g \circ f$ has $\mathcal{P}$.
\end{proof}

\noindent
The following lemma is the analogue of
Morphisms, Lemma \ref{morphisms-lemma-locally-P-characterize}.

\begin{lemma}
\label{lemma-local-source-target-characterize}
Let $S$ be a scheme.
Let $\mathcal{P}$ be a property of morphisms of algebraic spaces over $S$
which is smooth local on source-and-target. Let $f : X \to Y$ be a morphism
of algebraic spaces over $S$. The following are equivalent:
\begin{enumerate}
\item[(a)] $f$ has property $\mathcal{P}$,
\item[(b)] for every $x \in |X|$ there exists a smooth morphism of pairs
$a : (U, u) \to (X, x)$, a smooth morphism $b : V \to Y$, and
a morphism $h : U \to V$ such that $f \circ a = b \circ h$ and
$h$ has $\mathcal{P}$,
\item[(c)] for some commutative diagram
$$
\xymatrix{
U \ar[d]_a \ar[r]_h & V \ar[d]^b \\
X \ar[r]^f & Y
}
$$
with $a$, $b$ smooth and $a$ surjective the morphism $h$ has $\mathcal{P}$,
\item[(d)] for any commutative diagram
$$
\xymatrix{
U \ar[d]_a \ar[r]_h & V \ar[d]^b \\
X \ar[r]^f & Y
}
$$
with $b$ smooth and $U \to X \times_Y V$ smooth
the morphism $h$ has $\mathcal{P}$,
\item[(e)] there exists a smooth covering $\{Y_i \to Y\}_{i \in I}$ such
that each base change $Y_i \times_Y X \to Y_i$ has $\mathcal{P}$,
\item[(f)] there exists a smooth covering $\{X_i \to X\}_{i \in I}$ such
that each composition $X_i \to Y$ has $\mathcal{P}$,
\item[(g)] there exists a smooth covering $\{Y_i \to Y\}_{i \in I}$ and
for each $i \in I$ a smooth covering
$\{X_{ij} \to Y_i \times_Y X\}_{j \in J_i}$ such that each morphism
$X_{ij} \to Y_i$ has $\mathcal{P}$.
\end{enumerate}
\end{lemma}

\begin{proof}
The equivalence of (a) and (b) is part of
Definition \ref{definition-local-source-target}.
The equivalence of (a) and (e) is
Lemma \ref{lemma-local-source-target-implies} part (2).
The equivalence of (a) and (f) is
Lemma \ref{lemma-local-source-target-implies} part (1).
As (a) is now equivalent to (e) and (f) it follows that
(a) equivalent to (g).

\medskip\noindent
It is clear that (c) implies (b). If (b) holds, then for any
$x \in |X|$ we can choose a smooth morphism of pairs
$a_x : (U_x, u_x) \to (X, x)$, a smooth morphism $b_x : V_x \to Y$, and
a morphism $h_x : U_x \to V_x$ such that $f \circ a_x = b_x \circ h_x$ and
$h_x$ has $\mathcal{P}$. Then $h = \coprod h_x : \coprod U_x \to \coprod V_x$
with $a = \coprod a_x$ and $b = \coprod b_x$ is a diagram as in (c).
(Note that $h$ has property $\mathcal{P}$ as $\{V_x \to \coprod V_x\}$
is a smooth covering and $\mathcal{P}$ is smooth local on the target.)
Thus (b) is equivalent to (c).

\medskip\noindent
Now we know that (a), (b), (c), (e), (f), and (g) are equivalent.
Suppose (a) holds. Let $U, V, a, b, h$ be as in (d). Then
$X \times_Y V \to V$ has $\mathcal{P}$ as $\mathcal{P}$ is stable under
smooth base change, whence $U \to V$ has $\mathcal{P}$ as $\mathcal{P}$
is stable under precomposing with smooth morphisms. Conversely, if (d)
holds, then setting $U = X$ and $V = Y$ we see that $f$ has $\mathcal{P}$.
\end{proof}

\begin{lemma}
\label{lemma-smooth-local-source-target}
Let $S$ be a scheme.
Let $\mathcal{P}$ be a property of morphisms of algebraic spaces over $S$.
Assume
\begin{enumerate}
\item $\mathcal{P}$ is smooth local on the source,
\item $\mathcal{P}$ is smooth local on the target, and
\item $\mathcal{P}$ is stable under postcomposing with smooth morphisms:
if $f : X \to Y$ has $\mathcal{P}$ and $Y \to Z$ is a smooth morphism
then $X \to Z$ has $\mathcal{P}$.
\end{enumerate}
Then $\mathcal{P}$ is smooth local on the source-and-target.
\end{lemma}

\begin{proof}
Let $\mathcal{P}$ be a property of morphisms of algebraic spaces which
satisfies conditions (1), (2) and (3) of the lemma. By
Lemma \ref{lemma-precompose-property-local-source}
we see that $\mathcal{P}$ is stable under precomposing with
smooth morphisms. By
Lemma \ref{lemma-pullback-property-local-target}
we see that $\mathcal{P}$ is stable under smooth base change.
Hence it suffices to prove part (3) of
Definition \ref{definition-local-source-target}
holds.

\medskip\noindent
More precisely, suppose that $f : X \to Y$ is a morphism
of algebraic spaces over $S$ which satisfies
Definition \ref{definition-local-source-target} part (3)(b).
In other words, for every $x \in X$ there exists a smooth
morphism $a_x : U_x \to X$, a point $u_x \in |U_x|$ mapping to $x$,
a smooth morphism $b_x : V_x \to Y$, and a morphism $h_x : U_x \to V_x$
such that $f \circ a_x = b_x \circ h_x$ and $h_x$ has $\mathcal{P}$.
The proof of the lemma is complete once we show that $f$ has $\mathcal{P}$.
Set $U = \coprod U_x$, $a = \coprod a_x$, $V = \coprod V_x$,
$b = \coprod b_x$, and $h = \coprod h_x$. We obtain a
commutative diagram
$$
\xymatrix{
U \ar[d]_a \ar[r]_h & V \ar[d]^b \\
X \ar[r]^f & Y
}
$$
with $a$, $b$ smooth, $a$ surjective. Note that $h$ has $\mathcal{P}$
as each $h_x$ does and $\mathcal{P}$ is smooth local on the target.
Because $a$ is surjective and $\mathcal{P}$ is smooth local on the source,
it suffices to prove that $b \circ h$ has $\mathcal{P}$.
This follows as we assumed that $\mathcal{P}$ is stable under
postcomposing with a smooth morphism and as $b$ is smooth.
\end{proof}

\begin{remark}
\label{remark-list-local-source-target}
Using
Lemma \ref{lemma-smooth-local-source-target}
and the work done in the earlier sections of this chapter it is easy
to make a list of types of morphisms which are smooth local on the
source-and-target. In each case we list the lemma which implies
the property is smooth local on the source and the lemma which implies
the property is smooth local on the target. In each case the third assumption
of
Lemma \ref{lemma-smooth-local-source-target}
is trivial to check, and we omit it. Here is the list:
\begin{enumerate}
\item flat, see
Lemmas \ref{lemma-flat-fpqc-local-source} and
\ref{lemma-descending-property-flat},
\item locally of finite presentation, see
Lemmas \ref{lemma-locally-finite-presentation-fppf-local-source} and
\ref{lemma-descending-property-locally-finite-presentation},
\item locally finite type, see
Lemmas \ref{lemma-locally-finite-type-fppf-local-source} and
\ref{lemma-descending-property-locally-finite-type},
\item universally open, see
Lemmas \ref{lemma-universally-open-fppf-local-source} and
\ref{lemma-descending-property-universally-open},
\item syntomic, see
Lemmas \ref{lemma-syntomic-syntomic-local-source} and
\ref{lemma-descending-property-syntomic},
\item smooth, see
Lemmas \ref{lemma-smooth-smooth-local-source} and
\ref{lemma-descending-property-smooth},
\item add more here as needed.
\end{enumerate}
\end{remark}







\section{Properties of morphisms \'etale-smooth local on source-and-target}
\label{section-properties-etale-smooth-local-source-target}

\noindent
This section is the analogue of
Section \ref{section-properties-local-source-target}
for properties of morphisms which are \'etale local
on the source and smooth local on the target.
We give this property a ridiculously long name
in order to avoid using it too much.

\begin{definition}
\label{definition-etale-smooth-local-source-target}
Let $S$ be a scheme.
Let $\mathcal{P}$ be a property of morphisms of algebraic spaces over $S$.
We say $\mathcal{P}$ is {\it \'etale-smooth local on source-and-target} if
\begin{enumerate}
\item (stable under precomposing with \'etale maps)
if $f : X \to Y$ is \'etale and $g : Y \to Z$ has $\mathcal{P}$,
then $g \circ f$ has $\mathcal{P}$,
\item (stable under smooth base change)
if $f : X \to Y$ has $\mathcal{P}$ and $Y' \to Y$ is smooth, then
the base change $f' : Y' \times_Y X \to Y'$ has $\mathcal{P}$, and
\item (locality) given a morphism $f : X \to Y$ the following are
equivalent
\begin{enumerate}
\item $f$ has $\mathcal{P}$,
\item for every $x \in |X|$ there exists a commutative diagram
$$
\xymatrix{
U \ar[d]_a \ar[r]_h & V \ar[d]^b \\
X \ar[r]^f & Y
}
$$
with $b$ smooth and $U \to X \times_Y V$ \'etale
and $u \in |U|$ with $a(u) = x$ such that
$h$ has $\mathcal{P}$.
\end{enumerate}
\end{enumerate}
\end{definition}

\noindent
The above serves as our definition. In the lemmas below we will show that
this is equivalent to $\mathcal{P}$ being \'etale local on the target,
smooth local on the source, and stable under post-composing by
\'etale morphisms.

\begin{lemma}
\label{lemma-etale-smooth-local-source-target-implies}
Let $S$ be a scheme.
Let $\mathcal{P}$ be a property of morphisms of algebraic spaces over $S$
which is \'etale-smooth local on source-and-target. Then
\begin{enumerate}
\item $\mathcal{P}$ is \'etale local on the source,
\item $\mathcal{P}$ is smooth local on the target,
\item $\mathcal{P}$ is stable under postcomposing with \'etale morphisms:
if $f : X \to Y$ has $\mathcal{P}$ and $g : Y \to Z$ is \'etale, then
$g \circ f$ has $\mathcal{P}$, and
\item $\mathcal{P}$ has a permanence property: given $f : X \to Y$ and
$g : Y \to Z$ \'etale such that $g \circ f$ has $\mathcal{P}$, then
$f$ has $\mathcal{P}$.
\end{enumerate}
\end{lemma}

\begin{proof}
We write everything out completely.

\medskip\noindent
Proof of (1). Let $f : X \to Y$ be a morphism of algebraic spaces over $S$.
Let $\{X_i \to X\}_{i \in I}$ be an \'etale covering of $X$. If each composition
$h_i : X_i \to Y$ has $\mathcal{P}$, then for each $|x| \in X$ we can find
an $i \in I$ and a point $x_i \in |X_i|$ mapping to $x$. Then
$(X_i, x_i) \to (X, x)$ is an \'etale morphism of pairs, and
$\text{id}_Y : Y \to Y$ is a smooth morphism, and $h_i$ is as in part (3) of
Definition \ref{definition-etale-smooth-local-source-target}.
Thus we see that $f$ has $\mathcal{P}$.
Conversely, if $f$ has $\mathcal{P}$ then each $X_i \to Y$ has
$\mathcal{P}$ by
Definition \ref{definition-etale-smooth-local-source-target} part (1).

\medskip\noindent
Proof of (2). Let $f : X \to Y$ be a morphism of algebraic spaces over $S$.
Let $\{Y_i \to Y\}_{i \in I}$ be a smooth covering of $Y$.
Write $X_i = Y_i \times_Y X$ and $h_i : X_i \to Y_i$ for the base change
of $f$. If each  $h_i : X_i \to Y_i$ has $\mathcal{P}$, then for each
$x \in |X|$ we pick an $i \in I$ and a point $x_i \in |X_i|$ mapping to $x$.
Then $X_i \to X \times_Y Y_i$ is an \'etale morphism
(because it is an isomorphism), $Y_i \to Y$ is
smooth, and $h_i$ is as in part (3) of
Definition \ref{definition-local-source-target}.
Thus we see that $f$ has $\mathcal{P}$.
Conversely, if $f$ has $\mathcal{P}$, then each $X_i \to Y_i$ has
$\mathcal{P}$ by
Definition \ref{definition-local-source-target} part (2).

\medskip\noindent
Proof of (3). Assume $f : X \to Y$ has $\mathcal{P}$ and $g : Y \to Z$ is
\'etale. The morphism $X \to Y \times_Z X$ is \'etale as a morphism
between algebraic spaces \'etale over $X$ (
Properties of Spaces, Lemma \ref{spaces-properties-lemma-etale-permanence}).
Also $Y \to Z$ is \'etale hence a smooth morphism.
Thus the diagram
$$
\xymatrix{
X \ar[d] \ar[r]_f & Y \ar[d] \\
X \ar[r]^{g \circ f} & Z
}
$$
works for every $x \in |X|$ in part (3) of
Definition \ref{definition-local-source-target}
and we conclude that $g \circ f$ has $\mathcal{P}$.

\medskip\noindent
Proof of (4). Let $f : X \to Y$ be a morphism and $g : Y \to Z$ \'etale
such that $g \circ f$ has $\mathcal{P}$. Then by
Definition \ref{definition-etale-smooth-local-source-target} part (2)
we see that $\text{pr}_Y : Y \times_Z X \to Y$ has $\mathcal{P}$. But
the morphism $(f, 1) : X \to Y \times_Z X$ is \'etale as a section to the
\'etale projection $\text{pr}_X : Y \times_Z X \to X$, see
Morphisms of Spaces, Lemma \ref{spaces-morphisms-lemma-etale-permanence}.
Hence $f = \text{pr}_Y \circ (f, 1)$ has $\mathcal{P}$ by
Definition \ref{definition-etale-smooth-local-source-target} part (1).
\end{proof}

\begin{lemma}
\label{lemma-etale-smooth-local-source-target-characterize}
Let $S$ be a scheme.
Let $\mathcal{P}$ be a property of morphisms of algebraic spaces over $S$
which is etale-smooth local on source-and-target.
Let $f : X \to Y$ be a morphism
of algebraic spaces over $S$. The following are equivalent:
\begin{enumerate}
\item[(a)] $f$ has property $\mathcal{P}$,
\item[(b)] for every $x \in |X|$ there exists a smooth morphism $b : V \to Y$,
an \'etale morphism $a : U \to V \times_Y X$, and a point $u \in |U|$
mapping to $x$ such that $U \to V$ has $\mathcal{P}$,
\item[(c)] for some commutative diagram
$$
\xymatrix{
U \ar[d]_a \ar[r]_h & V \ar[d]^b \\
X \ar[r]^f & Y
}
$$
with $b$ smooth, $U \to V \times_Y X$ \'etale, and $a$ surjective
the morphism $h$ has $\mathcal{P}$,
\item[(d)] for any commutative diagram
$$
\xymatrix{
U \ar[d]_a \ar[r]_h & V \ar[d]^b \\
X \ar[r]^f & Y
}
$$
with $b$ smooth and $U \to X \times_Y V$ \'etale, the morphism $h$
has $\mathcal{P}$,
\item[(e)] there exists a smooth covering $\{Y_i \to Y\}_{i \in I}$ such
that each base change $Y_i \times_Y X \to Y_i$ has $\mathcal{P}$,
\item[(f)] there exists an \'etale covering $\{X_i \to X\}_{i \in I}$ such
that each composition $X_i \to Y$ has $\mathcal{P}$,
\item[(g)] there exists a smooth covering $\{Y_i \to Y\}_{i \in I}$ and
for each $i \in I$ an \'etale covering
$\{X_{ij} \to Y_i \times_Y X\}_{j \in J_i}$ such that each morphism
$X_{ij} \to Y_i$ has $\mathcal{P}$.
\end{enumerate}
\end{lemma}

\begin{proof}
The equivalence of (a) and (b) is part of
Definition \ref{definition-etale-smooth-local-source-target}.
The equivalence of (a) and (e) is
Lemma \ref{lemma-etale-smooth-local-source-target-implies} part (2).
The equivalence of (a) and (f) is
Lemma \ref{lemma-etale-smooth-local-source-target-implies} part (1).
As (a) is now equivalent to (e) and (f) it follows that
(a) equivalent to (g).

\medskip\noindent
It is clear that (c) implies (b). If (b) holds, then for any
$x \in |X|$ we can choose a smooth morphism a smooth morphism
$b_x : V_x \to Y$, an \'etale morphism $U_x \to V_x \times_Y X$,
and $u_x \in |U_x|$ mapping to $x$
such that $U_x \to V_x$ has $\mathcal{P}$.
Then $h = \coprod h_x : \coprod U_x \to \coprod V_x$
with $a = \coprod a_x$ and $b = \coprod b_x$ is a diagram as in (c).
(Note that $h$ has property $\mathcal{P}$ as $\{V_x \to \coprod V_x\}$
is a smooth covering and $\mathcal{P}$ is smooth local on the target.)
Thus (b) is equivalent to (c).

\medskip\noindent
Now we know that (a), (b), (c), (e), (f), and (g) are equivalent.
Suppose (a) holds. Let $U, V, a, b, h$ be as in (d). Then
$X \times_Y V \to V$ has $\mathcal{P}$ as $\mathcal{P}$ is stable under
smooth base change, whence $U \to V$ has $\mathcal{P}$ as $\mathcal{P}$
is stable under precomposing with \'etale morphisms. Conversely, if (d)
holds, then setting $U = X$ and $V = Y$ we see that $f$ has $\mathcal{P}$.
\end{proof}

\begin{lemma}
\label{lemma-etale-smooth-local-source-target}
Let $S$ be a scheme.
Let $\mathcal{P}$ be a property of morphisms of algebraic spaces over $S$.
Assume
\begin{enumerate}
\item $\mathcal{P}$ is \'etale local on the source,
\item $\mathcal{P}$ is smooth local on the target, and
\item $\mathcal{P}$ is stable under postcomposing with open immersions:
if $f : X \to Y$ has $\mathcal{P}$ and $Y \subset Z$ is an open embedding
then $X \to Z$ has $\mathcal{P}$.
\end{enumerate}
Then $\mathcal{P}$ is \'etale-smooth local on the source-and-target.
\end{lemma}

\begin{proof}
Let $\mathcal{P}$ be a property of morphisms of algebraic spaces which
satisfies conditions (1), (2) and (3) of the lemma. By
Lemma \ref{lemma-precompose-property-local-source}
we see that $\mathcal{P}$ is stable under precomposing with
\'etale morphisms. By
Lemma \ref{lemma-pullback-property-local-target}
we see that $\mathcal{P}$ is stable under smooth base change.
Hence it suffices to prove part (3) of
Definition \ref{definition-local-source-target}
holds.

\medskip\noindent
More precisely, suppose that $f : X \to Y$ is a morphism
of algebraic spaces over $S$ which satisfies
Definition \ref{definition-local-source-target} part (3)(b).
In other words, for every $x \in X$ there exists
a smooth morphism $b_x : V_x \to Y$,
an \'etale morphism $U_x \to V_x \times_Y X$, and
a point $u_x \in |U_x|$ mapping to $x$
such that $h_x : U_x \to V_x$ has $\mathcal{P}$.
The proof of the lemma is complete once we show that $f$ has $\mathcal{P}$.

\medskip\noindent
Let $a_x : U_x \to X$ be the composition $U_x \to V_x \times_Y X \to X$.
Set $U = \coprod U_x$, $a = \coprod a_x$, $V = \coprod V_x$,
$b = \coprod b_x$, and $h = \coprod h_x$. We obtain a
commutative diagram
$$
\xymatrix{
U \ar[d]_a \ar[r]_h & V \ar[d]^b \\
X \ar[r]^f & Y
}
$$
with $b$ smooth, $U \to V \times_Y X$ \'etale, $a$ surjective.
Note that $h$ has $\mathcal{P}$ as each $h_x$ does and $\mathcal{P}$
is smooth local on the target. In the next paragraph we prove
that we may assume $U, V, X, Y$ are schemes; we encourage the reader
to skip it.

\medskip\noindent
Let $X, Y, U, V, a, b, f, h$ be as in the previous paragraph. We have
to show $f$ has $\mathcal{P}$. Let $X' \to X$ be a surjective \'etale
morphism with $X_i$ a scheme. Set $U' = X' \times_X U$. Then
$U' \to X'$ is surjective and $U' \to X' \times_Y V$ is \'etale.
Since $\mathcal{P}$ is \'etale local on the source, we see that
$U' \to V$ has $\mathcal{P}$ and that it suffices to show that
$X' \to Y$ has $\mathcal{P}$. In other words, we may assume
that $X$ is a scheme. Next, choose a surjective \'etale morphism
$Y' \to Y$ with $Y'$ a scheme. Set $V' = V \times_Y Y'$,
$X' = X \times_Y Y'$, and $U' = U \times_Y Y'$. Then
$U' \to X'$ is surjective and $U' \to X' \times_{Y'} V'$ is \'etale.
Since $\mathcal{P}$ is smooth local on the target, we see that $U' \to V'$ has
$\mathcal{P}$ and that it suffices to prove $X' \to Y'$ has $\mathcal{P}$.
Thus we may assume both $X$ and $Y$ are schemes.
Choose a surjective \'etale morphism $V' \to V$
with $V'$ a scheme. Set $U' = U \times_V V'$.
Then $U' \to X$ is surjective and $U' \to X \times_Y V'$ is \'etale.
Since $\mathcal{P}$ is smooth local on the source, we see that
$U' \to V'$ has $\mathcal{P}$. Thus we may replace $U, V$ by
$U', V'$ and assume $X, Y, V$ are schemes.
Finally, we replace $U$ by a scheme surjective \'etale over $U$
and we see that we may assume $U, V, X, Y$ are all schemes.

\medskip\noindent
If $U, V, X, Y$ are schemes, then $f$ has $\mathcal{P}$
by Descent, Lemma \ref{descent-lemma-etale-tau-local-source-target}.
\end{proof}

\begin{remark}
\label{remark-list-etale-smooth-local-source-target}
Using Lemma \ref{lemma-etale-smooth-local-source-target}
and the work done in the earlier sections of this chapter it is easy
to make a list of types of morphisms which are smooth local on the
source-and-target. In each case we list the lemma which implies
the property is etale local on the source and the lemma which implies
the property is smooth local on the target. In each case the third assumption
of Lemma \ref{lemma-etale-smooth-local-source-target}
is trivial to check, and we omit it. Here is the list:
\begin{enumerate}
\item \'etale, see
Lemmas \ref{lemma-etale-etale-local-source} and
\ref{lemma-descending-property-etale},
\item locally quasi-finite, see
Lemmas \ref{lemma-locally-quasi-finite-etale-local-source} and
\ref{lemma-descending-property-quasi-finite},
\item unramified, see
Lemmas \ref{lemma-unramified-etale-local-source} and
\ref{lemma-descending-property-unramified}, and
\item add more here as needed.
\end{enumerate}
Of course any property listed in
Remark \ref{remark-list-local-source-target}
is a fortiori an example that could be listed here.
\end{remark}







\section{Descent data for spaces over spaces}
\label{section-descent-datum}

\noindent
This section is the analogue of Descent, Section
\ref{descent-section-descent-datum} for algebraic spaces.
Most of the arguments in this section are formal relying only
on the definition of a descent datum.

\begin{definition}
\label{definition-descent-datum}
Let $S$ be a scheme. Let $f : Y \to X$ be a morphism of algebraic spaces
over $S$.
\begin{enumerate}
\item Let $V \to Y$ be a morphism of algebraic spaces.
A {\it descent datum for $V/Y/X$} is an isomorphism
$\varphi : V \times_X Y \to Y \times_X V$ of algebraic spaces over
$Y \times_X Y$ satisfying the {\it cocycle condition} that the diagram
$$
\xymatrix{
V \times_X Y \times_X Y \ar[rd]^{\varphi_{01}} \ar[rr]_{\varphi_{02}} &
&
Y \times_X Y \times_X V\\
&
Y \times_X Y \times_X Y \ar[ru]^{\varphi_{12}}
}
$$
commutes (with obvious notation).
\item We also say that the pair $(V/Y, \varphi)$ is
a {\it descent datum relative to $Y \to X$}.
\item A {\it morphism $f : (V/Y, \varphi) \to (V'/Y, \varphi')$ of
descent data relative to $Y \to X$} is a morphism
$f : V \to V'$ of algebraic spaces over $Y$ such that
the diagram
$$
\xymatrix{
V \times_X Y \ar[r]_{\varphi} \ar[d]_{f \times \text{id}_Y} &
Y \times_X V \ar[d]^{\text{id}_Y \times f} \\
V' \times_X Y \ar[r]^{\varphi'} & Y \times_X V'
}
$$
commutes.
\end{enumerate}
\end{definition}

\begin{remark}
\label{remark-easier}
Let $S$ be a scheme.
Let $Y \to X$ be a morphism of algebraic spaces over $S$.
Let $(V/Y, \varphi)$ be a descent datum relative to $Y \to X$.
We may think of the isomorphism $\varphi$ as an isomorphism
$$
(Y \times_X Y) \times_{\text{pr}_0, Y} V
\longrightarrow
(Y \times_X Y) \times_{\text{pr}_1, Y} V
$$
of algebraic spaces over $Y \times_X Y$. So loosely speaking one may
think of $\varphi$ as a map
$\varphi : \text{pr}_0^*V \to \text{pr}_1^*V$\footnote{Unfortunately,
we have chosen the ``wrong'' direction for our arrow here. In
Definitions \ref{definition-descent-datum} and
\ref{definition-descent-datum-for-family-of-morphisms}
we should have the opposite direction to what was done in
Definition \ref{definition-descent-datum-quasi-coherent}
by the general principle that ``functions'' and ``spaces'' are dual.}.
The cocycle condition then says that
$\text{pr}_{02}^*\varphi =
\text{pr}_{12}^*\varphi \circ \text{pr}_{01}^*\varphi$.
In this way it is very similar to the case of a descent datum on
quasi-coherent sheaves.
\end{remark}

\noindent
Here is the definition in case you have a family of morphisms
with fixed target.

\begin{definition}
\label{definition-descent-datum-for-family-of-morphisms}
Let $S$ be a scheme.
Let $\{X_i \to X\}_{i \in I}$ be a family of morphisms
of algebraic spaces over $S$ with fixed target $X$.
\begin{enumerate}
\item A {\it descent datum $(V_i, \varphi_{ij})$ relative to the
family $\{X_i \to X\}$} is given by an algebraic space $V_i$ over $X_i$
for each $i \in I$, an isomorphism
$\varphi_{ij} : V_i \times_X X_j \to X_i \times_X V_j$
of algebraic spaces over $X_i \times_X X_j$ for each pair $(i, j) \in I^2$
such that for every triple of indices $(i, j, k) \in I^3$
the diagram
$$
\xymatrix{
V_i \times_X X_j \times_X X_k
\ar[rd]^{\text{pr}_{01}^*\varphi_{ij}}
\ar[rr]_{\text{pr}_{02}^*\varphi_{ik}} &
&
X_i \times_X X_j \times_X V_k\\
&
X_i \times_X V_j \times_X X_k
\ar[ru]^{\text{pr}_{12}^*\varphi_{jk}}
}
$$
of algebraic spaces over $X_i \times_X X_j \times_X X_k$ commutes
(with obvious notation).
\item A {\it morphism
$\psi : (V_i, \varphi_{ij}) \to (V'_i, \varphi'_{ij})$
of descent data} is given by a family $\psi = (\psi_i)_{i \in I}$
of morphisms $\psi_i : V_i \to V'_i$ of algebraic spaces over $X_i$
such that all the diagrams
$$
\xymatrix{
V_i \times_X X_j \ar[r]_{\varphi_{ij}} \ar[d]_{\psi_i \times \text{id}} &
X_i \times_X V_j \ar[d]^{\text{id} \times \psi_j} \\
V'_i \times_X X_j \ar[r]^{\varphi'_{ij}} & X_i \times_X V'_j
}
$$
commute.
\end{enumerate}
\end{definition}

\begin{remark}
\label{remark-easier-family}
Let $S$ be a scheme.
Let $\{X_i \to X\}_{i \in I}$ be a family of morphisms
of algebraic spaces over $S$ with fixed target $X$.
Let $(V_i, \varphi_{ij})$ be a descent datum relative to
$\{X_i \to X\}$. We may think of the isomorphisms $\varphi_{ij}$
as isomorphisms
$$
(X_i \times_X X_j) \times_{\text{pr}_0, X_i} V_i
\longrightarrow
(X_i \times_X X_j) \times_{\text{pr}_1, X_j} V_j
$$
of algebraic spaces over $X_i \times_X X_j$. So loosely speaking one may
think of $\varphi_{ij}$ as an isomorphism
$\text{pr}_0^*V_i \to \text{pr}_1^*V_j$ over $X_i \times_X X_j$.
The cocycle condition then says that
$\text{pr}_{02}^*\varphi_{ik} =
\text{pr}_{12}^*\varphi_{jk} \circ \text{pr}_{01}^*\varphi_{ij}$.
In this way it is very similar to the case of a descent datum on
quasi-coherent sheaves.
\end{remark}

\noindent
The reason we will usually work with the version of a family consisting
of a single morphism is the following lemma.

\begin{lemma}
\label{lemma-family-is-one}
Let $S$ be a scheme.
Let $\{X_i \to X\}_{i \in I}$ be a family of morphisms
of algebraic spaces over $S$ with fixed target $X$.
Set $Y = \coprod_{i \in I} X_i$.
There is a canonical equivalence of categories
$$
\begin{matrix}
\text{category of descent data } \\
\text{relative to the family } \{X_i \to X\}_{i \in I}
\end{matrix}
\longrightarrow
\begin{matrix}
\text{ category of descent data} \\
\text{ relative to } Y/X
\end{matrix}
$$
which maps $(V_i, \varphi_{ij})$ to $(V, \varphi)$ with
$V = \coprod_{i\in I} V_i$ and $\varphi = \coprod \varphi_{ij}$.
\end{lemma}

\begin{proof}
Observe that $Y \times_X Y = \coprod_{ij} X_i \times_X X_j$
and similarly for higher fibre products.
Giving a morphism $V \to Y$ is exactly the same as
giving a family $V_i \to X_i$. And giving a descent datum
$\varphi$ is exactly the same as giving a family $\varphi_{ij}$.
\end{proof}

\begin{lemma}
\label{lemma-pullback}
Pullback of descent data. Let $S$ be a scheme.
\begin{enumerate}
\item Let
$$
\xymatrix{
Y' \ar[r]_f \ar[d]_{a'} & Y \ar[d]^a \\
X' \ar[r]^h & X
}
$$
be a commutative diagram of algebraic spaces over $S$.
The construction
$$
(V \to Y, \varphi) \longmapsto f^*(V \to Y, \varphi) = (V' \to Y', \varphi')
$$
where $V' = Y' \times_Y V$ and where
$\varphi'$ is defined as the composition
$$
\xymatrix{
V' \times_{X'} Y' \ar@{=}[r] &
(Y' \times_Y V) \times_{X'} Y' \ar@{=}[r] &
(Y' \times_{X'} Y') \times_{Y \times_X Y} (V \times_X Y)
\ar[d]^{\text{id} \times \varphi} \\
Y' \times_{X'} V' \ar@{=}[r] &
Y' \times_{X'} (Y' \times_Y V) &
(Y' \times_X Y') \times_{Y \times_X Y} (Y \times_X V) \ar@{=}[l]
}
$$
defines a functor from the category of descent data
relative to $Y \to X$ to the category of descent data
relative to $Y' \to X'$.
\item Given two morphisms $f_i : Y' \to Y$, $i = 0, 1$ making the
diagram commute the functors $f_0^*$ and $f_1^*$ are
canonically isomorphic.
\end{enumerate}
\end{lemma}

\begin{proof}
We omit the proof of (1), but we remark that the morphism
$\varphi'$ is the morphism $(f \times f)^*\varphi$ in the
notation introduced in Remark \ref{remark-easier}.
For (2) we indicate which morphism
$f_0^*V \to f_1^*V$ gives the functorial isomorphism. Namely,
since $f_0$ and $f_1$ both fit into the commutative diagram
we see there is a unique morphism $r : Y' \to Y \times_X Y$
with $f_i = \text{pr}_i \circ r$. Then we take
\begin{eqnarray*}
f_0^*V & = &
Y' \times_{f_0, Y} V \\
& = &
Y' \times_{\text{pr}_0 \circ r, Y} V \\
& = &
Y' \times_{r, Y \times_X Y} (Y \times_X Y) \times_{\text{pr}_0, Y} V \\
& \xrightarrow{\varphi} &
Y' \times_{r, Y \times_X Y} (Y \times_X Y) \times_{\text{pr}_1, Y} V \\
& = &
Y' \times_{\text{pr}_1 \circ r, Y} V \\
& = &
Y' \times_{f_1, Y} V \\
& = & f_1^*V
\end{eqnarray*}
We omit the verification that this works.
\end{proof}

\begin{definition}
\label{definition-pullback-functor}
With $S, X, X', Y, Y', f, a, a', h$ as in Lemma \ref{lemma-pullback}
the functor
$$
(V, \varphi) \longmapsto f^*(V, \varphi)
$$
constructed in that lemma is called the {\it pullback functor} on descent data.
\end{definition}

\begin{lemma}
\label{lemma-pullback-family}
Let $S$ be a scheme. Let $\mathcal{U}' = \{X'_i \to X'\}_{i \in I'}$ and
$\mathcal{U} = \{X_j \to X\}_{i \in I}$ be families of morphisms with
fixed target. Let $\alpha : I' \to I$, $g : X' \to X$ and
$g_i : X'_i \to X_{\alpha(i)}$ be a morphism of families
of maps with fixed target, see
Sites, Definition \ref{sites-definition-morphism-coverings}.
\begin{enumerate}
\item Let $(V_i, \varphi_{ij})$ be a descent datum relative to the
family $\mathcal{U}$. The system
$$
\left(
g_i^*V_{\alpha(i)}, (g_i \times g_j)^*\varphi_{\alpha(i) \alpha(j)}
\right)
$$
(with notation as in Remark \ref{remark-easier-family})
is a descent datum relative to $\mathcal{U}'$.
\item This construction defines a functor between the category of
descent data relative to $\mathcal{U}$ and the category of
descent data relative to $\mathcal{U}'$.
\item Given a second $\beta : I' \to I$, $h : X' \to X$ and
$h'_i : X'_i \to X_{\beta(i)}$ morphism of families
of maps with fixed target, then if $g = h$ the two resulting functors
between descent data are canonically isomorphic.
\item These functors agree, via Lemma \ref{lemma-family-is-one},
with the pullback functors constructed in Lemma \ref{lemma-pullback}.
\end{enumerate}
\end{lemma}

\begin{proof}
This follows from Lemma \ref{lemma-pullback} via the
correspondence of Lemma \ref{lemma-family-is-one}.
\end{proof}

\begin{definition}
\label{definition-pullback-functor-family}
With $\mathcal{U}' = \{X'_i \to X'\}_{i \in I'}$,
$\mathcal{U} = \{X_i \to X\}_{i \in I}$, $\alpha : I' \to I$,
$g : X' \to X$, and $g_i : X'_i \to X_{\alpha(i)}$ as in
Lemma \ref{lemma-pullback-family} the functor
$$
(V_i, \varphi_{ij}) \longmapsto
(g_i^*V_{\alpha(i)}, (g_i \times g_j)^*\varphi_{\alpha(i) \alpha(j)})
$$
constructed in that lemma
is called the {\it pullback functor} on descent data.
\end{definition}

\noindent
If $\mathcal{U}$ and $\mathcal{U}'$ have the same target $X$,
and if $\mathcal{U}'$ refines $\mathcal{U}$ (see
Sites, Definition \ref{sites-definition-morphism-coverings})
but no explicit pair $(\alpha, g_i)$ is given, then we can still
talk about the pullback functor since we have seen in
Lemma \ref{lemma-pullback-family} that the choice of the pair does not matter
(up to a canonical isomorphism).

\begin{definition}
\label{definition-effective}
Let $S$ be a scheme. Let $f : Y \to X$ be a morphism of algebraic spaces over
$S$.
\begin{enumerate}
\item Given an algebraic space $U$ over $X$ we have the
{\it trivial descent datum} of $U$ relative to $\text{id} : X \to X$, namely
the identity morphism on $U$.
\item By Lemma \ref{lemma-pullback} we get a
{\it canonical descent datum} on $Y \times_X U$
relative to $Y \to X$ by pulling back the trivial
descent datum via $f$. We often
denote $(Y \times_X U, can)$ this descent datum.
\item A descent datum $(V, \varphi)$ relative to $Y/X$
is called {\it effective} if $(V, \varphi)$
is isomorphic to the canonical descent datum
$(Y \times_X U, can)$ for some algebraic space $U$ over $X$.
\end{enumerate}
\end{definition}

\noindent
Thus being effective means there exists an algebraic space $U$
over $X$ and an isomorphism $\psi : V \to Y \times_X U$
over $Y$ such that $\varphi$ is equal to the composition
$$
V \times_X Y \xrightarrow{\psi \times \text{id}_Y}
Y \times_X U \times_S Y =
Y \times_X Y \times_X U
\xrightarrow{\text{id}_Y \times \psi^{-1}}
Y \times_X V
$$
There is a slight problem here which is that this definition
(in spirit) conflicts with the definition given in
Descent, Definition \ref{descent-definition-effective}
in case $Y$ and $X$ are schemes. However, it will always be clear from
context which version we mean.

\begin{definition}
\label{definition-effective-family}
Let $S$ be a scheme.
Let $\{X_i \to X\}$ be a family of morphisms of algebraic spaces over $S$
with fixed target $X$.
\begin{enumerate}
\item  Given an algebraic space $U$ over $X$
we have a {\it canonical descent datum} on the family of
algebraic spaces $X_i \times_X U$ by pulling back the trivial
descent datum for $U$ relative to $\{\text{id} : S \to S\}$.
We denote this descent datum $(X_i \times_X U, can)$.
\item A descent datum $(V_i, \varphi_{ij})$
relative to $\{X_i \to S\}$ is called {\it effective}
if there exists an algebraic space $U$ over $X$ such that
$(V_i, \varphi_{ij})$ is isomorphic to $(X_i \times_X U, can)$.
\end{enumerate}
\end{definition}






\section{Descent data in terms of sheaves}
\label{section-descent-data-sheaves}

\noindent
This section is the analogue of Descent, Section
\ref{descent-section-descent-data-sheaves}.
It is slightly different as algebraic spaces are already sheaves.

\begin{lemma}
\label{lemma-descent-data-sheaves}
Let $S$ be a scheme. Let $\{X_i \to X\}_{i \in I}$ be an fppf
covering of algebraic spaces over $S$ (Topologies on Spaces,
Definition \ref{spaces-topologies-definition-fppf-covering}).
There is an equivalence of categories
$$
\left\{
\begin{matrix}
\text{descent data }(V_i, \varphi_{ij})\\
\text{relative to }\{X_i \to X\}
\end{matrix}
\right\}
\leftrightarrow
\left\{
\begin{matrix}
\text{sheaves }F\text{ on }(\Sch/S)_{fppf}\text{ endowed}\\
\text{with a map }F \to X\text{ such that each}\\
X_i \times_X F\text{ is an algebraic space}
\end{matrix}
\right\}.
$$
Moreover,
\begin{enumerate}
\item the algebraic space $X_i \times_X F$ on the right hand side
corresponds to $V_i$ on the left hand side, and
\item the sheaf $F$ is an algebraic space\footnote{We will see
later that this is always the case if $I$ is not too large, see
Bootstrap, Lemma \ref{bootstrap-lemma-descend-algebraic-space}.}
if and only if the
corresponding descent datum $(X_i, \varphi_{ij})$ is effective.
\end{enumerate}
\end{lemma}

\begin{proof}
Let us construct the functor from right to left.
Let $F \to X$ be a map of sheaves on $(\Sch/S)_{fppf}$ such that each
$V_i = X_i \times_X F$ is an algebraic space. We have the
projection $V_i \to X_i$.
Then both $V_i \times_X X_j$ and $X_i \times_X V_j$
represent the sheaf $X_i \times_X F \times_X X_j$
and hence we obtain an isomorphism
$$
\varphi_{ii'} : V_i \times_X X_j \to X_i \times_X V_j
$$
It is straightforward to see that the maps $\varphi_{ij}$
are morphisms over $X_i \times_X X_j$ and satisfy the
cocycle condition. The functor from right to left is given
by this construction $F \mapsto (V_i, \varphi_{ij})$.

\medskip\noindent
Let us construct a functor from left to right.
The isomorphisms $\varphi_{ij}$ give isomorphisms
$$
\varphi_{ij} : V_i \times_X X_j \longrightarrow X_i \times_X V_j
$$
over $X_i \times X_j$. Set $F$ equal to the coequalizer in the
following diagram
$$
\xymatrix{
\coprod_{i, i'} V_i \times_X X_j
\ar@<1ex>[rr]^-{\text{pr}_0}
\ar@<-1ex>[rr]_-{\text{pr}_1 \circ \varphi_{ij}}
& &
\coprod_i V_i \ar[r]
&
F
}
$$
The cocycle condition guarantees that $F$ comes with a map
$F \to X$ and that $X_i \times_X F$ is isomorphic to $V_i$.
The functor from left to right is given
by this construction $(V_i, \varphi_{ij}) \mapsto F$.

\medskip\noindent
We omit the verification that these constructions
are mutually quasi-inverse functors. The final statements
(1) and (2) follow from the constructions.
\end{proof}




\input{chapters}

\bibliography{my}
\bibliographystyle{amsalpha}


\end{document}
