\input{preamble}

% OK, start here.
%
\begin{document}

\title{Restricted Power Series}


\maketitle

\phantomsection
\label{section-phantom}

\tableofcontents

\section{Introduction}
\label{section-introduction}

\noindent
The title of this chapter is a bit misleading because the most
basic material on restricted power series is in the chapter on
formal algebraic spaces. For example
Formal Spaces, Section \ref{formal-spaces-section-restricted-power-series}
defines the restricted power series ring $A\{x_1, \ldots, x_n\}$
given a linearly topologized ring $A$. In
Formal Spaces, Section \ref{formal-spaces-section-tft}
we discuss the relationship between these restricted power
series rings and morphisms of finite type between
locally countably indexed formal algebraic spaces.

\medskip\noindent
Let $A$ be a Noetherian ring and let $I \subset A$ be an ideal. In the
first part of this chapter
(Sections \ref{section-two-categories} -- \ref{section-over-G-ring})
we discuss the category of $I$-adically complete algebras $B$
topologically of finite type over a Noetherian ring $A$.
It is shown that $B = A\{x_1, \ldots, x_n\}/J$ for some
(closed) ideal $J$ in the restricted power series ring
(where $A$ is endowed with the $I$-adic topology).
We show there is a good notion of a naive cotangent complex
$\NL_{B/A}^\wedge$. If some power of $I$ annihilates
$\NL_{B/A}^\wedge$, then we think of $\text{Spf}(B)$ as
a rig-\'etale formal algebraic space over $A$. This leads to
a definition of rig-\'etale morphisms of Noetherian
formal algebraic spaces. After a certain amount of work
we are able to prove the main result of the first part:
if $\text{Spf}(B)$ is rig-\'etale over $A$ as above,
then there exists a finite type $A$-algebra $C$
such that $B$ is isomorphic to the $I$-adic completion of $C$, see
Lemma \ref{lemma-approximate-by-etale-over-complement}.
One thing to note here is that we prove this without assuming
the ring $A$ is excellent or even a G-ring. In the last section
of the first part we show that under the assumption that
$A$ is a G-ring there is a straightforward
proof of the lemma based on Popescu's theorem.

\medskip\noindent
Many of the results discussed in the first part can be found in the paper
\cite{Elkik}. Other general references for this part are
\cite{EGA}, \cite{Abbes}, and \cite{Fujiwara-Kato}.

\medskip\noindent
In the second part of this chapter we use the main result of
the first part to prove Artin's result on dilatations from \cite{ArtinII}.
The result on contractions will be the subject of a later chapter
(insert future reference here).
The main existence theorem is the equivalence of categories in
Theorem \ref{theorem-dilatations-general}.
It is more general than Artin's result in that it shows that any
rig-\'etale morphism $f : W \to \Spec(A)_{/V(I)}$
is the completion of a morphism $Y \to \Spec(A)$
of algebraic spaces $X$ which is locally of finite type
and isomorphism away from $V(I)$. In Artin's work
the morphism $f$ is assumed proper and rig-surjective.
A special case of this is the main lemma mentioned
above and the general case follows from this by
a straightforward (somewhat lengthy) glueing procedure.
There are several lemmas modifying the main theorem
the final one of which is (almost) exactly the statement in
Artin's paper.
In the last section we apply the results to modifications
of $\Spec(A)$ before and after completion.









\section{Two categories}
\label{section-two-categories}

\noindent
Let $A$ be a ring and let $I \subset A$ be an ideal.
In this section ${}^\wedge$ will mean $I$-adic completion.
Set $A_n = A/I^n$ so that the $I$-adic completion of $A$ is
$A^\wedge = \lim A_n$. Let $\mathcal{C}$ be the
category
\begin{equation}
\label{equation-C}
\mathcal{C} =
\left\{
\begin{matrix}
\text{systems }(B_n, B_{n + 1} \to B_n)_{n \in \mathbf{N}}\text{ where }\\
B_n\text{ is a finite type }A_n\text{-algebra,}\\
B_{n + 1} \to B_n\text{ is an }A_{n + 1}\text{-algebra map}\\
\text{which induces }B_{n + 1}/I^nB_{n + 1} \cong B_n
\end{matrix}
\right\}
\end{equation}
Morphisms in $\mathcal{C}$ are given by systems of homomorphisms.
Let $\mathcal{C}'$ be the category
\begin{equation}
\label{equation-C-prime}
\mathcal{C}' =
\left\{
\begin{matrix}
A\text{-algebras }B\text{ which are }I\text{-adically complete}\\
\text{such that }B/IB\text{ is of finite type over }A/I
\end{matrix}
\right\}
\end{equation}
Morphisms in $\mathcal{C}'$ are $A$-algebra maps. There is a functor
\begin{equation}
\label{equation-from-complete-to-systems}
\mathcal{C}' \longrightarrow \mathcal{C},\quad
B \longmapsto (B/I^nB)
\end{equation}
Indeed, since $B/IB$ is of finite type over $A/I$ the ring maps
$A_n = A/I^n \to B/I^nB$ are of finite type
(apply Algebra, Lemma \ref{algebra-lemma-NAK}
to a ring map $A/I^n[x_1, \ldots, x_r] \to B/I^nB$
such that the images of $x_1, \ldots, x_r$ generate $B/IB$
over $A/I$).

\begin{lemma}
\label{lemma-topologically-finite-type}
Let $A$ be a ring and let $I \subset A$ be a finitely generated ideal.
The functor
$$
\mathcal{C} \longrightarrow \mathcal{C}',\quad
(B_n) \longmapsto B = \lim B_n
$$
is a quasi-inverse to (\ref{equation-from-complete-to-systems}).
The completions $A[x_1, \ldots, x_r]^\wedge$ are in $\mathcal{C}'$ and
any object of $\mathcal{C}'$ is of the form
$$
B = A[x_1, \ldots, x_r]^\wedge / J
$$
for some ideal $J \subset A[x_1, \ldots, x_r]^\wedge$.
\end{lemma}

\begin{proof}
Let $(B_n)$ be an object of $\mathcal{C}$. By
Algebra, Lemma \ref{algebra-lemma-limit-complete}
we see that $B = \lim B_n$ is $I$-adically complete
and $B/I^nB = B_n$. Hence we see that $B$ is an object of
$\mathcal{C}'$ and that we can recover the object $(B_n)$ 
by taking the quotients.
Conversely, if $B$ is an object of $\mathcal{C}'$, then
$B = \lim B/I^nB$ by assumption. Thus $B \mapsto (B/I^nB)$ is a quasi-inverse
to the functor of the lemma.

\medskip\noindent
Since $A[x_1, \ldots, x_r]^\wedge = \lim A_n[x_1, \ldots, x_r]$
it is an object of $\mathcal{C}'$ by the first statement of the lemma.
Finally, let $B$ be an object of $\mathcal{C}'$. Choose
$b_1, \ldots, b_r \in B$ whose images in $B/IB$ generate
$B/IB$ as an algebra over $A/I$. Since $B$ is $I$-adically
complete, the $A$-algebra map $A[x_1, \ldots, x_r] \to B$, $x_i \mapsto b_i$
extends to an $A$-algebra map $A[x_1, \ldots, x_r]^\wedge \to B$.
To finish the proof we have to show this map is surjective
which follows from Algebra, Lemma \ref{algebra-lemma-completion-generalities}
as our map $A[x_1, \ldots, x_r] \to B$ is surjective modulo $I$
and as $B = B^\wedge$.
\end{proof}

\noindent
We warn the reader that, in case $A$ is not Noetherian, the
quotient of an object of $\mathcal{C}'$ may not be an object
of $\mathcal{C}'$. See Examples, Lemma
\ref{examples-lemma-noncomplete-quotient}.
Next we show this does not happen when $A$ is Noetherian.

\begin{lemma}
\label{lemma-topologically-finite-type-Noetherian}
\begin{reference}
\cite[Proposition 7.5.5]{EGA1}
\end{reference}
Let $A$ be a Noetherian ring and let $I \subset A$ be an ideal. Then
\begin{enumerate}
\item every object of the category $\mathcal{C}'$, in particular the
completion $A[x_1, \ldots, x_r]^\wedge$, is Noetherian,
\item if $B$ is an object of $\mathcal{C}'$ and $J \subset B$ is an
ideal, then $B/J$ is an object of $\mathcal{C}'$.
\end{enumerate}
\end{lemma}

\begin{proof}
To see (1) by Lemma \ref{lemma-topologically-finite-type}
we reduce to the case of the completion of the polynomial ring.
This case follows from
Algebra, Lemma \ref{algebra-lemma-completion-Noetherian-Noetherian}
as $A[x_1, \ldots, x_r]$ is Noetherian
(Algebra, Lemma \ref{algebra-lemma-Noetherian-permanence}).
Part (2) follows from Algebra, Lemma \ref{algebra-lemma-completion-tensor}
which tells us that ever finite $B$-module is
$IB$-adically complete.
\end{proof}

\begin{remark}[Base change]
\label{remark-base-change}
Let $\varphi : A_1 \to A_2$ be a ring map and let
$I_i \subset A_i$ be ideals such that $\varphi(I_1^c) \subset I_2$
for some $c \geq 1$. This induces ring maps
$A_{1, cn} = A_1/I_1^{cn} \to A_2/I_2^n = A_{2, n}$ for all $n \geq 1$.
Let $\mathcal{C}_i$ be the category (\ref{equation-C}) for $(A_i, I_i)$.
There is a base change functor
\begin{equation}
\label{equation-base-change-systems}
\mathcal{C}_1 \longrightarrow \mathcal{C}_2,\quad
(B_n) \longmapsto (B_{cn} \otimes_{A_{1, cn}} A_{2, n})
\end{equation}
Let $\mathcal{C}_i'$ be the category (\ref{equation-C-prime}) for $(A_i, I_i)$.
If $I_2$ is finitely generated, then there is a base change functor
\begin{equation}
\label{equation-base-change-complete}
\mathcal{C}_1' \longrightarrow \mathcal{C}_2',\quad
B \longmapsto (B \otimes_{A_1} A_2)^\wedge
\end{equation}
because in this case the completion is complete
(Algebra, Lemma \ref{algebra-lemma-hathat-finitely-generated}).
If both $I_1$ and $I_2$ are finitely generated, then
the two base change functors agree via the functors
(\ref{equation-from-complete-to-systems})
which are equivalences by Lemma \ref{lemma-topologically-finite-type}.
\end{remark}

\begin{remark}[Base change by closed immersion]
\label{remark-take-bar}
Let $A$ be a Noetherian ring and $I \subset A$ an ideal.
Let $\mathfrak a \subset A$ be an ideal. Denote $\bar A = A/\mathfrak a$.
Let $\bar I \subset \bar A$ be an ideal such that
$I^c \bar A \subset \bar I$ and $\bar I^d \subset I\bar A$
for some $c, d \geq 1$. In this case the base change functor
(\ref{equation-base-change-complete}) for $(A, I)$ to $(\bar A, \bar I)$
is given by $B \mapsto \bar B = B/\mathfrak aB$. Namely, we have
\begin{equation}
\label{equation-base-change-to-closed}
\bar B = (B \otimes_A \bar A)^\wedge = (B/\mathfrak a B)^\wedge =
B/\mathfrak a B
\end{equation}
the last equality because any finite $B$-module is $I$-adically complete by
Algebra, Lemma \ref{algebra-lemma-completion-tensor}
and if annihilated by $\mathfrak a$ also $\bar I$-adically complete by
Algebra, Lemma \ref{algebra-lemma-change-ideal-completion}.
\end{remark}







\section{A naive cotangent complex}
\label{section-naive-cotangent-complex}

\noindent
Let $A$ be a Noetherian ring and let $I \subset A$ be a ideal.
Let $B$ be an $A$-algebra which is $I$-adically complete such
that $A/I \to B/IB$ is of finite type, i.e., an object of
(\ref{equation-C-prime}).
By Lemma \ref{lemma-topologically-finite-type-Noetherian} we can write
$$
B = A[x_1, \ldots, x_r]^\wedge / J
$$
for some finitely generated ideal $J$. For a choice of presentation as
above we define the naive cotangent complex in this setting by the formula
\begin{equation}
\label{equation-NL}
\NL^\wedge_{B/A} = (J/J^2 \longrightarrow \bigoplus B\text{d}x_i)
\end{equation}
with terms sitting in degrees $-1$ and $0$
where the map sends the residue class of $g \in J$ to the differential
$\text{d}g = \sum (\partial g/\partial x_i) \text{d}x_i$. Here
the partial derivative is taken by thinking of $g$ as a power series.
The following lemma shows that $\NL^\wedge_{B/A}$ is well defined
in $D(B)$, i.e., independent of the chosen presentation, although this
could be shown directly by comparing presentations as in
Algebra, Section \ref{algebra-section-netherlander}.

\begin{lemma}
\label{lemma-NL-is-limit}
Let $A$ be a Noetherian ring and let $I \subset A$ be a ideal.
Let $B$ be an object of (\ref{equation-C-prime}). Then
$\NL^\wedge_{B/A} = R\lim \NL_{B_n/A_n}$ in $D(B)$.
\end{lemma}

\begin{proof}
In fact, the presentation $B = A[x_1, \ldots, x_r]^\wedge / J$
defines presentations
$$
B_n = B/I^nB = A_n[x_1, \ldots, x_r]/J_n
$$
where
$$
J_n = JA_n[x_1, \ldots, x_r] =
J/(J \cap I^nA[x_1, \ldots, x_r]^\wedge)
$$
By Artin-Rees (Algebra, Lemma \ref{algebra-lemma-Artin-Rees})
in the Noetherian ring $A[x_1, \ldots, x_r]^\wedge$
(Lemma \ref{lemma-topologically-finite-type-Noetherian})
we see that we have canonical surjections
$$
J/I^nJ \to J_n \to J/I^{n - c}J,\quad n \geq c
$$
for some $c \geq 0$.
It follows that $\lim J_n/J_n^2 = J/J^2$ as any finite
$A[x_1, \ldots, x_r]^\wedge$-module is $I$-adically complete
(Algebra, Lemma \ref{algebra-lemma-completion-tensor}).
Thus
$$
\NL^\wedge_{B/A} =
\lim (J_n/J_n^2 \longrightarrow \bigoplus B_n \text{d}x_i)
$$
(termwise limit)
and the transition maps in the system are termwise surjective.
The two term complex $J_n/J_n^2 \longrightarrow \bigoplus B_n \text{d}x_i$
represents $\NL_{B_n/A_n}$ by
Algebra, Section \ref{algebra-section-netherlander}.
It follows that $\NL^\wedge_{B/A}$ represents
$R\lim \NL_{B_n/A_n}$ in the derived category by
More on Algebra, Lemma \ref{more-algebra-lemma-compute-Rlim-modules}.
\end{proof}

\begin{lemma}
\label{lemma-exact-sequence-NL}
Let $A$ be a Noetherian ring and let $I \subset A$ be a ideal.
Let $B \to C$ be morphism of (\ref{equation-C-prime}). Then
there is an exact sequence
$$
\xymatrix{
C \otimes_B H^0(\NL^\wedge_{B/A}) \ar[r] &
H^0(\NL^\wedge_{C/A}) \ar[r] &
H^0(\NL^\wedge_{C/B}) \ar[r] & 0 \\
H^{-1}(\NL^\wedge_{B/A} \otimes_B C) \ar[r] &
H^{-1}(\NL^\wedge_{C/A}) \ar[r] &
H^{-1}(\NL^\wedge_{C/B}) \ar[llu]
}
$$
\end{lemma}

\begin{proof}
Choose a presentation $B = A[x_1, \ldots, x_r]^\wedge/J$.
Note that $(B, IB)$ is a pair consisting of a Noetherian ring
and an ideal, and $C$ is in the corresponding category (\ref{equation-C-prime})
for this pair. Hence we can choose a presentation
$C = B[y_1, \ldots, y_s]^\wedge/J'$. Combinging these presentations
gives a presentation
$$
C = A[x_1, \ldots, x_r, y_1, \ldots, y_s]^\wedge/K
$$
Then the reader verifies that we obtain a commutative diagram
$$
\xymatrix{
0 \ar[r] &
\bigoplus C \text{d}x_i \ar[r] &
\bigoplus C \text{d}x_i \oplus \bigoplus C \text{d}y_j \ar[r] &
\bigoplus C \text{d}y_j \ar[r] &
0 \\
&
J/J^2 \otimes_B C \ar[r] \ar[u] &
K/K^2 \ar[r] \ar[u] &
J'/(J')^2 \ar[r] \ar[u] &
0
}
$$
with exact rows. Note that the vertical arrow on the left hand side
is the tensor product of the arrow defining $\NL^\wedge_{B/A}$ with
$\text{id}_C$. The lemma follows by applying the snake lemma
(Algebra, Lemma \ref{algebra-lemma-snake}).
\end{proof}

\begin{lemma}
\label{lemma-transitive-lci-at-end}
With assumptions as in Lemma \ref{lemma-exact-sequence-NL}
assume that $B/I^nB \to C/I^nC$ is a local complete intersection
homomorphism for all $n$. Then
$H^{-1}(\NL^\wedge_{B/A} \otimes_B C) \to H^{-1}(\NL^\wedge_{C/A})$
is injective.
\end{lemma}

\begin{proof}
By More on Algebra, Lemma \ref{more-algebra-lemma-transitive-lci-at-end}
we see that this holds for the map between naive cotangent complexes
of the situation modulo $I^n$ for all $n$. In other words, we obtain
a distinguished triangle in $D(C/I^nC)$ for every $n$. Using
Lemma \ref{lemma-NL-is-limit}
this implies the lemma; details omitted.
\end{proof}

\noindent
Maps in the derived category out of a complex such as (\ref{equation-NL})
are easy to understand by the result of the following lemma.

\begin{lemma}
\label{lemma-map-out-of-almost-free}
Let $R$ be a ring. Let $M^\bullet$ be a complex of modules over $R$
with $M^i = 0$ for $i > 0$ and $M^0$ a projective $R$-module.
Let $K^\bullet$ be a second complex.
\begin{enumerate}
\item If $K^i = 0$ for $i \leq -2$, then
$\Hom_{D(R)}(M^\bullet, K^\bullet) = \Hom_{K(R)}(M^\bullet, K^\bullet)$,
\item If $K^i = 0$ for $i \leq -3$ and
$\alpha \in \Hom_{D(R)}(M^\bullet, K^\bullet)$ composed with
$K^\bullet \to K^{-2}[2]$ comes from an $R$-module map
$a : M^{-2} \to K^{-2}$ with $a \circ d_M^{-3} = 0$, then
$\alpha$ can be represented by a map of complexes
$a^\bullet : M^\bullet \to K^\bullet$ with $a^{-2} = a$.
\item In (2) for any second map of complexes
$(a')^\bullet : M^\bullet \to K^\bullet$
representing $\alpha$ with $a = (a')^{-2}$
there exist $h' : M^0 \to K^{-1}$ and
$h : M^{-1} \to K^{-2}$ such that
$$
h \circ d_M^{-2} = 0, \quad
(a')^{-1} = a^{-1} + d_K^{-2} \circ h + h' \circ d_M^{-1},\quad
(a')^0 = a^0 + d_K^{-1} \circ h'
$$
\end{enumerate}
\end{lemma}

\begin{proof}
Set $F^0 = M^0$.
Choose a free $R$-module $F^{-1}$ and a surjection $F^{-1} \to M^{-1}$.
Choose a free $R$-module $F^{-2}$ and a surjection
$F^{-2} \to M^{-2} \times_{M^{-1}} F^{-1}$. Continuing in this
way we obtain a quasi-isomorphism $p^\bullet : F^\bullet \to M^\bullet$
which is termwise surjective and with $F^i$ free for all $i$.

\medskip\noindent
Proof of (1). By
Derived Categories, Lemma \ref{derived-lemma-morphisms-from-projective-complex}
we have
$$
\Hom_{D(R)}(M^\bullet, K^\bullet) = \Hom_{K(R)}(F^\bullet, K^\bullet)
$$
If $K^i = 0$ for $i \leq -2$, then any morphism of complexes
$F^\bullet \to K^\bullet$ factors through $p^\bullet$. Similarly, any
homotopy $\{h^i : F^i \to K^{i - 1}\}$ factors through $p^\bullet$.
Thus (1) holds.

\medskip\noindent
Proof of (2). Choose $b^\bullet : F^\bullet \to K^\bullet$ representing
$\alpha$. The composition of $\alpha$ with $K^\bullet \to K^{-2}[2]$ is
represented by $b^{-2} : F^{-2} \to K^{-2}$. As this is homotopic to
$a \circ p^{-2} : F^{-2} \to M^{-2} \to K^{-2}$, there is a map
$h : F^{-1} \to K^{-2}$ such that $b^{-2} = a \circ p^{-2} + h \circ d_F^{-2}$.
Adjusting $b^\bullet$ by $h$ viewed as a homotopy from $F^\bullet$
to $K^\bullet$, we find that $b^{-2} = a \circ p^{-2}$. Hence $b^{-2}$
factors through $p^{-2}$. Since $F^0 = M^0$ the kernel of $p^{-2}$
surjects onto the kernel of $p^{-1}$ (for example because the kernel
of $p^\bullet$ is an acyclic complex or by a diagram chase). Hence $b^{-1}$
necessarily factors through $p^{-1}$ as well and we see that (2)
holds for these factorizations and $a^0 = b^0$.

\medskip\noindent
Proof of (3) is omitted. Hint: There is a homotopy between
$a^\bullet \circ p^\bullet$ and $(a')^\bullet \circ p^\bullet$
and we argue as before that this homotopy factors through $p^\bullet$.
\end{proof}

\begin{lemma}
\label{lemma-zero-in-derived}
Let $R$ be a ring. Let $M^\bullet$ be a two term complex $M^{-1} \to M^0$
over $R$. If $\varphi, \psi \in \text{End}_{D(R)}(M^\bullet)$
are zero on $H^i(M^\bullet)$, then $\varphi \circ \psi = 0$.
\end{lemma}

\begin{proof}
Apply
Derived Categories, Lemma \ref{derived-lemma-trick-vanishing-composition}
to see that $\varphi \circ \psi$ factors through $\tau_{\leq -2}M^\bullet = 0$.
\end{proof}



\section{Rig-\'etale homomorphisms}
\label{section-rig-etale}

\noindent
In this and some of the later sections we will study ring maps as
in Lemma \ref{lemma-equivalent-with-artin}. Condition
(\ref{item-condition-artin}) is one of the conditions used in
\cite{ArtinII} to define modifications. Ring maps like this are sometimes
called rig-\'etale or rigid-\'etale ring maps in the literature. These and
the analogously defined rig-smooth ring maps were studied
in \cite{Elkik}. A detailed exposition can also be found in
\cite{Abbes}. Our main goal will be to show that rig-\'etale
ring maps are completions of finite type algebras, a result
very similar to results found in Elkik's paper \cite{Elkik}.

\begin{lemma}
\label{lemma-equivalent-with-artin}
Let $A$ be a Noetherian ring and let $I \subset A$ be an ideal.
Let $B$ be an object of (\ref{equation-C-prime}). The following are equivalent
\begin{enumerate}
\item
\label{item-zero-on-NL}
there exists a $c \geq 0$ such that multiplication by $a$
on $\NL^\wedge_{B/A}$ is zero in $D(B)$ for all $a \in I^c$,
\item
\label{item-zero-on-cohomology-NL}
there exits a $c \geq 0$ such that $H^i(\NL^\wedge_{B/A})$, $i = -1, 0$ is
annihilated by $I^c$,
\item
\label{item-zero-on-cohomology-NL-truncations}
there exists a $c \geq 0$ such that $H^i(\NL_{B_n/A_n})$, $i = -1, 0$ is
annihilated by $I^c$ for all $n \geq 1$,
\item
\label{item-condition-artin}
$B = A[x_1, \ldots, x_r]^\wedge/J$ and
for every $a \in I$ there exists a $c \geq 0$ such that
\begin{enumerate}
\item $a^c$ annihilates $H^0(\NL^\wedge_{B/A})$, and
\item there exist $f_1, \ldots, f_r \in J$ such that
$a^c J \subset (f_1, \ldots, f_r) + J^2$.
\end{enumerate}
\end{enumerate}
\end{lemma}

\begin{proof}
The equivalence of (1) and (2) follows from
Lemma \ref{lemma-zero-in-derived}.
The equivalence of (1) $+$ (2) and (3) follows from
Lemma \ref{lemma-NL-is-limit}. Some details omitted.

\medskip\noindent
Assume the equivalent conditions (1), (2), (3) holds and let
$B = A[x_1, \ldots, x_r]^\wedge/J$ be a presentation
(see Lemma \ref{lemma-topologically-finite-type}). Let $a \in I$.
Let $c$ be such that multiplication by $a^c$ is zero on $\NL^\wedge_{B/A}$
which exists by (1). By Lemma \ref{lemma-map-out-of-almost-free}
there exists a map $\alpha : \bigoplus B\text{d}x_i \to J/J^2$ such that
$\text{d} \circ \alpha$ and $\alpha \circ \text{d}$ are both
multiplication by $a^c$. Let $f_i \in J$ be an element whose
class modulo $J^2$ is equal to $\alpha(\text{d}x_i)$.
Then we see that (\ref{item-condition-artin})(a), (b) hold.

\medskip\noindent
Assume (\ref{item-condition-artin}) holds. Say $I = (a_1, \ldots, a_t)$.
Let $c_i \geq 0$ be the integer such that (\ref{item-condition-artin})(a), (b)
hold for $a_i^{c_i}$. Then we see that $I^{\sum c_i}$ annihilates
$H^0(\NL^\wedge_{B/A})$. Let $f_{i, 1}, \ldots, f_{i, r} \in J$
be as in (\ref{item-condition-artin})(b) for $a_i$.
Consider the composition
$$
B^{\oplus r} \to J/J^2 \to \bigoplus B\text{d}x_i
$$
where the $j$th basis vector is mapped to the class of $f_{i, j}$ in $J/J^2$.
By (\ref{item-condition-artin})(a) and (b) the cokernel of the composition
is annihilated by $a_i^{2c_i}$. Thus this map is surjective after inverting
$a_i^{c_i}$, and hence an isomorphism (Algebra, Lemma \ref{algebra-lemma-fun}).
Thus the kernel of $B^{\oplus r} \to \bigoplus B\text{d}x_i$ is
$a_i$-power torsion, and hence
$H^{-1}(\NL^\wedge_{B/A}) = \Ker(J/J^2 \to \bigoplus B\text{d}x_i)$
is $a_i$-power torsion. Since $B$ is Noetherian
(Lemma \ref{lemma-topologically-finite-type-Noetherian}),
all modules including $H^{-1}(\NL^\wedge_{B/A})$ are finite.
Thus $a_i^{d_i}$ annihilates $H^{-1}(\NL^\wedge_{B/A})$ for some $d_i \geq 0$.
It follows that $I^{\sum d_i}$ annihilates $H^{-1}(\NL^\wedge_{B/A})$
and we see that (2) holds.
\end{proof}

\begin{lemma}
\label{lemma-rig-etale}
Let $A$ be a Noetherian ring and let $I$ be an ideal.
Let $B$ be a finite type $A$-algebra.
\begin{enumerate}
\item If $\Spec(B) \to \Spec(A)$ is \'etale over $\Spec(A) \setminus V(I)$,
then $B^\wedge$ satisfies the equivalent conditions of
Lemma \ref{lemma-equivalent-with-artin}.
\item If $B^\wedge$ satisfies the equivalent conditions of
Lemma \ref{lemma-equivalent-with-artin},
then there exists $g \in 1 + IB$ such that $\Spec(B_g)$ is \'etale
over $\Spec(A) \setminus V(I)$.
\end{enumerate}
\end{lemma}

\begin{proof}
Assume $B^\wedge$ satisfies the equivalent conditions of
Lemma \ref{lemma-equivalent-with-artin}.
The naive cotangent complex $\NL_{B/A}$ is a complex of finite type
$B$-modules and hence $H^{-1}$ and $H^0$ are finite $B$-modules.
Completion is an exact functor on finite $B$-modules (Algebra,
Lemma \ref{algebra-lemma-completion-flat}) and $\NL^\wedge_{B^\wedge/A}$
is the completion of the complex $\NL_{B/A}$ (this is easy to see
by choosing presentations).
Hence the assumption implies there exists a $c \geq 0$ such that
$H^{-1}/I^nH^{-1}$ and $H^0/I^nH^0$ are annihilated by $I^c$
for all $n$. By Nakayama's lemma (Algebra, Lemma \ref{algebra-lemma-NAK})
this means that $I^cH^{-1}$ and $I^cH^0$ are annihilated by an element
of the form $g = 1 + x$ with $x \in IB$. After inverting $g$
(which does not change the quotients $B/I^nB$)
we see that $\NL_{B/A}$ has cohomology annihilated by $I^c$. Thus
$A \to B$ is \'etale at any prime of $B$ not lying over $V(I)$
by the definition of \'etale ring maps, see
Algebra, Definition \ref{algebra-definition-etale}.

\medskip\noindent
Conversely, assume that $\Spec(B) \to \Spec(A)$ is \'etale over
$\Spec(A) \setminus V(I)$. Then for every $a \in I$ there exists
a $c \geq 0$ such that multiplication by $a^c$ is zero $\NL_{B/A}$.
Since $\NL^\wedge_{B^\wedge/A}$ is the derived completion of
$\NL_{B/A}$ (see Lemma \ref{lemma-NL-is-limit}) it follows that
$B^\wedge$ satisfies the equivalent conditions of
Lemma \ref{lemma-equivalent-with-artin}.
\end{proof}

\begin{lemma}
\label{lemma-zero-after-modding-out}
Assume the map $(A_1, I_1) \to (A_2, I_2)$ is as in
Remark \ref{remark-base-change} with $A_1$ and $A_2$ Noetherian.
Let $B_1$ be in (\ref{equation-C-prime}) for $(A_1, I_1)$.
Let $B_2$ be the base change of $B_1$.
If multiplication by $f_1 \in B_1$ on $\NL^\wedge_{B_1/A_1}$
is zero in $D(B_1)$, then multiplication by
the image $f_2 \in B_2$ on $\NL^\wedge_{B_2/A_2}$ is zero
in $D(B_2)$.
\end{lemma}

\begin{proof}
Choose a presentation $B_1 = A_1[x_1, \ldots, x_r]^\wedge/J_1$.
Since
$A_2/I_2^n[x_1, \ldots, x_r] =
A_1/I_1^{cn}[x_1, \ldots, x_r] \otimes_{A_1/I_1^{cn}} A_2/I_2^n$
we have
$$
A_2[x_1, \ldots, x_r]^\wedge =
(A_1[x_1, \ldots, x_r]^\wedge \otimes_{A_1} A_2)^\wedge
$$
where we use $I_2$-adic completion on both sides (but of course
$I_1$-adic completion for $A_1[x_1, \ldots, x_r]^\wedge$).
Set $J_2 = J_1 A_2[x_1, \ldots, x_r]^\wedge$. Arguing similarly
we get the presentation
\begin{align*}
B_2
& =
(B_1 \otimes_{A_1} A_2)^\wedge \\
& =
\lim \frac{A_1/I_1^{cn}[x_1, \ldots, x_r]}{J_1(A_1/I_1^{cn}[x_1, \ldots, x_r])}
\otimes_{A_1/I_1^{cn}} A_2/I_2^n \\
& =
\lim \frac{A_2/I_2^n[x_1, \ldots, x_r]}{J_2(A_2/I_2^n[x_1, \ldots, x_r])} \\
& =
A_2[x_1, \ldots, x_r]^\wedge/J_2
\end{align*}
for $B_2$ over $A_2$. Consider the commutative diagram
$$
\xymatrix{
\NL^\wedge_{B_1/A_1} : \ar[d] &
J_1/J_1^2 \ar[r]_-{\text{d}} \ar[d] & \bigoplus B_1\text{d}x_i \ar[d] \\
\NL^\wedge_{B_2/A_2} : &
J_2/J_2^2 \ar[r] & \bigoplus B_2\text{d}x_i
}
$$
The induced arrow $J_1/J_1^2 \otimes_{B_1} B_2 \to J_2/J_2^2$
is surjective because $J_2$ is generated by the image of $J_1$.
By Lemma \ref{lemma-map-out-of-almost-free}
there is a map $\alpha_1 : \bigoplus B\text{d}x_i \to J_1/J_1^2$
such that $f_1 \text{id}_{\bigoplus B_1\text{d}x_i} = \text{d} \circ \alpha_1$
and $f_1 \text{id}_{J_1/J_1^2} = \alpha_1 \circ \text{d}$. We define
$\alpha_2 : \bigoplus B_1\text{d}x_i \to J_2/J_2^2$
by mapping $\text{d}x_i$ to the image of $\alpha_1(\text{d}x_i)$
in $J_2/J_2^2$. Because the image of the vertical arrows
contains generators of the modules $J_2/J_2^2$ and $\bigoplus B_2 \text{d}x_i$
it follows that $\alpha_2$ also defines a homotopy between
multiplication by $f_2$ and the zero map.
\end{proof}

\begin{lemma}
\label{lemma-fully-faithful-etale-over-complement}
Let $A$ be a Noetherian ring. Let $I \subset A$ be an ideal.
Let $B$ be a finite type $A$-algebra such that
$\Spec(B) \to \Spec(A)$ is \'etale over $\Spec(A) \setminus V(I)$.
Let $C$ be a Noetherian $A$-algebra. Then any $A$-algebra
map $B^\wedge \to C^\wedge$ of $I$-adic completions
comes from a unique $A$-algebra map
$$
B \longrightarrow C^h
$$
where $C^h$ is the henselization of the pair $(C, IC)$ as
in More on Algebra, Lemma \ref{more-algebra-lemma-henselization}.
Moreover, any $A$-algebra homomorphism $B \to C^h$ factors through
some \'etale $C$-algebra $C'$ such that $C/IC \to C'/IC'$ is an isomorphism.
\end{lemma}

\begin{proof}
Uniqueness follows from the fact that $C^h$ is a subring of
$C^\wedge$, see for example
More on Algebra, Lemma \ref{more-algebra-lemma-henselization-Noetherian-pair}.
The final assertion follows from the fact that $C^h$ is the filtered colimit
of these $C$-algebras $C'$, see proof of
More on Algebra, Lemma \ref{more-algebra-lemma-henselization}.
Having said this we now turn to the proof of existence.

\medskip\noindent
Let $\varphi : B^\wedge \to C^\wedge$ be the given map.
This defines a section
$$
\sigma : (B \otimes_A C)^\wedge \longrightarrow C^\wedge
$$
of the completion of the map $C \to B \otimes_A C$. We may
replace $(A, I, B, C, \varphi)$ by $(C, IC, B \otimes_A C, C, \sigma)$.
In this way we see that we may assume that $A = C$.

\medskip\noindent
Proof of existence in the case $A = C$. In this case the map
$\varphi : B^\wedge \to A^\wedge$ is necessarily surjective.
By Lemmas \ref{lemma-rig-etale} and \ref{lemma-exact-sequence-NL}
we see that the cohomology groups of
$\NL^\wedge_{A^\wedge/\!_\varphi B^\wedge}$
are annihilated by a power of $I$. Since $\varphi$ is surjective,
this implies that $\Ker(\varphi)/\Ker(\varphi)^2$ is annihilated
by a power of $I$. Hence $\varphi : B^\wedge \to A^\wedge$
is the completion of a finite type $B$-algebra $B \to D$, see
More on Algebra, Lemma \ref{more-algebra-lemma-quotient-by-idempotent}.
Hence $A \to D$ is a finite type algebra map which induces an isomorphism
$A^\wedge \to D^\wedge$. By
Lemma \ref{lemma-rig-etale} we may replace $D$ by a localization
and assume that $A \to D$ is \'etale away from $V(I)$.
Since $A^\wedge \to D^\wedge$ is an isomorphism, we see that
$\Spec(D) \to \Spec(A)$ is also \'etale in a neighbourhood of $V(ID)$
(for example by
More on Morphisms, Lemma
\ref{more-morphisms-lemma-check-smoothness-on-infinitesimal-nbhds}).
Thus $\Spec(D) \to \Spec(A)$ is \'etale. Therefore $D$ maps to
$A^h$ and the lemma is proved.
\end{proof}






\section{Rig-\'etale morphisms}
\label{section-rig-etale-morphisms}

\noindent
We can use the notion introduced in the previous section to define
a new type of morphism of locally Noetherian formal algebraic spaces.
Before we do so, we have to check it is a local property.

\begin{lemma}
\label{lemma-rig-etale-axioms}
For morphisms $A \to B$ of the category $\textit{WAdm}^{Noeth}$
(Formal Spaces, Section \ref{formal-spaces-section-morphisms-rings})
consider the condition $P=$``for some ideal of definition $I$ of $A$
the topology on $B$ is the $I$-adic topology, the ring map $A/I \to B/IB$
is of finite type and $A \to B$ satisfies the equivalent conditions of
Lemma \ref{lemma-equivalent-with-artin}''. Then $P$ is a local property, see
Formal Spaces, Remark \ref{formal-spaces-remark-variant-Noetherian}.
\end{lemma}

\begin{proof}
We have to show that Formal Spaces, Axioms (\ref{formal-spaces-item-axiom-1}),
(\ref{formal-spaces-item-axiom-2}), and (\ref{formal-spaces-item-axiom-3})
hold for maps between Noetherian adic rings. For a Noetherian adic ring
$A$ with ideal of definition $I$ we have
$A\{x_1, \ldots, x_r\} = A[x_1, \ldots, x_r]^\wedge$ as topological
$A$-algebras (see Formal Spaces, Remark
\ref{formal-spaces-remark-I-adic-completion-and-restricted-power-series}).
We will use without further mention that we know the axioms hold
for the property ``$B$ is a quotient of $A[x_1, \ldots, x_r]^\wedge$'', see
Formal Spaces, Lemma
\ref{formal-spaces-lemma-quotient-restricted-power-series}.

\medskip\noindent
Let a diagram as in
Formal Spaces, Diagram (\ref{formal-spaces-equation-localize})
be given with $A$ and $B$ in the category $\textit{WAdm}^{Noeth}$.
Pick an ideal of definition $I \subset A$. By the remarks above
the topology on each ring in the diagram is the $I$-adic topology.
Since $A \to A'$ and $B \to B'$ are \'etale we see that
$\NL^\wedge_{(A')^\wedge/A}$ and $\NL^\wedge_{(B')^\wedge/B}$
are zero. By Lemmas \ref{lemma-exact-sequence-NL} and
\ref{lemma-transitive-lci-at-end} we get
$$
H^i(\NL^\wedge_{(B')^\wedge/(A')^\wedge})
\cong
H^i(\NL^\wedge_{(B')^\wedge/A})
\quad\text{and}\quad
H^i(\NL^\wedge_{B/A} \otimes_B (B')^\wedge) \cong
H^i(\NL^\wedge_{(B')^\wedge/A})
$$
for $i = -1, 0$. Since $B$ is Noetherian the ring map
$B \to B' \to (B')^\wedge$ is flat
(Algebra, Lemma \ref{algebra-lemma-completion-flat})
hence the tensor product comes out. Moreover, as $B$ is
$I$-adically complete, then if $B \to B'$ is faithfully flat,
so is $B \to (B')^\wedge$. From these observations
Formal Spaces, Axioms (\ref{formal-spaces-item-axiom-1})
and (\ref{formal-spaces-item-axiom-2}) follow immediately.

\medskip\noindent
We omit the proof of Formal Spaces, Axiom (\ref{formal-spaces-item-axiom-3}).
\end{proof}

\begin{definition}
\label{definition-rig-etale}
Let $S$ be a scheme. Let $f : X \to Y$ be a morphism of locally
Noetherian formal algebraic spaces over $S$. We say $f$ is
{\it rig-\'etale} if $f$ satisfies the equivalent conditions of
Formal Spaces, Lemma
\ref{formal-spaces-lemma-property-defines-property-morphisms}
(in the setting of locally Noetherian formal algebraic spaces, see
Formal Spaces, Remark \ref{formal-spaces-remark-variant-adic-star})
for the property $P$ of Lemma \ref{lemma-rig-etale-axioms}.
\end{definition}

\noindent
To be sure, a rig-\'etale morphism is locally of finite type.

\begin{lemma}
\label{lemma-rig-etale-finite-type}
A rig-\'etale morphism of locally Noetherian formal algebraic spaces
is locally of finite type.
\end{lemma}

\begin{proof}
The property $P$ in Lemma \ref{lemma-rig-etale-axioms}
implies the equivalent conditions (a), (b), (c), and (d) in
Formal Spaces, Lemma
\ref{formal-spaces-lemma-quotient-restricted-power-series}.
Hence this follows from
Formal Spaces, Lemma \ref{formal-spaces-lemma-finite-type-local-property}.
\end{proof}








\section{Glueing rings along a principal ideal}
\label{section-approximation-principal}

\noindent
In this situation we prove some results about the categories
$\mathcal{C}$ and $\mathcal{C}'$ of
Section \ref{section-two-categories}
in case $A$ is a Noetherian ring and $I = (a)$ is a principal
ideal.

\begin{remark}[Linear approximation]
\label{remark-linear-approximation}
Let $A$ be a ring and $I \subset A$ be a finitely generated ideal.
Let $C$ be an $I$-adically complete $A$-algebra.
Let $\psi : A[x_1, \ldots, x_r]^\wedge \to C$ be a continuous
$A$-algebra map. Suppose given $\delta_i \in C$, $i = 1, \ldots, r$.
Then we can consider
$$
\psi' : A[x_1, \ldots, x_r]^\wedge \to C,\quad
x_i \longmapsto \psi(x_i) + \delta_i
$$
see Formal Spaces, Remark \ref{formal-spaces-remark-universal-property}.
Then we have
$$
\psi'(g) = \psi(g) + \sum \psi(\partial g/\partial x_i)\delta_i + \xi
$$
with error term $\xi \in (\delta_i\delta_j)$. This follows by
writing $g$ as a power series and working term by term. Convergence
is automatic as the coefficients of $g$ tend to zero.
Details omitted.
\end{remark}

\begin{lemma}
\label{lemma-get-morphism-nonzerodivisor}
Let $A$ be a Noetherian ring and $I = (a)$ a principal ideal.
Let $B$ be an objects of (\ref{equation-C-prime}).
Assume given an integer $c \geq 0$ such that
multiplication by $a^c$ on $\NL^\wedge_{B/A}$ is zero in $D(B)$.
Let $C$ be an $I$-adically complete $A$-algebra such that
$a$ is a nonzerodivisor on $C$. Let $n > 2c$. For any $A_n$-algebra
map $\psi_n : B/a^nB \to C/a^nC$ there exists an $A$-algebra
map $\varphi : B \to C$ such that
$\psi_n \bmod a^{n - c} = \varphi \bmod a^{n - c}$.
\end{lemma}

\begin{proof}
Choose a presentation $B = A[x_1, \ldots, x_r]^\wedge/J$. Choose
a lift
$$
\psi : A[x_1, \ldots, x_r]^\wedge \to C
$$
of $\psi_n$. Then $\psi(J) \subset a^nC$ and $\psi(J^2) \subset a^{2n}C$
which determines a linear map
$$
J/J^2 \longrightarrow a^nC/a^{2n}C,\quad g \longmapsto \psi(g)
$$
By assumption and Lemma \ref{lemma-map-out-of-almost-free}
there is a $B$-module map
$\bigoplus B\text{d}x_i \to a^nC/a^{2n}C$,
$\text{d}x_i \mapsto \delta_i$ such that
$a^c \psi(g) = \sum \psi(\partial g/\partial x_i) \delta_i$
for all $g \in J$. Write $\delta_i = - a^c \delta'_i$ for some
$\delta'_i \in a^{n - c}C$. Since $a$ is a nonzerodivisor
on $C$ we see that $\psi(g) = - \sum \psi(\partial g/\partial x_i) \delta'_i$
in $C/a^{2n - c}C$.
Then we look at the map
$$
\psi' : A[x_1, \ldots, x_r]^\wedge \to C,\quad
x_i \longmapsto \psi(x_i) + \delta'_i
$$
A computation with power series (see Remark \ref{remark-linear-approximation})
shows that $\psi'(J) \subset a^{2n - 2c}C$. Since $n > 2c$
we see that $n' = 2n - 2c = n + (n - 2c) > n$. Thus we obtain a morphism
$\psi_{n'} : B/a^{n'}B \to C/a^{n'}C$ agreeing with $\psi_n$ modulo
$a^{n - c}$. Continuing in this fashion and taking the limit
into $C = \lim C/a^tC$ we obtain the lemma.
\end{proof}

\begin{lemma}
\label{lemma-get-morphism-principal}
Let $A$ be a Noetherian ring and $I = (a)$ a principal ideal.
Let $B$ be an object of (\ref{equation-C-prime}).
Assume given an integer $c \geq 0$ such that
multiplication by $a^c$ on $\NL^\wedge_{B/A}$ is zero in $D(B)$.
Let $C$ be an $I$-adically complete $A$-algebra.
Assume given an integer $d \geq 0$ such that $C[a^\infty] \cap a^dC = 0$.
Let $n > \max(2c, c + d)$. For any $A_n$-algebra map
$\psi_n : B/a^nB \to C/a^nC$ there exists an $A$-algebra map
$\varphi : B \to C$ such
that $\psi_n \bmod a^{n - c} = \varphi \bmod a^{n - c}$.
\end{lemma}

\noindent
If $C$ is Noetherian we have $C[a^\infty] = C[a^e]$ for some
$e \geq 0$. By Artin-Rees (Algebra, Lemma \ref{algebra-lemma-Artin-Rees})
there exists an integer $f$ such that
$a^nC \cap C[a^\infty] \subset a^{n - f}C[a^\infty]$ for all $n \geq f$.
Then $d = e + f$ is an integer as in the lemma. This argument
works in particular if $C$ is an object of (\ref{equation-C-prime})
by Lemma \ref{lemma-topologically-finite-type-Noetherian}.

\begin{proof}
Let $C \to C'$ be the quotient of $C$ by $C[a^\infty]$. The $A$-algebra
$C'$ is $I$-adically complete by
Algebra, Lemma \ref{algebra-lemma-quotient-complete}
and the fact that $\bigcap (C[a^\infty] + a^nC) = C[a^\infty]$
because for $n \geq d$ the sum $C[a^\infty] + a^nC$ is direct.
For $m \geq d$ the diagram
$$
\xymatrix{
0 \ar[r] &
C[a^\infty] \ar[r] \ar[d] &
C \ar[r] \ar[d] & C' \ar[r] \ar[d] & 0 \\
0 \ar[r] &
C[a^\infty] \ar[r] &
C/a^m C \ar[r] & C'/a^m C' \ar[r] & 0
}
$$
has exact rows. Thus $C$ is the fibre product of $C'$ and
$C/a^mC$ over $C'/a^mC'$. Thus the lemma now follows formally from
the lifting result of Lemma \ref{lemma-get-morphism-nonzerodivisor}.
\end{proof}

\begin{lemma}
\label{lemma-approximate-principal}
\begin{reference}
The rig-\'etale case of \cite[III Theorem 7]{Elkik}
which handles the rig-smooth case.
\end{reference}
Let $A$ be a Noetherian ring and $I = (a)$ a principal ideal.
Let $B$ be an object of (\ref{equation-C-prime}).
Assume given an integer $c \geq 0$ such that
multiplication by $a^c$ on $\NL^\wedge_{B/A}$ is zero in $D(B)$.
Then there exists a finite type $A$-algebra $C$ and an
isomorphism $B \cong C^\wedge$.
\end{lemma}

\begin{proof}
Choose a presentation $B = A[x_1, \ldots, x_r]^\wedge/J$.
By Lemma \ref{lemma-map-out-of-almost-free} we can find a map
$\alpha : \bigoplus B\text{d}x_i \to J/J^2$ such that
$\text{d} \circ \alpha$ and $\alpha \circ \text{d}$ are both
multiplication by $a^c$. Pick an element $f_i \in J$ whose
class modulo $J^2$ is equal to $\alpha(\text{d}x_i)$.
Then we see that $\text{d}f_i = a^c \text{d}x_i$ in $\bigoplus \text{d}x_i$.
In particular we have a ring map
$$
A[x_1, \ldots, x_r]^\wedge/
(f_1, \ldots, f_r, \Delta(f_1, \ldots, f_r) - a^{rc})
\longrightarrow B
$$
where $\Delta(f_1, \ldots, f_r) \in A[x_1, \ldots, x_r]^\wedge$
is the determinant of the matrix of partial derivatives of the $f_i$.

\medskip\noindent
Pick a large integer $N$. Pick $F_1, \ldots, F_r \in A[x_1, \ldots, x_r]$
such that $F_i - f_i \in I^NA[x_1, \ldots, x_r]^\wedge$. Set
$$
C = A[x_1, \ldots, x_r, z]/
(F_1, \ldots, F_r, z\Delta(F_1, \ldots, F_r) - a^{rc})
$$
We claim that multiplication by $a^{2rc}$ is zero on $\NL_{C/A}$ in $D(C)$.
Namely, the determinant of the matrix of the partial derivatives
of the $r + 1$ generators of the ideal of $C$ with respect to the variables
$x_1, \ldots, x_{r + 1}, z$ is $\Delta(F_1, \ldots, F_r)^2$. Since
$\Delta(F_1, \ldots, F_r)$ divides $a^{rc}$ we in $C$ the
claim follows for example from
Algebra, Lemma \ref{algebra-lemma-matrix-left-inverse}.
Let $C^\wedge$ be the $I$-adic completion of $C$. Since
$\NL^\wedge_{C^\wedge/A}$ is the $I$-adic completion of $\NL_{C/A}$
we conclude that multiplication by $a^{2rc}$ is zero on
$\NL^\wedge_{C^\wedge/A}$ as well.

\medskip\noindent
By construction there is a (surjective) map $\psi_N : C/I^NC \to B/I^NB$
sending $x_i$ to $x_i$ and $z$ to $1$. By
Lemma \ref{lemma-get-morphism-principal} (with the roles of $B$ and $C$
reversed) for $N$ large enough we get a map $\varphi : C^\wedge \to B$ which
agrees with $\psi_N$ modulo $I^{N - 2rc}$.

\medskip\noindent
Since $\varphi : C^\wedge \to B$ is surjective modulo $I$ we see that it is
surjective (for example use
Algebra, Lemma \ref{algebra-lemma-completion-generalities}).
By construction and assumption the naive cotangent complexes
$\NL^\wedge_{C^\wedge/A}$ and $\NL^\wedge_{B/A}$
have cohomology annihilated by a fixed power of $a$. Thus the same thing
is true for $\NL^\wedge_{B/C^\wedge}$ by Lemma \ref{lemma-exact-sequence-NL}.
Since $\varphi$ is surjective we conclude that
$\Ker(\varphi)/\Ker(\varphi)^2$ is annihilated by a power of $a$.
The result of the lemma now follows from
More on Algebra, Lemma \ref{more-algebra-lemma-quotient-by-idempotent}.
\end{proof}









\section{Glueing rings along an ideal}
\label{section-approximation}

\noindent
Let $A$ be a Noetherian ring. Let $I \subset A$ be an ideal.
In this section we study $I$-adically complete $A$-algebras
which are, in some vague sense, \'etale over the complement of
$V(I)$ in $\Spec(A)$.

\begin{lemma}
\label{lemma-get-morphism-general}
Let $A$ be a Noetherian ring. Let $I \subset A$ be an ideal.
Let $t$ be the minimal number of generators for $I$.
Let $C$ be a Noetherian $I$-adically complete $A$-algebra.
There exists an integer $d \geq 0$ depending only on
$I \subset A \to C$ with the following property: given
\begin{enumerate}
\item $c \geq 0$ and $B$ in (\ref{equation-C-prime}) such that for $a \in I^c$
multiplication by $a$ on $\NL^\wedge_{B/A}$ is zero in $D(B)$,
\item an integer $n > 2t\max(c, d)$,
\item an $A/I^n$-algebra map $\psi_n : B/I^nB \to C/I^nC$,
\end{enumerate}
there exists a map $\varphi : B \to C$ of $A$-algebras such
that $\psi_n \bmod I^{m - c} = \varphi \bmod I^{m - c}$
with $m = \lfloor \frac{n}{t} \rfloor$.
\end{lemma}

\begin{proof}
We prove this lemma by induction on the number of generators of $I$.
Say $I = (a_1, \ldots, a_t)$. If $t = 0$, then $I = 0$ and there
is nothing to prove. If $t = 1$, then the lemma follows from
Lemma \ref{lemma-get-morphism-principal} because
$2\max(c, d) \geq \max(2c, c + d)$. Assume $t > 1$.

\medskip\noindent
Set $m = \lfloor \frac{n}{t} \rfloor$ as in the lemma.
Set $\bar A = A/(a_t^m)$. Consider the ideal
$\bar I = (\bar a_1, \ldots, \bar a_{t - 1})$ in $\bar A$.
Set $\bar C = C/(a_t^m)$. Note that $\bar C$ is a $\bar I$-adically
complete Noetherian $\bar A$-algebra (use
Algebra, Lemmas \ref{algebra-lemma-completion-tensor} and
\ref{algebra-lemma-change-ideal-completion}).
Let $\bar d$ be the integer for $\bar I \subset \bar A \to \bar C$
which exists by induction hypothesis.

\medskip\noindent
Let $d_1 \geq 0$ be an integer such that $C[a_t^\infty] \cap a_t^{d_1}C = 0$
as in Lemma \ref{lemma-get-morphism-principal} (see discussion following
the lemma and before the proof).

\medskip\noindent
We claim the lemma holds with $d = \max(\bar d, d_1)$.
To see this, let $c, B, n, \psi_n$ be as in the lemma.

\medskip\noindent
Note that $\bar I \subset I\bar A$. Hence by
Lemma \ref{lemma-zero-after-modding-out}
multiplication by an element of $\bar I^c$
on the cotangent complex of $\bar B = B/(a_t^m)$
is zero in $D(\bar B)$. Also, we have
$$
\bar I^{n - m + 1} \supset I^n \bar A
$$
Thus $\psi_n$ gives rise to a map
$$
\bar \psi_{n - m + 1} :
\bar B/\bar I^{n - m + 1}\bar B
\longrightarrow
\bar C/\bar I^{n - m + 1}\bar C
$$
Since $n > 2t\max(c, d)$ and $d \geq \bar d$ we see that
$$
n - m + 1 \geq (t - 1)n/t > 2(t - 1)\max(c, d) \geq 2(t - 1)\max(c, \bar d)
$$
Hence we can find a morphism $\varphi_m : \bar B \to \bar C$
agreeing with $\bar \psi_{n - m + 1}$ modulo the ideal
$\bar I^{m' - c}$ where $m' = \lfloor \frac{n - m + 1}{t - 1} \rfloor$.

\medskip\noindent
Since $m \geq n/t > 2\max(c, d) \geq 2\max(c, d_1) \geq \max(2c, c+ d_1)$,
we can apply Lemma \ref{lemma-get-morphism-principal} for
the ring map $A \to B$ and the ideal $(a_t)$ to
find a morphism $\varphi : B \to C$ agreeing modulo
$a_t^{m - c}$ with $\varphi_m$.

\medskip\noindent
All in all we find $\varphi : B \to C$ which agrees with
$\psi_n$ modulo
$$
(a_t^{m - c}) + (a_1, \ldots, a_{t - 1})^{m' - c}
\subset I^{\min(m - c, m' - c)}
$$
We leave it to the reader to see that
$\min(m - c, m' - c) = m - c$. This concludes the proof.
\end{proof}

\begin{lemma}
\label{lemma-lift-approximation}
Let $A$ be a Noetherian ring and $I \subset A$ an ideal.
Let $J \subset A$ be a nilpotent ideal. Consider a diagram
$$
\xymatrix{
C \ar[r] & C/JC \\
& B_0 \ar[u] \\
A \ar[r] \ar[uu] & A/J \ar[u]
}
$$
whose vertical arrows are of finite type such that
\begin{enumerate}
\item $\Spec(C) \to \Spec(A)$ is \'etale over $\Spec(A) \setminus V(I)$,
\item $\Spec(B_0) \to \Spec(A/J)$ is \'etale over
$\Spec(A/J) \setminus V((I + J)/J)$, and
\item $B_0 \to C/JC$ is \'etale and induces an isomorphism
$B_0/IB_0 = C/(I + J)C$.
\end{enumerate}
Then we can fill in the diagram
$$
\xymatrix{
C \ar[r] & C/JC \\
B \ar[u] \ar[r] & B_0 \ar[u] \\
A \ar[r] \ar[u] & A/J \ar[u]
}
$$
with $A \to B$ of finite type, $B/JB = B_0$, $B \to C$ \'etale, and
$\Spec(B) \to \Spec(A)$ \'etale over $\Spec(A) \setminus V(I)$.
\end{lemma}

\begin{proof}[First proof]
This proof uses algebraic spaces to construct $B$. Set
$X = \Spec(A)$, $X_0 = \Spec(A/J)$, $Y_0 = \Spec(B_0)$,
$Z = \Spec(C)$, $Z_0 = \Spec(C/JC)$. Furthermore, denote
$U \subset X$, $U_0 \subset X_0$, $V_0 \subset Y_0$,
$W \subset Z$, $W_0 \subset Z_0$ the complement of the
vanishing set of $I$. The conditions in the lemma guarantee
that
$$
\xymatrix{
W_0 \ar[r] \ar[d] & Z_0 \ar[d] \\
V_0 \ar[r] & Y_0
}
$$
is an elementary distinguished square. In addition we know that
$W_0 \to U_0$ and $V_0 \to U_0$ are \'etale. The morphism
$X_0 \subset X$ is a finite order thickening.
By the topological invariance of the \'etale site
we can find a unique \'etale morphism $V \to X$
with $V_0 = V \times_X X_0$ and we can lift the given morphism
$W_0 \to V_0$ to a unique morphism $W \to V$.
See More on Morphisms of Spaces, Theorem
\ref{spaces-more-morphisms-theorem-topological-invariance}.
By Pushouts of Spaces, Lemma
\ref{spaces-pushouts-lemma-construct-elementary-distinguished-square}
we can construct an elementary distinguished square
$$
\xymatrix{
W \ar[r] \ar[d] & Z \ar[d] \\
V \ar[r] & Y
}
$$
in the category of algebraic spaces over $X$. Since the base change
of an elementary distinguished square is an elementary distinguished
square (Derived Categories of Spaces, Lemma
\ref{spaces-perfect-lemma-make-more-elementary-distinguished-squares})
and since elementary distinguished squares are pushouts
(Pushouts of Spaces, Lemma
\ref{spaces-pushouts-lemma-elementary-distinguished-square-pushout})
we see that the base change of this diagram by
$X_0 \to X$ gives the previous diagram.
It follows that $Y$ is affine by
Limits of Spaces, Proposition \ref{spaces-limits-proposition-affine}.
Write $Y = \Spec(B)$. Then $B$ fits into the desired diagram and
satisfies all the properties required of it.
\end{proof}

\begin{proof}[Second proof]
This proof uses a little bit of deformation theory to construct $B$.
By induction on the smallest $n$ such that $J^n = 0$ we reduce
to the case $J^2 = 0$. Denote by a subscript zero the base change
of objects to $A_0 = A/J$. Since $J^2 = 0$ we see that $JC$
is a $C_0$-module.

\medskip\noindent
Consider the canonical map
$$
\gamma : J \otimes_{A_0} C_0 \longrightarrow JC
$$
Since $\Spec(C) \to \Spec(A)$ is \'etale over the complement
of $V(I)$ (and hence flat) we see that $\gamma$ is an isomorphism
away from $V(IC_0)$, see
More on Morphisms, Lemma \ref{more-morphisms-lemma-deform}.
In particular, the kernel and cokernel of $\gamma$ are annihilated by
a power of $I$ (use that $C_0$ is Noetherian and that the modules in
question are finite). Observe that $J \otimes_{A_0} C_0 =
(J \otimes_{A_0} B_0) \otimes_{B_0} C_0$. Hence by
More on Algebra, Lemma \ref{more-algebra-lemma-application-formal-glueing}
there exists a unique $B_0$-module homomorphism
$$
c : J \otimes_{A_0} B_0 \to N
$$
with $c \otimes \text{id}_{C_0} = \gamma$ and $\Ker(\gamma) = \Ker(c)$
and $\Coker(\gamma) = \Coker(c)$. Moreover, $N$ is a finite $B_0$-module, see
More on Algebra, Remark \ref{more-algebra-remark-formal-glueing-algebras}.

\medskip\noindent
Choose a presentation $B_0 = A[x_1, \ldots, x_r]/K$. To construct $B$
we try to find the dotted arrow $m$ fitting into the following
pushout diagram
$$
\xymatrix{
0 \ar[r] & N \ar@{..>}[r] & B \ar@{..>}[r] & B_0 \ar[r] & 0 \\
0 \ar[r] & K/K^2 \ar[r] \ar@{..>}[u]_m &
A[x_1, \ldots, x_r]/K^2 \ar[r] \ar@{..>}[u] &
A[x_1, \ldots, x_r]/K \ar@{=}[u] \ar[r] & 0 \\
& J \otimes_{A_0} B_0 \ar[u] \ar@/^2pc/[uu] |!{[lu];[u]}\hole
}
$$
where the curved arrow is the map $c$ constructed above and the
map $J \otimes_{A_0} B_0 \to K/K^2$ is the obvious one.

\medskip\noindent
As $B_0 \to C_0$ is \'etale we can write
$C_0 = B_0[y_1, \ldots, y_r]/(g_{0, 1}, \ldots, g_{0, r})$
such that the determinant of the partial derivatives of the $g_{0, j}$
is invertible in $C_0$, see
Algebra, Lemma \ref{algebra-lemma-etale-standard-smooth}.
We combine this with the chosen presentation of $B_0$ to
get a presentation $C_0 = A[x_1, \ldots, x_r, y_1, \ldots, y_s]/L$.
Choose a lift $\psi : A[x_i, y_j] \to C$
of the map to $C_0$. Then it is the case that $C$ fits into the diagram
$$
\xymatrix{
0 \ar[r] & JC \ar[r] & C \ar[r] & C_0 \ar[r] & 0 \\
0 \ar[r] & L/L^2 \ar[r] \ar[u]_\mu &
A[x_i, y_j]/L^2 \ar[r] \ar[u] &
A[x_i, y_j]/L \ar@{=}[u] \ar[r] & 0 \\
& J \otimes_{A_0} C_0 \ar[u] \ar@/^2pc/[uu] |!{[lu];[u]}\hole
}
$$
where the curved arrow is the map $\gamma$ constructed above and the
map $J \otimes_{A_0} C_0 \to L/L^2$ is the obvious one.
By our choice of presentations and the fact that $C_0$ is
a complete intersection over $B_0$ we have
$$
L/L^2 = K/K^2 \otimes_{B_0} C_0 \oplus \bigoplus C_0 g_j
$$
where $g_j \in L$ is any lift of $g_{0, j}$, see
More on Algebra, Lemma \ref{more-algebra-lemma-transitive-lci-at-end}.

\medskip\noindent
Consider the three term complex
$$
K^\bullet : J \otimes_{A_0} B_0 \to K/K^2 \to \bigoplus B_0 \text{d}x_i
$$
where the second arrow is the differential in the naive cotangent
complex of $B_0$ over $A$ for the given presentation and the last
term is placed in degree $0$. Since
$\Spec(B_0) \to \Spec(A_0)$ is \'etale away from $V(I)$
the cohomology modules of this complex are supported on
$V(IB_0)$. Namely, for $a \in I$ after inverting $a$
we can apply 
More on Algebra, Lemma \ref{more-algebra-lemma-transitive-lci-at-end}
for the ring maps $A_a \to A_{0, a} \to B_{0, a}$
and use that $\NL_{A_{0, a}/A_a} = J_a$ and
$\NL_{B_{0, a}/A_{0, a}} = 0$ (some details omitted).
Hence these cohomology groups are annihilated by a power of $I$.

\medskip\noindent
Similarly, consider the three term complex
$$
L^\bullet : J \otimes_{A_0} C_0 \to L/L^2 \to
\bigoplus C_0 \text{d}x_i \oplus \bigoplus C_0 \text{d}y_j
$$
By our direct sum decomposition of $L/L^2$ above and the fact
that the determinant of the partial derivatives of the $g_{0, j}$
is invertible in $C_0$ we see that the natural map
$K^\bullet \to L^\bullet$ induces a quasi-isomorphism
$$
K^\bullet \otimes_{B_0} C_0 \longrightarrow L^\bullet
$$
Applying
Dualizing Complexes, Lemma \ref{dualizing-lemma-neighbourhood-extensions}
we find that
\begin{equation}
\label{equation-go-down}
\Hom_{D(B_0)}(K^\bullet, E) =
\Hom_{D(C_0)}(L^\bullet, E \otimes_{B_0} C_0)
\end{equation}
for any object $E \in D(B_0)$.

\medskip\noindent
The maps $\text{id}_{J \otimes_{A_0} C_0}$ and $\mu$ define
an element in
$$
\Hom_{D(C_0)}(L^\bullet, (J \otimes_{A_0} C_0 \to  JC))
$$
(the target two term complex is placed in degree $-2$ and $-1$)
such that the composition with the map to $J \otimes_{A_0} C_0[2]$
is the element in $\Hom_{D(C_0)}(L^\bullet, J \otimes_{A_0} C_0[2])$
corresponding to $\text{id}_{J \otimes_{A_0} C_0}$. Picture
$$
\xymatrix{
J \otimes_{A_0} C_0 \ar[r] \ar[d]_{\text{id}_{J \otimes_{A_0} C_0}} &
L/L^2 \ar[r] \ar[d]^\mu &
\bigoplus C_0 \text{d}x_i \oplus \bigoplus C_0 \text{d}y_j \\
J \otimes_{A_0} C_0 \ar[r]^-\gamma &
JC
}
$$
Applying (\ref{equation-go-down}) we obtain a unique element
$$
\xi \in \Hom_{D(B_0)}(K^\bullet, (J \otimes_{A_0} B_0 \to N))
$$
Its composition with the map to $J \otimes_{A_0} B_0[2]$
is the element in $\Hom_{D(C_0)}(K^\bullet, J \otimes_{A_0} B_0[2])$
corresponding to $\text{id}_{J \otimes_{A_0} B_0}$. By
Lemma \ref{lemma-map-out-of-almost-free}
we can find a map of complexes $K^\bullet \to (J \otimes_{A_0} B_0 \to N)$
representing $\xi$ and equal to $\text{id}_{J \otimes_{A_0} B_0}$
in degree $-2$. Denote $m : K/K^2 \to N$ the degree $-1$ part
of this map. Picture
$$
\xymatrix{
J \otimes_{A_0} B_0 \ar[r] \ar[d]_{\text{id}_{J \otimes_{A_0} B_0}} &
K/K^2 \ar[r] \ar[d]^m &
\bigoplus B_0 \text{d}x_i \\
J \otimes_{A_0} B_0 \ar[r]^-c &
N
}
$$
Thus we can use $m$ to create an algebra $B$ by push out as
explained above. However, we may still have to change $m$ a bit to
make sure that $B$ maps to $C$ in the correct manner.

\medskip\noindent
Denote $m \otimes \text{id}_{C_0} \oplus 0 : L/L^2 \to JC$
the map coming from the direct sum decomposition of $L/L^2$
(see above), using that $N \otimes_{B_0} C_0 = JC$, and using $0$
on the second factor. By our choice of $m$ above the maps of complexes
$(\text{id}_{J \otimes_{A_0} C_0}, \mu, 0)$ and
$(\text{id}_{J \otimes_{A_0} C_0}, m \otimes \text{id}_{C_0} \oplus 0, 0)$
define the same element of
$\Hom_{D(C_0)}(L^\bullet, (J \otimes_{A_0} C_0 \to  JC))$.
By Lemma \ref{lemma-map-out-of-almost-free} there exist maps
$h : L^{-1} \to J \otimes_{A_0} C_0$ and $h' : L^0 \to JC$
which define a homotopy between
$(\text{id}_{J \otimes_{A_0} C_0}, \mu, 0)$ and
$(\text{id}_{J \otimes_{A_0} C_0}, m \otimes \text{id}_{C_0} \oplus 0, 0)$.
Picture
$$
\xymatrix{
J \otimes_{A_0} C_0 \ar[r] \ar[d]_{\text{id}_{J \otimes_{A_0} C_0}} &
K/K^2 \otimes_{B_0} C_0 \oplus \bigoplus C_0 g_j \ar@{..>}[ld]_h
\ar[r] \ar@<-1ex>[d]_\mu \ar[d]^{m \otimes \text{id}_{C_0} \oplus 0} &
\bigoplus C_0 \text{d}x_i \oplus \bigoplus C_0 \text{d}y_j \ar@{..>}[ld]_{h'}\\
J \otimes_{A_0} C_0 \ar[r]^-\gamma &
JC
}
$$
Since $h$ precomposed with $d_L^{-2}$ is zero it defines
an element in $\Hom_{D(C_0)}(L^\bullet, J \otimes_{A_0} C_0[1])$
which comes from a unique element $\chi$ of
$\Hom_{D(B_0)}(K^\bullet, J \otimes_{A_0} B_0[1])$
by (\ref{equation-go-down}).
Applying Lemma \ref{lemma-map-out-of-almost-free} again we represent $\chi$
by a map $g : K/K^2 \to J \otimes_{A_0} B_0$.
Then the base change $g \otimes \text{id}_{C_0}$ and $h$ differ
by a homotopy $h'' : L^0 \to J \otimes_{A_0} C$.
Hence if we modify $m$ into $m + c \circ g$, then
we find that $m \otimes \text{id}_{C_0} \oplus 0$ and $\mu$ just differ by
a map $h' : L^0 \to JC$.

\medskip\noindent
Changing our choice of the map $\psi : A[x_i, y_j] \to C$
by sending $x_i$ to $\psi(x_i) + h'(\text{d}x_i)$ and sending
$y_j$ to $\psi(y_j) + h'(\text{d}y_j)$, we find a commutative
diagram
$$
\xymatrix{
N \ar[r] & JC \\
K/K^2 \ar[r] \ar[u]_m & L/L^2 \ar[u]_\mu \\
J \otimes_{A_0} B_0 \ar[u] \ar@/^2pc/[uu]^c \ar[r] &
J \otimes_{A_0} C_0 \ar[u] \ar@/_2pc/[uu]_\gamma
}
$$
At this point we can define $B$ as the pushout in the first
commutative diagram of the proof. The commutativity of the
diagram just displayed, shows that there is an $A$-algebra
map $B \to C$ compatible with the given map $N = JB \to JC$.
As $N \otimes_{B_0} C_0 = JC$ it follows from
More on Morphisms, Lemma \ref{more-morphisms-lemma-deform}
that $B \to C$ is flat.
From this it easily follows that it is \'etale.
We omit the proof of the other properties as they are mostly
self evident at this point.
\end{proof}

\begin{lemma}
\label{lemma-approximate}
Let $A$ be a Noetherian ring. Let $I \subset A$ be an ideal.
Let $B$ be an object of (\ref{equation-C-prime}).
Assume there is an integer $c \geq 0$ such that for $a \in I^c$
multiplication by $a$ on $\NL^\wedge_{B/A}$ is zero in $D(B)$.
Then there exists a finite type $A$-algebra $C$ and an
isomorphism $B \cong C^\wedge$.
\end{lemma}

\noindent
In Section \ref{section-over-G-ring} we will give a simpler
proof of this result in case $A$ is a G-ring.

\begin{proof}
We prove this lemma by induction on the number of generators of $I$.
Say $I = (a_1, \ldots, a_t)$. If $t = 0$, then $I = 0$ and there
is nothing to prove. If $t = 1$, then the lemma follows from
Lemma \ref{lemma-approximate-principal}. Assume $t > 1$.

\medskip\noindent
For any $m \geq 1$ set $\bar A_m = A/(a_t^m)$. Consider the ideal
$\bar I_m = (\bar a_1, \ldots, \bar a_{t - 1})$ in $\bar A_m$.
Let $B_m = B/(a_t^m)$ be the base change of $B$ for the
map $(A, I) \to (\bar A_m, \bar I_m)$, see
(\ref{equation-base-change-to-closed}).
By Lemma \ref{lemma-zero-after-modding-out}
the assumption of the lemma holds for
$\bar I_m \subset \bar A_m \to B_m$.

\medskip\noindent
By induction hypothesis (on $t$) we can find a finite type
$\bar A_m$-algebra $C_m$ and a map $C_m \to B_m$ which induces an
isomorphism $C_m^\wedge \cong B_m$
where the completion is with respect to $\bar I_m$.
By Lemma \ref{lemma-rig-etale} we may assume that
$\Spec(C_m) \to \Spec(\bar A_m)$ is \'etale
over $\Spec(\bar A_m) \setminus V(\bar I_m)$.

\medskip\noindent
We claim that we may choose $A_m \to C_m \to B_m$ as in the previous
paragraph such that moreover there are isomorphisms
$C_m/(a_t^{m - 1}) \to C_{m - 1}$ compatible with the given
$A$-algebra structure and the maps to $B_{m - 1} = B_m/(a_t^{m - 1})$.
Namely, first fix a choice of $A_1 \to C_1 \to B_1$.
Suppose we have found $C_{m - 1} \to C_{m - 2} \to \ldots \to C_1$
with the desired properties.
Note that $C_m/(a_t^{m - 1})$ is \'etale over
$\Spec(\bar A_{m - 1}) \setminus V(\bar I_{m - 1})$.
Hence by Lemma \ref{lemma-fully-faithful-etale-over-complement}
there exists an \'etale extension $C_{m - 1} \to C'_{m - 1}$
which induces an isomorphism modulo $\bar I_{m - 1}$ and an
$\bar A_{m - 1}$-algebra map $C_m/(a_t^{m - 1}) \to C'_{m - 1}$
inducing the isomorphism $B_m/(a_t^{m - 1}) \to B_{m - 1}$ on completions.
Note that $C_m/(a_t^{m - 1}) \to C'_{m - 1}$ is \'etale over the complement
of $V(\bar I_{m - 1})$ by
Morphisms, Lemma \ref{morphisms-lemma-etale-permanence}
and over $V(\bar I_{m - 1})$ induces an isomorphism on completions
hence is \'etale there too (for example by More on Morphisms, Lemma
\ref{more-morphisms-lemma-check-smoothness-on-infinitesimal-nbhds}).
Thus $C_m/(a_t^{m - 1}) \to C'_{m - 1}$ is \'etale. By the
topological invariance of \'etale morphisms
(\'Etale Morphisms, Theorem \ref{etale-theorem-remarkable-equivalence})
there exists an \'etale ring map $C_m \to C'_m$ such that
$C_m/(a_t^{m - 1}) \to C'_{m - 1}$ is isomorphic to
$C_m/(a_t^{m - 1}) \to C'_m/(a_t^{m - 1})$. Observe that the
$\bar I_m$-adic completion of $C'_m$ is equal to the $\bar I_m$-adic
completion of $C_m$, i.e., to $B_m$ (details omitted).
We apply Lemma \ref{lemma-lift-approximation} to the diagram
$$
\xymatrix{
 & C'_m \ar[r] & C'_m/(a_t^{m - 1}) \\
C''_m \ar@{..>}[ru] \ar@{..>}[rr] & & C_{m - 1} \ar[u] \\
 & \bar A_m \ar[r] \ar[uu] \ar@{..>}[lu] & \bar A_{m - 1} \ar[u]
}
$$
to see that there exists a ``lift'' of $C''_m$ of $C_{m - 1}$
to an algebra over $\bar A_m$ with all the desired properties.

\medskip\noindent
By construction $(C_m)$ is an object of the category
(\ref{equation-C}) for the principal ideal $(a_t)$.
Thus the inverse limit $B' = \lim C_m$ is an $(a_t)$-adically
complete $A$-algebra such that $B'/a_t B'$ is of finite type
over $A/(a_t)$, see Lemma \ref{lemma-topologically-finite-type}.
By construction the $I$-adic completion of $B'$ is isomorphic to $B$
(details omitted). Consider the complex $\NL^\wedge_{B'/A}$ constructed
using the $(a_t)$-adic topology. Choosing a presentation for $B'$
(which induces a similar presentation for $B$) the reader immediately
sees that $\NL^\wedge_{B'/A} \otimes_{B'} B = \NL^\wedge_{B/A}$.
Since $a_t \in I$ and since the cohomology modules of
$\NL^\wedge_{B'/A}$ are finite $B'$-modules (hence complete for the
$a_t$-adic topology), we conclude that $a_t^c$ acts as zero on
these cohomologies as the same thing is true by assumption for
$\NL^\wedge_{B/A}$. Thus multiplication by $a_t^{2c}$ is zero
on $\NL^\wedge_{B'/A}$ by Lemma \ref{lemma-zero-in-derived}.
Hence finally, we may apply Lemma \ref{lemma-approximate-principal}
to $(a_t) \subset A \to B'$ to finish the proof.
\end{proof}

\begin{lemma}
\label{lemma-approximate-by-etale-over-complement}
Let $A$ be a Noetherian ring. Let $I \subset A$ be an ideal.
Let $B$ be an $I$-adically complete $A$-algebra with $A/I \to B/IB$
of finite type. The equivalent conditions of
Lemma \ref{lemma-equivalent-with-artin} are also equivalent to
\begin{enumerate}
\item[(5)]
\label{item-algebraize}
there exists a finite type $A$-algebra $C$ with
$\Spec(C) \to \Spec(A)$ is \'etale over $\Spec(A) \setminus V(I)$
such that $B \cong C^\wedge$.
\end{enumerate}
\end{lemma}

\begin{proof}
First, assume conditions (1) -- (4) hold. Then there exists
a finite type $A$-algebra $C$ with such that $B \cong C^\wedge$
by Lemma \ref{lemma-approximate}. In other words, $B_n = C/I^nC$.
The naive cotangent complex $\NL_{C/A}$ is a complex of finite type
$C$-modules and hence $H^{-1}$ and $H^0$ are finite $C$-modules.
By assumption there exists a $c \geq 0$ such that
$H^{-1}/I^nH^{-1}$ and $H^0/I^nH^0$ are annihilated by $I^c$
for some $n$. By Nakayama's lemma this means that
$I^cH^{-1}$ and $I^cH^0$ are annihilated by an element of the
form $f = 1 + x$ with $x \in IC$. After inverting $f$
(which does not change the quotients $B_n = C/I^nC$)
we see that $\NL_{C/A}$ has cohomology annihilated by $I^c$. Thus
$A \to C$ is \'etale at any prime of $C$ not lying over $V(I)$
by the definition of \'etale ring maps, see
Algebra, Definition \ref{algebra-definition-etale}.

\medskip\noindent
Conversely, assume that $A \to C$ of finite type is given such that
$\Spec(C) \to \Spec(A)$ is \'etale over $\Spec(A) \setminus V(I)$.
Then for every $a \in I$ there exists an $c \geq 0$ such that
multiplication by $a^c$ is zero $\NL_{C/A}$.
Since $\NL^\wedge_{C^\wedge/A}$ is the derived completion of
$\NL_{C/A}$ (see Lemma \ref{lemma-NL-is-limit}) it follows that
$B = C^\wedge$ satisfies the equivalent conditions of
Lemma \ref{lemma-equivalent-with-artin}.
\end{proof}








\section{In case the base ring is a G-ring}
\label{section-over-G-ring}

\noindent
If the base ring $A$ is a Noetherian G-ring, then some of the material
above simplifies somewhat and we obtain some additional results.

\begin{proof}[Proof of Lemma \ref{lemma-approximate} in case $A$ is a G-ring]
This proof is easier in that it does not depend on the somewhat
delicate deformation theory argument given in the proof of
Lemma \ref{lemma-lift-approximation}, but of course it requires
a very strong assumption on the Noetherian ring $A$.

\medskip\noindent
Choose a presentation $B = A[x_1, \ldots, x_r]^\wedge/J$.
Choose generators $g_1, \ldots, g_m \in J$.
Choose generators $k_1, \ldots, k_t$ of the module
of relations between $g_1, \ldots, g_m$, i.e., such that
$$
(A[x_1, \ldots, x_r]^\wedge)^{\oplus t} \xrightarrow{k_1, \ldots, k_t}
(A[x_1, \ldots, x_r]^\wedge)^{\oplus m} \xrightarrow{g_1, \ldots, g_m}
A[x_1, \ldots, x_r]^\wedge
$$
is exact in the middle. Write $k_i = (k_{i1}, \ldots, k_{im})$ so that we have
\begin{equation}
\label{equation-relations-straight-up}
\sum k_{ij}g_j = 0
\end{equation}
for $i = 1, \ldots, t$.
Let $I^c = (a_1, \ldots, a_s)$. For each $l \in \{1, \ldots, s\}$
we know that multiplication by $a_l$ on $\NL^\wedge_{B/A}$ is zero
in $D(B)$. By Lemma \ref{lemma-map-out-of-almost-free} we can find a map
$\alpha_l : \bigoplus B\text{d}x_i \to J/J^2$ such that
$\text{d} \circ \alpha_l$ and $\alpha_l \circ \text{d}$ are both
multiplication by $a_l$. Pick an element $f_{l, i} \in J$ whose
class modulo $J^2$ is equal to $\alpha_l(\text{d}x_i)$.
Then we have for all $l = 1, \ldots, s$ and $i = 1, \ldots, r$ that
\begin{equation}
\label{equation-derivatives}
\sum\nolimits_{i'} (\partial f_{l, i}/ \partial x_{i'}) \text{d}x_{i'} =
a_l \text{d}x_i + \sum h_{l, i}^{j', i'} g_{j'} \text{d}x_{i'}
\end{equation}
for some $h_{l, i}^{j', i'} \in A[x_1, \ldots, x_r]^\wedge$.
We also have for $j = 1, \ldots, m$ and $l = 1, \ldots, s$ that
\begin{equation}
\label{equation-ci}
a_l g_j = \sum h_{l, j}^if_{l, i} + \sum h_{l, j}^{j', j''}g_{j'} g_{j''}
\end{equation}
for some $h_{l, j}^i$ and $h_{l, j}^{j', j''}$ in
$A[x_1, \ldots, x_r]^\wedge$. Of course, since $f_{l, i} \in J$
we can write for $l = 1, \ldots, s$ and $i = 1, \ldots, r$
\begin{equation}
\label{equation-in-ideal}
f_{l, i} = \sum h_{l, i}^jg_j
\end{equation}
for some $h_{l, i}^j$ in $A[x_1, \ldots, x_r]^\wedge$.

\medskip\noindent
Let $A[x_1, \ldots, x_r]^h$ be the henselization of the
pair $(A[x_1, \ldots, x_r], IA[x_1, \ldots, x_r])$, see
More on Algebra, Lemma \ref{more-algebra-lemma-henselization}.
Since $A$ is a Noetherian G-ring, so is $A[x_1, \ldots, x_r]$, see
More on Algebra, Proposition
\ref{more-algebra-proposition-finite-type-over-G-ring}.
Hence we have approximation for the map
$A[x_1, \ldots, x_r]^h \to A[x_1, \ldots, x_r]^\wedge$
with respect to the ideal generated by $I$, see
Smoothing Ring Maps, Lemma \ref{smoothing-lemma-henselian-pair}.
Choose a large integer $M$. Choose
$$
G_j, K_{ij}, F_{l, i}, H_{l, j}^i, H_{l, j}^{j', j''}, H_{l, i}^j
\in A[x_1, \ldots, x_r]^h
$$
such that analogues of equations (\ref{equation-relations-straight-up}),
(\ref{equation-ci}), and (\ref{equation-in-ideal})
hold for these elements in $A[x_1, \ldots, x_r]^h$, i.e.,
$$
\sum K_{ij}G_j = 0,\quad
a_l G_j = \sum H_{l, j}^iF_{l, i} +
\sum H_{l, j}^{j', j''} G_{j'} G_{j''},\quad
F_{l, i} = \sum H_{l, i}^j G_j
$$
and such that we have
$$
G_j - g_j, K_{ij} - k_{ij}, F_{l, i} - f_{l, i},
H_{l, j}^i - h_{l, j}^i, H_{l, j}^{j', j''} - h_{l, j}^{j', j''},
H_{l, i}^j - h_{l, i}^j
\in I^M A[x_1, \ldots, x_r]^h
$$
where we take liberty of thinking of $A[x_1, \ldots, x_r]^h$ as a
subring of $A[x_1, \ldots, x_r]^\wedge$.
Note that we cannot guarantee that the analogue of
(\ref{equation-derivatives}) holds
in $A[x_1, \ldots, x_r]^h$, because it is not a polynomial equation.
But since taking partial derivatives is $A$-linear, we do get
the analogue modulo $I^M$. More precisely, we see that
\begin{equation}
\label{equation-derivatives-analogue}
\sum\nolimits_{i'} (\partial F_{l, i}/ \partial x_{i'}) \text{d}x_{i'}
- a_l \text{d}x_i - \sum h_{l, i}^{j', i'} G_{j'} \text{d}x_{i'}
\in I^MA[x_1, \ldots, x_r]^\wedge
\end{equation}
for $l = 1, \ldots, s$ and $i = 1, \ldots, r$.

\medskip\noindent
With these choices, consider the ring
$$
C^h = A[x_1, \ldots, x_r]^h/(G_1, \ldots, G_r)
$$
and denote $C^\wedge$ its $I$-adic completion, namely
$$
C^\wedge = A[x_1, \ldots, x_r]^\wedge/J',\quad
J' = (G_1, \ldots, G_r)A[x_1, \ldots, x_r]^\wedge
$$
In the following paragraphs we establish the fact that $C^\wedge$
is isomorphic to $B$. Then in the final paragraph we deal with
show that $C^h$ comes from a finite type algebra
over $A$ as in the statement of the lemma.

\medskip\noindent
First consider the cokernel
$$
\Omega = \Coker(J'/(J')^2 \longrightarrow \bigoplus C^\wedge \text{d}x_i)
$$
This $C^\wedge$ module is generated by the images of the elements
$\text{d}x_i$. Since $F_{l, i} \in J'$ by the analogue of
(\ref{equation-in-ideal}) we see from
(\ref{equation-derivatives-analogue}) we see
that $a_l \text{d}x_i \in I^M\Omega$. As $I^c = (a_l)$ we see that
$I^c \Omega \subset I^M \Omega$. Since $M > c$ we conclude that
$I^c \Omega = 0$ by Algebra, Lemma \ref{algebra-lemma-NAK}.

\medskip\noindent
Next, consider the kernel
$$
H_1 = \Ker(J'/(J')^2 \longrightarrow \bigoplus C^\wedge \text{d}x_i)
$$
By the analogue of (\ref{equation-ci}) we see that
$a_l J' \subset (F_{l, i}) + (J')^2$. On the other hand, the
determinant $\Delta_l$ of the matrix $(\partial F_{l, i}/ \partial x_{i'})$
satisfies $\Delta_l = a_l^r \bmod I^M C^\wedge$ by
(\ref{equation-derivatives-analogue}). It follows that
$a_l^{r + 1} H_1 \subset I^M H_1$ (some details omitted; use
Algebra, Lemma \ref{algebra-lemma-matrix-left-inverse}).
Now $(a_1^{r + 1}, \ldots, a_s^{r + 1}) \supset I^{(sr + 1)c}$.
Hence $I^{(sr + 1)c}H_1 \subset I^M H_1$ and since $M > (sr + 1)c$
we conclude that $I^{(sr + 1)c}H_1 = 0$.

\medskip\noindent
By Lemma \ref{lemma-zero-in-derived}
we conclude that multiplication by an element
of $I^{2(sr + 1)c}$ on $\NL^\wedge_{C^\wedge/A}$ is zero
(note that the bound does not depend on $M$ or the choice
of the approximation, as long as $M$ is large enough).
Since $G_j - g_j$ is in the ideal generated by $I^M$
we see that there is an isomorphism
$$
\psi_M : C^\wedge/I^MC^\wedge \to B/I^MB
$$
As $M$ is large enough we can use
Lemma \ref{lemma-get-morphism-general}
with $d = d(I \subset A \to B)$,
with $C^\wedge$ playing the role of $B$,
with $2(rs + 1)c$ instead of $c$,
to find a morphism
$$
\psi : C^\wedge \longrightarrow B
$$
which agrees with $\psi_M$ modulo $I^{q - 2(rs + 1)c}$ where
$q$ is the quotient of $M$ by the number of generators of $I$.
We claim $\psi$ is an isomorphism. Since $C^\wedge$ and $B$
are $I$-adically complete the map $\psi$ is surjective
because it is surjective modulo $I$ (see
Algebra, Lemma \ref{algebra-lemma-completion-generalities}).
On the other hand, as $M$ is large enough we see that
$$
\text{Gr}_I(C^\wedge) \cong \text{Gr}_I(B)
$$
as graded $\text{Gr}_I(A[x_1, \ldots, x_r]^\wedge)$-modules
by More on Algebra, Lemma \ref{more-algebra-lemma-approximate-complex-graded}.
Since $\psi$ is compatible with this isomorphism as it
agrees with $\psi_M$ modulo $I$, this means that $\text{Gr}_I(\psi)$ is an
isomorphism. As $C^\wedge$ and $B$ are
$I$-adically complete, it follows that $\psi$ is an isomorphism.

\medskip\noindent
This paragraph serves to deal with the issue that $C^h$
is not of finite type over $A$. Namely, the ring
$A[x_1, \ldots, x_r]^h$ is a filtered colimit of
\'etale $A[x_1, \ldots, x_r]$ algebras $A'$ such that
$A/I[x_1, \ldots, x_r] \to A'/IA'$ is an isomorphism
(see proof of More on Algebra, Lemma \ref{more-algebra-lemma-henselization}).
Pick an $A'$ such that $G_1, \ldots, G_m$ are the
images of $G'_1, \ldots, G'_m \in A'$.
Setting $C = A'/(G'_1, \ldots, G'_m)$ we get the finite
type algebra we were looking for.
\end{proof}

\noindent
The following lemma isn't true in general if $A$ is not a G-ring
but just Noetherian. Namely, if $(A, \mathfrak m)$ is local
and $I = \mathfrak m$, then the lemma is equivalent to 
Artin approximation for $A^h$ (as in
Smoothing Ring Maps, Theorem \ref{smoothing-theorem-approximation-property})
which does not hold for every Noetherian local ring.

\begin{lemma}
\label{lemma-fully-faithfulness}
Let $A$ be a Noetherian G-ring. Let $I \subset A$ be an ideal.
Let $B, C$ be finite type $A$-algebras. For any $A$-algebra map
$\varphi : B^\wedge \to C^\wedge$ of $I$-adic completions and any
$N \geq 1$ there exist
\begin{enumerate}
\item an \'etale ring map $C \to C'$ which induces
an isomorphism $C/IC \to C'/IC'$,
\item an $A$-algebra map $\varphi : B \to C'$
\end{enumerate}
such that $\varphi$ and $\psi$ agree modulo $I^N$
into $C^\wedge = (C')^\wedge$.
\end{lemma}

\begin{proof}
The statement of the lemma makes sense as $C \to C'$ is flat
(Algebra, Lemma \ref{algebra-lemma-etale}) hence induces an isomorphism
$C/I^nC \to C'/I^nC'$ for all $n$
(More on Algebra, Lemma \ref{more-algebra-lemma-neighbourhood-isomorphism})
and hence an isomorphism on completions.
Let $C^h$ be the henselization of the pair $(C, IC)$, see
More on Algebra, Lemma \ref{more-algebra-lemma-henselization}.
Then $C^h$ is the filtered colimit of the algebras $C'$
and the maps
$C \to C' \to C^h$ induce isomorphism on completions (More on Algebra,
Lemma \ref{more-algebra-lemma-henselization-Noetherian-pair}).
Thus it suffices to prove there exists an $A$-algebra map
$B \to C^h$ which is congruent to $\psi$ modulo $I^N$.
Write $B = A[x_1, \ldots, x_n]/(f_1, \ldots, f_m)$.
The ring map $\psi$ corresponds to elements
$\hat c_1, \ldots, \hat c_n \in C^\wedge$ with
$f_j(\hat c_1, \ldots, \hat c_n) = 0$ for $j = 1, \ldots, m$.
Namely, as $A$ is a Noetherian G-ring, so is $C$, see
More on Algebra, Proposition
\ref{more-algebra-proposition-finite-type-over-G-ring}.
Thus Smoothing Ring Maps,
Lemma \ref{smoothing-lemma-henselian-pair}
applies to give elements $c_1, \ldots, c_n \in C^h$ such
that $f_j(c_1, \ldots, c_n) = 0$ for $j = 1, \ldots, m$
and such that $\hat c_i - c_i \in I^NC^h$.
This determines the map $B \to C^h$ as desired.
\end{proof}








\section{Rig-surjective morphisms}
\label{section-rig-surjective}

\noindent
For morphisms locally of finite type between locally Noetherian formal
algebraic spaces a definition borrowed from \cite{ArtinII} can be used. See
Remark \ref{remark-rig-surjective-more-general} for a discussion
of what to do in more general cases.

\begin{definition}
\label{definition-rig-surjective}
Let $S$ be a scheme. Let $f : X \to Y$ be a morphism of formal
algebraic spaces over $S$. Assume that $X$ and $Y$ are locally
Noetherian and that $f$ is locally of finite type. We say
$f$ is {\it rig-surjective} if for every solid diagram
$$
\xymatrix{
\text{Spf}(R') \ar@{..>}[r] \ar@{..>}[d] & X \ar[d]^f \\
\text{Spf}(R) \ar[r]^-p & Y
}
$$
where $R$ is a complete discrete valuation ring and where
$p$ is an adic morphism there exists an
extension of complete discrete valuation rings $R \subset R'$
and a morphism $\text{Spf}(R') \to X$ making the displayed diagram commute.
\end{definition}

\noindent
We prove a few lemmas to explain what this means.

\begin{lemma}
\label{lemma-composition-rig-surjective}
\begin{slogan}
Rig-surjectivity of locally finite type morphisms is preserved under
composition
\end{slogan}
Let $S$ be a scheme. Let $f : X \to Y$ and $g : Y \to Z$ be morphisms of formal
algebraic spaces over $S$. Assume $X$, $Y$, $Z$ are locally Noetherian and
$f$ and $g$ locally of finite type. Then if $f$ and $g$ are rig-surjective,
so is $g \circ f$.
\end{lemma}

\begin{proof}
Follows in a straightforward manner from the definitions
(and Formal Spaces, Lemma \ref{formal-spaces-lemma-composition-finite-type}).
\end{proof}

\begin{lemma}
\label{lemma-base-change-rig-surjective}
Let $S$ be a scheme. Let $f : X \to Y$ and $Z \to Y$ be morphisms
of formal algebraic spaces over $S$. Assume $X$, $Y$, $Z$ are locally
Noetherian and $f$ and $g$ locally of finite type. If $f$ is
rig-surjective, then the base change $Z \times_Y X \to Z$ is too.
\end{lemma}

\begin{proof}
Follows in a straightforward manner from the definitions (and
Formal Spaces, Lemmas \ref{formal-spaces-lemma-fibre-product-Noetherian} and
\ref{formal-spaces-lemma-base-change-finite-type}).
\end{proof}

\begin{lemma}
\label{lemma-permanence-rig-surjective}
Let $S$ be a scheme. Let $f : X \to Y$ and $g : Y \to Z$ be morphisms of
formal algebraic spaces over $S$. Assume $X$, $Y$, $Z$ locally Noetherian
and $f$ and $g$ locally of finite type. If $g \circ f : X \to Z$
is rig-surjective, so is $g : Y \to Z$.
\end{lemma}

\begin{proof}
Immediate from the definition.
\end{proof}

\begin{lemma}
\label{lemma-etale-covering-rig-surjective}
Let $S$ be a scheme. Let $f : X \to Y$ be a morphism of formal algebraic
spaces which is representable by algebraic spaces, \'etale, and surjective.
Assume $X$ and $Y$ locally Noetherian. Then $f$ is rig-surjective.
\end{lemma}

\begin{proof}
Let $p : \text{Spf}(R) \to Y$ be an adic morphism where $R$ is a complete
discrete valuation ring. Let $Z = \text{Spf}(R) \times_Y X$. Then
$Z \to \text{Spf}(R)$ is representable by algebraic spaces, \'etale, and
surjective. Hence $Z$ is nonempty. Pick a nonempty affine formal algebraic
space $V$ and an \'etale morphism $V \to Z$ (possible by our definitions).
Then $V \to \text{Spf}(R)$ corresponds to $R \to A^\wedge$ where
$R \to A$ is an \'etale ring map, see Formal Spaces, Lemma
\ref{formal-spaces-lemma-etale}. Since $A^\wedge \not = 0$
(as $V \not = \emptyset$) we can find a maximal ideal $\mathfrak m$
of $A$ lying over $\mathfrak m_R$. Then $A_\mathfrak m$ is a discrete
valuation ring (More on Algebra, Lemma
\ref{more-algebra-lemma-Dedekind-etale-extension}).
Then $R' = A_\mathfrak m^\wedge$ is a complete discrete valuation ring
(More on Algebra, Lemma \ref{more-algebra-lemma-completion-dvr}).
Applying Formal Spaces, Lemma
\ref{formal-spaces-lemma-morphism-between-formal-spectra}.
we find the desired morphism $\text{Spf}(R') \to V \to Z \to X$.
\end{proof}

\begin{remark}
\label{remark-upshot}
Let $S$ be a scheme. Let $f : X \to Y$ be a morphism of
locally Noetherian formal algebraic spaces which is locally of finite type.
The upshot of the lemmas above is that we may check whether
$f : X \to Y$ is rig-surjective, \'etale locally on $Y$. For example,
suppose that $\{Y_i \to Y\}$ is a covering as in
Formal Spaces, Definition \ref{formal-spaces-definition-formal-algebraic-space}.
Then $f$ is rig-surjective if and only if $f_i : X \times_Y Y_i \to Y_i$ is
rig-surjective. Namely, if $f$ is rig-surjective, so is any base change
(Lemma \ref{lemma-base-change-rig-surjective}).
Conversely, if all $f_i$ are rig-surjective, so is
$\coprod f_i : \coprod X \times_Y Y_i \to \coprod Y_i$.
By Lemma \ref{lemma-etale-covering-rig-surjective}
the morphism $\coprod Y_i \to Y$ is rig-surjective.
Hence $\coprod X \times_Y Y_i \to Y$ is rig-surjective
(Lemma \ref{lemma-composition-rig-surjective}).
Since this morphism factors through $X \to Y$ we see that $X \to Y$
is rig-surjective by Lemma \ref{lemma-permanence-rig-surjective}.
\end{remark}

\begin{lemma}
\label{lemma-completion-proper-surjective-rig-surjective}
Let $S$ be a scheme. Let $f : X \to Y$ be a proper surjective morphism
of locally Noetherian algebraic spaces over $S$. Let $T \subset |Y|$
be a closed subset and let $T' = |f|^{-1}(T) \subset |X|$.
Then $X_{/T'} \to Y_{/T}$ is rig-surjective.
\end{lemma}

\begin{proof}
The statement makes sense by
Formal Spaces, Lemmas \ref{formal-spaces-lemma-formal-completion-types} and
\ref{formal-spaces-lemma-map-completions-finite-type}.
Let $Y_j \to Y$ be a jointly surjective family of \'etale morphism
where $Y_j$ is an affine scheme for each $j$.
Denote $T_j \subset Y_j$ the inverse image of $T$.
Then $\{(Y_j)_{/T_j} \to Y_{/T}\}$ is a covering as in
Formal Spaces, Definition \ref{formal-spaces-definition-formal-algebraic-space}.
Moreover, setting $X_j = Y_j \times_Y X$ and $T'_j \subset |X_j|$
the inverse image of $T$, we have
$$
(X_j)_{/T'_j} = (Y_j)_{/T_j} \times_{(Y_{/T})} X_{/T'}
$$
By the discussion in Remark \ref{remark-upshot} we reduce to the case where
$Y$ is an affine Noetherian scheme treated in the next paragraph.

\medskip\noindent
Assume $Y = \Spec(A)$ where $A$ is a Noetherian ring. This implies that
$Y_{/T} = \text{Spf}(A^\wedge)$ where $A^\wedge$ is the $I$-adic completion
of $A$ for some ideal $I \subset A$. Let
$p : \text{Spf}(R) \to \text{Spf}(A^\wedge)$
be an adic morphism where $R$ is a complete discrete valuation ring.
Let $K$ be the field of fractions of $R$.
Consider the composition $A \to A^\wedge \to R$.
Since $X \to Y$ is surjective, the fibre $X_K = \Spec(K) \times_Y X$
is nonempty. Thus we may choose an affine scheme $U$ and an \'etale
morphism $U \to X$ such that $U_K$ is nonempty.
Let $u \in U_K$ be a closed point (possible as $U_K$ is affine). By
Morphisms, Lemma \ref{morphisms-lemma-closed-point-fibre-locally-finite-type}
the residue field $L = \kappa(u)$ is a finite extension of $K$. Let
$R' \subset L$ be the integral closure of $R$ in $L$. By
More on Algebra, Remark \ref{more-algebra-remark-finite-separable-extension}
we see that $R'$ is a discrete valuation ring.
Because $X \to Y$ is proper we see that the given morphism
$\Spec(L) = u \to U_K \to X_K \to X$ extends to a morphism
$\Spec(R') \to X$ over the given morphism $\Spec(R) \to Y$
(Morphisms of Spaces,
Lemma \ref{spaces-morphisms-lemma-characterize-proper}).
By commutativity of the diagram the induced morphisms
$\Spec(R'/\mathfrak m_{R'}^n) \to X$ are points of $X_{/T'}$
and we find
$$
\text{Spf}((R')^\wedge) = \colim \Spec(R'/\mathfrak m_{R'}^n)
\longrightarrow X_{/T'}
$$
as desired (note that $(R')^\wedge$ is a complete discrete valuation ring
by More on Algebra, Lemma \ref{more-algebra-lemma-completion-dvr};
in fact in the current situation $R' = (R')^\wedge$ but we do not
need this).
\end{proof}

\begin{lemma}
\label{lemma-faithfully-flat-rig-surjective}
Let $A$ be a Noetherian ring complete with respect to an ideal $I$.
Let $B$ be an $I$-adically complete $A$-algebra.
If $A/I^n \to B/I^nB$ is of finite type and flat for all $n$ and
faithfully flat for $n = 1$, then $\text{Spf}(B) \to \text{Spf}(A)$
is rig-surjective.
\end{lemma}

\begin{proof}
We will use without further mention that morphisms between formal spectra
are given by continuous maps between the corresponding topological rings, see
Formal Spaces, Lemma \ref{formal-spaces-lemma-morphism-between-formal-spectra}.
Let $\varphi : A \to R$ be a continuous map into a complete discrete
valuation ring $A$. This implies that $\varphi(I) \subset \mathfrak m_R$.
On the other hand, since we only need to produce the lift
$\varphi' : B' \to R'$ in the case that $\varphi$ corresponds to an adic
morphism, we may assume that $\varphi(I) \not = 0$. Thus we may consider
the base change $C = B \widehat{\otimes}_A R$, see
Remark \ref{remark-base-change} for example.
Then $C$ is an $\mathfrak m_R$-adically complete $R$-algebra
such that $C/\mathfrak m_R^n C$ is of finite type and flat over
$R/\mathfrak m_R^n$ and such that $C/\mathfrak m_R C$ is nonzero.
Pick any maximal ideal $\mathfrak m \subset C$ lying over
$\mathfrak m_R$. By flatness (which implies going down) we see that
$\Spec(C_\mathfrak m) \setminus V(\mathfrak m_R C_\mathfrak m)$
is a nonempty open. Hence
We can pick a prime $\mathfrak q \subset \mathfrak m$
such that $\mathfrak q$ defines a closed point of
$\Spec(C_\mathfrak m) \setminus \{\mathfrak m\}$ and such that
$\mathfrak q \not \in V(IC_\mathfrak m)$, see
Properties, Lemma \ref{properties-lemma-complement-closed-point-Jacobson}.
Then $C/\mathfrak q$ is a dimension $1$-local domain and we can find
$C/\mathfrak q \subset R'$ with $R'$ a discrete valuation ring
(Algebra, Lemma \ref{algebra-lemma-exists-dvr}).
By construction $\mathfrak m_R R' \subset \mathfrak m_{R'}$
and we see that $C \to R'$ extends to a continuous map
$C \to (R')^\wedge$ (in fact we can pick $R'$ such that
$R' = (R')^\wedge$ in our current situation but we do not need this).
Since the completion of a discrete valuation ring is a discrete
valuation ring, we see that the assumption gives a commutative
diagram of rings
$$
\xymatrix{
(R')^\wedge & C \ar[l] & B \ar[l] \\
R \ar[u] & R \ar[l] \ar[u] & A \ar[l] \ar[u]
}
$$
which gives the desired lift.
\end{proof}

\begin{lemma}
\label{lemma-flat-rig-surjective}
Let $A$ be a Noetherian ring complete with respect to an ideal $I$.
Let $B$ be an $I$-adically complete $A$-algebra. Assume that
\begin{enumerate}
\item the $I$-torsion in $A$ is $0$,
\item $A/I^n \to B/I^nB$ is flat and of finite type for all $n$.
\end{enumerate}
Then $\text{Spf}(B) \to \text{Spf}(A)$ is rig-surjective if and only
if $A/I \to B/IB$ is faithfully flat.
\end{lemma}

\begin{proof}
Faithful flatness implies rig-surjectivity by
Lemma \ref{lemma-faithfully-flat-rig-surjective}.
To prove the converse we will use without further mention that the
vanishing of $I$-torsion is equivalent to the vanishing of $I$-power torsion
(More on Algebra, Lemma \ref{more-algebra-lemma-torsion-free}).
We will also use without further mention that morphisms between
formal spectra are given by continuous maps between the corresponding
topological rings, see
Formal Spaces, Lemma \ref{formal-spaces-lemma-morphism-between-formal-spectra}.

\medskip\noindent
Assume $\text{Spf}(B) \to \text{Spf}(A)$ is rig-surjective.
Choose a maximal ideal $I \subset \mathfrak m \subset A$.
The open $U = \Spec(A_\mathfrak m) \setminus V(I_\mathfrak m)$
of $\Spec(A_\mathfrak m)$ is nonempty as the $I_\mathfrak m$-torsion of
$A_\mathfrak m$ is zero
(use Algebra, Lemma \ref{algebra-lemma-Noetherian-power-ideal-kills-module}).
Thus we can find a prime $\mathfrak q \subset A_\mathfrak m$ which defines
a point of $U$ (i.e., $IA_\mathfrak m \not \subset \mathfrak q$)
and which corresponds to a closed point
of $\Spec(A_\mathfrak m) \setminus \{\mathfrak m\}$, see
Properties, Lemma \ref{properties-lemma-complement-closed-point-Jacobson}.
Then $A_\mathfrak m/\mathfrak q$ is a dimension $1$ local domain.
Thus we can find an injective local homomorphism of local rings
$A_\mathfrak m/\mathfrak q \subset R$ where $R$ is a discrete valuation ring
(Algebra, Lemma \ref{algebra-lemma-exists-dvr}).
By construction $IR \subset \mathfrak m_R$ and we see that
$A \to R$ extends to a continuous map $A \to R^\wedge$.
Since the completion of a discrete valuation ring is a discrete
valuation ring, we see that the assumption gives a commutative
diagram of rings
$$
\xymatrix{
R' & B \ar[l] \\
R^\wedge \ar[u] & A \ar[l] \ar[u]
}
$$
Thus we find a prime ideal of $B$ lying over $\mathfrak m$. It follows
that $\Spec(B/IB) \to \Spec(A/I)$ is surjective, whence $A/I \to B/IB$
is faithfully flat
(Algebra, Lemma \ref{algebra-lemma-ff-rings}).
\end{proof}

\begin{remark}
\label{remark-rig-surjective-more-general}
The condition as formulated in Definition \ref{definition-rig-surjective}
is not right for morphisms of locally adic* formal algebraic spaces.
For example, if $A = (\bigcup_{n \geq 1} k[t^{1/n}])^\wedge$
where the completion is the $t$-adic completion, then
there are no adic morphisms $\text{Spf}(R) \to \text{Spf}(A)$
where $R$ is a complete discrete valuation ring.
Thus any morphism $X \to \text{Spf}(A)$ would be rig-surjective,
but since $A$ is a domain and $t \in A$ is not zero, we want to
think of $A$ as having at least one ``rig-point'', and we do not
want to allow $X = \emptyset$. To cover this
particular case, one can consider adic morphisms
$$
\text{Spf}(R) \longrightarrow Y
$$
where $R$ is a valuation ring complete with respect to a principal
ideal $J$ whose radical is $\mathfrak m_R = \sqrt{J}$.
In this case the value group of $R$ can be embedded into
$(\mathbf{R}, +)$ and one obtains the point of view used by
Berkovich in defining an analytic space associated to $Y$, see
\cite{Berkovich}. Another approach is championed by Huber. In his theory,
one drops the hypothesis that $\Spec(R/J)$ is a singleton, see
\cite{Huber-continuous-valuations}.
\end{remark}

\begin{lemma}
\label{lemma-monomorphism-rig-surjective}
Let $S$ be a scheme. Let $f : X \to Y$ be a morphism of formal algebraic
spaces. Assume $X$ and $Y$ are locally Noetherian, $f$ locally of finite
type, and $f$ a monomorphism. Then $f$ is rig surjective if and only if
every adic morphism $\text{Spf}(R) \to Y$ where $R$ is a complete discrete
valuation ring factors through $X$.
\end{lemma}

\begin{proof}
One direction is trivial. For the other, suppose that $\text{Spf}(R) \to Y$
is an adic morphism such that there exists an extension of complete
discrete valuation rings $R \subset R'$ with
$\text{Spf}(R') \to \text{Spf}(R) \to X$ factoring through $Y$. Then
$\Spec(R'/\mathfrak m_R^n R') \to \Spec(R/\mathfrak m_R^n)$ is surjective
and flat, hence the morphisms $\Spec(R/\mathfrak m_R^n) \to X$ factor
through $X$ as $X$ satisfies the sheaf condition for fpqc coverings, see
Formal Spaces, Lemma \ref{formal-spaces-lemma-sheaf-fpqc}.
In other words, $\text{Spf}(R) \to Y$ factors through $X$.
\end{proof}





\section{Algebraization}
\label{section-algebraization}

\noindent
In this section we prove a generalization of the result on dilatations
from the paper of Artin \cite{ArtinII}. We first reformulate the algebra
results proved above into the language of formal algebraic spaces.

\medskip\noindent
Let $S$ be a scheme. Let $V$ be a locally Noetherian formal algebraic space
over $S$. We denote $\mathcal{C}_V$ the category of formal algebraic
spaces $W$ over $V$ such that the structure morphism $W \to V$ is rig-\'etale.

\medskip\noindent
Let $S$ be a scheme. Let $X$ be an algebraic space over $S$.
Let $T \subset |X|$ be a closed subset. Recall that $X_{/T}$ denotes
the formal completion of $X$ along $T$, see 
Formal Spaces, Section \ref{formal-spaces-section-completion}.
More generally, for any algebraic space $Y$ over $X$ we
denote $Y_{/T}$ the completion of $Y$ along the inverse
image of $T$ in $|Y|$, so that $Y_{/T}$ is a formal algebraic space
over $X_{/T}$.

\begin{lemma}
\label{lemma-etale-gives-rig-etale}
Let $S$ be a scheme. Let $X$ be a locally Noetherian algebraic space over $S$.
Let $T \subset |X|$ be a closed subset. If $Y \to X$ is morphism of
algebraic spaces which is locally of finite type and \'etale over
$X \setminus T$, then $Y_{/T} \to X_{/T}$ is rig-\'etale, i.e., $Y_{/T}$
is an object of $\mathcal{C}_{X_{/T}}$ defined above.
\end{lemma}

\begin{proof}
Choose a surjective \'etale morphism $U \to X$ with $U = \coprod U_i$
a disjoint union of affine schemes, see Properties of Spaces, Lemma
\ref{spaces-properties-lemma-cover-by-union-affines}.
For each $i$ choose a surjective \'etale morphism $V_i \to Y \times_X U_i$
where $V_i = \coprod V_{ij}$ is a disjoint union of affines.
Write $U_i = \Spec(A_i)$ and $V_{ij} = \Spec(B_{ij})$.
Let $I_i \subset A_i$ be an ideal cutting out the inverse image
of $T$ in $U_i$. Then we may apply
Lemma \ref{lemma-rig-etale}
to see that the map of $I_i$-adic completions
$A_i^\wedge \to B_{ij}^\wedge$ has the property
$P$ of Lemma \ref{lemma-rig-etale-axioms}.
Since $\{\text{Spf}(A_i^\wedge) \to X_{/T}\}$ and
$\{\text{Spf}(B_{ij}) \to Y_{/T}\}$ are coverings as in
Formal Spaces, Definition
\ref{formal-spaces-definition-formal-algebraic-space}
we see that $Y_{/T} \to X_{/T}$ is rig-\'etale by definition.
\end{proof}

\begin{lemma}
\label{lemma-algebraize-morphism-rig-etale}
Let $X$ be a Noetherian affine scheme. Let $T \subset X$ be a closed subset.
Let $U$ be an affine scheme and let $U \to X$ a finite type morphism
\'etale over $X \setminus T$. Let $V$ be a Noetherian affine scheme over $X$.
For any morphism $c' : V_{/T} \to U_{/T}$ over $X_{/T}$ there exists
an \'etale morphism $b : V' \to V$ of affine schemes which induces an
isomorphism $b_{/T} : V'_{/T} \to V_{/T}$ and a morphism $a : V' \to U$
such that $c' = a_{/T} \circ b_{/T}^{-1}$.
\end{lemma}

\begin{proof}
This is a reformulation of
Lemma \ref{lemma-fully-faithful-etale-over-complement}.
\end{proof}

\begin{lemma}
\label{lemma-algebraize-rig-etale-affine}
Let $X$ be a Noetherian affine scheme. Let $T \subset X$ be a closed subset.
Let $W \to X_{/T}$ be a rig-\'etale morphism of formal algebraic
spaces with $W$ an affine formal algebraic space. Then there exists
an affine scheme $U$, a finite type morphism $U \to X$ \'etale over
$X \setminus T$ such that $W \cong U_{/T}$.
Moreover, if $W \to X_{/T}$ is \'etale, then $U \to X$ is \'etale.
\end{lemma}

\begin{proof}
The existence of $U$ is a restatement of
Lemma \ref{lemma-approximate-by-etale-over-complement}.
The final statement follows from
More on Morphisms, Lemma
\ref{more-morphisms-lemma-check-smoothness-on-infinitesimal-nbhds}.
\end{proof}

\noindent
Let $S$ be a scheme. Let $X$ be a locally Noetherian algebraic space over $S$
and let $T \subset |X|$ be a closed subset. Let us denote
$\mathcal{C}_{X, T}$ the category of algebraic spaces $Y$ over $X$
such that the structure morphism $f : Y \to X$
is locally of finite type and an isomorphism over the complement of
$T$. Formal completion defines a functor
\begin{equation}
\label{equation-completion-functor}
F_{X, T} : \mathcal{C}_{X, T} \longrightarrow \mathcal{C}_{X_{/T}},\quad
(f : Y \to X) \longmapsto (f_{/T} : Y_{/T} \to X_{/T})
\end{equation}
see Lemma \ref{lemma-etale-gives-rig-etale}.

\begin{lemma}
\label{lemma-factor}
Let $S$ be a scheme. Let $f : X \to Y$ and $g : Z \to Y$ be morphisms
of algebraic spaces. Let $T \subset |X|$ be closed.
Assume that
\begin{enumerate}
\item $X$ is locally Noetherian,
\item $g$ is a monomorphism and locally of finite type,
\item $f|_{X \setminus T} : X \setminus T \to Y$ factors through $g$, and
\item $f_{/T} : X_{/T} \to Y$ factors through $g$,
\end{enumerate}
then $f$ factors through $g$.
\end{lemma}

\begin{proof}
Consider the fibre product $E = X \times_Y Z \to X$.
By assumption the open immersion $X \setminus T \to X$
factors through $E$ and any morphism $\varphi : X' \to X$ with
$|\varphi|(|X'|) \subset T$ factors through $E$ as well, see
Formal Spaces, Section \ref{formal-spaces-section-completion}.
By More on Morphisms of Spaces, Lemma
\ref{spaces-more-morphisms-lemma-check-smoothness-on-infinitesimal-nbhds}
this implies that $E \to X$ is \'etale at every point of $E$
mapping to a point of $T$. Hence $E \to X$ is an \'etale
monomorphism, hence an open immersion
(Morphisms of Spaces, Lemma
\ref{spaces-morphisms-lemma-etale-universally-injective-open}).
Then it follows that $E = X$ since our assumptions imply that $|X| = |E|$.
\end{proof}

\begin{lemma}
\label{lemma-faithful}
Let $S$ be a scheme. Let $X$, $Y$ be locally Noetherian algebraic spaces
over $S$. Let $T \subset |X|$ and $T' \subset |Y|$ be closed subsets.
Let $a, b : X \to Y$ be morphisms of algebraic spaces over $S$ such
that $a|_{X \setminus T} = b|_{X \setminus T}$, such that
$|a|(T) \subset T'$ and $|b|(T) \subset T'$, and such that
$a_{/T} = b_{/T}$ as morphisms $X_{/T} \to Y_{/T'}$.
Then $a = b$.
\end{lemma}

\begin{proof}
Let $E$ be the equalizer of $a$ and $b$. Then $E$ is an algebraic space
and $E \to X$ is locally of finite type and a monomorphism, see
Morphisms of Spaces, Lemma \ref{spaces-morphisms-lemma-properties-diagonal}.
Our assumptions imply we can apply Lemma \ref{lemma-factor} to the two
morphisms $f = \text{id} : X \to X$ and $g : E \to X$ and the closed
subset $T$ of $|X|$.
\end{proof}

\begin{lemma}
\label{lemma-equivalence-relation}
Let $S$ be a scheme. Let $X$ be a locally Noetherian algebraic space
over $S$. Let $T \subset |X|$ be a closed subset.
Let $s, t : R \to U$ be two morphisms of algebraic spaces over $X$.
Assume
\begin{enumerate}
\item $R$, $U$ are locally of finite type over $X$,
\item the base change of $s$ and $t$ to $X \setminus T$
is an \'etale equivalence relation, and
\item the formal completion
$(t_{/T}, s_{/T}) : R_{/T} \to U_{/T} \times_{X_{/T}} U_{/T}$
is an equivalence relation too.
\end{enumerate}
Then $(t, s) : R \to U \times_X U$ is an \'etale equivalence relation.
\end{lemma}

\begin{proof}
The morphisms $s, t : R \to U$ are \'etale over $X \setminus T$
by assumption. Since the formal completions of the maps
$s, t : R \to U$ are \'etale, we see that $s$ and $t$ are \'etale
for example by More on Morphisms, Lemma
\ref{more-morphisms-lemma-check-smoothness-on-infinitesimal-nbhds}.
Applying Lemma \ref{lemma-factor} to the morphisms
$\text{id} : R \times_{U \times_X U} R \to R \times_{U \times_X U} R$
and $\Delta : R \to R \times_{U \times_X U} R$ we conclude that
$(t, s)$ is a monomorphism. Applying it again to
$(t \circ \text{pr}_0, s \circ \text{pr}_1) :
R \times_{s, U, t} R \to U \times_X U$ and $(t, s) : R \to U \times_X U$
we find that ``transitivity'' holds. We omit the proof of
the other two axioms of an equivalence relation.
\end{proof}

\begin{remark}
\label{remark-smash-away-from-T}
Let $S$, $X$, and $T \subset |X|$ be as in (\ref{equation-completion-functor}).
Let $U \to X$ be an algebraic space over $X$ such that $U \to X$ is
locally of finite type and \'etale outside of $T$. We will construct a
factorization
$$
U \longrightarrow Y \longrightarrow X
$$
with $Y$ in $\mathcal{C}_{X, T}$ such that $U_{/T} \to Y_{/T}$ is an
isomorphism. We may assume the image of $U \to X$ contains
$X \setminus T$, otherwise we replace $U$ by $U \amalg (X \setminus T)$.
For an algebraic space $Z$ over $X$, let us denote $Z^\circ$
the open subspace which is the inverse image of $X \setminus T$.
Let
$$
R = U \amalg_{U^\circ} (U \times_X U)^\circ
$$
be the pushout of $U^\circ \to U$ and the diagonal morphism
$U^\circ \to U^\circ \times_X U^\circ = (U \times_X U)^\circ$.
Since $U^\circ \to X$ is \'etale, the diagonal is an open immersion
and we see that $R$ is an algebraic space (this follows for example
from Spaces, Lemma \ref{spaces-lemma-glueing-algebraic-spaces}).
The two projections $(U \times_X U)^\circ \to U$ extend to $R$
and we obtain two \'etale morphisms $s, t : R \to U$. Checking on each
piece separatedly we find that $R$ is an \'etale equivalence
relation on $U$. Set $Y = U/R$ which is an algebraic space by
Bootstrap, Theorem \ref{bootstrap-theorem-final-bootstrap}.
Since $U^\circ \to X \setminus T$ is a surjective \'etale morphism
and since $R^\circ = U^\circ \times_{X \setminus T} U^\circ$
we see that $Y^\circ \to X \setminus T$ is an isomorphism.
In other words, $Y \to X$ is an object of $\mathcal{C}_{X, T}$.
On the other hand, the morphism $U \to Y$ induces an isomorphism
$U_{/T} \to Y_{/T}$. Namely, the formal completion of $R$
along the inverse image of $T$ is equal to the formal completion of
$U$ along the inverse image of $T$ by our choice of $R$. By
our construction of the formal completion in
Formal Spaces, Section \ref{formal-spaces-section-completion}
we conclude that $U_{/T} = Y_{/T}$.
\end{remark}

\begin{lemma}
\label{lemma-dilatations-affine}
Let $S$ be a scheme. Let $X$ be a Noetherian affine algebraic space over $S$.
Let $T \subset |X|$ be a closed subset. Then the functor $F_{X, T}$ is an
equivalence.
\end{lemma}

\noindent
Before we prove this lemma let us discuss an example. Suppose that
$S = \Spec(k)$, $X = \mathbf{A}^1_k$, and $T = \{0\}$. Then
$X_{/T} = \text{Spf}(k[[x]])$. Let $W = \text{Spf}(k[[x]] \times k[[x]])$.
Then the corresponding $Y$ is the affine line with zero doubled
(Schemes, Example \ref{schemes-example-affine-space-zero-doubled}).
Moreover, this is the output of the construction in
Remark \ref{remark-smash-away-from-T}
starting with $U = X \amalg X$.

\begin{proof}
For any scheme or algebraic space $Z$ over $X$, let us denote
$Z_0 \subset Z$ the inverse image of $T$ with the induced reduced
closed subscheme or subspace structure. Note that $Z_0 = (Z_{/T})_{red}$
is the reduction of the formal completion.

\medskip\noindent
The functor $F_{X, T}$ is faithful by Lemma \ref{lemma-faithful}.

\medskip\noindent
Let $Y, Y'$ be objects of $\mathcal{C}_{X, T}$ and let
$a' : Y_{/T} \to Y'_{/T}$ be a morphism in $\mathcal{C}_{X_{/T}}$.
To prove $F_{X, T}$ is fully faithful, we will construct a morphism
$a : Y \to Y'$ in $\mathcal{C}_{X, T}$ such that $a' = a_{/T}$.

\medskip\noindent
Let $U$ be an affine scheme and let $U \to Y$ be an \'etale morphism.
Because $U$ is affine, $U_0$ is affine and the image of
$U_0 \to Y_0 \to Y'_0$ is a quasi-compact subspace of $|Y'_0|$.
Thus we can choose an affine scheme $V$ and an \'etale morphism
$V \to Y'$ such that the image of $|V_0| \to |Y'_0|$
contains this quasi-compact subset. Consider the formal algebraic space
$$
W = U_{/T} \times_{Y'_{/T}} V_{/T}
$$
By our choice of $V$ the above, the map $W \to U_{/T}$ is surjective.
Thus there exists an affine formal algebraic space $W'$ and an \'etale
morphism $W' \to W$ such that $W' \to W \to U_{/T}$ is surjective.
Then $W' \to U_{/T}$ is \'etale.
By Lemma \ref{lemma-algebraize-rig-etale-affine} $W' = U'_{/T}$
for $U' \to U$ \'etale and $U'$ affine. Write $V = \Spec(C)$. By
Lemma \ref{lemma-algebraize-morphism-rig-etale} there exists an
\'etale morphism $U'' \to U'$ of affines which is an isomorphism on
completions and a morphism $U'' \to V$ whose completion is
the composition $U''_{/T} \to U'_{/T} \to W \to V_{/T}$.
Thus we get
$$
Y \longleftarrow U'' \longrightarrow Y'
$$
over $X$ agreeing with the given map on formal completions such that
the image of $U''_0 \to Y_0$ is the same as the image of $U_0 \to Y_0$.

\medskip\noindent
Taking a disjoint union of $U''$ as constructed in the previous
paragraph, we find a scheme $U$, an \'etale morphism $U \to Y$,
and a morphism $b : U \to Y'$ over $X$, such that the diagram
$$
\xymatrix{
U_{/T} \ar[d] \ar[rd]^{b_{/T}} \\
Y_{/T} \ar[r]^{a'} & Y'_{/T}
}
$$
is commutative and such that $U_0 \to Y_0$ is surjective.
Taking a disjoint union with the open $X \setminus T$ (which is also
open in $Y$ and $Y'$), we find that we may even assume that $U \to Y$
is a surjective \'etale morphism. Let $R = U \times_Y U$.
Then the two compositions $R \to U \to Y'$ agree both over $X \setminus T$
and after formal completion along $T$, whence are equal by
Lemma \ref{lemma-faithful}. This means exactly that $b$ factors as
$U \to Y \to Y'$ to give us our desired morphism $a : Y \to Y'$.

\medskip\noindent
Essential surjectivity. Let $W$ be an object of $\mathcal{C}_{X_{/T}}$.
We prove $W$ is in the essential image in a number of steps.

\medskip\noindent
Step 1: $W$ is an affine formal algebraic space. Then we can find
$U \to X$ of finite type and \'etale over $X \setminus T$ such that
$U_{/T}$ is isomorphic to $W$, see
Lemma \ref{lemma-algebraize-rig-etale-affine}.
Thus we see that $W$ is in the essential image by the construction
in Remark \ref{remark-smash-away-from-T}.

\medskip\noindent
Step 2: $W$ is separated. Choose $\{W_i \to W\}$ as in
Formal Spaces, Definition \ref{formal-spaces-definition-formal-algebraic-space}.
By Step 1 the formal algebraic spaces $W_i$ and $W_i \times_W W_j$
are in the essential image.
Say $W_i = (Y_i)_{/T}$ and $W_i \times_W W_j = (Y_{ij})_{/T}$.
By fully faithfulness we obtain morphisms $t_{ij} : Y_{ij} \to Y_i$
and $s_{ij} : Y_{ij} \to Y_j$ matching the projections
$W_i \times_W W_j \to W_i$ and $W_i \times_W W_j \to W_j$.
Set $R = \coprod Y_{ij}$ and $U = \coprod Y_i$ and denote
$s = \coprod s_{ij} : R \to U$ and $t = \coprod t_{ij} : R \to U$.
Applying Lemma \ref{lemma-equivalence-relation}
we find that $(t, s) : R \to U \times_X U$ is an \'etale equivalence relation.
Thus we can take the quotient $Y = U/R$ and it is an algebraic
space, see 
Bootstrap, Theorem \ref{bootstrap-theorem-final-bootstrap}.
Since completion commutes with fibre products and taking
quotient sheaves, we find that $Y_{/T} \cong W$ in $\mathcal{C}_{X_{/T}}$.

\medskip\noindent
Step 3: $W$ is general. Choose $\{W_i \to W\}$ as in
Formal Spaces, Definition \ref{formal-spaces-definition-formal-algebraic-space}.
The formal algebraic spaces $W_i$ and $W_i \times_W W_j$ are separated.
Hence by Step 2  the formal algebraic spaces $W_i$ and $W_i \times_W W_j$
are in the essential image. Then we argue exactly as in the previous
paragraph to see that $W$ is in the essential image as well.
This concludes the proof.
\end{proof}

\begin{theorem}
\label{theorem-dilatations-general}
Let $S$ be a scheme. Let $X$ be a locally Noetherian algebraic space over $S$.
Let $T \subset |X|$ be a closed subset. The functor $F_{X, T}$
(\ref{equation-completion-functor})
$$
\left\{
\begin{matrix}
\text{algebraic spaces }Y\text{ locally of finite}\\
\text{type over }X\text{ such that }Y \to X\\
\text{is an isomorphism over }X \setminus T
\end{matrix}
\right\}
\longrightarrow
\left\{
\begin{matrix}
\text{formal algebraic spaces }W\text{ endowed} \\
\text{with a rig-\'etale morphism }W \to X_{/T}
\end{matrix}
\right\}
$$
given by formal completion is an equivalence.
\end{theorem}

\begin{proof}
The theorem is essentially a formal consequence of
Lemma \ref{lemma-dilatations-affine}. We give the details but
we encourage the reader to think it through for themselves.
Let $g : U \to X$ be a surjective \'etale morphism with $U = \coprod U_i$
and each $U_i$ affine. Denote $F_{U, T}$ the functor for $U$ and the
inverse image of $T$ in $|U|$.

\medskip\noindent
Since $U = \coprod U_i$ both the category $\mathcal{C}_{U, T}$ and
the category $\mathcal{C}_{U_{/T}}$ decompose as a product of categories,
one for each $i$. Since the functors $F_{U_i, T}$ are equivalences
for all $i$ by the lemma we find that the same is true for $F_{U, T}$.

\medskip\noindent
Since $F_{U, T}$ is faithful, it follows that $F_{X, T}$
is faithful too. Namely, if $a, b : Y \to Y'$ are morphisms
in $\mathcal{C}_{X, T}$ such that $a_{/T} = b_{/T}$, then we find
on pulling back that the base changes
$a_U, b_U : U \times_X Y \to U \times_X Y'$ are equal.
Since $U \times_X Y \to Y$ is surjective \'etale, this implies that $a = b$.

\medskip\noindent
At this point we know that $F_{X, T}$ is faithful for every situation
as in the theorem. Let $R = U \times_X U$ where $U$ is as above.
Let $t, s : R \to U$ be the projections.
Since $X$ is Noetherian, so is $R$. Thus the functor $F_{R, T}$
(defined in the obvious manner) is
faithful. Let $Y \to X$ and $Y' \to X$ be objects of $\mathcal{C}_{X, T}$.
Let $a' : Y_{/T} \to Y'_{/T}$ be a morphism in the category
$\mathcal{C}_{X_{/T}}$. Taking the base change to $U$ we obtain a
morphism $a'_U : (U \times_X Y)_{/T} \to (U \times_X Y')_{/T}$
in the category $\mathcal{C}_{U_{/T}}$. Since the functor $F_{U, T}$
is fully faithful we obtain a morphism $a_U : U \times_X Y \to U \times_X Y'$
with $F_{U, T}(a_U) = a'_U$. Since $s^*(a'_U) = t^*(a'_U)$ and since
$F_{R, T}$ is faithful, we find that $s^*(a_U) = t^*(a_U)$.
Since
$$
\xymatrix{
R \times_X Y \ar@<1ex>[r] \ar@<-1ex>[r] &
U \times_X Y \ar[r] & Y
}
$$
is an equalizer diagram of sheaves, we find that $a_U$ descends to
a morphism $a : Y \to Y'$. We omit the proof that $F_{X, T}(a) = a'$.

\medskip\noindent
At this point we know that $F_{X, T}$ is faithful for every situation
as in the theorem. To finish the proof we show that $F_{X, T}$ is
essentially surjective. Let $W \to X_{/T}$ be an object of
$\mathcal{C}_{X_{/T}}$. Then $U \times_X W$ is an object of
$\mathcal{C}_{U_{/T}}$. By the affine case we find an object
$V \to U$ of $\mathcal{C}_{U, T}$ and an isomorphism
$\alpha : F_{U, T}(V) \to U \times_X W$ in $\mathcal{C}_{U_{/T}}$.
By fully faithfulness of $F_{R, T}$ we find a unique morphism
$h : s^*V \to t^*V$ in the category $\mathcal{C}_{R, T}$ such that
$F_{R, T}(h)$ corresponds, via the isomorphism $\alpha$, to the
canonical descent datum on $U \times_X W$ in the category
$\mathcal{C}_{R_{/T}}$. Using faithfulness of our functor on
$R \times_{s, U, t} R$ we see that $h$ satisfies the cocycle
condition. We conclude, for example by the much more general
Bootstrap, Lemma \ref{bootstrap-lemma-descend-algebraic-space},
that there exists an object $Y \to X$ of $\mathcal{C}_{X, T}$
and an isomorphism $\beta : U \times_X Y \to V$ such that the
descent datum $h$ corresponds, via $\beta$, to the canonical descent
datum on $U \times_X Y$. We omit the verification that
$F_{X, T}(Y)$ is isomorphic to $W$; hint: in the category of formal
algebraic spaces there is descent for morphisms along \'etale coverings.
\end{proof}

\noindent
We are often interested as to whether the output of the construction
of Theorem \ref{theorem-dilatations-general} is a separated algebraic space.
In the next few lemmas we match properties of $Y \to X$ and the corresponding
completion $Y_{/T} \to X_{/T}$.

\begin{lemma}
\label{lemma-output-quasi-compact}
Let $S$ be a scheme. Let $X$ be a locally Noetherian algebraic space over $S$.
Let $T \subset |X|$ be a closed subset. Let $W \to X_{/T}$ be an object
of the category $\mathcal{C}_{X_{/T}}$ and let $Y \to X$ be the object
corresponding to $W$ via Theorem \ref{theorem-dilatations-general}.
Then $Y \to X$ is quasi-compact if and only if $W \to X_{/T}$ is so.
\end{lemma}

\begin{proof}
These conditions may be checked after base change to an affine
scheme \'etale over $X$, resp.\ a formal affine algebraic space
\'etale over $X_{/T}$, see
Morphisms of Spaces, Lemma \ref{spaces-morphisms-lemma-quasi-compact-local}
as well as Formal Spaces, Lemma
\ref{formal-spaces-lemma-characterize-quasi-compact-morphism}.
If $U \to X$ ranges over \'etale morphisms with $U$ affine, then
the formal completions $U_{/T} \to X_{/T}$ give a family
of formal affine coverings as in Formal Spaces, Definition
\ref{formal-spaces-definition-formal-algebraic-space}.
Thus we may and do assume $X$ is affine.

\medskip\noindent
Let $V \to Y$ be a surjective \'etale morphism
where $V = \coprod_{j \in J} V_j$ is a disjoint union of affines. Then
$V_{/T} \to Y_{/T} = W$ is a surjective \'etale morphism.
Thus if $Y$ is quasi-compact, we can choose $J$ is finite, and
we conclude that $W$ is quasi-compact. Conversely, if $W$ is
quasi-compact, then we can find a finite subset $J' \subset J$
such that $\coprod_{j \in J'} (V_j)_{/T} \to W$ is surjective.
Then it follows that
$$
(X \setminus T) \amalg \coprod\nolimits_{j \in J'} V_j \longrightarrow Y
$$
is surjective. This either follows from the construction of
$Y$ in the proof of Lemma \ref{lemma-dilatations-affine}
or it follows since we have
$$
|Y| = |X \setminus T| \amalg |W_{red}|
$$
as $Y_{/T} = W$.
\end{proof}

\begin{lemma}
\label{lemma-output-quasi-separated}
Let $S$ be a scheme. Let $X$ be a locally Noetherian algebraic space over $S$.
Let $T \subset |X|$ be a closed subset. Let $W \to X_{/T}$ be an object
of the category $\mathcal{C}_{X_{/T}}$ and let $Y \to X$ be the object
corresponding to $W$ via Theorem \ref{theorem-dilatations-general}.
Then $Y \to X$ is quasi-separated if and only if $W \to X_{/T}$ is so.
\end{lemma}

\begin{proof}
These conditions may be checked after base change to an affine
scheme \'etale over $X$, resp.\ a formal affine algebraic space
\'etale over $X_{/T}$, see
Morphisms of Spaces, Lemma
\ref{spaces-morphisms-lemma-separated-local}
as well as
Formal Spaces, Lemma
\ref{formal-spaces-lemma-separated-local},
If $U \to X$ ranges over \'etale morphisms with $U$ affine, then
the formal completions $U_{/T} \to X_{/T}$ give a family
of formal affine coverings as in Formal Spaces, Definition
\ref{formal-spaces-definition-formal-algebraic-space}.
Thus we may and do assume $X$ is affine.

\medskip\noindent
Let $V \to Y$ be a surjective \'etale morphism
where $V = \coprod_{j \in J} V_j$ is a disjoint union of affines.
Then $Y$ is quasi-separated if and only if $V_j \times_Y V_{j'}$ is
quasi-compact for all $j, j' \in J$. Similarly, $W$ is quasi-separated
if and only if
$(V_j \times_Y V_{j'})_{/T} = (V_j)_{/T} \times_{Y_{/T}} (V_{j'})_{/T}$
is quasi-compact for all $j, j' \in J$. Since $X$ is Noetherian affine,
we see that
$$
(V_j \times_Y V_{j'}) \times_X (X \setminus T)
$$
is quasi-compact. Hence we conclude the equivalence holds by the equality
$$
|V_j \times_Y V_{j'}| =
|(V_j \times_Y V_{j'}) \times_X (X \setminus T)| \amalg
|(V_j \times_Y V_{j'})_{/T}|
$$
and the fact that the second summand is closed in the left hand side.
\end{proof}

\begin{lemma}
\label{lemma-output-separated}
Let $S$ be a scheme. Let $X$ be a locally Noetherian algebraic space over $S$.
Let $T \subset |X|$ be a closed subset. Let $W \to X_{/T}$ be an object
of the category $\mathcal{C}_{X_{/T}}$ and let $Y \to X$ be the object
corresponding to $W$ via Theorem \ref{theorem-dilatations-general}.
Then $Y \to X$ is separated if and only if $W \to X_{/T}$ is
separated and $\Delta : W \to W \times_{X_{/T}} W$ is rig-surjective.
\end{lemma}

\begin{proof}
These conditions may be checked after base change to an affine
scheme \'etale over $X$, resp.\ a formal affine algebraic space
\'etale over $X_{/T}$, see
Morphisms of Spaces, Lemma \ref{spaces-morphisms-lemma-separated-local}
as well as
Formal Spaces, Lemma \ref{formal-spaces-lemma-separated-local}.
If $U \to X$ ranges over \'etale morphisms with $U$ affine, then
the formal completions $U_{/T} \to X_{/T}$ give a family
of formal affine coverings as in Formal Spaces, Definition
\ref{formal-spaces-definition-formal-algebraic-space}.
Thus we may and do assume $X$ is affine.
In the proof of both directions we may assume that $Y \to X$ and
$W \to X_{/T}$ are quasi-separated by
Lemma \ref{lemma-output-quasi-separated}.

\medskip\noindent
Proof of easy direction. Assume $Y \to X$ is separated.
Then $Y \to Y \times_X Y$ is a closed immersion and it follows that
$W \to W \times_{X_{/T}} W$ is a closed immersion too, i.e., we see
that $W \to X_{/T}$ is separated. Let
$$
p : \text{Spf}(R) \longrightarrow W \times_{X_{/T}} W = (Y \times_X Y)_{/T}
$$
be an adic morphism where $R$ is a complete discrete valuation ring with
fraction field $K$. The composition into $Y \times_X Y$ corresponds to
a morphism $g : \Spec(R) \to Y \times_X Y$, see
Formal Spaces, Lemma \ref{formal-spaces-lemma-map-into-algebraic-space}.
Since $p$ is an adic morphism, so is the composition $\text{Spf}(R) \to X$.
Thus we see that $g(\Spec(K))$ is a point of
$$
(Y \times_X Y) \times_X (X \setminus T) \cong
X \setminus T \cong
Y \times_X (X \setminus T)
$$
(small detail omitted). Hence this lifts to a $K$-point of $Y$ and
we obtain a commutative diagram
$$
\xymatrix{
\Spec(K) \ar[r] \ar[d] & Y \ar[d] \\
\Spec(R) \ar[r] \ar@{-->}[ru] & Y \times_X Y
}
$$
Since $Y \to X$ was assumed separated we find the dotted arrow exists
(Cohomology of Spaces, Lemma
\ref{spaces-cohomology-lemma-check-separated-dvr}).
Applying the functor completion along $T$ we find that $p$ can be
lifted to a morphism into $W$, i.e., $W \to W \times_{X_{/T}} W$ is
rig-surjective.

\medskip\noindent
Proof of hard direction. Assume $W \to X_{/T}$ separated and
$W \to W \times_{X_{/T}} W$ rig-surjective. By
Cohomology of Spaces, Lemma
\ref{spaces-cohomology-lemma-check-separated-dvr} and
Remark \ref{spaces-cohomology-remark-variant}
it suffices to show that given any commutative diagram
$$
\xymatrix{
\Spec(K) \ar[r] \ar[d] & Y \ar[d] \\
\Spec(R) \ar[r]^g \ar@{-->}[ru] & Y \times_X Y
}
$$
where $R$ is a complete discrete valuation ring with fraction field $K$,
there is at most one dotted arrow making the diagram commute. Let
$h : \Spec(R) \to X$ be the composition of $g$ with the morphism
$Y \times_X Y \to X$. There are three cases:
Case I: $h(\Spec(R)) \subset (X \setminus T)$. This case is trivial
because $Y \times_X (X \setminus T) = X \setminus T$.
Case II: $h$ maps $\Spec(R)$ into $T$. This case follows
from our assumption that $W \to X_{/T}$ is separated. Namely,
if $T$ denotes the reduced induced closed subspace structure
on $T$, then $h$ factors through $T$ and
$$
W \times_{X_{/T}} T = Y \times_X T \longrightarrow T
$$
is separated by assumption (and for example
Formal Spaces, Lemma \ref{formal-spaces-lemma-separated-local})
which implies we get the lifting property by
Cohomology of Spaces, Lemma \ref{spaces-cohomology-lemma-check-separated-dvr}
applied to the displayed arrow. Case III: $h(\Spec(K))$ is not in $T$
but $h$ maps the closed point of $\Spec(R)$ into $T$. In this case
the corresponding morphism
$$
g_{/T} :
\text{Spf}(R)
\longrightarrow
(Y \times_X Y)_{/T} =
W \times_{X_{/T}} W
$$
is an adic morphism (detail omitted). Hence our assumption that
$W \to W \times_{X_{/T}} W$ be rig-surjective implies we can lift
$g_{/T}$ to a morphism $e : \text{Spf}(R) \to W = Y_{/T}$ (see
Lemma \ref{lemma-monomorphism-rig-surjective}
for why we do not need to extend $R$).
Algebraizing the composition $\text{Spf}(R) \to Y$ using
Formal Spaces, Lemma \ref{formal-spaces-lemma-map-into-algebraic-space}
we find a morphism $\Spec(R) \to Y$ lifting $g$ as desired.
\end{proof}

\begin{lemma}
\label{lemma-output-proper}
Let $S$ be a scheme. Let $X$ be a locally Noetherian algebraic space over $S$.
Let $T \subset |X|$ be a closed subset. Let $W \to X_{/T}$ be an object
of the category $\mathcal{C}_{X_{/T}}$ and let $Y \to X$ be the object
corresponding to $W$ via Theorem \ref{theorem-dilatations-general}.
Then $Y \to X$ is proper if and only if the following conditions hold
\begin{enumerate}
\item $W \to X_{/T}$ is proper,
\item $W \to X_{/T}$ is rig-surjective, and
\item $\Delta : W \to W \times_{X_{/T}} W$ is rig-surjective.
\end{enumerate}
\end{lemma}

\begin{proof}
These conditions may be checked after base change to an affine
scheme \'etale over $X$, resp.\ a formal affine algebraic space
\'etale over $X_{/T}$, see
Morphisms of Spaces, Lemma \ref{spaces-morphisms-lemma-proper-local}
as well as
Formal Spaces, Lemma \ref{formal-spaces-lemma-proper-local}.
If $U \to X$ ranges over \'etale morphisms with $U$ affine, then
the formal completions $U_{/T} \to X_{/T}$ give a family
of formal affine coverings as in Formal Spaces, Definition
\ref{formal-spaces-definition-formal-algebraic-space}.
Thus we may and do assume $X$ is affine.
In the proof of both directions we may assume that $Y \to X$
and $W \to X_{/T}$ are separated and quasi-compact and that
$W \to W \times_{X_{/T}} W$ is rig-surjective by
Lemmas \ref{lemma-output-quasi-compact} and \ref{lemma-output-separated}.

\medskip\noindent
Proof of the easy direction. Assume $Y \to X$ is proper.
Then $Y_{/T} = Y \times_X X_{/T} \to X_{/T}$ is proper too.
Let
$$
p : \text{Spf}(R) \longrightarrow X_{/T}
$$
be an adic morphism where $R$ is a complete discrete valuation ring with
fraction field $K$. Then $p$ corresponds to a morphism $g : \Spec(R) \to X$,
see
Formal Spaces, Lemma \ref{formal-spaces-lemma-map-into-algebraic-space}.
Since $p$ is an adic morphism, we have $p(\Spec(K)) \not \in T$.
Since $Y \to X$ is an isomorphism over $X \setminus T$ we can lift
to $X$ and obtain a commutative diagram
$$
\xymatrix{
\Spec(K) \ar[r] \ar[d] & Y \ar[d] \\
\Spec(R) \ar[r] \ar@{-->}[ru] & X
}
$$
Since $Y \to X$ was assumed proper we find the dotted arrow exists.
(Cohomology of Spaces, Lemma
\ref{spaces-cohomology-lemma-check-proper-dvr}).
Applying the functor completion along $T$ we find that $p$ can be
lifted to a morphism into $W$, i.e., $W \to X_{/T}$ is
rig-surjective.

\medskip\noindent
Proof of hard direction. Assume $W \to X_{/T}$ proper,
$W \to W \times_{X_{/T}} W$ rig-surjective, and $W \to X_{/T}$
rig-surjective. By
Cohomology of Spaces, Lemma \ref{spaces-cohomology-lemma-check-proper-dvr} and
Remark \ref{spaces-cohomology-remark-variant}
it suffices to show that given any commutative diagram
$$
\xymatrix{
\Spec(K) \ar[r] \ar[d] & Y \ar[d] \\
\Spec(R) \ar[r]^g \ar@{-->}[ru] & X
}
$$
where $R$ is a complete discrete valuation ring with fraction field $K$,
there is a dotted arrow making the diagram commute. Let
$h : \Spec(R) \to X$ be the composition of $g$ with the morphism
$Y \times_X Y \to X$. There are three cases:
Case I: $h(\Spec(R)) \subset (X \setminus T)$. This case is trivial
because $Y \times_X (X \setminus T) = X \setminus T$.
Case II: $h$ maps $\Spec(R)$ into $T$. This case follows
from our assumption that $W \to X_{/T}$ is proper. Namely,
if $T$ denotes the reduced induced closed subspace structure
on $T$, then $h$ factors through $T$ and
$$
W \times_{X_{/T}} T = Y \times_X T \longrightarrow T
$$
is proper by assumption which implies we get the lifting property by
Cohomology of Spaces, Lemma \ref{spaces-cohomology-lemma-check-proper-dvr}
applied to the displayed arrow. Case III: $h(\Spec(K))$ is not in $T$
but $h$ maps the closed point of $\Spec(R)$ into $T$. In this case
the corresponding morphism
$$
g_{/T} : \text{Spf}(R) \longrightarrow Y_{/T} = W
$$
is an adic morphism (detail omitted). Hence our assumption that
$W \to X_{/T}$ be rig-surjective implies we can lift
$g_{/T}$ to a morphism $e : \text{Spf}(R') \to W = Y_{/T}$
for some extension of complete discrete valuation rings $R \subset R'$.
Algebraizing the composition $\text{Spf}(R') \to Y$ using
Formal Spaces, Lemma \ref{formal-spaces-lemma-map-into-algebraic-space}
we find a morphism $\Spec(R') \to Y$ lifting $g$. By the discussion
in Cohomology of Spaces, Remark \ref{spaces-cohomology-remark-variant}
this is sufficient to conclude that $Y \to X$ is proper.
\end{proof}





\section{Application to modifications}
\label{section-modifications}

\noindent
Let $A$ be a Noetherian ring and let $I \subset A$ be an ideal. We set
$S = \Spec(A)$ and $U = S \setminus V(I)$. In this section
we will consider the category
\begin{equation}
\label{equation-modification}
\left\{
f : X \longrightarrow S
\quad \middle| \quad
\begin{matrix}
X\text{ is an algebraic space}\\
f\text{ is locally of finite type}\\
f^{-1}(U) \to U\text{ is an isomorphism}
\end{matrix}
\right\}
\end{equation}
A morphism from $X/S$ to $X'/S$ will be a morphism of algebraic spaces
$X \to X'$ compatible with the structure morphisms over $S$.

\medskip\noindent
Let $A \to B$ be a homomorphism of Noetherian rings and let $J \subset B$
be an ideal such that $J = \sqrt{I B}$. Then base
change along the morphism $\Spec(B) \to \Spec(A)$ gives a functor
from the category (\ref{equation-modification}) for $A$
to the category (\ref{equation-modification}) for $B$.

\begin{lemma}
\label{lemma-Noetherian-local-ring}
Let $(A, I)$ be a pair consisting of a Noetherian ring and an ideal $I$.
Let $A^\wedge$ be the $I$-adic completion of $A$. Then base change defines
an equivalence of categories between the category (\ref{equation-modification})
for $A$ with the category (\ref{equation-modification}) for the completion
$A^\wedge$.
\end{lemma}

\begin{proof}
Set $S = \Spec(A)$ as in (\ref{equation-modification}) and $T = V(I)$.
Similarly, write $S' = \Spec(A^\wedge)$ and $T' = V(IA^\wedge)$.
The morphism $S' \to S$ defines an isomorphism $S'_{/T'} \to S_{/T}$
of formal completions. Let $\mathcal{C}_{S, T}$, $\mathcal{C}_{S_{/T}}$,
$\mathcal{C}_{S'_{/T'}}$, and $\mathcal{C}_{S', T'}$ be the corresponding
categories as used in (\ref{equation-completion-functor}).
By Theorem \ref{theorem-dilatations-general} (in fact we only need
the affine case treated in Lemma \ref{lemma-dilatations-affine})
we see that
$$
\mathcal{C}_{S, T} = \mathcal{C}_{S_{/T}} =
\mathcal{C}_{S_{/T'}'} = \mathcal{C}_{S', T'}
$$
Since $\mathcal{C}_{S, T}$ is the category (\ref{equation-modification})
for $A$ and $\mathcal{C}_{S', T'}$ the category (\ref{equation-modification})
for $A^\wedge$ this proves the lemma.
\end{proof}

\begin{lemma}
\label{lemma-Noetherian-local-ring-properties}
Notation and assumptions as in Lemma \ref{lemma-Noetherian-local-ring}.
Let $f : X \to \Spec(A)$ correspond to $g : Y \to \Spec(A^\wedge)$
via the equivalence. Then $f$ is quasi-compact, quasi-separated, separated,
proper, finite, and add more here if and only if $g$ is so.
\end{lemma}

\begin{proof}
You can deduce this for the statements
quasi-compact, quasi-separated, separated, and proper
by using Lemmas \ref{lemma-output-quasi-compact}
\ref{lemma-output-quasi-separated},
\ref{lemma-output-separated},
\ref{lemma-output-quasi-separated}, and
\ref{lemma-output-proper}
to translate the corresponding property into a property
of the formal completion and using the argument of the proof
of Lemma \ref{lemma-Noetherian-local-ring}.
However, there is a direct argument using fpqc descent as follows.
First, note that $\{U \to \Spec(A), \Spec(A^\wedge) \to \Spec(A)\}$ is an
fpqc covering with $U = \Spec(A) \setminus V(I)$ as before.
The base change of $f$ by $U \to \Spec(A)$ is $\text{id}_U$
by definition of our category (\ref{equation-modification}).
Let $P$ be a property of morphisms of algebraic spaces which
is fpqc local on the base (Descent on Spaces, Definition
\ref{spaces-descent-definition-property-morphisms-local})
such that $P$ holds for identity morphisms.
Then we see that $P$ holds for $f$ if and only if $P$ holds for $g$.
This applies to $P$ equal to
quasi-compact, quasi-separated, separated, proper, and finite
by
Descent on Spaces, Lemmas
\ref{spaces-descent-lemma-descending-property-quasi-compact},
\ref{spaces-descent-lemma-descending-property-quasi-separated},
\ref{spaces-descent-lemma-descending-property-separated},
\ref{spaces-descent-lemma-descending-property-proper}, and
\ref{spaces-descent-lemma-descending-property-finite}.
\end{proof}

\begin{lemma}
\label{lemma-equivalence-to-completion}
Let $A \to B$ be a local map of local Noetherian rings such that
\begin{enumerate}
\item $A \to B$ is flat,
\item $\mathfrak m_B = \mathfrak m_A B$, and
\item $\kappa(\mathfrak m_A) = \kappa(\mathfrak m_B)$
\end{enumerate}
(equivalently, $A \to B$ induces an isomorphism on completions, see
More on Algebra, Lemma \ref{more-algebra-lemma-flat-unramified}).
Then the base change functor from the category
(\ref{equation-modification}) for $(A, \mathfrak m_A)$ to the category
(\ref{equation-modification}) for $(B, \mathfrak m_B)$
is an equivalence.
\end{lemma}

\begin{proof}
This follows immediately from Lemma \ref{lemma-Noetherian-local-ring}.
\end{proof}

\begin{lemma}
\label{lemma-dominate-by-admissible-blowup}
Let $(A, \mathfrak m, \kappa)$ be a Noetherian local ring.
Let $f : X \to S$ be an object of (\ref{equation-modification}).
Then there exists a $U$-admissible blowup $S' \to S$
which dominates $X$.
\end{lemma}

\begin{proof}
Special case of More on Morphisms of Spaces,
Lemma \ref{spaces-more-morphisms-lemma-dominate-modification-by-blowup}.
\end{proof}




\input{chapters}

\bibliography{my}
\bibliographystyle{amsalpha}

\end{document}

