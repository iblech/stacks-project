\input{preamble}

% OK, start here.
%
\begin{document}

\title{Pushouts of Algebraic Spaces}


\maketitle

\phantomsection
\label{section-phantom}

\tableofcontents

\section{Introduction}
\label{section-introduction}

\noindent
The goal of this chapter is to discuss pushouts in the category of
algebraic spaces. This can be done with varying assumptions.
A fairly general pushout construction is given in \cite{Temkin-Tyomkin}:
one of the morphisms is affine and the other is a closed immersion.
We discuss a particular case of this in Section \ref{section-pushouts}
where we assume one of the morphisms is affine and the other is a
thickening, a situation that often comes up in deformation theory.

\medskip\noindent
In Sections \ref{section-formal-glueing} and
\ref{section-formal-glueing-spaces} we discuss diagrams
$$
\xymatrix{
f^{-1}(X \setminus Z) \ar[r] \ar[d] & Y \ar[d]^f \\
X \setminus Z \ar[r] & X
}
$$
where $f$ is a quasi-compact and quasi-separated morphism of
algebraic spaces, $Z \to X$ is a closed immersion of finite presentation,
the map $f^{-1}(Z) \to Z$ is an isomorphism, and
$f$ is flat along $f^{-1}(Z)$. In this situation we glue
quasi-coherent modules on $X \setminus Z$ and $Y$
(in Section \ref{section-formal-glueing}) to quasi-coherent modules on $X$
and we glue algebraic spaces over $X \setminus Z$ and $Y$
(in Section \ref{section-formal-glueing-spaces}) to algebraic spaces over $X$.

\medskip\noindent
In Section \ref{section-coequalizer-glue} we discuss how proper birational
morphisms of Noetherian algebraic spaces give rise to coequalizer diagrams
in algebraic spaces in some sense.





\section{Pushouts in the category of algebraic spaces}
\label{section-pushouts}

\noindent
This section is analogue of
More on Morphisms, Section \ref{more-morphisms-section-pushouts}.
We first prove a general result on colimits and algebraic spaces.
To do this we discuss a bit of notation. Let $S$ be a scheme.
Let $\mathcal{I} \to (\Sch/S)_{fppf}$, $i \mapsto X_i$
be a diagram (see Categories, Section \ref{categories-section-limits}).
For each $i$ we may consider the small \'etale site $X_{i, \etale}$.
For each morphism $i \to j$ of $\mathcal{I}$ we have the morphism
$X_i \to X_j$ and hence a pullback functor
$X_{j, \etale} \to X_{i, \etale}$.
Hence we obtain a pseudo functor from $\mathcal{I}^{opp}$ into
the $2$-category of categories. Denote
$$
\lim_i X_{i, \etale}
$$
the $2$-limit (see insert future reference here). What does this mean
concretely? An object of this limit is a system of \'etale morphisms
$U_i \to X_i$ over $\mathcal{I}$ such that for each $i \to j$ in
$\mathcal{I}$ the diagram
$$
\xymatrix{
U_i \ar[r] \ar[d] & U_j \ar[d] \\
X_i \ar[r] & X_j
}
$$
is cartesian. Morphisms between objects are defined in the obvious manner.
Suppose that $f_i : X_i \to T$ is a family of morphisms such that
for each $i \to j$ the composition $X_i \to X_j \to T$ is equal to $f_i$.
Then we get a functor $T_\etale \to \lim X_{i, \etale}$.
With this notation in hand we can formulate our lemma.

\begin{lemma}
\label{lemma-colimit-agrees}
Let $S$ be a scheme. Let $\mathcal{I} \to (\Sch/S)_{fppf}$, $i \mapsto X_i$
be a diagram as above. Assume that
\begin{enumerate}
\item $X = \colim X_i$ exists in the category of schemes,
\item $\coprod X_i \to X$ is surjective,
\item if $U \to X$ is \'etale and $U_i = X_i \times_X U$, then
$U = \colim U_i$ in the category of schemes, and
\item every object $(U_i \to X_i)$ of $\lim X_{i, \etale}$
with $U_i \to X_i$ separated is in the essential image
the functor $X_\etale \to \lim X_{i, \etale}$.
\end{enumerate}
Then $X = \colim X_i$ in the category of algebraic spaces over $S$ also.
\end{lemma}

\begin{proof}
Let $Z$ be an algebraic space over $S$. Suppose that $f_i : X_i \to Z$ is
a family of morphisms such that for each $i \to j$ the composition
$X_i \to X_j \to Z$ is equal to $f_i$. We have to construct a morphism
of algebraic spaces $f : X \to Z$ such that we can recover $f_i$ as
the composition $X_i \to X \to Z$. Let $W \to Z$ be a surjective
\'etale morphism of a scheme to $Z$. We may assume that $W$ is a
disjoint union of affines and in particular we may assume that
$W \to Z$ is separated. For each $i$ set
$U_i = W \times_{Z, f_i} X_i$ and denote $h_i : U_i \to W$ the projection.
Then $U_i \to X_i$ forms an object of $\lim X_{i, \etale}$
with $U_i \to X_i$ separated. By
assumption (4) we can find an \'etale morphism $U \to X$ and (functorial)
isomorphisms $U_i = X_i \times_X U$. By assumption (3) there exists a morphism
$h : U \to W$ such that the compositions $U_i \to U \to W$ are $h_i$.
Let $g : U \to Z$ be the composition of $h$ with the map $W \to Z$. To
finish the proof we have to show that $g : U \to Z$ descends to a morphism
$X \to Z$. To do this, consider the morphism
$(h, h) : U \times_X U \to W \times_S W$.
Composing with $U_i \times_{X_i} U_i \to U \times_X U$ we obtain
$(h_i, h_i)$ which factors through $W \times_Z W$. Since $U \times_X U$
is the colimit of the schemes $U_i \times_{X_i} U_i$ by (3) we see
that $(h, h)$ factors through $W \times_Z W$. Hence the two compositions
$U \times_X U \to U \to W \to Z$ are equal. Because each $U_i \to X_i$ is
surjective and assumption (2) we see that $U \to X$ is surjective.
As $Z$ is a sheaf for the \'etale topology, we conclude that
$g : U \to Z$ descends to $f : X \to Z$ as desired.
\end{proof}

\begin{lemma}
\label{lemma-pushout-along-thickening-schemes}
Let $S$ be a scheme. Let $X \to X'$ be a thickening of schemes
over $S$ and let $X \to Y$ be an affine morphism of schemes over $S$.
Let $Y' = Y \amalg_X X'$ be the pushout in the category of schemes (see
More on Morphisms, Lemma \ref{more-morphisms-lemma-pushout-along-thickening}).
Then $Y'$ is also a pushout in the category of algebraic spaces over $S$.
\end{lemma}

\begin{proof}
This is an immediate consequence of Lemma \ref{lemma-colimit-agrees} and
More on Morphisms, Lemmas
\ref{more-morphisms-lemma-pushout-along-thickening},
\ref{more-morphisms-lemma-equivalence-categories-schemes-over-pushout}, and
\ref{more-morphisms-lemma-equivalence-categories-schemes-over-pushout-flat}.
\end{proof}

\begin{lemma}
\label{lemma-pushout-along-closed-immersion-and-integral}
In More on Morphisms, Situation
\ref{more-morphisms-situation-pushout-along-closed-immersion-and-integral}
let $Y \amalg_Z X$ be the pushout in the category of schemes
(More on Morphisms, Proposition
\ref{more-morphisms-proposition-pushout-along-closed-immersion-and-integral}).
Then $Y \amalg_Z X$
is also a pushout in the category of algebraic spaces over $S$.
\end{lemma}

\begin{proof}
This is a consequence of Lemma \ref{lemma-colimit-agrees}, the proposition
mentioned in the lemma and More on Morphisms, Lemmas
\ref{more-morphisms-lemma-pushout-functor} and
\ref{more-morphisms-lemma-pushout-functor-equivalence-flat}.
Conditions (1) and (2) of Lemma \ref{lemma-colimit-agrees}
follow immediately. To see (3) and (4) note that an \'etale morphism
is locally quasi-finite and use that the equivalence of categories of
More on Morphisms, Lemma
\ref{more-morphisms-lemma-pushout-functor-equivalence-flat}
is constructed using the pushout construction of
More on Morphisms, Lemmas \ref{more-morphisms-lemma-pushout-functor}.
Minor details omitted.
\end{proof}

\begin{lemma}
\label{lemma-pushout-along-thickening}
Let $S$ be a scheme. Let $X \to X'$ be a thickening of algebraic spaces
over $S$ and let $X \to Y$ be an affine morphism of algebraic spaces over $S$.
Then there exists a pushout
$$
\xymatrix{
X \ar[r] \ar[d]_f
&
X' \ar[d]^{f'}
\\
Y \ar[r]
&
Y \amalg_X X'
}
$$
in the category of algebraic spaces over $S$. Moreover $Y' = Y \amalg_X X'$
is a thickening of $Y$ and
$$
\mathcal{O}_{Y'} = \mathcal{O}_Y \times_{f_*\mathcal{O}_X} f'_*\mathcal{O}_{X'}
$$
as sheaves on $Y_\etale = (Y')_\etale$.
\end{lemma}

\begin{proof}
Choose a scheme $V$ and a surjective \'etale morphism $V \to Y$.
Set $U = V \times_Y X$. This is a scheme affine over $V$ with a
surjective \'etale morphism $U \to X$. By More on Morphisms of Spaces,
Lemma \ref{spaces-more-morphisms-lemma-thickening-equivalence}
there exists a $U' \to X'$
surjective \'etale with $U = U' \times_{X'} X$. In particular the
morphism of schemes $U \to U'$ is a thickening too. Apply
More on Morphisms, Lemma \ref{more-morphisms-lemma-pushout-along-thickening}
to obtain a pushout $V' = V \amalg_U U'$ in the category of schemes.

\medskip\noindent
We repeat this procedure to construct a pushout
$$
\xymatrix{
U \times_X U \ar[d] \ar[r] & U' \times_{X'} U' \ar[d] \\
V \times_Y V \ar[r] & R'
}
$$
in the category of schemes. Consider the morphisms
$$
U \times_X U \to U \to V',\quad
U' \times_{X'} U' \to U' \to V',\quad
V \times_Y V \to V \to V'
$$
where we use the first projection in each case. Clearly these glue to
give a morphism $t' : R' \to V'$ which is \'etale by
More on Morphisms, Lemma
\ref{more-morphisms-lemma-equivalence-categories-schemes-over-pushout-flat}.
Similarly, we obtain $s' : R' \to V'$ \'etale.
The morphism $j' = (t', s') : R' \to V' \times_S V'$ is unramified
(as $t'$ is \'etale) and a monomorphism when restricted to the closed
subscheme $V \times_Y V \subset R'$. As $V \times_Y V \subset R'$ is
a thickening it follows that $j'$ is a monomorphism too. Finally, $j'$
is an equivalence relation as we can use the functoriality of pushouts
of schemes to construct a morphism $c' : R' \times_{s', V', t'} R' \to R'$
(details omitted). At this point we set $Y' = U'/R'$, see
Spaces, Theorem \ref{spaces-theorem-presentation}.

\medskip\noindent
We have morphisms $X' = U'/U' \times_{X'} U' \to V'/R' = Y'$ and
$Y = V/V \times_Y V \to V'/R' = Y'$.
By construction these fit into the commutative diagram
$$
\xymatrix{
X \ar[r] \ar[d]_f & X' \ar[d]^{f'} \\
Y \ar[r] & Y'
}
$$
Since $Y \to Y'$ is a thickening we have
$Y_\etale = (Y')_\etale$, see More on Morphisms of Spaces,
Lemma \ref{spaces-more-morphisms-lemma-thickening-equivalence}.
The commutativity of the diagram gives a map of sheaves
$$
\mathcal{O}_{Y'}
\longrightarrow
\mathcal{O}_Y \times_{f_*\mathcal{O}_X} f'_*\mathcal{O}_{X'}
$$
on this set. By More on Morphisms, Lemma
\ref{more-morphisms-lemma-pushout-along-thickening}
this map is an isomorphism when we restrict to
the scheme $V'$, hence it is an isomorphism.

\medskip\noindent
To finish the proof we show that the diagram above is a pushout in
the category of algebraic spaces. To see this, let $Z$ be an algebraic
space and let $a' : X' \to Z$ and $b : Y \to Z$ be morphisms of
algebraic spaces. By
Lemma \ref{lemma-pushout-along-thickening-schemes}
we obtain a unique morphism $h : V' \to Z$ fitting into the commutative
diagrams
$$
\vcenter{
\xymatrix{
U' \ar[d] \ar[r] & V' \ar[d]^h \\
X' \ar[r]^{a'} & Z
}
}
\quad\text{and}\quad
\vcenter{
\xymatrix{
V \ar[r] \ar[d] & V' \ar[d]^h \\
Y \ar[r]^b & Z
}
}
$$
The uniqueness shows that $h \circ t' = h \circ s'$. Hence $h$ factors
uniquely as $V' \to Y' \to Z$ and we win.
\end{proof}

\noindent
In the following lemma we use the fibre product of categories as
defined in
Categories, Example \ref{categories-example-2-fibre-product-categories}.

\begin{lemma}
\label{lemma-categories-spaces-over-pushout}
Let $S$ be a base scheme. Let $X \to X'$ be a thickening of algebraic spaces
over $S$ and let $X \to Y$ be an affine morphism of algebraic spaces over $S$.
Let $Y' = Y \amalg_X X'$ be the pushout (see
Lemma \ref{lemma-pushout-along-thickening}). Base change gives a functor
$$
F :
(\textit{Spaces}/Y')
\longrightarrow
(\textit{Spaces}/Y) \times_{(\textit{Spaces}/Y')} (\textit{Spaces}/X')
$$
given by $V' \longmapsto (V' \times_{Y'} Y, V' \times_{Y'} X', 1)$ which
sends $(\Sch/Y')$ into $(\Sch/Y) \times_{(\Sch/Y')} (\Sch/X')$.
The functor $F$ has a left adjoint
$$
G :
(\textit{Spaces}/Y) \times_{(\textit{Spaces}/Y')} (\textit{Spaces}/X')
\longrightarrow
(\textit{Spaces}/Y')
$$
which sends the triple $(V, U', \varphi)$ to the pushout
$V \amalg_{(V \times_Y X)} U'$ in the category of algebraic spaces over $S$.
The functor $G$ sends $(\Sch/Y) \times_{(\Sch/Y')} (\Sch/X')$ into $(\Sch/Y')$.
\end{lemma}

\begin{proof}
The proof is completely formal.
Since the morphisms $X \to X'$ and $X \to Y$ are representable it
is clear that $F$ sends $(\Sch/Y')$ into
$(\Sch/Y) \times_{(\Sch/Y')} (\Sch/X')$.

\medskip\noindent
Let us construct $G$. Let $(V, U', \varphi)$ be an object of the fibre
product category. Set $U = U' \times_{X'} X$. Note that $U \to U'$ is a
thickening. Since $\varphi : V \times_Y X \to U' \times_{X'} X = U$ is an
isomorphism we have a morphism $U \to V$ over $X \to Y$ which identifies
$U$ with the fibre product $X \times_Y V$. In particular $U \to V$ is
affine, see
Morphisms of Spaces, Lemma \ref{spaces-morphisms-lemma-base-change-affine}.
Hence we can apply Lemma \ref{lemma-pushout-along-thickening}
to get a pushout $V' = V \amalg_U U'$. Denote $V' \to Y'$ the morphism
we obtain in virtue of the fact that $V'$ is a pushout and because
we are given morphisms $V \to Y$ and $U' \to X'$ agreeing on $U$
as morphisms into $Y'$. Setting $G(V, U', \varphi) = V'$
gives the functor $G$.

\medskip\noindent
If $(V, U', \varphi)$ is an object of $(\Sch/Y) \times_{(\Sch/Y')} (\Sch/X')$
then $U = U' \times_{X'} X$ is a scheme too and we can form the pushout
$V' = V \amalg_U U'$ in the category of schemes by
More on Morphisms, Lemma \ref{more-morphisms-lemma-pushout-along-thickening}.
By Lemma \ref{lemma-pushout-along-thickening-schemes}
this is also a pushout in the category of schemes, hence
$G$ sends $(\Sch/Y) \times_{(\Sch/Y')} (\Sch/X')$ into $(\Sch/Y')$.

\medskip\noindent
Let us prove that $G$ is a left adjoint to $F$. Let $Z$ be an algebraic space
over $Y'$. We have to show that
$$
\Mor(V', Z) = \Mor((V, U', \varphi), F(Z))
$$
where the morphism sets are taking in their respective categories.
Let $g' : V' \to Z$ be a morphism. Denote $\tilde g$, resp.\ $\tilde f'$
the composition of $g'$ with the morphism $V \to V'$, resp.\ $U' \to V'$.
Base change $\tilde g$, resp.\ $\tilde f'$ by $Y \to Y'$, resp.\ $X' \to Y'$
to get a morphism $g : V \to Z \times_{Y'} Y$,
resp.\ $f' : U' \to Z \times_{Y'} X'$. Then $(g, f')$ is an element
of the right hand side of the equation above (details omitted).
Conversely, suppose that $(g, f') : (V, U', \varphi) \to F(Z)$ is an
element of the right hand side.
We may consider the composition $\tilde g : V \to Z$,
resp.\ $\tilde f' : U' \to Z$ of $g$, resp.\ $f$ by
$Z \times_{Y'} X' \to Z$, resp.\ $Z \times_{Y'} Y \to Z$.
Then $\tilde g$ and $\tilde f'$ agree as morphism from $U$ to $Z$.
By the universal property of pushout, we obtain a morphism
$g' : V' \to Z$, i.e., an element of the left hand side.
We omit the verification that these constructions are mutually inverse.
\end{proof}

\begin{lemma}
\label{lemma-diagram}
Let $S$ be a scheme. Let
$$
\xymatrix{
A \ar[r] \ar[d] & C \ar[d] \ar[r] & E \ar[d] \\
B \ar[r] & D \ar[r] & F
}
$$
be a commutative diagram of algebraic spaces over $S$.
Assume that $A, B, C, D$ and $A, B, E, F$ form cartesian squares
and that $B \to D$ is surjective \'etale.
Then $C, D, E, F$ is a cartesian square.
\end{lemma}

\begin{proof}
This is formal.
\end{proof}

\begin{lemma}
\label{lemma-equivalence-categories-spaces-over-pushout}
In the situation of Lemma \ref{lemma-categories-spaces-over-pushout}
the functor $F \circ G$ is isomorphic to the identity functor.
\end{lemma}

\begin{proof}
We will prove that $F \circ G$ is isomorphic to the identity by
reducing this to the corresponding statement of
More on Morphisms, Lemma
\ref{more-morphisms-lemma-equivalence-categories-schemes-over-pushout}.

\medskip\noindent
Choose a scheme $Y_1$ and a surjective \'etale morphism
$Y_1 \to Y$. Set $X_1 = Y_1 \times_Y X$. This is a scheme affine over
$Y_1$ with a surjective \'etale morphism $X_1 \to X$. By
More on Morphisms of Spaces,
Lemma \ref{spaces-more-morphisms-lemma-thickening-equivalence}
there exists a $X'_1 \to X'$
surjective \'etale with $X_1 = X_1' \times_{X'} X$. In particular the
morphism of schemes $X_1 \to X_1'$ is a thickening too. Apply
More on Morphisms, Lemma \ref{more-morphisms-lemma-pushout-along-thickening}
to obtain a pushout $Y_1' = Y_1 \amalg_{X_1} X_1'$ in the category of
schemes. In the proof of Lemma \ref{lemma-pushout-along-thickening}
we constructed
$Y'$ as a quotient of an \'etale equivalence relation on $Y_1'$
such that we get a commutative diagram
\begin{equation}
\label{equation-cube}
\vcenter{
\xymatrix{
& X \ar[rr] \ar'[d][dd] & & X' \ar[dd] \\
X_1 \ar[rr] \ar[dd] \ar[ru] & & X_1' \ar[dd] \ar[ru] & \\
& Y \ar'[r][rr] & & Y' \\
Y_1 \ar[rr] \ar[ru] & & Y_1' \ar[ru]
}
}
\end{equation}
where all squares except the front and back squares are cartesian
(the front and back squares are pushouts) and the northeast arrows
are surjective \'etale. Denote $F_1$, $G_1$ the
functors constructed in
More on Morphisms, Lemma
\ref{more-morphisms-lemma-equivalence-categories-schemes-over-pushout}
for the front square. Then the diagram of categories
$$
\xymatrix{
(\Sch/Y_1') \ar@<-1ex>[r]_-{F_1} \ar[d] &
(\Sch/Y_1) \times_{(\Sch/Y_1')} (\Sch/X_1') \ar[d] \ar@<-1ex>[l]_-{G_1} \\
(\textit{Spaces}/Y') \ar@<-1ex>[r]_-F &
(\textit{Spaces}/Y) \times_{(\textit{Spaces}/Y')} (\textit{Spaces}/X')
\ar@<-1ex>[l]_-G
}
$$
is commutative by simple considerations regarding base change functors
and the agreement of pushouts in schemes with pushouts in
spaces of Lemma \ref{lemma-pushout-along-thickening-schemes}.

\medskip\noindent
Let $(V, U', \varphi)$ be an object of
$(\textit{Spaces}/Y) \times_{(\textit{Spaces}/Y')} (\textit{Spaces}/X')$.
Denote $U = U' \times_{X'} X$ so that $G(V, U', \varphi) = V \amalg_U U'$.
Choose a scheme $V_1$ and a surjective \'etale morphism
$V_1 \to Y_1 \times_Y V$. Set $U_1 = V_1 \times_Y X$. Then
$$
U_1 = V_1 \times_Y X
\longrightarrow
(Y_1 \times_Y V) \times_Y X =
X_1 \times_Y V = X_1 \times_X X \times_Y V = X_1 \times_X U
$$
is surjective \'etale too. By
More on Morphisms of Spaces, Lemma
\ref{spaces-more-morphisms-lemma-thickening-equivalence}
there exists a thickening $U_1 \to U_1'$ and a surjective \'etale morphism
$U_1' \to X_1' \times_{X'} U'$ whose base change to $X_1 \times_X U$ is the
displayed morphism. At this point $(V_1, U'_1, \varphi_1)$ is an object of
$(\Sch/Y_1) \times_{(\Sch/Y_1')} (\Sch/X_1')$. In the proof of
Lemma \ref{lemma-pushout-along-thickening} we constructed
$G(V, U', \varphi) = V \amalg_U U'$ as a quotient of an \'etale equivalence
relation on $G_1(V_1, U_1', \varphi_1) = V_1 \amalg_{U_1} U_1'$
such that we get a commutative diagram
\begin{equation}
\label{equation-cube-over}
\vcenter{
\xymatrix{
& U \ar[rr] \ar'[d][dd] & & U' \ar[dd] \\
U_1 \ar[rr] \ar[dd] \ar[ru] & & U_1' \ar[dd] \ar[ru] & \\
& V \ar'[r][rr] & & G(V, U', \varphi) \\
V_1 \ar[rr] \ar[ru] & & G_1(V_1, U_1', \varphi_1) \ar[ru]
}
}
\end{equation}
where all squares except the front and back squares are cartesian
(the front and back squares are pushouts) and the northeast arrows
are surjective \'etale. In particular
$$
G_1(V_1, U_1', \varphi_1) \to G(V, U', \varphi)
$$
is surjective \'etale.

\medskip\noindent
Finally, we come to the proof of the lemma. We have to show that the adjunction
mapping $(V, U', \varphi) \to F(G(V, U', \varphi))$ is an isomorphism. We know
$(V_1, U_1', \varphi_1) \to F_1(G_1(V_1, U_1', \varphi_1))$ is an isomorphism
by More on Morphisms, Lemma
\ref{more-morphisms-lemma-equivalence-categories-schemes-over-pushout}.
Recall that $F$ and $F_1$ are given by base change.
Using the properties of (\ref{equation-cube-over})
and Lemma \ref{lemma-diagram}
we see that
$V \to G(V, U', \varphi) \times_{Y'} Y$ and
$U' \to G(V, U', \varphi) \times_{Y'} X'$ are isomorphisms, i.e.,
$(V, U', \varphi) \to F(G(V, U', \varphi))$ is an isomorphism.
\end{proof}

\begin{lemma}
\label{lemma-space-over-pushout-flat-modules}
Let $S$ be a base scheme.
Let $X \to X'$ be a thickening of algebraic spaces over $S$
and let $X \to Y$ be an affine morphism of algebraic spaces over $S$.
Let $Y' = Y \amalg_X X'$ be the pushout
(see Lemma \ref{lemma-pushout-along-thickening}).
Let $V' \to Y'$ be a morphism of algebraic spaces over $S$. Set
$V = Y \times_{Y'} V'$, $U' = X' \times_{Y'} V'$, and $U = X \times_{Y'} V'$.
There is an equivalence of categories between
\begin{enumerate}
\item quasi-coherent $\mathcal{O}_{V'}$-modules flat over $Y'$, and
\item the category of triples $(\mathcal{G}, \mathcal{F}', \varphi)$ where
\begin{enumerate}
\item $\mathcal{G}$ is a quasi-coherent $\mathcal{O}_V$-module flat over $Y$,
\item $\mathcal{F}'$ is a quasi-coherent $\mathcal{O}_{U'}$-module flat
over $X$, and
\item $\varphi : (U \to V)^*\mathcal{G} \to (U \to U')^*\mathcal{F}'$
is an isomorphism of $\mathcal{O}_U$-modules.
\end{enumerate}
\end{enumerate}
The equivalence maps $\mathcal{G}'$ to
$((V \to V')^*\mathcal{G}', (U' \to V')^*\mathcal{G}', can)$.
Suppose $\mathcal{G}'$ corresponds to the triple
$(\mathcal{G}, \mathcal{F}', \varphi)$. Then
\begin{enumerate}
\item[(a)] $\mathcal{G}'$ is a finite type $\mathcal{O}_{V'}$-module if and
only if $\mathcal{G}$ and $\mathcal{F}'$ are finite type
$\mathcal{O}_Y$ and $\mathcal{O}_{U'}$-modules.
\item[(b)] if $V' \to Y'$ is locally of finite presentation, then
$\mathcal{G}'$ is an $\mathcal{O}_{V'}$-module of finite
presentation if and only if $\mathcal{G}$ and $\mathcal{F}'$ are
$\mathcal{O}_Y$ and $\mathcal{O}_{U'}$-modules of finite presentation.
\end{enumerate}
\end{lemma}

\begin{proof}
A quasi-inverse functor assigns to the triple
$(\mathcal{G}, \mathcal{F}', \varphi)$ the fibre product
$$
(V \to V')_*\mathcal{G}
\times_{(U \to V')_*\mathcal{F}}
(U' \to V')_*\mathcal{F}'
$$
where $\mathcal{F} = (U \to U')^*\mathcal{F}'$. This works, because on
affines \'etale over $V'$ and $Y'$ we recover the equivalence of
More on Algebra, Lemma
\ref{more-algebra-lemma-relative-flat-module-over-fibre-product}.
Details omitted.

\medskip\noindent
Parts (a) and (b) reduce by \'etale localization
(Properties of Spaces, Section
\ref{spaces-properties-section-properties-modules})
to the case where $V'$ and $Y'$ are affine in which case the result
follows from
More on Algebra, Lemmas
\ref{more-algebra-lemma-relative-finite-module-over-fibre-product} and
\ref{more-algebra-lemma-relative-finitely-presented-module-over-fibre-product}.
\end{proof}


\begin{lemma}
\label{lemma-equivalence-categories-spaces-pushout-flat}
In the situation of
Lemma \ref{lemma-equivalence-categories-spaces-over-pushout}.
If $V' = G(V, U', \varphi)$ for some triple $(V, U', \varphi)$, then
\begin{enumerate}
\item $V' \to Y'$ is locally of finite type if and only if $V \to Y$ and
$U' \to X'$ are locally of finite type,
\item $V' \to Y'$ is flat if and only if $V \to Y$ and $U' \to X'$ are flat,
\item $V' \to Y'$ is flat and locally of finite presentation if and only if
$V \to Y$ and $U' \to X'$ are flat and locally of finite presentation,
\item $V' \to Y'$ is smooth if and only if $V \to Y$ and $U' \to X'$ are smooth,
\item $V' \to Y'$ is \'etale if and only if $V \to Y$ and $U' \to X'$
are \'etale, and
\item add more here as needed.
\end{enumerate}
If $W'$ is flat over $Y'$, then the adjunction mapping
$G(F(W')) \to W'$ is an isomorphism. Hence $F$ and $G$ define mutually
quasi-inverse functors between the category of spaces flat over $Y'$
and the category of triples $(V, U', \varphi)$ with $V \to Y$
and $U' \to X'$ flat.
\end{lemma}

\begin{proof}
Choose a diagram (\ref{equation-cube}) as in the proof of
Lemma \ref{lemma-equivalence-categories-spaces-over-pushout}.

\medskip\noindent
Proof of (1) -- (5). Let $(V, U', \varphi)$ be an object of
$(\textit{Spaces}/Y) \times_{(\textit{Spaces}/Y')} (\textit{Spaces}/X')$.
Construct a diagram (\ref{equation-cube-over}) as in the proof of
Lemma \ref{lemma-equivalence-categories-spaces-over-pushout}.
Then the base change of $G(V, U', \varphi) \to Y'$ to
$Y'_1$ is $G_1(V_1, U_1', \varphi_1) \to Y_1'$. Hence (1) -- (5)
follow immediately from the corresponding statements of
More on Morphisms, Lemma
\ref{more-morphisms-lemma-equivalence-categories-schemes-over-pushout-flat}
for schemes.

\medskip\noindent
Suppose that $W' \to Y'$ is flat. Choose a scheme $W'_1$ and a surjective
\'etale morphism $W'_1 \to Y_1' \times_{Y'} W'$. Observe that $W'_1 \to W'$
is surjective \'etale as a composition of surjective \'etale morphisms. We
know that $G_1(F_1(W_1')) \to W_1'$ is an isomorphism by
More on Morphisms, Lemma
\ref{more-morphisms-lemma-equivalence-categories-schemes-over-pushout-flat}
applied to $W'_1$ over $Y'_1$ and the front of the diagram (with functors
$G_1$ and $F_1$ as in the proof of
Lemma \ref{lemma-equivalence-categories-spaces-over-pushout}).
Then the construction of $G(F(W'))$ (as a pushout, i.e.,
as constructed in Lemma \ref{lemma-pushout-along-thickening}) shows that
$G_1(F_1(W'_1)) \to G(F(W))$ is surjective \'etale. Whereupon we conclude
that $G(F(W)) \to W$ is \'etale, see for example
Properties of Spaces, Lemma \ref{spaces-properties-lemma-etale-local}.
But $G(F(W)) \to W$ is an isomorphism on underlying reduced
algebraic spaces (by construction), hence it is an isomorphism.
\end{proof}









\section{Pushouts and derived categories}
\label{section-pushouts-derived}

\noindent
In this section we discuss the behaviour of the derived
category of modules under pushouts.

\begin{lemma}
\label{lemma-pushout-along-thickening-derived}
Let $S$ be a scheme. Consider a pushout
$$
\xymatrix{
X \ar[r]_i \ar[d]_f & X' \ar[d]^{f'}
\\
Y \ar[r]^j & Y'
}
$$
in the category of algebraic spaces over $S$
as in Lemma \ref{lemma-pushout-along-thickening}.
Assume $i$ is a thickening. Then the essential
image of the functor\footnote{All functors given by derived pullback.}
$$
D(\mathcal{O}_{Y'}) \longrightarrow
D(\mathcal{O}_Y) \times_{D(\mathcal{O}_X)} D(\mathcal{O}_{X'})
$$
contains every triple $(M, K', \alpha)$ where $M \in D(\mathcal{O}_Y)$
and $K' \in D(\mathcal{O}_{X'})$ are pseudo-coherent.
\end{lemma}

\begin{proof}
Let $(M, K', \alpha)$ be an object of the target of the functor
of the lemma. Here $\alpha : Lf^*M \to Li^*K'$
is an isomorphism which is adjoint to a map $\beta : M \to Rf_*Li^*K'$.
Thus we obtain maps
$$
Rj_*M \xrightarrow{Rj_*\beta}
Rj_*Rf_*Li^*K' = Rf'_*Ri_*Li^*K' \leftarrow Rf'_*K'
$$
where the arrow pointing left comes from $K' \to Ri_*Li^*K'$.
Choose a distinguished triangle
$$
M' \to Rj_*M \oplus Rf'_*K' \to Rj_*Rf_*Li^*K' \to M'[1]
$$
in $D(\mathcal{O}_{Y'})$. The first arrow defines canonical maps
$Lj^*M' \to M$ and $L(f')^*M' \to K'$ compatible with $\alpha$.
Thus it suffices to show that the maps
$Lj^*M' \to M$ and $L(f')^*M' \to K$ are isomorphisms.
This we may check \'etale locally on $Y'$, hence we may
assume $Y'$ is \'etale.

\medskip\noindent
Assume $Y'$ affine and $M \in D(\mathcal{O}_Y)$
and $K' \in D(\mathcal{O}_{X'})$ are pseudo-coherent.
Say our pushout corresponds to the fibre product
$$
\xymatrix{
B & B' \ar[l] \\
A \ar[u] & A' \ar[l] \ar[u]
}
$$
of rings where $B' \to B$ is surjective with locally nilpotent kernel $I$
(and hence $A' \to A$ is surjective with locally nilpotent kernel $I$ as well).
The assumption on $M$ and $K'$ imply that $M$ comes from a pseudo-coherent
object of $D(A)$ and $K'$ comes from a pseudo-coherent object of $D(B')$, see
Derived Categories of Spaces, Lemmas
\ref{spaces-perfect-lemma-pseudo-coherent},
\ref{spaces-perfect-lemma-derived-quasi-coherent-small-etale-site}, and
\ref{spaces-perfect-lemma-descend-pseudo-coherent}
and
Derived Categories of Schemes, Lemma
\ref{perfect-lemma-affine-compare-bounded} and
\ref{perfect-lemma-pseudo-coherent-affine}.
Moreover, pushforward and derived pullback agree with the
corresponding operations on derived categories of modules, see
Derived Categories of Spaces, Remark
\ref{spaces-perfect-remark-match-total-direct-images}
and
Derived Categories of Schemes, Lemmas
\ref{perfect-lemma-quasi-coherence-pushforward} and
\ref{perfect-lemma-quasi-coherence-pullback}.
This reduces us to the statement formulated in the next paragraph.
(To be sure these references show
the object $M'$ lies $D_\QCoh(\mathcal{O}_{Y'})$
as this is a triangulated subcategory of $D(\mathcal{O}_{Y'})$.)

\medskip\noindent
Given a diagram of rings as above and a triple
$(M, K', \alpha)$ where $M \in D(A)$, $K' \in D(B')$ are
pseudo-coherent and
$\alpha : M \otimes_A^\mathbf{L} B \to K' \otimes_{B'}^\mathbf{L} B$
is an isomorphism suppose we have distinguished triangle
$$
M' \to M \oplus K' \to K' \otimes_{B'}^\mathbf{L} B \to M'[1]
$$
in $D(A')$. Goal: show that the induced maps
$M' \otimes_{A'}^\mathbf{L} A \to M$ and
$M' \otimes_{A'}^\mathbf{L} B' \to K'$ are isomorphisms.
To do this, choose a bounded above complex
$E^\bullet$ of finite free $A$-modules representing $M$.
Since $(B', I)$ is a henselian pair
(More on Algebra, Lemma \ref{more-algebra-lemma-locally-nilpotent-henselian})
with $B = B'/I$ we may apply More on Algebra, Lemma
\ref{more-algebra-lemma-lift-complex-finite-projectives}
to see that there exists a bounded above complex $P^\bullet$
of free $B'$-modules such that $\alpha$ is represented
by an isomorphism $E^\bullet \otimes_A B \cong P^\bullet \otimes_{B'} B$.
Then we can consider the short exact sequence
$$
0 \to L^\bullet \to
E^\bullet \oplus P^\bullet \to P^\bullet \otimes_{B'} B \to 0
$$
of complexes of $B'$-modules.
More on Algebra, Lemma
\ref{more-algebra-lemma-finitely-presented-module-over-fibre-product}
implies $L^\bullet$ is a bounded above complex of
finite projective $A'$-modules
(in fact it is rather easy to show directly that $L^n$ is finite free
in our case) and that we have
$L^\bullet \otimes_{A'} A = E^\bullet$ and
$L^\bullet \otimes_{A'} B' = P^\bullet$.
The short exact sequence gives a distinguished triangle
$$
L^\bullet \to M \oplus K' \to K' \otimes_{B'}^\mathbf{L} B \to (L^\bullet)[1]
$$
in $D(A')$ (Derived Categories, Section
\ref{derived-section-canonical-delta-functor}) which is isomorphic
to the given distinguished triangle by general properties of
triangulated categories (Derived Categories, Section
\ref{derived-section-elementary-results}). In other words, $L^\bullet$
represents $M'$ compatibly with the given maps. Thus the maps
$M' \otimes_{A'}^\mathbf{L} A \to M$ and
$M' \otimes_{A'}^\mathbf{L} B' \to K'$ are
isomorphisms because we just saw that the corresponding
thing is true for $L^\bullet$.
\end{proof}







\section{Constructing elementary distinguished squares}
\label{section-elementary-dsitnguished-squares}

\noindent
Elementary distinguished squares were defined in
Derived Categories of Spaces, Section \ref{spaces-perfect-section-induction}.

\begin{lemma}
\label{lemma-elementary-distinguished-square-pushout}
Let $S$ be a scheme. Let $(U \subset W, f : V \to W)$ be
an elementary distinguished square. Then
$$
\xymatrix{
U \times_W V \ar[r] \ar[d] &
V \ar[d]^f \\
U \ar[r] & W
}
$$
is a pushout in the category of algebraic spaces over $S$.
\end{lemma}

\begin{proof}
Observe that $U \amalg V \to W$ is a surjective \'etale morphism.
The fibre product
$$
(U \amalg V) \times_W (U \amalg V)
$$
is the disjoint union of four pieces, namely
$U = U \times_W U$, $U \times_W V$, $V \times_W U$,
and $V \times_W V$.
There is a surjective \'etale morphism
$$
V \amalg (U \times_W V) \times_U (U \times_W V) \longrightarrow V \times_W V
$$
because $f$ induces an isomorphism over $W \setminus U$
(part of the definition of being an elementary distinguished square).
Let $B$ be an algebraic space over $S$ and let
$g : V \to B$ and $h : U \to B$ be morphisms over
$S$ which agree after restricting to $U \times_W V$.
Then the description of
$(U \amalg V) \times_W (U \amalg V)$ given above
shows that $h \amalg g : U \amalg V \to B$
equalizes the two projections. Since $B$ is a sheaf
for the \'etale topology we obtain a unique
factorization of $h \amalg g$ through $W$ as desired.
\end{proof}

\begin{lemma}
\label{lemma-construct-elementary-distinguished-square}
Let $S$ be a scheme. Let $V$, $U$ be algebraic spaces over $S$.
Let $V' \subset V$ be an open subspace and let $f' : V' \to U$ be a
separated \'etale morphism of algebraic spaces over $S$.
Then there exists a pushout
$$
\xymatrix{
V' \ar[r] \ar[d] &
V \ar[d]^f \\
U \ar[r] & W
}
$$
in the category of algebraic spaces over $S$ and moreover
$(U \subset W, f : V \to W)$ is an elementary distinguished square.
\end{lemma}

\begin{proof}
We are going to construct $W$ as the quotient of an \'etale
equivalence relation $R$ on $U \amalg V$. Such a quotient is an
algebraic space for example by
Bootstrap, Theorem \ref{bootstrap-theorem-final-bootstrap}.
Moreover, the proof of
Lemma \ref{lemma-elementary-distinguished-square-pushout} tells us to take
$$
R = U \amalg V' \amalg V' \amalg V \amalg
(V' \times_U V' \setminus \Delta_{V'/U}(V'))
$$
Since we assumed $V' \to U$ is separated, the image of
$\Delta_{V'/U}$ is closed and hence the complement is an
open substack. The morphism $j : R \to (U \amalg V) \times_S (U \amalg V)$
is given by
$$
u,\ v',\ v',\ v,\ (v'_1, v'_2) \mapsto
(u, u),\ (f'(v'), v'),\ (v', f'(v')),\ (v, v),\ (v'_1, v'_2)
$$
with obvious notation. It is immediately verified that this is a
monomorphism, an equivalence relation, and that the induced morphisms
$s, t : R \to U \amalg V$ are \'etale. Let
$W = (U \amalg V)/R$ be the quotient algebraic space.
We obtain a commutative diagram as in the statement of the lemma.
To finish the proof it suffices to show that this diagram is
an elementary distinguished square, since then
Lemma \ref{lemma-elementary-distinguished-square-pushout}
implies that it is a pushout.
Thus we have to show that $U \to W$ is open and that
$f$ is \'etale and is an isomorphism over $W \setminus U$.
This follows from the choice of $R$; we omit the details.
\end{proof}






\section{Formal glueing of quasi-coherent modules}
\label{section-formal-glueing}

\noindent
This section is the analogue of
More on Algebra, Section \ref{more-algebra-section-formal-glueing}.
In the case of morphisms of schemes, the result can be found in
the paper by Joyet \cite{Joyet}; this is a good place to start reading.
For a discussion of applications to descent problems for stacks, see the
paper by Moret-Bailly \cite{MB}. In the case of an affine
morphism of schemes there is a statement in the appendix of the paper
\cite{Ferrand-Raynaud} but one needs to add the hypothesis
that the closed subscheme is cut out by a finitely generated
ideal (as in the paper by Joyet) since otherwise the result does not hold.
A generalization of this material to (higher) derived categories
with potential applications to nonflat situations
can be found in \cite[Section 5]{Bhatt-Algebraize}.

\medskip\noindent
We start with a lemma on abelian sheaves supported on closed subsets.

\begin{lemma}
\label{lemma-stalk-pushforward-with-support}
Let $S$ be a scheme. Let $f : Y \to X$ be a morphism of algebraic spaces
over $S$. Let $Z \subset X$ closed subspace such that $f^{-1}Z \to Z$ is
integral and universally injective. Let $\overline{y}$ be a geometric point
of $Y$ and $\overline{x} = f(\overline{y})$. We have
$$
(Rf_*Q)_{\overline{x}} = Q_{\overline{y}}
$$
in $D(\textit{Ab})$ for any object $Q$ of $D(Y_\etale)$ supported
on $|f^{-1}Z|$.
\end{lemma}

\begin{proof}
Consider the commutative diagram of algebraic spaces
$$
\xymatrix{
f^{-1}Z \ar[r]_{i'} \ar[d]_{f'} & Y \ar[d]_f \\
Z \ar[r]^i & X
}
$$
By Cohomology of Spaces, Lemma
\ref{spaces-cohomology-lemma-complexes-with-support-on-closed} we can write
$Q = Ri'_*K'$ for some object $K'$ of $D(f^{-1}Z_\etale)$.
By Morphisms of Spaces, Lemma
\ref{spaces-morphisms-lemma-integral-universally-injective-push-pull}
we have $K' = (f')^{-1}K$ with $K = Rf'_*K'$.
Then we have $Rf_*Q = Rf_*Ri'_*K' = Ri_*Rf'_*K' = Ri_*K$.
Let $\overline{z}$ be the geometric point of $Z$ corresponding
to $\overline{x}$ and let $\overline{z}'$ be the geometric point
of $f^{-1}Z$ corresponding to $\overline{y}$. We obtain
the result of the lemma as follows
$$
Q_{\overline{y}} = (Ri'_*K')_{\overline{y}} = K'_{\overline{z}'} =
(f')^{-1}K_{\overline{z}'} = K_{\overline{z}} = Ri_*K_{\overline{x}} =
Rf_*Q_{\overline{x}}
$$
The middle equality holds because of the description of the stalk
of a pullback given in
Properties of Spaces, Lemma \ref{spaces-properties-lemma-stalk-pullback}.
\end{proof}

\begin{lemma}
\label{lemma-stalk-formal-glueing}
Let $S$ be a scheme. Let $f : Y \to X$ be a morphism of algebraic spaces
over $S$. Let $Z \subset X$ closed subspace such that $f^{-1}Z \to Z$ is
integral and universally injective. Let $\overline{y}$ be a geometric point
of $Y$ and $\overline{x} = f(\overline{y})$. Let $\mathcal{G}$
be an abelian sheaf on $Y$. Then the map of two term complexes
$$
\left(f_*\mathcal{G}_{\overline{x}} \to
(f \circ j')_*(\mathcal{G}|_V)_{\overline{x}}\right)
\longrightarrow
\left(\mathcal{G}_{\overline{y}} \to j'_*(\mathcal{G}|_V)_{\overline{y}}\right)
$$
induces an isomorphism on kernels and an injection on cokernels.
Here $V = Y \setminus f^{-1}Z$ and $j' : V \to Y$ is the inclusion.
\end{lemma}

\begin{proof}
Choose a distinguished triangle
$$
\mathcal{G} \to Rj'_*\mathcal{G}|_V \to Q \to \mathcal{G}[1]
$$
n $D(Y_\etale)$. The cohomology sheaves of $Q$
are supported on $|f^{-1}Z|$. We apply $Rf_*$ and we obtain
$$
Rf_*\mathcal{G} \to Rf_*Rj'_*\mathcal{G}|_V \to Rf_*Q
\to Rf_*\mathcal{G}[1]
$$
Taking stalks at $\overline{x}$ we obtain an exact sequence
$$
0 \to
(R^{-1}f_*Q)_{\overline{x}} \to
f_*\mathcal{G}_{\overline{x}} \to
(f \circ j')_*(\mathcal{G}|_V)_{\overline{x}} \to
(R^0f_*Q)_{\overline{x}}
$$
We can compare this with the exact sequence
$$
0 \to
H^{-1}(Q)_{\overline{y}} \to
\mathcal{G}_{\overline{y}} \to
j'_*(\mathcal{G}|_V)_{\overline{y}} \to
H^0(Q)_{\overline{y}}
$$
Thus we see that the lemma follows because
$Q_{\overline{y}} = Rf_*Q_{\overline{x}}$ by
Lemma \ref{lemma-stalk-pushforward-with-support}.
\end{proof}

\begin{lemma}
\label{lemma-stalk-of-pushforward}
Let $S$ be a scheme. Let $X$ be an algebraic space over $S$.
Let $f : Y \to X$ be a quasi-compact and quasi-separated morphism.
Let $\overline{x}$ be a geometric point of $X$ and let
$\Spec(\mathcal{O}_{X, \overline{x}}) \to X$
be the canonical morphism. For a quasi-coherent module
$\mathcal{G}$ on $Y$ we have
$$
f_*\mathcal{G}_{\overline{x}} =
\Gamma(Y \times_X \Spec(\mathcal{O}_{X, \overline{x}}), p^*\mathcal{F})
$$
where $p : Y \times_X \Spec(\mathcal{O}_{X, \overline{x}}) \to Y$
is the projection.
\end{lemma}

\begin{proof}
Observe that $f_*\mathcal{G}_{\overline{x}} =
\Gamma(\Spec(\mathcal{O}_{X, \overline{x}}), h^*f_*\mathcal{G})$
where $h : \Spec(\mathcal{O}_{X, \overline{x}}) \to X$.
Hence the result is true because $h$ is flat so that
Cohomology of Spaces, Lemma
\ref{spaces-cohomology-lemma-flat-base-change-cohomology}
applies.
\end{proof}

\begin{lemma}
\label{lemma-stalk-of-module-with-support}
Let $S$ be a scheme. Let $X$ be an algebraic space over $S$.
Let $i : Z \to X$ be a closed immersion of finite presentation.
Let $Q \in D_\QCoh(\mathcal{O}_X)$ be supported on $|Z|$.
Let $\overline{x}$ be a geometric point of $X$ and let
$I_{\overline{x}} \subset \mathcal{O}_{X, \overline{x}}$ be the stalk of
the ideal sheaf of $Z$. Then the cohomology modules
$H^n(Q_{\overline{x}})$ are $I_{\overline{x}}$-power torsion
(see More on Algebra, Definition
\ref{more-algebra-definition-f-power-torsion}).
\end{lemma}

\begin{proof}
Choose an affine scheme $U$ and an \'etale morphism $U \to X$ such
that $\overline{x}$ lifts to a geometric point $\overline{u}$
of $U$. Then we can replace $X$ by $U$, $Z$ by $U \times_X Z$,
$Q$ by the restriction $Q|_U$, and $\overline{x}$ by $\overline{u}$.
Thus we may assume that $X = \Spec(A)$ is affine. Let $I \subset A$
be the ideal defining $Z$. Since $i : Z \to X$ is of finite presentation,
the ideal $I = (f_1, \ldots, f_r)$ is finitely generated.
The object $Q$ comes from a complex of $A$-modules $M^\bullet$, see
Derived Categories of Spaces, Lemma
\ref{spaces-perfect-lemma-derived-quasi-coherent-small-etale-site}
and
Derived Categories of Schemes, Lemma
\ref{perfect-lemma-affine-compare-bounded}.
Since the cohomology sheaves of $Q$ are supported on $Z$
we see that the localization $M^\bullet_f$ is acyclic for each $f \in I$.
Take $x \in H^p(M^\bullet)$. By the above we can find $n_i$ such
that $f_i^{n_i} x = 0$ in $H^p(M^\bullet)$ for each $i$.
Then with $n = \sum n_i$ we see that $I^n$ annihilates $x$.
Thus $H^p(M^\bullet)$ is $I$-power torsion. Since the ring
map $A \to \mathcal{O}_{X, \overline{x}}$ is flat and since
$I_{\overline{x}} = I\mathcal{O}_{X, \overline{x}}$ we conclude.
\end{proof}

\begin{lemma}
\label{lemma-formal-glueing-on-closed}
Let $S$ be a scheme. Let $f : Y \to X$ be a morphism of algebraic spaces
over $S$. Let $Z \subset X$ be a closed subspace. Assume $f^{-1}Z \to Z$
is an isomorphism and that $f$ is flat in every point of $f^{-1}Z$. For any
$Q$ in $D_\QCoh(\mathcal{O}_Y)$ supported on $|f^{-1}Z|$ we have
$Lf^*Rf_*Q = Q$.
\end{lemma}

\begin{proof}
We show the canonical map $Lf^*Rf_*Q \to Q$ is an isomorphism
by checking on stalks at $\overline{y}$. If $\overline{y}$ is not
in $f^{-1}Z$, then both sides are zero and the result is true.
Assume the image $\overline{x}$ of $\overline{y}$ is in $Z$.
By Lemma \ref{lemma-stalk-pushforward-with-support} we have
$Rf_*Q_{\overline{x}} = Q_{\overline{y}}$ and since $f$ is flat
at $\overline{y}$ we see that
$$
(Lf^*Rf_*Q)_{\overline{y}} =
(Rf_*Q)_{\overline{x}}
\otimes_{\mathcal{O}_{X, \overline{x}}}
\mathcal{O}_{Y, \overline{y}} =
Q_{\overline{y}} \otimes_{\mathcal{O}_{X, \overline{x}}}
\mathcal{O}_{Y, \overline{y}}
$$
Thus we have to check that the canonical map
$$
Q_{\overline{y}} \otimes_{\mathcal{O}_{X, \overline{x}}}
\mathcal{O}_{Y, \overline{y}}
\longrightarrow Q_{\overline{y}}
$$
is an isomorphism in the derived category. Let
$I_{\overline{x}} \subset \mathcal{O}_{X, \overline{x}}$ be the
stalk of the ideal sheaf defining $Z$. Since $Z \to X$ is locally of
finite presentation this ideal is finitely generated and the
cohomology groups of $Q_{\overline{y}}$
are $I_{\overline{y}} = I_{\overline{x}}\mathcal{O}_{Y, \overline{y}}$-power
torsion by Lemma \ref{lemma-stalk-of-module-with-support} applied to $Q$ on $Y$.
It follows that they are also $I_{\overline{x}}$-power torsion.
The ring map
$\mathcal{O}_{X, \overline{x}} \to \mathcal{O}_{Y, \overline{y}}$
is flat and induces an isomorphism after dividing by
$I_{\overline{x}}$ and $I_{\overline{y}}$ because we assumed
that $f^{-1}Z \to Z$ is an isomorphism. Hence we see that
the cohomology modules of
$Q_{\overline{y}} \otimes_{\mathcal{O}_{X, \overline{x}}}
\mathcal{O}_{Y, \overline{y}}$
are equal to the cohomology modules of $Q_{\overline{y}}$ by
More on Algebra, Lemma \ref{more-algebra-lemma-neighbourhood-isomorphism}
which finishes the proof.
\end{proof}

\begin{situation}
\label{situation-formal-glueing}
Here $S$ is a base scheme, $f : Y \to X$ is a quasi-compact
and quasi-separated morphism of algebraic spaces over $S$, and
$Z \to X$ is a closed immersion of finite presentation. We assume that
$f^{-1}(Z) \to Z$ is an isomorphism and that $f$ is flat in every
point $x \in |f^{-1}Z|$. We set $U = X \setminus Z$ and
$V = Y \setminus f^{-1}(Z)$.
Picture
$$
\xymatrix{
V \ar[r]_{j'} \ar[d]_{f|_V} & Y \ar[d]^f \\
U \ar[r]^j & X
}
$$
\end{situation}

\noindent
In Situation \ref{situation-formal-glueing} we define
$\textit{QCoh}(Y \to X, Z)$ as the category of
triples $(\mathcal{H}, \mathcal{G}, \varphi)$ where
$\mathcal{H}$ is a quasi-coherent sheaf of
$\mathcal{O}_U$-modules, $\mathcal{G}$ is a quasi-coherent sheaf
of $\mathcal{O}_Y$-modules, and
$\varphi : f^*\mathcal{H} \to \mathcal{G}|_V$ is an isomorphism
of $\mathcal{O}_V$-modules. There is a canonical
functor
\begin{equation}
\label{equation-formal-glueing-modules}
\QCoh(\mathcal{O}_X) \longrightarrow \textit{QCoh}(Y \to X, Z)
\end{equation}
which maps $\mathcal{F}$ to the system
$(\mathcal{F}|_U, f^*\mathcal{F}, can)$.
By analogy with the proof given in the affine case, we construct
a functor in the opposite direction. To an object
$(\mathcal{H}, \mathcal{G}, \varphi)$ we assign the $\mathcal{O}_X$-module
\begin{equation}
\label{equation-reverse}
\Ker(j_*\mathcal{H} \oplus f_*\mathcal{G} \to (f \circ j')_*\mathcal{G}|_V)
\end{equation}
Observe that $j$ and $j'$ are quasi-compact morphisms as
$Z \to X$ is of finite presentation. Hence $f_*$, $j_*$, and $(f \circ j')_*$
transform quasi-coherent modules into quasi-coherent modules
(Morphisms of Spaces, Lemma \ref{spaces-morphisms-lemma-pushforward}).
Thus the module (\ref{equation-reverse}) is quasi-coherent.

\begin{lemma}
\label{lemma-adjoint}
In Situation \ref{situation-formal-glueing}.
The functor (\ref{equation-reverse}) is right adjoint to
the functor (\ref{equation-formal-glueing-modules}).
\end{lemma}

\begin{proof}
This follows easily from the adjointness of $f^*$ to $f_*$
and $j^*$ to $j_*$. Details omitted.
\end{proof}

\begin{lemma}
\label{lemma-reverse-commutes-with-flat-base-change}
In Situation \ref{situation-formal-glueing}.
Let $X' \to X$ be a flat morphism of algebraic spaces.
Set $Z' = X' \times_X Z$ and $Y' = X' \times_X Y$.
The pullbacks $\QCoh(\mathcal{O}_X) \to \QCoh(\mathcal{O}_{X'})$
and $\QCoh(Y \to X, Z) \to \QCoh(Y' \to X', Z')$ are compatible
with the functors (\ref{equation-reverse}) and
\ref{equation-formal-glueing-modules}).
\end{lemma}

\begin{proof}
This is true because pullback commutes with pullback and because
flat pullback commutes with pushforward along quasi-compact
and quasi-separated morphisms, see
Cohomology of Spaces, Lemma
\ref{spaces-cohomology-lemma-flat-base-change-cohomology}.
\end{proof}

\begin{proposition}
\label{proposition-formal-glueing-modules}
In Situation \ref{situation-formal-glueing} the functor
(\ref{equation-formal-glueing-modules}) is an equivalence
with quasi-inverse given by (\ref{equation-reverse}).
\end{proposition}

\begin{proof}
We first treat the special case where $X$ and $Y$ are affine schemes
and where the morphism $f$ is flat. Say $X = \Spec(R)$ and $Y = \Spec(S)$.
Then $f$ corresponds to a flat ring map $R \to S$. Moreover, $Z \subset X$
is cut out by a finitely generated ideal $I \subset R$. Choose generators
$f_1, \ldots, f_t \in I$. By the description of quasi-coherent modules
in terms of modules
(Schemes, Section \ref{schemes-section-quasi-coherent-affine}),
we see that the category $\textit{QCoh}(Y \to X, Z)$
is canonically equivalent to the category
$\text{Glue}(R \to S, f_1, \ldots, f_t)$
of More on Algebra, Remark \ref{more-algebra-remark-glueing-data}
such that the functors
(\ref{equation-formal-glueing-modules}) and (\ref{equation-reverse})
correspond to the functors $\text{Can}$ and $H^0$.
Hence the result follows from
More on Algebra, Proposition \ref{more-algebra-proposition-equivalence}
in this case.

\medskip\noindent
We return to the general case.
Let $\mathcal{F}$ be a quasi-coherent module on $X$.
We will show that
$$
\alpha :
\mathcal{F}
\longrightarrow
\Ker\left(j_*\mathcal{F}|_U \oplus f_*f^*\mathcal{F} \to
(f \circ j')_*f^*\mathcal{F}|_V\right)
$$
is an isomorphism. Let $(\mathcal{H}, \mathcal{G}, \varphi)$
be an object of $\QCoh(Y \to X, Z)$. We will show that
$$
\beta :
f^*\Ker\left(
j_*\mathcal{H} \oplus f_*\mathcal{G} \to (f \circ j')_*\mathcal{G}|_V
\right)
\longrightarrow
\mathcal{G}
$$
and
$$
\gamma :
j^*\Ker\left(
j_*\mathcal{H} \oplus f_*\mathcal{G} \to (f \circ j')_*\mathcal{G}|_V
\right)
\longrightarrow
\mathcal{H}
$$
are isomorphisms. To see these statements are true it suffices to
look at stalks. Let $\overline{y}$ be a geometric point of $Y$ mapping
to the geometric point $\overline{x}$ of $X$.

\medskip\noindent
Fix an object $(\mathcal{H}, \mathcal{G}, \varphi)$ of $\QCoh(Y \to X, Z)$.
By Lemma \ref{lemma-stalk-formal-glueing}
and a diagram chase (omitted) the canonical map
$$
\Ker(j_*\mathcal{H} \oplus f_*\mathcal{G} \to
(f \circ j')_*\mathcal{G}|_V)_{\overline{x}}
\longrightarrow
\Ker(
j_*\mathcal{H}_{\overline{x}} \oplus \mathcal{G}_{\overline{y}}
\to
j'_*\mathcal{G}_{\overline{y}}
)
$$
is an isomorphism.

\medskip\noindent
In particular, if $\overline{y}$ is a geometric point of $V$, then
we see that $j'_*\mathcal{G}_{\overline{y}} = \mathcal{G}_{\overline{y}}$
and hence that this kernel is equal to $\mathcal{H}_{\overline{x}}$.
This easily implies that $\alpha_{\overline{x}}$, $\beta_{\overline{x}}$,
and $\beta_{\overline{y}}$ are isomorphisms in this case.

\medskip\noindent
Next, assume that $\overline{y}$ is a point of $f^{-1}Z$.
Let $I_{\overline{x}} \subset \mathcal{O}_{X, \overline{x}}$,
resp.\ $I_{\overline{y}} \subset \mathcal{O}_{Y, \overline{y}}$
be the stalk of the ideal cutting out $Z$, resp.\ $f^{-1}Z$.
Then $I_{\overline{x}}$ is a finitely generated ideal,
$I_{\overline{y}} = I_{\overline{x}}\mathcal{O}_{Y, \overline{y}}$,
and $\mathcal{O}_{X, \overline{x}} \to \mathcal{O}_{Y, \overline{y}}$
is a flat local homomorphism inducing an isomorphism
$\mathcal{O}_{X, \overline{x}}/I_{\overline{x}} =
\mathcal{O}_{Y, \overline{y}}/I_{\overline{y}}$.
At this point we can bootstrap using the diagram of categories
$$
\xymatrix{
\QCoh(\mathcal{O}_X) \ar[r]_-{(\ref{equation-formal-glueing-modules})} \ar[d] &
\QCoh(Y \to X, Z) \ar[d] \ar@/_2pc/[l]^{(\ref{equation-reverse})} \\
\text{Mod}_{\mathcal{O}_{X, \overline{x}}} \ar[r]^-{\text{Can}} &
\text{Glue}(\mathcal{O}_{X, \overline{x}} \to \mathcal{O}_{Y, \overline{y}},
f_1, \ldots, f_t) \ar@/^2pc/[l]_{H^0}
}
$$
Namely, as in the first paragraph of the proof we identify
$$
\text{Glue}(\mathcal{O}_{X, \overline{x}} \to \mathcal{O}_{Y, \overline{y}},
f_1, \ldots, f_t)
=
\QCoh(\Spec(\mathcal{O}_{Y, \overline{y}}) \to
\Spec(\mathcal{O}_{X, \overline{x}}), V(I_{\overline{x}}))
$$
The right vertical functor is given by pullback, and it is clear that
the inner square is commutative. Our computation of the stalk of the
kernel in the third paragraph of the proof combined with
Lemma \ref{lemma-stalk-of-pushforward} implies that
the outer square (using the curved arrows) commutes. Thus we
conclude using the case of a flat morphism of affine schemes
which we handled in the first paragraph of the proof.
\end{proof}

\begin{lemma}
\label{lemma-derived-equivalent}
In Situation \ref{situation-formal-glueing} the functor
$Rf_*$ induces an equivalence between $D_{\QCoh, |f^{-1}Z|}(\mathcal{O}_Y)$
and $D_{\QCoh, |Z|}(\mathcal{O}_X)$ with quasi-inverse given by
$Lf^*$.
\end{lemma}

\begin{proof}
Since $f$ is quasi-compact and quasi-separated we see that $Rf_*$
defines a functor from $D_{\QCoh, |f^{-1}Z|}(\mathcal{O}_Y)$
to $D_{\QCoh, |Z|}(\mathcal{O}_X)$, see
Derived Categories of Spaces, Lemma
\ref{spaces-perfect-lemma-quasi-coherence-direct-image}.
By Derived Categories of Spaces, Lemma
\ref{spaces-perfect-lemma-quasi-coherence-pullback}
we see that $Lf^*$ maps $D_{\QCoh, |Z|}(\mathcal{O}_X)$
into $D_{\QCoh, |f^{-1}Z|}(\mathcal{O}_Y)$.
In Lemma \ref{lemma-formal-glueing-on-closed} we have seen that
$Lf^*Rf_*Q = Q$ for $Q$ in $D_{\QCoh, |f^{-1}Z|}(\mathcal{O}_Y)$.
By the dual of Derived Categories, Lemma
\ref{derived-lemma-fully-faithful-adjoint-kernel-zero}
to finish the proof it suffices to show that $Lf^*K = 0$
implies $K = 0$ for $K$ in $D_{\QCoh, |Z|}(\mathcal{O}_X)$.
This follows from the fact that $f$ is flat at all points of
$f^{-1}Z$ and the fact that $f^{-1}Z \to Z$ is surjective.
\end{proof}

\begin{lemma}
\label{lemma-dominate-by-fpqc-covering}
In Situation \ref{situation-formal-glueing} there exists an
fpqc covering $\{X_i \to X\}_{i \in I}$ refining the
family $\{U \to X, Y \to X\}$.
\end{lemma}

\begin{proof}
For the definition and general properties of fpqc coverings we refer to
Topologies, Section \ref{topologies-section-fpqc}. In particular, we can
first choose an \'etale covering $\{X_i \to X\}$ with $X_i$ affine and by
base changing $Y$, $Z$, and $U$ to each $X_i$ we reduce to the case where
$X$ is affine. In this case $U$ is quasi-compact and hence a finite union
$U = U_1 \cup \ldots \cup U_n$ of affine opens. 
Then $Z$ is quasi-compact hence also $f^{-1}Z$ is quasi-compact.
Thus we can choose an affine scheme $W$ and an \'etale morphism
$h : W \to Y$ such that $h^{-1}f^{-1}Z \to f^{-1}Z$ is surjective.
Say $W = \Spec(B)$ and $h^{-1}f^{-1}Z = V(J)$ where $J \subset B$
is an ideal of finite type.
By Pro-\'etale Cohomology, Lemma \ref{proetale-lemma-localization}
there exists a localization $B \to B'$ such that points of
$\Spec(B')$ correspond exactly to points of $W = \Spec(B)$
specializing to $h^{-1}f^{-1}Z = V(J)$. It follows that the
composition $\Spec(B') \to \Spec(B) = W \to Y \to X$ is flat
as by assumption $f : Y \to X$ is flat at all the points of $f^{-1}Z$. Then
$\{\Spec(B') \to X, U_1 \to X, \ldots, U_n \to X\}$
is an fpqc covering by
Topologies, Lemma \ref{topologies-lemma-recognize-fpqc-covering}.
\end{proof}




\section{Formal glueing of algebraic spaces}
\label{section-formal-glueing-spaces}

\noindent
In Situation \ref{situation-formal-glueing} we consider the category
$\textit{Spaces}(X \to Y, Z)$
of commutative diagrams of algebraic spaces over $S$ of the form
$$
\xymatrix{
U' \ar[d] & V' \ar[l] \ar[d] \ar[r] & Y' \ar[d] \\
U & V \ar[l] \ar[r] & Y
}
$$
where both squares are cartesian. There is a canonical functor
\begin{equation}
\label{equation-formal-glueing-spaces}
\textit{Spaces}/X \longrightarrow \textit{Spaces}(Y \to X, Z)
\end{equation}
which maps $X' \to X$ to the morphisms
$U \times_X X' \leftarrow V \times_X X' \rightarrow Y \times_X X'$.

\begin{lemma}
\label{lemma-equivalence-on-affine}
In Situation \ref{situation-formal-glueing} the functor
(\ref{equation-formal-glueing-spaces}) restricts to an
equivalence
\begin{enumerate}
\item from the category of algebraic spaces affine over $X$
to the full subcategory of $\textit{Spaces}(Y \to X, Z)$ consisting
of $(U' \leftarrow V' \rightarrow Y')$ with $U' \to U$, $V' \to V$,
and $Y' \to Y$ affine,
\item from the category of closed immersions $X' \to X$
to the full subcategory of $\textit{Spaces}(Y \to X, Z)$ consisting
of $(U' \leftarrow V' \rightarrow Y')$ with $U' \to U$, $V' \to V$,
and $Y' \to Y$ closed immersions, and
\item same statement as in (2) for finite morphisms.
\end{enumerate}
\end{lemma}

\begin{proof}
The category of algebraic spaces affine over $X$ is equivalent to the
category of quasi-coherent sheaves $\mathcal{A}$ of $\mathcal{O}_X$-algebras.
The full subcategory of $\textit{Spaces}(Y \to X, Z)$ consisting of
$(U' \leftarrow V' \rightarrow Y')$ with $U' \to U$, $V' \to V$,
and $Y' \to Y$ affine is equivalent to the category of
algebra objects of $\QCoh(Y \to X, Z)$. In both cases this follows
from Morphisms of Spaces, Lemma
\ref{spaces-morphisms-lemma-affine-equivalence-algebras}
with quasi-inverse given by the relative spectrum construction
(Morphisms of Spaces, Definition
\ref{spaces-morphisms-definition-relative-spec})
which commutes with arbitrary base change. Thus part (1) of the
lemma follows from Proposition \ref{proposition-formal-glueing-modules}.

\medskip\noindent
Fully faithfulness in part (2) follows from part (1). For essential
surjectivity, we reduce by part (1) to proving that $X' \to X$
is a closed immersion if and only if both $U \times_X X' \to U$ and
$Y \times_X X' \to Y$ are closed immersions. By
Lemma \ref{lemma-dominate-by-fpqc-covering}
$\{U \to X, Y \to X\}$ can be refined by an fpqc covering.
Hence the result follows from
Descent on Spaces, Lemma
\ref{spaces-descent-lemma-descending-property-closed-immersion}.

\medskip\noindent
For (3) use the argument proving (2) and
Descent on Spaces, Lemma
\ref{spaces-descent-lemma-descending-property-finite}.
\end{proof}

\begin{lemma}
\label{lemma-reflects-isomorphisms}
In Situation \ref{situation-formal-glueing} the functor
(\ref{equation-formal-glueing-spaces}) reflects isomorphisms.
\end{lemma}

\begin{proof}
By a formal argument with base change, this reduces to the following
question: A morphism $a : X' \to X$ of algebraic spaces such that
$U \times_X X' \to U$ and $Y \times_X X' \to Y$ are isomorphisms, is
an isomorphism. The family $\{U \to X, Y \to X\}$ can be refined by
an fpqc covering by Lemma \ref{lemma-dominate-by-fpqc-covering}.
Hence the result follows from
Descent on Spaces, Lemma
\ref{spaces-descent-lemma-descending-property-isomorphism}.
\end{proof}

\begin{lemma}
\label{lemma-fully-faithful-on-separated}
In Situation \ref{situation-formal-glueing} the functor
(\ref{equation-formal-glueing-spaces}) is fully faithful
on algebraic spaces separated over $X$. More precisely, it induces
a bijection
$$
\Mor_X(X'_1, X'_2)
\longrightarrow
\Mor_{\textit{Spaces}(Y \to X, Z)}(F(X'_1), F(X'_2))
$$
whenever $X'_2 \to X$ is separated.
\end{lemma}

\begin{proof}
Since $X'_2 \to X$ is separated, the graph $i : X'_1 \to X'_1 \times_X X'_2$
of a morphism $X'_1 \to X'_2$ over $X$ is a closed immersion, see
Morphisms of Spaces, Lemma \ref{spaces-morphisms-lemma-semi-diagonal}.
Moreover a closed immersion $i : T \to X'_1 \times_X X'_2$ is the graph of a
morphism if and only if $\text{pr}_1 \circ i$ is an isomorphism.
The same is true for
\begin{enumerate}
\item the graph of a morphism $U \times_X X'_1 \to U \times_X X'_2$ over $U$,
\item the graph of a morphism $V \times_X X'_1 \to V \times_X X'_2$ over $V$,
and
\item the graph of a morphism $Y \times_X X'_1 \to Y \times_X X'_2$ over $Y$.
\end{enumerate}
Moreover, if morphisms as in (1), (2), (3) fit together to form a
morphism in the category $\textit{Spaces}(Y \to X, Z)$, then these
graphs fit together to give an object of
$\textit{Spaces}(Y \times_X (X'_1 \times_X X'_2) \to X'_1 \times_X X'_2,
Z \times_X (X'_1 \times_X X'_2))$
whose triple of morphisms are closed immersions. The proof is finished
by applying Lemmas \ref{lemma-equivalence-on-affine} and
\ref{lemma-reflects-isomorphisms}.
\end{proof}







\section{Coequalizers and glueing}
\label{section-coequalizer-glue}

\noindent
Let $X$ be a Noetherian algebraic space and $Z \to X$ a closed subscheme.
Let $X' \to X$ be the blowing up in $Z$. In this section we show that
$X$ can be recovered from $X'$, $Z_n$ and glueing data where $Z_n$
is the $n$th infinitesimal neighbourhood of $Z$ in $X$.

\begin{lemma}
\label{lemma-coequalizer}
Let $S$ be a scheme. Let
$$
\xymatrix{
Y \ar[rr]_g \ar[rd] & & X \ar[ld] \\
& B
}
$$
be a commutative diagram of algebraic spaces over $S$. Assume
$B$ Noetherian, $g$ proper and surjective, and $X \to B$ separated
of finite type. Let $R = Y \times_X Y$ with projection morphisms
$t, s : R \to Y$.  There exists a coequalizer $X'$ of $s, t : R \to Y$
in the category of algebraic spaces separated over $B$. The morphism
$X' \to X$ is a finite universal homeomorphism.
\end{lemma}

\begin{proof}
Denote $h : R \to X$ the given morphism. The sheaves
$$
g_*\mathcal{O}_Y
\quad\text{and}\quad
h_*\mathcal{O}_R
$$
are coherent $\mathcal{O}_X$-algebras
(Cohomology of Spaces, Lemma
\ref{spaces-cohomology-lemma-proper-pushforward-coherent}).
The $X$-morphisms $s$, $t$ induce $\mathcal{O}_X$-algebra maps
$s^\sharp, t^\sharp$ from the first to the second.
Set
$$
\mathcal{A} = \text{Equalizer}\left(s^\sharp, t^\sharp :
g_*\mathcal{O}_Y \longrightarrow h_*\mathcal{O}_R\right)
$$
Then $\mathcal{A}$ is a coherent $\mathcal{O}_X$-algebra and we
can define
$$
X' = \underline{\Spec}_X(\mathcal{A})
$$
as in Morphisms of Spaces, Definition
\ref{spaces-morphisms-definition-relative-spec}.
By Morphisms of Spaces, Remark
\ref{spaces-morphisms-remark-factorization-quasi-compact-quasi-separated}
and functoriality of the $\underline{\Spec}$ construction
there is a factorization
$$
Y \longrightarrow X' \longrightarrow X
$$
and the morphism $g' : Y \to X'$ equalizes $s$ and $t$.
Since $\mathcal{A}$ is a coherent $\mathcal{O}_X$-module it is clear that
$X' \to X$ is a finite morphism of algebraic spaces. Since the
surjective morphism $g : Y \to X$ factors through $X'$ we see that
$X' \to X$ is surjective.

\medskip\noindent
To check that $X' \to X$ is a universal homeomorphism, it suffices
to check that it is universally injective (as we've already seen that
it is universally surjective and universally closed). To check this it
suffices to check that $|X' \times_X U| \to |U|$ is injective, for all
$U \to X$ \'etale, see
More on Morphisms of Spaces, Lemma
\ref{spaces-more-morphisms-lemma-check-universally-injective}.
It suffices to check this in all cases where $U$ is an affine scheme
(minor detail omitted). Since the construction of $X'$
commutes with \'etale localization, we may replace $U$ by $X$.
Hence it suffices to check that $|X'| \to |X|$ is injective
when $X$ is moreover an affine scheme. First observe that
$|Y| \to |X'|$ is surjective, because $g' : Y \to X'$ is proper
by Morphisms of Spaces, Lemma
\ref{spaces-morphisms-lemma-universally-closed-permanence}
(hence the image is closed) and
$\mathcal{O}_{X'} \subset g'_*\mathcal{O}_Y$ by construction.
Thus if $x_1, x_2 \in |X'|$ map to the same point in $|X|$, then
we can lift $x_1, x_2$ to points $y_1, y_2 \in |Y|$ mapping to the
same point of $|X|$. Then we can find an $r \in |R|$ with
$s(r) = y_1$ and $t(r) = y_2$, see
Properties of Spaces, Lemma \ref{spaces-properties-lemma-points-cartesian}).
Since $g'$ coequalizes $s$ and $t$ we conclude that $x_1 = x_2$ as desired.

\medskip\noindent
To prove that $X'$ is the coequalizer, let $W \to B$ be a separated morphism
of algebraic spaces over $S$ and let $a : Y \to W$ be a morphism over $B$
which equalizes $s$ and $t$. We will show that $a$ factors in a unique manner
through the morphism $g' : Y \to X'$. We will first reduce this to the
case where $W \to B$ is separated of finite type by a limit argument
(we recommend the reader skip this argument). Since $Y$ is quasi-compact
we can find a quasi-compact open subspace $W' \subset W$ such that $a$
factors through $W'$. After replacing $W$ by $W'$ we may assume $W$ is
quasi-compact. By Limits of Spaces, Lemma
\ref{spaces-limits-lemma-relative-approximation}
we can write $W = \lim_{i \in I} W_i$ as a cofiltered limit with
affine transition morphisms with $W_i$ of finite type over $B$. After
shrinking $I$ we may assume $W_i \to B$ is separated as well, see
Limits of Spaces, Lemma \ref{spaces-limits-lemma-descend-separated-morphism}.
Since $W = \lim W_i$ we have $a = \lim a_i$ for some morphisms
$a_i : Y \to W_i$. If we can prove $a_i$ factors through $g'$
for all $i$, then the same thing is true for $a$.
This proves the reduction to the case of a finite type $W$.

\medskip\noindent
Assume we have $a : Y \to W$ equalizing $s$ and $t$ with $W \to B$ separated
and of finite type. Consider
$$
\Gamma \subset X \times_B W
$$
the scheme theoretic image of $(g, a) : Y \to X \times_B W$.
Since $g$ is proper we conclude $Y \to \Gamma$ is surjective and
the projection $p : \Gamma \to X$ is proper, see
Morphisms of Spaces, Lemma
\ref{spaces-morphisms-lemma-scheme-theoretic-image-is-proper}.
Since both $g$ and $a$ equalize $s$ and $t$, the morphism $Y \to \Gamma$
also equalizes $s$ and $t$.

\medskip\noindent
We claim that $p : \Gamma \to X$ is a universal homeomorphism.
As in the proof of the corresponding fact for $X' \to X$, it
suffices to show that $p$ is universally injective. By
More on Morphisms of Spaces, Lemma
\ref{spaces-more-morphisms-lemma-check-universally-injective}
it suffices to check $|\Gamma \times_X U| \to |U|$ is injective
for every $U \to X$ \'etale. It suffices to check this for
$U$ affine (minor details omitted). Taking scheme
theoretic image commutes with \'etale localization
(Morphisms of Spaces, Lemma
\ref{spaces-morphisms-lemma-quasi-compact-scheme-theoretic-image}).
Hence we may replace $X$ by $V$ and we conclude it suffices
to show that $|\Gamma| \to |X|$ is injective.
If $\gamma_1, \gamma_2 \in |\Gamma|$ map to the same point in $|X|$, then
we can lift $\gamma_1, \gamma_2$ to points $y_1, y_2 \in |Y|$ mapping to the
same point of $|X|$ (by surjectivity of $Y \to \Gamma$ we've seen above).
Then we can find an $r \in |R|$ with $s(r) = y_1$ and $t(r) = y_2$, see
Properties of Spaces, Lemma \ref{spaces-properties-lemma-points-cartesian}).
Since $Y \to \Gamma$ coequalizes $s$ and $t$ we conclude that
$\gamma_1 = \gamma_2$ as desired.

\medskip\noindent
As a proper universal homeomorphism the morphism $p$ is finite
(see for example More on Morphisms of Spaces, Lemma
\ref{spaces-more-morphisms-lemma-characterize-finite}).
We conclude that
$$
\Gamma = \underline{\Spec}(p_*\mathcal{O}_\Gamma).
$$
Since $Y \to \Gamma$ equalizes $s$ and $t$ the map
$p_*\mathcal{O}_\Gamma \to g_*\mathcal{O}_Y$ factors through
$\mathcal{A}$ and we obtain a morphism
$X' \to \Gamma$ by functoriality of the $\underline{\Spec}$ construction.
We can compose this morphism with the projection
$q : \Gamma \to W$ to get the desired morphism $X' \to W$.
We omit the proof of uniqueness of the factorization.
\end{proof}

\noindent
We will work in the following situation.

\begin{situation}
\label{situation-coequalizer-glue}
Let $S$ be a scheme. Let $X \to B$ be a separated finite type morphism of
algebraic spaces over $S$ with $B$ Noetherian. Let
$Z \to X$ be a closed immersion and let $U \subset X$ be the complementary
open subspace. Finally, let $f : X' \to X$ be a proper morphism of algebraic
spaces such that $f^{-1}(U) \to U$ is an isomorphism.
\end{situation}

\begin{lemma}
\label{lemma-coequalizer-glue}
In Situation \ref{situation-coequalizer-glue} let
$Y = X' \amalg Z$ and $R = Y \times_X Y$ with projections $t, s : R \to Y$.
There exists a coequalizer $X_1$ of $s, t : R \to Y$ in the category
of algebraic spaces separated over $B$. The morphism
$X_1 \to X$ is a finite universal homeomorphism, an isomorphism
over $U$ and $Z \to X$ lifts to $X_1$.
\end{lemma}

\begin{proof}
Existence of $X_1$ and the fact that $X_1 \to X$ is a finite
universal homeomorphism is a special case of Lemma \ref{lemma-coequalizer}.
The formation of $X_1$ commutes with \'etale localization on $X$
(see proof of Lemma \ref{lemma-coequalizer}).
Thus the morphisms $X_n \to X$ are isomorphisms over $U$.
It is immediate from the construction that $Z \to X$ lifts to $X_1$.
\end{proof}

\noindent
In Situation \ref{situation-coequalizer-glue} for $n \geq 1$ let
$Z_n \subset X$ be the $n$th order infinitesimal neighbourhood
of $Z$ in $X$, i.e., the closed subscheme defined by the $n$th
power of the sheaf of ideals cutting out $Z$. Consider $Y_n = X' \amalg Z_n$
and $R_n = Y_n \times_X Y_n$ and the coequalizer
$$
\xymatrix{
R_n \ar@<1ex>[r] \ar@<-1ex>[r] & Y_n \ar[r] & X_n \ar[r] & X
}
$$
as in Lemma \ref{lemma-coequalizer-glue}. The maps $Y_n \to Y_{n + 1}$
and $R_n \to R_{n + 1}$ induce morphisms
\begin{equation}
\label{equation-system-coequalizers}
X_1 \to X_2 \to X_3 \to \ldots \to X
\end{equation}
Each of these morphisms is a universal homeomorphism as the morphisms
$X_n \to X$ are universal homeomorphisms.

\begin{lemma}
\label{lemma-essentially-constant}
In (\ref{equation-system-coequalizers}) for all $n$ large enough, there
exists an $m$ such that $X_n \to X_{n + m}$ factors through a
closed immersion $X \to X_{n + m}$.
\end{lemma}

\begin{proof}
Let's look a bit more closely at the construction of $X_n$
and how it changes as we increase $n$. We have
$X_n = \underline{\Spec}(\mathcal{A}_n)$
where $\mathcal{A}_n$ is the equalizer of $s_n^\sharp$ and $t_n^\sharp$
going from $g_{n , *}\mathcal{O}_{Y_n}$ to $h_{n, *}\mathcal{O}_{R_n}$.
Here $g_n : Y_n = X' \amalg Z_n \to X$ and $h_n : R_n = Y_n \times_X Y_n \to X$
are the given morphisms. Let $\mathcal{I} \subset \mathcal{O}_X$ be the
coherent sheaf of ideals corresponding to $Z$. Then
$$
g_{n, *}\mathcal{O}_{Y_n} =
f_*\mathcal{O}_{X'} \times \mathcal{O}_X/\mathcal{I}^n
$$
Similarly, we have a decomposition
$$
R_n = X' \times_X X' \amalg X" \times_X Z_n \amalg Z_n \times_X Z_n
$$
Denote $f_n : X' \times_X Z_n \to X$ the restriction of $f$
and denote
$$
\mathcal{A} = \text{Equalizer}(
\xymatrix{
f_*\mathcal{O}_{X'} \ar@<1ex>[r] \ar@<-1ex>[r] &
(f \times f)_*\mathcal{O}_{X' \times_X X'}
}
)
$$
Then we see that
$$
\mathcal{A}_n =
\text{Equalizer}(
\xymatrix{
\mathcal{A} \times \mathcal{O}_X/\mathcal{I}^n \ar@<1ex>[r] \ar@<-1ex>[r] &
f_{n, *}\mathcal{O}_{X' \times_X Z_n}
}
)
$$
We have canonical maps
$$
\mathcal{O}_X \to \ldots \to \mathcal{A}_3 \to \mathcal{A}_2 \to \mathcal{A}_1
$$
of coherent $\mathcal{O}_X$-algebras. The statement of the lemma means that
for $n$ large enough there exists an $m \geq 0$ such that the image of
$\mathcal{A}_{n + m} \to \mathcal{A}_n$ is isomorphic to $\mathcal{O}_X$.

\medskip\noindent
Since $X_n \to X$ is an isomorphism over $U$ we see that the kernel
of $\mathcal{O}_X \to \mathcal{A}_n$ is supported on $|Z|$.
Since $X$ is Noetherian, the sequence of kernels
$\mathcal{J}_n = \Ker(\mathcal{O}_X \to \mathcal{A}_n)$ stabilizes
(Cohomology of Spaces, Lemma \ref{spaces-cohomology-lemma-acc-coherent}).
Say $\mathcal{J}_{n_0} = \mathcal{J}_{n_0 + 1} = \ldots = \mathcal{J}$.
By Cohomology of Spaces, Lemma
\ref{spaces-cohomology-lemma-power-ideal-kills-sheaf}
we find that $\mathcal{I}^t \mathcal{J} = 0$ for some $t \geq 0$.
On the other hand, there is an $\mathcal{O}_X$-algebra map
$\mathcal{A}_n \to \mathcal{O}_X/\mathcal{I}^n$
and hence $\mathcal{J} \subset \mathcal{I}^n$ for all $n$.
By Artin-Rees (Cohomology of Spaces, Lemma
\ref{spaces-cohomology-lemma-Artin-Rees}) we find that
$\mathcal{J} \cap \mathcal{I}^n \subset \mathcal{I}^{n - c}\mathcal{J}$
for some $c  \geq 0$ and all $n \gg 0$. We conclude that $\mathcal{J} = 0$.

\medskip\noindent
Pick $n \geq n_0$ as in the previous paragraph. Then
$\mathcal{O}_X \to \mathcal{A}_n$ is injective. Hence it now
suffices to find $m \geq 0$ such that the image of
$\mathcal{A}_{n + m} \to \mathcal{A}_n$ is equal
to the image of $\mathcal{O}_X$. Observe that $\mathcal{A}_n$
sits in a short exact sequence
$$
0 \to \Ker(\mathcal{A} \to f_{n, *}\mathcal{O}_{X' \times_X Z_n})
\to \mathcal{A}_n \to \mathcal{O}_X/\mathcal{I}^n \to 0
$$
and similarly for $\mathcal{A}_{n + m}$. Hence it suffices to show
$$
\Ker(\mathcal{A} \to f_{n + m, *}\mathcal{O}_{X' \times_X Z_{n + m}})
\subset
\Im(\mathcal{I}^n \to \mathcal{A})
$$
for some $m \geq 0$. To do this we may work \'etale locally on
$X$ and since $X$ is Noetherian we may assume that $X$ is
a Noetherian affine scheme. Say $X = \Spec(R)$ and $\mathcal{I}$
corresponds to the ideal $I \subset R$. Let $\mathcal{A} = \widetilde{A}$
for a finite $R$-algebra $A$. Let $f_*\mathcal{O}_{X'} = \widetilde{B}$
for a finite $R$-algebra $B$. Then $R \to A \subset B$ and these maps
become isomorphisms on inverting any element of $I$.

\medskip\noindent
Note that $f_{n, *}\mathcal{O}_{X' \times_X Z_n}$
is equal to $f_*(\mathcal{O}_{X'}/I^n\mathcal{O}_{X'})$
in the notation used in Cohomology of Spaces, Section
\ref{spaces-cohomology-section-theorem-formal-functions}.
By Cohomology of Spaces, Lemma
\ref{spaces-cohomology-lemma-ML-cohomology-powers-ideal}
we see that there exists a $c \geq 0$ such that
$$
\Ker(B \to \Gamma(X, f_*(\mathcal{O}_{X'}/I^{n + m + c}\mathcal{O}_{X'}))
$$
is contained in $I^{n + m}B$. On the other hand, as $R \to B$ is
finite and an isomorphism after inverting any element of $I$
we see that $I^{n + m}B \subset \Im(I^n \to B)$ for $m$ large enough
(can be chosen independent of $n$). This finishes the proof as $A \subset B$.
\end{proof}

\begin{remark}
\label{remark-essentially-constant}
The meaning of Lemma \ref{lemma-essentially-constant}
is the system $X_1 \to X_2 \to X_3 \to \ldots$ is essentially
constant with value $X$. See Categories, Definition
\ref{categories-definition-essentially-constant-diagram}.
\end{remark}








\input{chapters}

\bibliography{my}
\bibliographystyle{amsalpha}

\end{document}

