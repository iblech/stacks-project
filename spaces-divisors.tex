\input{preamble}

% OK, start here.
%
\begin{document}

\title{Divisors on Algebraic Spaces}


\maketitle

\phantomsection
\label{section-phantom}

\tableofcontents

\section{Introduction}
\label{section-introduction}

\noindent
In this chapter we study divisors on algebraic spaces and related topics.
A basic reference for algebraic spaces is \cite{Kn}.









\section{Associated and weakly associated points}
\label{section-associated}

\noindent
In the case of schemes we have introduced two competing notions
of associated points. Namely, the usual associated points
(Divisors, Section \ref{divisors-section-associated})
and the weakly associated points
(Divisors, Section \ref{divisors-section-weakly-associated}).
For a general algebraic space the notion of an associated point
is basically useless and we don't even bother to introduce it.
If the algebraic space is locally Noetherian, then we allow ourselves
to use the phrase ``associated point'' instead of
``weakly associated point'' as the notions are the same for
Noetherian schemes (Divisors, Lemma \ref{divisors-lemma-ass-weakly-ass}).
Before we make our definition, we need a lemma.

\begin{lemma}
\label{lemma-associated}
Let $S$ be a scheme. Let $X$ be an algebraic space over $S$.
Let $\mathcal{F}$ be a quasi-coherent $\mathcal{O}_X$-module.
Let $x \in |X|$. The following are equivalent
\begin{enumerate}
\item for some \'etale morphism $f : U \to X$ with $U$ a scheme
and $u \in U$ mapping to $x$, the point $u$ is weakly associated
to $f^*\mathcal{F}$,
\item for every \'etale morphism $f : U \to X$ with $U$ a scheme
and $u \in U$ mapping to $x$, the point $u$ is weakly associated
to $f^*\mathcal{F}$,
\item the maximal ideal of $\mathcal{O}_{X, \overline{x}}$
is a weakly associated prime of the stalk $\mathcal{F}_{\overline{x}}$.
\end{enumerate}
If $X$ is locally Noetherian, then these are also equivalent to
\begin{enumerate}
\item[(4)] for some \'etale morphism $f : U \to X$ with $U$ a scheme
and $u \in U$ mapping to $x$, the point $u$ is associated
to $f^*\mathcal{F}$,
\item[(5)] for every \'etale morphism $f : U \to X$ with $U$ a scheme
and $u \in U$ mapping to $x$, the point $u$ is associated
to $f^*\mathcal{F}$,
\item[(6)] the maximal ideal of $\mathcal{O}_{X, \overline{x}}$
is an associated prime of the stalk $\mathcal{F}_{\overline{x}}$.
\end{enumerate}
\end{lemma}

\begin{proof}
Choose a scheme $U$ with a point $u$ and an \'etale morphism
$f : U \to X$ mapping $u$ to $x$. Lift $\overline{x}$ to a geometric
point of $U$ over $u$. Recall that
$\mathcal{O}_{X, \overline{x}} = \mathcal{O}_{U, u}^{sh}$
where the strict henselization is with respect to our chosen
lift of $\overline{x}$, see
Properties of Spaces, Lemma
\ref{spaces-properties-lemma-describe-etale-local-ring}.
Finally, we have
$$
\mathcal{F}_{\overline{x}} =
(f^*\mathcal{F})_u \otimes_{\mathcal{O}_{U, u}}
\mathcal{O}_{X, \overline{x}} =
(f^*\mathcal{F})_u \otimes_{\mathcal{O}_{U, u}}
\mathcal{O}_{U, u}^{sh}
$$
by
Properties of Spaces, Lemma \ref{spaces-properties-lemma-stalk-quasi-coherent}.
Hence the equivalence of (1), (2), and (3) follows from
More on Flatness, Lemma \ref{flat-lemma-weakly-associated-henselization}.
If $X$ is locally Noetherian, then
any $U$ as above is locally Noetherian,
hence we see that (1), resp.\ (2) are equivalent to (4), resp.\ (5) by
Divisors, Lemma \ref{divisors-lemma-ass-weakly-ass}.
On the other hand, in the locally Noetherian case the
local ring $\mathcal{O}_{X, \overline{x}}$ is Noetherian too
(Properties of Spaces, Lemma
\ref{spaces-properties-lemma-Noetherian-local-ring-Noetherian}).
Hence the equivalence of (3) and (6) by the same lemma
(or by Algebra, Lemma \ref{algebra-lemma-ass-weakly-ass}).
\end{proof}

\begin{definition}
\label{definition-weakly-associated}
Let $S$ be a scheme. Let $X$ be an algebraic space over $S$.
Let $\mathcal{F}$ be a quasi-coherent sheaf on $X$.
Let $x \in |X|$.
\begin{enumerate}
\item We say $x$ is {\it weakly associated} to $\mathcal{F}$
if the equivalent conditions (1), (2), and (3) of
Lemma \ref{lemma-associated} are satisfied.
\item We denote $\text{WeakAss}(\mathcal{F})$ the set of weakly associated
points of $\mathcal{F}$.
\item The {\it weakly associated points of $X$} are the weakly associated
points of $\mathcal{O}_X$.
\end{enumerate}
If $X$ is locally Noetherian we will say
{\it $x$ is associated to $\mathcal{F}$}
if and only if $x$ is weakly associated to $\mathcal{F}$ and we set
$\text{Ass}(\mathcal{F}) = \text{WeakAss}(\mathcal{F})$.
Finally (still assuming $X$ is locally Noetherian),
we will say {\it $x$ is an associated point of $X$} if and only if
$x$ is a weakly associated point of $X$.
\end{definition}

\noindent
At this point we can prove the obligatory lemmas.

\begin{lemma}
\label{lemma-weakly-ass-support}
Let $S$ be a scheme. Let $X$ be an algebraic space over $S$.
Let $\mathcal{F}$ be a quasi-coherent $\mathcal{O}_X$-module.
Then $\text{WeakAss}(\mathcal{F}) \subset \text{Supp}(\mathcal{F})$.
\end{lemma}

\begin{proof}
This is immediate from the definitions. The support of an abelian sheaf
on $X$ is defined in Properties of Spaces, Definition
\ref{spaces-properties-definition-support}.
\end{proof}

\begin{lemma}
\label{lemma-ses-weakly-ass}
Let $S$ be a scheme. Let $X$ be an algebraic space over $S$.
Let $0 \to \mathcal{F}_1 \to \mathcal{F}_2 \to \mathcal{F}_3 \to 0$
be a short exact sequence of quasi-coherent sheaves on $X$.
Then
$\text{WeakAss}(\mathcal{F}_2) \subset
\text{WeakAss}(\mathcal{F}_1) \cup \text{WeakAss}(\mathcal{F}_3)$
and
$\text{WeakAss}(\mathcal{F}_1) \subset \text{WeakAss}(\mathcal{F}_2)$.
\end{lemma}

\begin{proof}
For every geometric point $\overline{x} \in X$
the sequence of stalks
$0 \to \mathcal{F}_{1, \overline{x}} \to
\mathcal{F}_{2, \overline{x}} \to
\mathcal{F}_{3, \overline{x}} \to 0$
is a short exact sequence of $\mathcal{O}_{X, \overline{x}}$-modules.
Hence the lemma follows from
Algebra, Lemma \ref{algebra-lemma-weakly-ass}.
\end{proof}

\begin{lemma}
\label{lemma-weakly-ass-zero}
Let $S$ be a scheme. Let $X$ be an algebraic space over $S$.
Let $\mathcal{F}$ be a quasi-coherent $\mathcal{O}_X$-module.
Then
$$
\mathcal{F} = (0) \Leftrightarrow \text{WeakAss}(\mathcal{F}) = \emptyset
$$
\end{lemma}

\begin{proof}
Choose a scheme $U$ and a surjective \'etale morphism $f : U \to X$.
Then $\mathcal{F}$ is zero if and only if $f^*\mathcal{F}$ is zero.
Hence the lemma follows from the definition and the lemma in the
case of schemes, see
Divisors, Lemma \ref{divisors-lemma-weakly-ass-zero}.
\end{proof}

\begin{lemma}
\label{lemma-minimal-support-in-weakly-ass}
Let $S$ be a scheme. Let $X$ be an algebraic space over $S$.
Let $\mathcal{F}$ be a quasi-coherent $\mathcal{O}_X$-module.
Let $x \in |X|$. If
\begin{enumerate}
\item $x \in \text{Supp}(\mathcal{F})$
\item $x$ is a codimension $0$ point of $X$
(Properties of Spaces, Definition
\ref{spaces-properties-definition-dimension-local-ring}).
\end{enumerate}
Then $x \in \text{WeakAss}(\mathcal{F})$. If $\mathcal{F}$
is a finite type $\mathcal{O}_X$-module with scheme theoretic support $Z$
(Morphisms of Spaces, Definition
\ref{spaces-morphisms-definition-scheme-theoretic-support})
and $x$ is a codimension $0$ point of $Z$, then
$x \in \text{WeakAss}(\mathcal{F})$.
\end{lemma}

\begin{proof}
Since $x \in \text{Supp}(\mathcal{F})$ the stalk
$\mathcal{F}_{\overline{x}}$ is not zero. Hence
$\text{WeakAss}(\mathcal{F}_{\overline{x}})$
is nonempty by
Algebra, Lemma \ref{algebra-lemma-weakly-ass-zero}.
On the other hand, the spectrum of $\mathcal{O}_{X, \overline{x}}$
is a singleton. Hence $x$ is a weakly associated point of
$\mathcal{F}$ by definition. The final statement follows
as $\mathcal{O}_{X, \overline{x}} \to \mathcal{O}_{Z, \overline{z}}$
is a surjection, the spectrum of $\mathcal{O}_{Z, \overline{z}}$
is a singleton, and $\mathcal{F}_{\overline{x}}$ is a nonzero
module over $\mathcal{O}_{Z, \overline{z}}$.
\end{proof}

\begin{lemma}
\label{lemma-minimal-support-in-weakly-ass-decent}
Let $S$ be a scheme. Let $X$ be an algebraic space over $S$.
Let $\mathcal{F}$ be a quasi-coherent $\mathcal{O}_X$-module.
Let $x \in |X|$. If
\begin{enumerate}
\item $X$ is decent (for example quasi-separated or locally separated),
\item $x \in \text{Supp}(\mathcal{F})$
\item $x$ is not a specialization of another point in
$\text{Supp}(\mathcal{F})$.
\end{enumerate}
Then $x \in \text{WeakAss}(\mathcal{F})$.
\end{lemma}

\begin{proof}
(A quasi-separated algebraic space is decent, see
Decent Spaces, Section \ref{decent-spaces-section-reasonable-decent}.
A locally separated algebraic space is decent, see
Decent Spaces, Lemma \ref{decent-spaces-lemma-locally-separated-decent}.)
Choose a scheme $U$, a point $u \in U$, and an \'etale morphism
$f : U \to X$ mapping $u$ to $x$. By
Decent Spaces, Lemma
\ref{decent-spaces-lemma-decent-no-specializations-map-to-same-point}
if $u' \leadsto u$ is a nontrivial specialization, then
$f(u') \not = x$. Hence we see that $u \in \text{Supp}(f^*\mathcal{F})$
is not a specialization of another point of
$\text{Supp}(f^*\mathcal{F})$.
Hence $u \in \text{WeakAss}(f^*\mathcal{F})$ by
Divisors, Lemma \ref{lemma-minimal-support-in-weakly-ass}.
\end{proof}

\begin{lemma}
\label{lemma-finite-ass}
Let $S$ be a scheme. Let $X$ be a locally Noetherian algebraic space over $S$.
Let $\mathcal{F}$ be a coherent $\mathcal{O}_X$-module.
Then $\text{Ass}(\mathcal{F}) \cap W$ is finite for
every quasi-compact open $W \subset |X|$.
\end{lemma}

\begin{proof}
Choose a quasi-compact scheme $U$ and an \'etale morphism $U \to X$
such that $W$ is the image of $|U| \to |X|$. Then $U$ is a
Noetherian scheme and we may apply
Divisors, Lemma \ref{divisors-lemma-finite-ass} to conclude.
\end{proof}

\begin{lemma}
\label{lemma-restriction-injective-open-contains-weakly-ass}
Let $S$ be a scheme. Let $X$ be an algebraic space over $S$.
Let $\mathcal{F}$ be a quasi-coherent $\mathcal{O}_X$-module.
If $U \to X$ is an \'etale morphism such that
$\text{WeakAss}(\mathcal{F}) \subset \Im(|U| \to |X|)$, then
$\Gamma(X, \mathcal{F}) \to \Gamma(U, \mathcal{F})$ is injective.
\end{lemma}

\begin{proof}
Let $s \in \Gamma(X, \mathcal{F})$ be a section which restricts to zero on $U$.
Let $\mathcal{F}' \subset \mathcal{F}$ be the image of the map
$\mathcal{O}_X \to \mathcal{F}$ defined by $s$. Then $\mathcal{F}'|_U = 0$.
This implies that
$\text{WeakAss}(\mathcal{F}') \cap \Im(|U| \to |X|) = \emptyset$
(by the definition of weakly associated points).
On the other hand,
$\text{WeakAss}(\mathcal{F}') \subset \text{WeakAss}(\mathcal{F})$
by Lemma \ref{lemma-ses-weakly-ass}. We conclude
$\text{Ass}(\mathcal{F}') = \emptyset$.
Hence $\mathcal{F}' = 0$ by Lemma \ref{lemma-weakly-ass-zero}.
\end{proof}

\begin{lemma}
\label{lemma-weakass-pushforward}
Let $S$ be a scheme. Let $f : X \to Y$ be a quasi-compact and quasi-separated
morphism of algebraic spaces over $S$. Let $\mathcal{F}$ be a quasi-coherent
$\mathcal{O}_X$-module. Let $y \in |Y|$ be a point which is not in the
image of $|f|$. Then $y$ is not weakly associated to $f_*\mathcal{F}$.
\end{lemma}

\begin{proof}
By Morphisms of Spaces, Lemma \ref{spaces-morphisms-lemma-pushforward}
the $\mathcal{O}_Y$-module $f_*\mathcal{F}$ is quasi-coherent hence
the lemma makes sense.
Choose an affine scheme $V$, a point $v \in V$, and an \'etale morphism
$V \to Y$ mapping $v$ to $y$. We may replace
$f : X \to Y$, $\mathcal{F}$, $y$ by
$X \times_Y V \to V$, $\mathcal{F}|_{X \times_Y V}$, $v$.
Thus we may assume $Y$ is an affine scheme.
In this case $X$ is quasi-compact, hence we can choose
an affine scheme $U$ and a surjective \'etale morphism $U \to X$.
Denote $g : U \to Y$ the composition.
Then $f_*\mathcal{F} \subset g_*(\mathcal{F}|_U)$.
By Lemma \ref{lemma-ses-weakly-ass}
we reduce to the case of schemes which is
Divisors, Lemma \ref{divisors-lemma-weakass-pushforward}.
\end{proof}

\begin{lemma}
\label{lemma-check-injective-on-weakass}
Let $S$ be a scheme. Let $X$ be an algebraic space over $S$.
Let $\varphi : \mathcal{F} \to \mathcal{G}$ be a map of
quasi-coherent $\mathcal{O}_X$-modules. Assume that for every
$x \in |X|$ at least one of the following happens
\begin{enumerate}
\item $\mathcal{F}_{\overline{x}} \to \mathcal{G}_{\overline{x}}$
is injective, or
\item $x \not \in \text{WeakAss}(\mathcal{F})$.
\end{enumerate}
Then $\varphi$ is injective.
\end{lemma}

\begin{proof}
The assumptions imply that $\text{WeakAss}(\Ker(\varphi)) = \emptyset$
and hence $\Ker(\varphi) = 0$ by Lemma \ref{lemma-weakly-ass-zero}.
\end{proof}










\section{Morphisms and weakly associated points}
\label{section-morphisms-weakly-associated}

\begin{lemma}
\label{lemma-weakly-ass-reverse-functorial}
Let $S$ be a scheme.
Let $f : X \to Y$ be an affine morphism of algebraic spaces over $S$.
Let $\mathcal{F}$ be a quasi-coherent $\mathcal{O}_X$-module.
Then we have
$$
\text{WeakAss}_S(f_*\mathcal{F}) \subset f(\text{WeakAss}_X(\mathcal{F}))
$$
\end{lemma}

\begin{proof}
Choose a scheme $V$ and a surjective \'etale morphism $V \to Y$.
Set $U = X \times_Y V$. Then $U \to V$ is an affine morphism
of schemes. By our definition of weakly associated points
the problem is reduced to the morphism of schemes $U \to V$. This case is
treated in Divisors, Lemma \ref{divisors-lemma-weakly-ass-reverse-functorial}.
\end{proof}

\begin{lemma}
\label{lemma-ass-functorial-equal}
Let $S$ be a scheme.
Let $f : X \to Y$ be an affine morphism of algebraic spaces over $S$.
Let $\mathcal{F}$ be a quasi-coherent $\mathcal{O}_X$-module.
If $X$ is locally Noetherian, then we have
$$
\text{WeakAss}_Y(f_*\mathcal{F}) =
f(\text{WeakAss}_X(\mathcal{F}))
$$
\end{lemma}

\begin{proof}
Choose a scheme $V$ and a surjective \'etale morphism $V \to Y$.
Set $U = X \times_Y V$. Then $U \to V$ is an affine morphism
of schemes and $U$ is locally Noetherian.
By our definition of weakly associated points
the problem is reduced to the morphism of schemes $U \to V$. This case is
treated in Divisors, Lemma \ref{divisors-lemma-ass-functorial-equal}.
\end{proof}

\begin{lemma}
\label{lemma-weakly-associated-finite}
Let $S$ be a scheme.
Let $f : X \to Y$ be a finite morphism of algebraic spaces over $S$.
Let $\mathcal{F}$ be a quasi-coherent $\mathcal{O}_X$-module.
Then $\text{WeakAss}(f_*\mathcal{F}) = f(\text{WeakAss}(\mathcal{F}))$.
\end{lemma}

\begin{proof}
Choose a scheme $V$ and a surjective \'etale morphism $V \to Y$.
Set $U = X \times_Y V$. Then $U \to V$ is a finite morphism
of schemes. By our definition of weakly associated points
the problem is reduced to the morphism of schemes $U \to V$. This case is
treated in Divisors, Lemma \ref{divisors-lemma-weakly-associated-finite}.
\end{proof}

\begin{lemma}
\label{lemma-weakly-ass-pullback}
Let $S$ be a scheme. Let $f : X \to Y$ be a morphism of algebraic spaces
over $S$. Let $\mathcal{G}$ be a quasi-coherent $\mathcal{O}_Y$-module.
Let $x \in |X|$ and $y = f(x) \in |Y|$. If
\begin{enumerate}
\item $y \in \text{WeakAss}_S(\mathcal{G})$,
\item $f$ is flat at $x$, and
\item the dimension of the local ring of the fibre of $f$ at $x$
is zero (Morphisms of Spaces, Definition
\ref{spaces-morphisms-definition-dimension-fibre}),
\end{enumerate}
then $x \in \text{WeakAss}(f^*\mathcal{G})$.
\end{lemma}

\begin{proof}
Choose a scheme $V$, a point $v \in V$, and an \'etale morphism $V \to Y$
mapping $v$ to $y$. Choose a scheme $U$, a point $u \in U$, and an
\'etale morphism $U \to V \times_Y X$ mapping $v$ to a point lying over
$v$ and $x$. This is possible because there is a $t \in |V \times_Y X|$
mapping to $(v, y)$ by Properties of Spaces, Lemma
\ref{spaces-properties-lemma-points-cartesian}.
By definition we see that the dimension of $\mathcal{O}_{U_v, u}$ is zero.
Hence $u$ is a generic point of the fiber $U_v$.
By our definition of weakly associated points
the problem is reduced to the morphism of schemes $U \to V$.
This case is treated in
Divisors, Lemma \ref{divisors-lemma-weakly-ass-pullback}.
\end{proof}

\begin{lemma}
\label{lemma-weakly-ass-change-fields}
Let $K/k$ be a field extension. Let $X$ be an algebraic space over $k$.
Let $\mathcal{F}$ be a quasi-coherent $\mathcal{O}_X$-module.
Let $y \in X_K$ with image $x \in X$. If $y$ is a weakly
associated point of the pullback $\mathcal{F}_K$, then $x$
is a weakly associated point of $\mathcal{F}$.
\end{lemma}

\begin{proof}
This is the translation of
Divisors, Lemma \ref{divisors-lemma-weakly-ass-change-fields}
into the language of algebraic spaces. We omit the details of the
translation.
\end{proof}

\begin{lemma}
\label{lemma-finite-flat-weak-assassin-up-down}
Let $S$ be a scheme.
Let $f : X \to Y$ be a finite flat morphism of algebraic spaces.
Let $\mathcal{G}$ be a quasi-coherent $\mathcal{O}_Y$-module.
Let $x \in |X|$ be a point with image $y \in |Y|$. Then
$$
x \in \text{WeakAss}(g^*\mathcal{G})
\Leftrightarrow
y \in \text{WeakAss}(\mathcal{G})
$$
\end{lemma}

\begin{proof}
Follows immediately from the case of schemes
(More on Flatness, Lemma \ref{flat-lemma-finite-flat-weak-assassin-up-down})
by \'etale localization.
\end{proof}

\begin{lemma}
\label{lemma-etale-weak-assassin-up-down}
Let $S$ be a scheme.
Let $f : X \to Y$ be an \'etale morphism of algebraic spaces.
Let $\mathcal{G}$ be a quasi-coherent $\mathcal{O}_Y$-module.
Let $x \in |X|$ be a point with image $y \in |Y|$. Then
$$
x \in \text{WeakAss}(f^*\mathcal{G})
\Leftrightarrow
y \in \text{WeakAss}(\mathcal{G})
$$
\end{lemma}

\begin{proof}
This is immediate from the definition of weakly associated points
and in fact the corresponding lemma for the case of schemes
(More on Flatness, Lemma \ref{flat-lemma-etale-weak-assassin-up-down})
is the basis for our definition.
\end{proof}







\section{Relative weak assassin}
\label{section-relative-weak-assassin}

\noindent
We need a couple of lemmas to define this gadget.

\begin{lemma}
\label{lemma-locally-noetherian-fibre}
Let $S$ be a scheme. Let $f : X \to Y$ be a morphism of algebraic spaces
over $S$. Let $y \in |Y|$. The following are equivalent
\begin{enumerate}
\item for some scheme $V$, point $v \in V$, and \'etale morphism $V \to Y$
mapping $v$ to $y$, the algebraic space $X_v$ is locally Noetherian,
\item for every scheme $V$, point $v \in V$, and \'etale morphism $V \to Y$
mapping $v$ to $y$, the algebraic space $X_v$ is locally Noetherian, and
\item there exists a field $k$ and a morphism $\Spec(k) \to Y$ representing
$y$ such that $X_k$ is locally Noetherian.
\end{enumerate}
If there exists a field $k_0$ and a monomorphism $\Spec(k_0) \to Y$
representing $y$, then these are also equivalent to
\begin{enumerate}
\item[(4)] the algebraic space $X_{k_0}$ is locally Noetherian.
\end{enumerate}
\end{lemma}

\begin{proof}
Observe that $X_v = v \times_Y X = \Spec(\kappa(v)) \times_Y X$.
Hence the implications (2) $\Rightarrow$ (1) $\Rightarrow$ (3) are clear.
Assume that $\Spec(k) \to Y$ is a morphism from the spectrum of a field
such that $X_k$ is locally Noetherian. Let $V \to Y$ be an \'etale morphism
from a scheme $V$ and let $v \in V$ a point mapping to $y$.
Then the scheme $v \times_Y \Spec(k)$ is nonempty. Choose a
point $w \in v \times_Y \Spec(k)$. Consider the morphisms
$$
X_v \longleftarrow X_w \longrightarrow X_k
$$
Since $V \to Y$ is \'etale and since $w$ may be viewed as a point of
$V \times_Y \Spec(k)$, we see that $\kappa(w) \supset k$
is a finite separable extension of fields
(Morphisms, Lemma \ref{morphisms-lemma-etale-over-field}).
Thus $X_w \to X_k$ is a finite \'etale morphism as a base change of
$w \to \Spec(k)$. Hence $X_w$ is locally Noetherian
(Morphisms of Spaces, Lemma
\ref{spaces-morphisms-lemma-locally-finite-type-locally-noetherian}).
The morphism $X_w \to X_v$ is a surjective, affine, flat morphism
as a base change of the surjective, affine, flat morphism $w \to v$.
Then the fact that $X_w$ is locally Noetherian implies that
$X_v$ is locally Noetherian. This can be seen by picking a
surjective \'etale morphism $U \to X$ and then using that
$U_w \to U_v$ is surjective, affine, and flat. Working
affine locally on the scheme $U_v$ we conclude
that $U_w$ is locally Noetherian by
Algebra, Lemma \ref{algebra-lemma-descent-Noetherian}.

\medskip\noindent
Finally, it suffices to prove that (3) implies (4)
in case we have a monomorphism $\Spec(k_0) \to Y$ in the class of $y$.
Then $\Spec(k) \to Y$ factors as $\Spec(k) \to \Spec(k_0) \to Y$.
The argument given above then shows that $X_k$ being
locally Noetherian impies that $X_{k_0}$ is locally Noetherian.
\end{proof}

\begin{definition}
\label{definition-locally-Noetherian-fibre}
Let $S$ be a scheme. Let $f : X \to Y$ be a morphism of algebraic spaces
over $S$. Let $y \in |Y|$. We say {\it the fibre of $f$ over $y$ is
locally Noetherian} if the equivalent conditions (1), (2), and (3)
of Lemma \ref{lemma-locally-noetherian-fibre} are satisfied.
We say {\it the fibres of $f$ are locally Noetherian} if this
holds for every $y \in |Y|$.
\end{definition}

\noindent
Of course, the usual way to guarantee locally Noetherian fibres is
to assume the morphism is locally of finite type.

\begin{lemma}
\label{lemma-locally-finite-type-locally-Noetherian-fibres}
Let $S$ be a scheme. Let $f : X \to Y$ be a morphism of algebraic spaces
over $S$. If $f$ is locally of finite type, then 
the fibres of $f$ are locally Noetherian.
\end{lemma}

\begin{proof}
This follows from Morphisms of Spaces, Lemma
\ref{spaces-morphisms-lemma-locally-finite-type-locally-noetherian}
and the fact that the spectrum of a field is Noetherian.
\end{proof}

\begin{lemma}
\label{lemma-relative-assassin}
Let $S$ be a scheme. Let $f : X \to Y$ be a morphism of algebraic spaces
over $S$. Let $x \in |X|$ and $y = f(x) \in |Y|$.
Let $\mathcal{F}$ be a quasi-coherent
$\mathcal{O}_X$-module. Consider commutative diagrams
$$
\xymatrix{
X \ar[d] & X \times_Y V \ar[d] \ar[l] & X_v \ar[d] \ar[l] \\
Y & V \ar[l] & v \ar[l]
}
\quad
\xymatrix{
X \ar[d] & U \ar[d] \ar[l] & U_v \ar[d] \ar[l] \\
Y & V \ar[l] & v \ar[l]
}
\quad
\xymatrix{
x \ar@{|->}[d] &
x' \ar@{|->}[d] \ar@{|->}[l] &
u \ar@{|->}[ld] \ar@{|->}[l] \\
y &
v \ar@{|->}[l]
}
$$
where $V$ and $U$ are schemes, $V \to Y$ and $U \to X \times_Y V$
are \'etale, $v \in V$, $x' \in |X_v|$, $u \in U$ are points
related as in the last diagram.
Denote $\mathcal{F}|_{X_v}$ and $\mathcal{F}|_{U_v}$
the pullbacks of $\mathcal{F}$.
The following are equivalent
\begin{enumerate}
\item for some $V, v, x'$ as above $x'$ is a weakly associated
point of $\mathcal{F}|_{X_v}$,
\item for every $V \to Y, v, x'$ as above $x'$ is a weakly associated
point of $\mathcal{F}|_{X_v}$,
\item for some $U, V, u, v$ as above $u$ is a weakly associated
point of $\mathcal{F}|_{U_v}$,
\item for every $U, V, u, v$ as above $u$ is a weakly associated
point of $\mathcal{F}|_{U_v}$,
\item for some field $k$ and morphism $\Spec(k) \to Y$ representing $y$
and some $t \in |X_k|$ mapping to $x$, the point $t$ is a weakly
associated point of $\mathcal{F}|_{X_k}$.
\end{enumerate}
If there exists a field $k_0$ and a monomorphism $\Spec(k_0) \to Y$
representing $y$, then these are also equivalent to
\begin{enumerate}
\item[(6)] $x_0$ is a weakly associated point of $\mathcal{F}|_{X_{k_0}}$
where $x_0 \in |X_{k_0}|$ is the unique point mapping to $x$.
\end{enumerate}
If the fibre of $f$ over $y$ is locally Noetherian, then in
conditions (1), (2), (3), (4), and (6) we may replace
``weakly associated'' with ``associated''.
\end{lemma}

\begin{proof}
Observe that given $V, v, x'$ as in the lemma we can find
$U \to X \times_Y V$ and $u \in U$ mapping to $x'$
and then the morphism $U_v \to X_v$ is \'etale.
Thus it is clear that (1) and (3) are equivalent
as well as (2) and (4). Each of these implies (5).
We will show that (5) implies (2).
Suppose given $V, v, x'$ as well as $\Spec(k) \to X$ and $t \in |X_k|$
such that the point $t$ is a weakly
associated point of $\mathcal{F}|_{X_k}$.
We can choose a point $w \in v \times_Y \Spec(k)$.
Then we obtain the morphisms
$$
X_v \longleftarrow X_w \longrightarrow X_k
$$
Since $V \to Y$ is \'etale and since $w$ may be viewed as a point of
$V \times_Y \Spec(k)$, we see that $\kappa(w) \supset k$
is a finite separable extension of fields
(Morphisms, Lemma \ref{morphisms-lemma-etale-over-field}).
Thus $X_w \to X_k$ is a finite \'etale morphism as a base change of
$w \to \Spec(k)$. Thus any point $x''$ of $X_w$ lying over $t$
is a weakly associated point of $\mathcal{F}|_{X_w}$ by
Lemma \ref{lemma-etale-weak-assassin-up-down}.
We may pick $x''$ mapping to $x'$
(Properties of Spaces, Lemma \ref{spaces-properties-lemma-points-cartesian}).
Then Lemma \ref{lemma-weakly-ass-change-fields}
implies that $x'$ is a weakly associated
point of $\mathcal{F}|_{X_v}$.

\medskip\noindent
To finish the proof it suffices to show that the equivalent
conditions (1) -- (5) imply (6) if we are given
$\Spec(k_0) \to Y$ as in (6). In this case the morphism
$\Spec(k) \to Y$ of (5) factors uniquely as $\Spec(k) \to \Spec(k_0) \to Y$.
Then $x_0$ is the image of $t$ under the morphism $X_k \to X_{k_0}$.
Hence the same lemma as above shows that (6) is true.
\end{proof}

\begin{definition}
\label{definition-relative-weak-assassin}
Let $S$ be a scheme. Let $f : X \to Y$ be a morphism of algebraic spaces
over $S$. Let $\mathcal{F}$ be a quasi-coherent $\mathcal{O}_X$-module.
The {\it relative weak assassin of $\mathcal{F}$ in $X$ over $Y$}
is the set $\text{WeakAss}_{X/Y}(\mathcal{F}) \subset |X|$
consisting of those $x \in |X|$ such that the equivalent conditions of
Lemma \ref{lemma-relative-assassin} are satisfied.
If the fibres of $f$ are locally Noetherian
(Definition \ref{definition-locally-Noetherian-fibre})
then we use the notation $\text{Ass}_{X/Y}(\mathcal{F})$.
\end{definition}

\noindent
With this notation we can formulate some of the results
already proven for schemes.

\begin{lemma}
\label{lemma-bourbaki}
Let $S$ be a scheme.
Let $f : X \to Y$ be a morphism of algebraic spaces over $S$.
Let $\mathcal{F}$ be a quasi-coherent $\mathcal{O}_X$-module.
Let $\mathcal{G}$ be a quasi-coherent $\mathcal{O}_Y$-module.
Assume
\begin{enumerate}
\item $\mathcal{F}$ is flat over $Y$,
\item $X$ and $Y$ are locally Noetherian, and
\item the fibres of $f$ are locally Noetherian.
\end{enumerate}
Then
$$
\text{Ass}_X(\mathcal{F} \otimes_{\mathcal{O}_X} f^*\mathcal{G}) =
\{x \in \text{Ass}_{X/Y}(\mathcal{F})\text{ such that }
f(x) \in \text{Ass}_Y(\mathcal{G}) \}
$$
\end{lemma}

\begin{proof}
Via \'etale localization, this is an immediate consequence of the result
for schemes, see
Divisors, Lemma \ref{divisors-lemma-bourbaki}.
The result for schemes is more general only because
we haven't defined associated points for
non-Noetherian algebraic spaces (hence we need to assume $X$
and the fibres of $X \to Y$ are locally Noetherian to even
be able to formulate this result).
\end{proof}

\begin{lemma}
\label{lemma-base-change-relative-assassin}
Let $S$ be a scheme. Let
$$
\xymatrix{
X' \ar[d]_{f'} \ar[r]_{g'} & X \ar[d]^f \\
Y' \ar[r]^g & Y
}
$$
be a cartesian diagram of algebraic spaces over $S$.
Let $\mathcal{F}$ be a quasi-coherent $\mathcal{O}_X$-module
and set $\mathcal{F}' = (g')^*\mathcal{F}$.
If $f$ is locally of finite type, then
\begin{enumerate}
\item $x' \in \text{Ass}_{X'/Y'}(\mathcal{F}')
\Rightarrow g'(x') \in \text{Ass}_{X/Y}(\mathcal{F})$
\item if $x \in \text{Ass}_{X/Y}(\mathcal{F})$, then given
$y' \in |Y'|$ with $f(x) = g(y')$, there exists an
$x' \in \text{Ass}_{X'/Y'}(\mathcal{F}')$
with $g'(x') = x$ and $f'(x') = y'$.
\end{enumerate}
\end{lemma}

\begin{proof}
This follows from the case of schemes by \'etale localization.
We write out the details completely. Choose a scheme
$V$ and a surjective \'etale morphism $V \to Y$.
Choose a scheme $U$ and a surjective
\'etale morphism $U \to V \times_Y X$. Choose a scheme $V'$
and a surjective \'etale morphism $V' \to V \times_Y Y'$.
Then $U' = V' \times_V U$ is a scheme and the morphism
$U' \to X'$ is surjective and \'etale.

\medskip\noindent
Proof of (1). Choose $u' \in U'$ mapping to $x'$.
Denote $v' \in V'$ the image of $u'$.
Then $x' \in \text{Ass}_{X'/Y'}(\mathcal{F}')$ is
equivalent to $u' \in \text{Ass}(\mathcal{F}|_{U'_{v'}})$
by definition (writing $\text{Ass}$ instead of $\text{WeakAss}$
makes sense as $U'_{v'}$ is locally Noetherian).
Applying Divisors, Lemma \ref{divisors-lemma-base-change-relative-assassin}
we see that the image $u \in U$ of $u'$ is in
$\text{Ass}(\mathcal{F}|_{U_v})$ where $v \in V$ is the image of $u$.
This in turn means $g'(x') \in \text{Ass}_{X/Y}(\mathcal{F})$.

\medskip\noindent
Proof of (2). Choose $u \in U$ mapping to $x$.
Denote $v \in V$ the image of $u$.
Then $x \in \text{Ass}_{X/Y}(\mathcal{F})$ is
equivalent to $u \in \text{Ass}(\mathcal{F}|_{U_v})$
by definition. Choose a point $v' \in V'$ mapping
to $y' \in |Y'|$ and to $v \in V$ (possible by
Properties of Spaces, Lemma \ref{spaces-properties-lemma-points-cartesian}).
Let $t \in \Spec(\kappa(v') \otimes_{\kappa(v)} \kappa(u))$
be a generic point of an irreducible component.
Let $u' \in U'$ be the image of $t$.
Applying Divisors, Lemma \ref{divisors-lemma-base-change-relative-assassin}
we see that $u' \in \text{Ass}(\mathcal{F}'|_{U'_{v'}})$.
This in turn means $x' \in \text{Ass}_{X'/Y'}(\mathcal{F}')$
where $x' \in |X'|$ is the image of $u'$.
\end{proof}

\begin{lemma}
\label{lemma-base-change-relative-assassin-quasi-finite}
With notation and assumptions as in
Lemma \ref{lemma-base-change-relative-assassin}.
Assume $g$ is locally quasi-finite, or more generally that
for every $y' \in |Y'|$ the transcendence degree of $y'/g(y')$ is $0$.
Then $\text{Ass}_{X'/Y'}(\mathcal{F}')$ is the inverse image of
$\text{Ass}_{X/Y}(\mathcal{F})$.
\end{lemma}

\begin{proof}
The transcendence degree of a point over its image is defined in
Morphisms of Spaces, Definition
\ref{spaces-morphisms-definition-dimension-fibre}.
Let $x' \in |X'|$ with image $x \in |X|$.
Choose a scheme $V$ and a surjective \'etale morphism $V \to Y$.
Choose a scheme $U$ and a surjective
\'etale morphism $U \to V \times_Y X$. Choose a scheme $V'$
and a surjective \'etale morphism $V' \to V \times_Y Y'$.
Then $U' = V' \times_V U$ is a scheme and the morphism
$U' \to X'$ is surjective and \'etale.
Choose $u \in U$ mapping to $x$.
Denote $v \in V$ the image of $u$.
Then $x \in \text{Ass}_{X/Y}(\mathcal{F})$ is
equivalent to $u \in \text{Ass}(\mathcal{F}|_{U_v})$
by definition. Choose a point $u' \in U'$ mapping
to $x' \in |X'|$ and to $u \in U$ (possible by
Properties of Spaces, Lemma \ref{spaces-properties-lemma-points-cartesian}).
Let $v' \in V'$ be the image of $u'$.
Then $x' \in \text{Ass}_{X'/Y'}(\mathcal{F}')$ is
equivalent to $u' \in \text{Ass}(\mathcal{F}'|_{U'_{v'}})$
by definition.
Now the lemma follows from the discussion in
Divisors, Remark \ref{divisors-remark-base-change-relative-assassin}
applied to $u' \in \Spec(\kappa(v') \otimes_{\kappa(v)} \kappa(u))$.
\end{proof}

\begin{lemma}
\label{lemma-relative-weak-assassin-finite}
Let $S$ be a scheme.
Let $f : X \to Y$ be a morphism of algebraic spaces over $S$.
Let $i : Z \to X$ be a finite morphism.
Let $\mathcal{G}$ be a quasi-coherent $\mathcal{O}_Z$-module.
Then $\text{WeakAss}_{X/Y}(i_*\mathcal{G}) =
i(\text{WeakAss}_{Z/Y}(\mathcal{G}))$.
\end{lemma}

\begin{proof}
Follows from the case of schemes
(Divisors, Lemma \ref{divisors-lemma-relative-weak-assassin-finite})
by \'etale localization. Details omitted.
\end{proof}

\begin{lemma}
\label{lemma-relative-assassin-constructible}
Let $Y$ be a scheme. Let $X$ be an algebraic space of finite presentation
over $Y$. Let $\mathcal{F}$ be a quasi-coherent $\mathcal{O}_X$-module
of finite presentation. Let $U \subset X$ be an open subspace
such that $U \to Y$ is quasi-compact. Then the set
$$
E = \{y \in Y \mid \text{Ass}_{X_y}(\mathcal{F}_y) \subset |U_y|\}
$$
is locally constructible in $Y$.
\end{lemma}

\begin{proof}
Note that since $Y$ is a scheme, it makes sense to take the fibres
$X_y = \Spec(\kappa(y)) \times_Y X$. (Also, by our definitions, the
set $\text{Ass}_{X_y}(\mathcal{F}_y)$ is exactly the fibre of
$\text{Ass}_{X/Y}(\mathcal{F}) \to Y$ over $y$, but we won't need this.)
The question is local on $Y$, indeed, we have to show that
$E$ is constructible if $Y$ is affine.
In this case $X$ is quasi-compact. Choose an affine scheme $W$
and a surjective \'etale morphism $\varphi : W \to X$.
Then $\text{Ass}_{X_y}(\mathcal{F}_y)$ is the image of
$\text{Ass}_{W_y}(\varphi^*\mathcal{F}_y)$ for all $y \in Y$.
Hence the lemma follows from the case of schemes for
the open $\varphi^{-1}(U) \subset W$ and the morphism $W \to Y$.
The case of schemes is
More on Morphisms, Lemma
\ref{more-morphisms-lemma-relative-assassin-constructible}.
\end{proof}









\section{Fitting ideals}
\label{section-fitting-ideals}

\noindent
This section is the continuation of the discussion in
Divisors, Section \ref{divisors-section-fitting-ideals}.
Let $S$ be a scheme. Let $X$ be an algebraic space over $S$.
Let $\mathcal{F}$ be a finite type, quasi-coherent $\mathcal{O}_X$-module.
In this situation we can construct the Fitting ideals
$$
0 = \text{Fit}_{-1}(\mathcal{F}) \subset \text{Fit}_0(\mathcal{F}) \subset
\text{Fit}_1(\mathcal{F}) \subset \ldots \subset \mathcal{O}_X
$$
as the sequence of quasi-coherent sheaves ideals characterized by the
following property: for every affine $U = \Spec(A)$ \'etale over $X$
if $\mathcal{F}|_U$ corresponds to the $A$-module $M$, then
$\text{Fit}_i(\mathcal{F})|_U$
corresponds to the ideal $\text{Fit}_i(M) \subset A$.
This is well defined and a quasi-coherent sheaf of ideals because
if $A \to B$ is an \'etale ring map, then the $i$th Fitting ideal
of $M \otimes_A B$ over $B$ is equal to $\text{Fit}_i(M) B$ by
More on Algebra, Lemma \ref{more-algebra-lemma-fitting-ideal-basics} part (3).
More precisely (perhaps), the existence of the quasi-coherent sheaves of ideals
$\text{Fit}_0(\mathcal{O}_X)$ follows (for example) from
the description of quasi-coherent sheaves in
Properties of Spaces, Lemma
\ref{spaces-properties-lemma-characterize-quasi-coherent-small-etale}
and the pullback property given in
Divisors, Lemma \ref{divisors-lemma-base-change-fitting-ideal}.

\medskip\noindent
The advantage of constructing the Fitting ideals in this way
is that we see immediately that formation of Fitting ideals
commutes with \'etale localization hence many properties of
the Fitting ideals immediately reduce to the corresponding
properties in the case of schemes. Often we will use the
discussion in Properties of Spaces, Section
\ref{spaces-properties-section-properties-modules}
to do the translation between properties of quasi-coherent sheaves
on schemes and on algebraic spaces.

\begin{lemma}
\label{lemma-base-change-fitting-ideal}
Let $S$ be a scheme.
Let $f : X \to Y$ be a morphism of algebraic spaces over $S$.
Let $\mathcal{F}$ be a finite type, quasi-coherent $\mathcal{O}_Y$-module.
Then
$f^{-1}\text{Fit}_i(\mathcal{F}) \cdot \mathcal{O}_X =
\text{Fit}_i(f^*\mathcal{F})$.
\end{lemma}

\begin{proof}
Reduces to
Divisors, Lemma \ref{divisors-lemma-base-change-fitting-ideal}
by \'etale localization.
\end{proof}

\begin{lemma}
\label{lemma-fitting-ideal-of-finitely-presented}
Let $S$ be a scheme. Let $X$ be an algebraic space over $S$.
Let $\mathcal{F}$ be a finitely presented $\mathcal{O}_X$-module.
Then $\text{Fit}_r(\mathcal{F})$ is a quasi-coherent ideal of finite type.
\end{lemma}

\begin{proof}
Reduces to
Divisors, Lemma \ref{divisors-lemma-fitting-ideal-of-finitely-presented}
by \'etale localization.
\end{proof}

\begin{lemma}
\label{lemma-on-subscheme-cut-out-by-Fit-0}
Let $S$ be a scheme. Let $X$ be an algebraic space over $S$.
Let $\mathcal{F}$ be a finite type, quasi-coherent $\mathcal{O}_X$-module.
Let $Z_0 \subset X$ be the closed subspace cut out by
$\text{Fit}_0(\mathcal{F})$.
Let $Z \subset X$ be the scheme theoretic support of $\mathcal{F}$.
Then
\begin{enumerate}
\item $Z \subset Z_0 \subset X$ as closed subspaces,
\item $|Z| = |Z_0| = \text{Supp}(\mathcal{F})$ as closed subsets of $|X|$,
\item there exists a finite type, quasi-coherent $\mathcal{O}_{Z_0}$-module
$\mathcal{G}_0$ with
$$
(Z_0 \to X)_*\mathcal{G}_0 = \mathcal{F}.
$$
\end{enumerate}
\end{lemma}

\begin{proof}
Recall that formation of $Z$ commutes with \'etale localization, see
Morphisms of Spaces, Definition
\ref{spaces-morphisms-definition-scheme-theoretic-support}
(which uses Morphisms of Spaces, Lemma
\ref{spaces-morphisms-lemma-scheme-theoretic-support}
to define $Z$). Hence (1) and (2) follow from the case of schemes, see
Divisors, Lemma \ref{divisors-lemma-on-subscheme-cut-out-by-Fit-0}.
To get $\mathcal{G}_0$ as in part (3) we can use
that we have $\mathcal{G}$ on $Z$ as in Morphisms of Spaces, Lemma
\ref{spaces-morphisms-lemma-scheme-theoretic-support}
and set $\mathcal{G}_0 = (Z \to Z_0)_*\mathcal{G}$.
\end{proof}

\begin{lemma}
\label{lemma-fitting-ideal-generate-locally}
Let $S$ be a scheme. Let $X$ be an algebraic space over $S$.
Let $\mathcal{F}$ be a finite type, quasi-coherent
$\mathcal{O}_X$-module. Let $x \in |X|$. Then $\mathcal{F}$ can be
generated by $r$ elements in an \'etale neighbourhood of $x$ if and only
if $\text{Fit}_r(\mathcal{F})_{\overline{x}} = \mathcal{O}_{X, \overline{x}}$.
\end{lemma}

\begin{proof}
Reduces to
Divisors, Lemma \ref{divisors-lemma-fitting-ideal-generate-locally}
by \'etale localization (as well as the description of the local
ring in Properties of Spaces, Section
\ref{spaces-properties-section-stalks-structure-sheaf}
and the fact that the strict henselization of a local ring
is faithfully flat to see that the equality over the strict
henselization is equivalent to the equality over the local ring).
\end{proof}

\begin{lemma}
\label{lemma-fitting-ideal-finite-locally-free}
Let $S$ be a scheme. Let $X$ be an algebraic space over $S$.
Let $\mathcal{F}$ be a finite type, quasi-coherent
$\mathcal{O}_X$-module. Let $r \geq 0$. The following are equivalent
\begin{enumerate}
\item $\mathcal{F}$ is finite locally free of rank $r$
\item $\text{Fit}_{r - 1}(\mathcal{F}) = 0$ and
$\text{Fit}_r(\mathcal{F}) = \mathcal{O}_X$, and
\item $\text{Fit}_k(\mathcal{F}) = 0$ for $k < r$ and
$\text{Fit}_k(\mathcal{F}) = \mathcal{O}_X$ for $k \geq r$.
\end{enumerate}
\end{lemma}

\begin{proof}
Reduces to
Divisors, Lemma \ref{divisors-lemma-fitting-ideal-finite-locally-free}
by \'etale localization.
\end{proof}

\begin{lemma}
\label{lemma-locally-free-rank-r-pullback}
Let $S$ be a scheme. Let $X$ be an algebraic space over $S$.
Let $\mathcal{F}$ be a finite type, quasi-coherent
$\mathcal{O}_X$-module. The closed subspaces
$$
X = Z_{-1} \supset Z_0 \supset Z_1 \supset Z_2 \ldots
$$
defined by the Fitting ideals of $\mathcal{F}$ have the following
properties
\begin{enumerate}
\item The intersection $\bigcap Z_r$ is empty.
\item The functor $(\Sch/X)^{opp} \to \textit{Sets}$ defined by the rule
$$
T \longmapsto
\left\{
\begin{matrix}
\{*\} & \text{if }\mathcal{F}_T\text{ is locally generated by }
\leq r\text{ sections} \\
\emptyset & \text{otherwise}
\end{matrix}
\right.
$$
is representable by the open subspace $X \setminus Z_r$.
\item The functor $F_r : (\Sch/X)^{opp} \to \textit{Sets}$ defined by the rule
$$
T \longmapsto
\left\{
\begin{matrix}
\{*\} & \text{if }\mathcal{F}_T\text{ locally free rank }r\\
\emptyset & \text{otherwise}
\end{matrix}
\right.
$$
is representable by the locally closed subspace $Z_{r - 1} \setminus Z_r$
of $X$.
\end{enumerate}
If $\mathcal{F}$ is of finite presentation, then
$Z_r \to X$, $X \setminus Z_r \to X$, and $Z_{r - 1} \setminus Z_r \to X$
are of finite presentation.
\end{lemma}

\begin{proof}
Reduces to
Divisors, Lemma \ref{divisors-lemma-locally-free-rank-r-pullback}
by \'etale localization.
\end{proof}

\begin{lemma}
\label{lemma-finite-presentation-module}
Let $S$ be a scheme. Let $X$ be an algebraic space over $S$.
Let $\mathcal{F}$ be an $\mathcal{O}_X$-module
of finite presentation. Let $X = Z_{-1} \subset Z_0 \subset Z_1 \subset \ldots$
be as in Lemma \ref{lemma-locally-free-rank-r-pullback}.
Set $X_r = Z_{r - 1} \setminus Z_r$.
Then $X' = \coprod_{r \geq 0} X_r$ represents the functor
$$
F_{flat} : \Sch/X \longrightarrow \textit{Sets},\quad\quad
T \longmapsto
\left\{
\begin{matrix}
\{*\} & \text{if }\mathcal{F}_T\text{ flat over }T\\
\emptyset & \text{otherwise}
\end{matrix}
\right.
$$
Moreover, $\mathcal{F}|_{X_r}$ is locally free of rank $r$ and the
morphisms $X_r \to X$ and $X' \to X$ are of finite presentation.
\end{lemma}

\begin{proof}
Reduces to
Divisors, Lemma \ref{divisors-lemma-finite-presentation-module}
by \'etale localization.
\end{proof}














\section{Effective Cartier divisors}
\label{section-effective-Cartier-divisors}

\noindent
For some reason it seem convenient to define the notion of an effective
Cartier divisor before anything else. Note that in
Morphisms of Spaces, Section \ref{spaces-morphisms-section-closed-immersions}
we discussed the correspondence between closed subspaces and quasi-coherent
sheaves of ideals. Moreover, in
Properties of Spaces, Section
\ref{spaces-properties-section-properties-modules}, we discussed properties
of quasi-coherent modules, in particular ``locally generated by $1$ element''.
These references show that the following definition is
compatible with the definition for schemes.

\begin{definition}
\label{definition-effective-Cartier-divisor}
Let $S$ be a scheme. Let $X$ be an algebraic space over $S$.
\begin{enumerate}
\item A {\it locally principal closed subspace} of $X$ is a closed subspace
whose sheaf of ideals is locally generated by $1$ element.
\item An {\it effective Cartier divisor} on $X$ is a closed subspace
$D \subset X$ such that the ideal sheaf $\mathcal{I}_D \subset \mathcal{O}_X$
is an invertible $\mathcal{O}_X$-module.
\end{enumerate}
\end{definition}

\noindent
Thus an effective Cartier divisor is a locally principal closed subspace,
but the converse is not always true. Effective Cartier divisors are closed
subspaces of pure codimension $1$ in the strongest possible sense. Namely
they are locally cut out by a single element which is not a zerodivisor.
In particular they are nowhere dense.

\begin{lemma}
\label{lemma-characterize-effective-Cartier-divisor}
Let $S$ be a scheme. Let $X$ be an algebraic space over $S$.
Let $D \subset X$ be a closed subspace.
The following are equivalent:
\begin{enumerate}
\item The subspace $D$ is an effective Cartier divisor on $X$.
\item For some scheme $U$ and surjective \'etale morphism $U \to X$
the inverse image $D \times_X U$ is an effective Cartier divisor on $U$.
\item For every scheme $U$ and every \'etale morphism $U \to X$
the inverse image $D \times_X U$ is an effective Cartier divisor on $U$.
\item For every $x \in |D|$ there exists an \'etale morphism
$(U, u) \to (X, x)$ of pointed algebraic spaces such that $U = \Spec(A)$
and $D \times_X U = \Spec(A/(f))$ with $f \in A$ not a zerodivisor.
\end{enumerate}
\end{lemma}

\begin{proof}
The equivalence of (1) -- (3) follows from
Definition \ref{definition-effective-Cartier-divisor}
and the references preceding it.
Assume (1) and let $x \in |D|$. Choose a scheme $W$ and a
surjective \'etale morphism
$W \to X$. Choose $w \in D \times_X W$ mapping to $x$.
By (3) $D \times_X W$ is an effective Cartier
divisor on $W$. Hence we can find affine \'etale neighbourhood $U$
by choosing an affine open neighbourhood of $w$ in $W$ as in
Divisors, Lemma \ref{divisors-lemma-characterize-effective-Cartier-divisor}.

\medskip\noindent
Assume (4). Then we see that $\mathcal{I}_D|_U$ is invertible by
Divisors, Lemma \ref{divisors-lemma-characterize-effective-Cartier-divisor}.
Since we can find an \'etale covering of $X$ by the collection of
all such $U$ and $X \setminus D$, we conclude that
$\mathcal{I}_D$ is an invertible $\mathcal{O}_X$-module.
\end{proof}

\begin{lemma}
\label{lemma-complement-locally-principal-closed-subscheme}
Let $S$ be a scheme. Let $X$ be an algebraic space over $S$.
Let $Z \subset X$ be a locally principal closed
subspace. Let $U = X \setminus Z$. Then $U \to X$ is an affine morphism.
\end{lemma}

\begin{proof}
The question is \'etale local on $X$, see
Morphisms of Spaces, Lemmas \ref{spaces-morphisms-lemma-affine-local}
and
Lemma \ref{lemma-characterize-effective-Cartier-divisor}.
Thus this follows from the case of schemes which is
Divisors, Lemma
\ref{divisors-lemma-complement-locally-principal-closed-subscheme}.
\end{proof}

\begin{lemma}
\label{lemma-complement-effective-Cartier-divisor}
Let $S$ be a scheme. Let $X$ be an algebraic space over $S$.
Let $D \subset X$ be an effective Cartier divisor.
Let $U = X \setminus D$. Then $U \to X$ is an affine morphism and $U$
is scheme theoretically dense in $X$.
\end{lemma}

\begin{proof}
Affineness is Lemma \ref{lemma-complement-locally-principal-closed-subscheme}.
The density question is \'etale local on $X$ by
Morphisms of Spaces, Definition
\ref{spaces-morphisms-definition-scheme-theoretically-dense}.
Thus this follows from the case of schemes which is
Divisors, Lemma
\ref{divisors-lemma-complement-effective-Cartier-divisor}.
\end{proof}

\begin{lemma}
\label{lemma-effective-Cartier-makes-dimension-drop}
Let $S$ be a scheme. Let $X$ be an algebraic space over $S$.
Let $D \subset X$ be an effective Cartier divisor.
Let $x \in |D|$.
If $\dim_x(X) < \infty$, then $\dim_x(D) < \dim_x(X)$.
\end{lemma}

\begin{proof}
Both the definition of an effective Cartier divisor and of the
dimension of an algebraic space at a point
(Properties of Spaces, Definition
\ref{spaces-properties-definition-dimension-at-point})
are \'etale local. Hence this lemma follows from the case of schemes
which is
Divisors, Lemma \ref{divisors-lemma-effective-Cartier-makes-dimension-drop}.
\end{proof}

\begin{definition}
\label{definition-sum-effective-Cartier-divisors}
Let $S$ be a scheme. Let $X$ be an algebraic space over $S$.
Given effective Cartier divisors
$D_1$, $D_2$ on $X$ we set $D = D_1 + D_2$ equal to the
closed subspace of $X$ corresponding to the quasi-coherent
sheaf of ideals
$\mathcal{I}_{D_1}\mathcal{I}_{D_2} \subset \mathcal{O}_S$.
We call this the {\it sum of the effective Cartier divisors
$D_1$ and $D_2$}.
\end{definition}

\noindent
It is clear that we may define the sum $\sum n_iD_i$ given
finitely many effective Cartier divisors $D_i$ on $X$
and nonnegative integers $n_i$.

\begin{lemma}
\label{lemma-sum-effective-Cartier-divisors}
The sum of two effective Cartier divisors is an effective
Cartier divisor.
\end{lemma}

\begin{proof}
Omitted. \'Etale locally this reduces to the following simple
algebra fact: if $f_1, f_2 \in A$ are nonzerodivisors of a ring $A$, then
$f_1f_2 \in A$ is a nonzerodivisor.
\end{proof}

\begin{lemma}
\label{lemma-sum-closed-subschemes-effective-Cartier}
Let $S$ be a scheme. Let $X$ be an algebraic space over $S$.
Let $Z, Y$ be two closed subspaces of $X$
with ideal sheaves $\mathcal{I}$ and $\mathcal{J}$. If $\mathcal{I}\mathcal{J}$
defines an effective Cartier divisor $D \subset X$, then $Z$ and $Y$
are effective Cartier divisors and $D = Z + Y$.
\end{lemma}

\begin{proof}
By Lemma \ref{lemma-characterize-effective-Cartier-divisor}
this reduces to the case of schemes which is
Divisors, Lemma \ref{divisors-lemma-sum-closed-subschemes-effective-Cartier}.
\end{proof}

\noindent
Recall that we have defined the inverse image of a closed subspace
under any morphism of algebraic spaces in
Morphisms of Spaces, Definition
\ref{spaces-morphisms-definition-inverse-image-closed-subspace}.

\begin{lemma}
\label{lemma-pullback-locally-principal}
Let $S$ be a scheme.
Let $f : X' \to X$ be a morphism of algebraic spaces over $S$.
Let $Z \subset X$ be a locally principal closed subspace.
Then the inverse image $f^{-1}(Z)$ is a locally principal closed
subspace of $X'$.
\end{lemma}

\begin{proof}
Omitted.
\end{proof}

\begin{definition}
\label{definition-pullback-effective-Cartier-divisor}
Let $S$ be a scheme.
Let $f : X' \to X$ be a morphism of algebraic spaces over $S$.
Let $D \subset X$
be an effective Cartier divisor. We say the {\it pullback of
$D$ by $f$ is defined} if the closed subspace $f^{-1}(D) \subset X'$
is an effective Cartier divisor. In this case we denote it either
$f^*D$ or $f^{-1}(D)$ and we call it the
{\it pullback of the effective Cartier divisor}.
\end{definition}

\noindent
The condition that $f^{-1}(D)$ is an effective Cartier divisor
is often satisfied in practice.

\begin{lemma}
\label{lemma-pullback-effective-Cartier-defined}
Let $S$ be a scheme.
Let $f : X \to Y$ be a morphism of algebraic spaces over $S$.
Let $D \subset Y$ be an effective Cartier divisor.
The pullback of $D$ by $f$ is defined in each of the following cases:
\begin{enumerate}
\item $f$ is flat, and
\item add more here as needed.
\end{enumerate}
\end{lemma}

\begin{proof}
Omitted.
\end{proof}

\begin{lemma}
\label{lemma-pullback-effective-Cartier-divisors-additive}
Let $S$ be a scheme.
Let $f : X' \to X$ be a morphism of algebraic spaces over $S$.
Let $D_1$, $D_2$ be effective Cartier divisors on $X$.
If the pullbacks of $D_1$ and $D_2$ are defined then the
pullback of $D = D_1 + D_2$ is defined and
$f^*D = f^*D_1 + f^*D_2$.
\end{lemma}

\begin{proof}
Omitted.
\end{proof}




\section{Effective Cartier divisors and invertible sheaves}
\label{section-effective-Cartier-invertible}

\noindent
Since an effective Cartier divisor has an invertible ideal sheaf
(Definition \ref{definition-effective-Cartier-divisor}) the
following definition makes sense.

\begin{definition}
\label{definition-invertible-sheaf-effective-Cartier-divisor}
Let $S$ be a scheme. Let $X$ be an algebraic space over $S$
and let $D \subset X$ be an effective Cartier divisor with ideal
sheaf $\mathcal{I}_D$.
\begin{enumerate}
\item The {\it invertible sheaf $\mathcal{O}_X(D)$ associated to $D$}
is defined by
$$
\mathcal{O}_X(D) =
\SheafHom_{\mathcal{O}_X}(\mathcal{I}_D, \mathcal{O}_X) =
\mathcal{I}_D^{\otimes -1}.
$$
\item The canonical section, usually denoted $1$ or $1_D$, is the
global section of $\mathcal{O}_X(D)$ corresponding to
the inclusion mapping $\mathcal{I}_D \to \mathcal{O}_X$.
\item We write
$\mathcal{O}_X(-D) = \mathcal{O}_X(D)^{\otimes -1} = \mathcal{I}_D$.
\item Given a second effective Cartier divisor $D' \subset X$ we define
$\mathcal{O}_X(D - D') =
\mathcal{O}_X(D) \otimes_{\mathcal{O}_X} \mathcal{O}_X(-D')$.
\end{enumerate}
\end{definition}

\noindent
Some comments. We will see below that the assignment
$D \mapsto \mathcal{O}_X(D)$ turns addition of effective Cartier
divisors (Definition \ref{definition-sum-effective-Cartier-divisors})
into addition in the Picard group of $X$
(Lemma \ref{lemma-invertible-sheaf-sum-effective-Cartier-divisors}).
However, the expression $D - D'$ in the definition above does not
have any geometric meaning. More precisely, we can think of the
set of effective Cartier divisors on $X$ as a commutative monoid
$\text{EffCart}(X)$ whose zero element is the empty effective Cartier divisor.
Then the assignment $(D, D') \mapsto \mathcal{O}_X(D - D')$ defines
a group homomorphism
$$
\text{EffCart}(X)^{gp} \longrightarrow \Pic(X)
$$
where the left hand side is the group completion of
$\text{EffCart}(X)$. In other words, when we write $\mathcal{O}_X(D - D')$
we may think of $D - D'$ as an element of $\text{EffCart}(X)^{gp}$.

\begin{lemma}
\label{lemma-conormal-effective-Cartier-divisor}
Let $S$ be a scheme. Let $X$ be an algebraic space over $S$.
Let $D \subset X$ be an effective Cartier divisor.
Then for the conormal sheaf we have $\mathcal{C}_{D/X} = \mathcal{I}_D|D =
\mathcal{O}_X(D)^{\otimes -1}|_D$.
\end{lemma}

\begin{proof}
Omitted.
\end{proof}

\begin{lemma}
\label{lemma-invertible-sheaf-sum-effective-Cartier-divisors}
Let $S$ be a scheme. Let $X$ be an algebraic space over $S$.
Let $D_1$, $D_2$ be effective Cartier divisors on $X$.
Let $D = D_1 + D_2$.
Then there is a unique isomorphism
$$
\mathcal{O}_X(D_1) \otimes_{\mathcal{O}_X} \mathcal{O}_X(D_2)
\longrightarrow
\mathcal{O}_X(D)
$$
which maps $1_{D_1} \otimes 1_{D_2}$ to $1_D$.
\end{lemma}

\begin{proof}
Omitted.
\end{proof}

\begin{definition}
\label{definition-regular-section}
Let $S$ be a scheme. Let $X$ be an algebraic space over $S$.
Let $\mathcal{L}$ be an invertible sheaf on $X$.
A global section $s \in \Gamma(X, \mathcal{L})$ is called a
{\it regular section} if the map $\mathcal{O}_X \to \mathcal{L}$,
$f \mapsto fs$ is injective.
\end{definition}

\begin{lemma}
\label{lemma-regular-section-structure-sheaf}
Let $S$ be a scheme.
Let $X$ be an algebraic space over $S$.
Let $f \in \Gamma(X, \mathcal{O}_X)$.
The following are equivalent:
\begin{enumerate}
\item $f$ is a regular section, and
\item for any $x \in X$ the image $f \in \mathcal{O}_{X, \overline{x}}$
is not a zerodivisor.
\item for any affine $U = \Spec(A)$ \'etale over $X$
the restriction $f|_U$ is a nonzerodivisor of $A$, and
\item there exists a scheme $U$ and a surjective \'etale morphism
$U \to X$ such that $f|_U$ is a regular section of $\mathcal{O}_U$.
\end{enumerate}
\end{lemma}

\begin{proof}
Omitted.
\end{proof}

\noindent
Note that a global section $s$ of an invertible $\mathcal{O}_X$-module
$\mathcal{L}$ may be seen as an $\mathcal{O}_X$-module map
$s : \mathcal{O}_X \to \mathcal{L}$. Its dual is therefore a
map $s : \mathcal{L}^{\otimes -1} \to \mathcal{O}_X$.
(See Modules on Sites, Lemma \ref{sites-modules-lemma-constructions-invertible}
for the dual invertible sheaf.)

\begin{definition}
\label{definition-zero-scheme-s}
Let $S$ be a scheme. Let $X$ be an algebraic space over $S$.
Let $\mathcal{L}$ be an invertible sheaf.
Let $s \in \Gamma(X, \mathcal{L})$.
The {\it zero scheme} of $s$ is the closed subspace $Z(s) \subset X$
defined by the quasi-coherent sheaf of ideals
$\mathcal{I} \subset \mathcal{O}_X$ which is the image of the
map $s : \mathcal{L}^{\otimes -1} \to \mathcal{O}_X$.
\end{definition}

\begin{lemma}
\label{lemma-zero-scheme}
Let $S$ be a scheme. Let $X$ be an algebraic space over $S$.
Let $\mathcal{L}$ be an invertible $\mathcal{O}_X$-module.
Let $s \in \Gamma(X, \mathcal{L})$.
\begin{enumerate}
\item Consider closed immersions $i : Z \to X$ such that
$i^*s \in \Gamma(Z, i^*\mathcal{L}))$ is zero
ordered by inclusion. The zero scheme $Z(s)$ is the
maximal element of this ordered set.
\item For any morphism of algebraic spaces $f : Y \to X$ over $S$
we have $f^*s = 0$ in $\Gamma(Y, f^*\mathcal{L})$ if and only if
$f$ factors through $Z(s)$.
\item The zero scheme $Z(s)$ is a locally principal closed subspace of $X$.
\item The zero scheme $Z(s)$ is an effective Cartier divisor on $X$
if and only if $s$ is a regular section of $\mathcal{L}$.
\end{enumerate}
\end{lemma}

\begin{proof}
Omitted.
\end{proof}

\begin{lemma}
\label{lemma-characterize-OD}
Let $S$ be a scheme. Let $X$ be an algebraic space over $S$.
\begin{enumerate}
\item If $D \subset X$ is an effective Cartier divisor, then
the canonical section $1_D$ of $\mathcal{O}_X(D)$ is regular.
\item Conversely, if $s$ is a regular section of the invertible
sheaf $\mathcal{L}$, then there exists a unique effective
Cartier divisor $D = Z(s) \subset X$ and a unique isomorphism
$\mathcal{O}_X(D) \to \mathcal{L}$ which maps $1_D$ to $s$.
\end{enumerate}
The constructions
$D \mapsto (\mathcal{O}_X(D), 1_D)$ and $(\mathcal{L}, s) \mapsto Z(s)$
give mutually inverse maps
$$
\left\{
\begin{matrix}
\text{effective Cartier divisors on }X
\end{matrix}
\right\}
\leftrightarrow
\left\{
\begin{matrix}
\text{pairs }(\mathcal{L}, s)\text{ consisting of an invertible}\\
\mathcal{O}_X\text{-module and a regular global section}
\end{matrix}
\right\}
$$
\end{lemma}

\begin{proof}
Omitted.
\end{proof}





\section{Effective Cartier divisors on Noetherian spaces}
\label{section-Noetherian-effective-Cartier}

\noindent
In the locally Noetherian setting most of the discussion of
effective Cartier divisors and regular sections simplifies somewhat.

\begin{lemma}
\label{lemma-effective-Cartier-divisor-Sk}
Let $S$ be a scheme and let $X$ be a locally Noetherian algebraic space
over $S$. Let $D \subset X$ be an effective Cartier divisor. If $X$ is
$(S_k)$, then $D$ is $(S_{k - 1})$.
\end{lemma}

\begin{proof}
By our definition of the property $(S_k)$ for algebraic spaces
(Properties of Spaces, Section
\ref{spaces-properties-section-types-properties})
and
Lemma \ref{lemma-characterize-effective-Cartier-divisor}
this follows from the case of schemes
(Divisors, Lemma \ref{divisors-lemma-effective-Cartier-divisor-Sk}).
\end{proof}

\begin{lemma}
\label{lemma-normal-effective-Cartier-divisor-S1}
Let $S$ be a scheme and let $X$ be a locally Noetherian normal
algebraic space over $S$. Let $D \subset X$ be an
effective Cartier divisor. Then $D$ is $(S_1)$.
\end{lemma}

\begin{proof}
By our definition of normality for algebraic spaces
(Properties of Spaces, Section
\ref{spaces-properties-section-types-properties})
and
Lemma \ref{lemma-characterize-effective-Cartier-divisor}
this follows from the case of schemes
(Divisors, Lemma \ref{divisors-lemma-normal-effective-Cartier-divisor-S1}).
\end{proof}

\noindent
The following lemma can sometimes be used to produce effective
Cartier divisors.

\begin{lemma}
\label{lemma-complement-open-affine-effective-cartier-divisor}
Let $S$ be a scheme. Let $X$ be a regular Noetherian separated algebraic space
over $S$. Let $U \subset X$ be a dense affine open. Then there exists an
effective Cartier divisor $D \subset X$ with $U = X \setminus D$.
\end{lemma}

\begin{proof}
We claim that the reduced induced algebraic space structure $D$
on $X \setminus U$ (Properties of Spaces, Definition
\ref{spaces-properties-definition-reduced-induced-space})
is the desired effective Cartier divisor. The construction
of $D$ commutes with \'etale localization, see proof of
Properties of Spaces, Lemma
\ref{spaces-properties-lemma-reduced-closed-subspace}.
Let $X' \to X$ be a surjective \'etale morphism with $X'$ affine.
Since $X$ is separated, we see that $U' = X' \times_X U$ is
affine. Since $|X'| \to |X|$ is open, we see that $U'$
is dense in $X'$. Since $D' = X' \times_X D$ is the reduced induced
scheme structure on $X' \setminus U'$, we conclude that
$D'$ is an effective Cartier divisor by
Divisors, Lemma
\ref{divisors-lemma-complement-open-affine-effective-cartier-divisor}
and its proof. This is what we had to show.
\end{proof}

\begin{lemma}
\label{lemma-Noetherian-regular-separated-pic-effective-Cartier}
Let $S$ be a scheme. Let $X$ be a regular Noetherian separated algebraic space
over $S$. Then every invertible $\mathcal{O}_X$-module is isomorphic to
$$
\mathcal{O}_X(D - D') =
\mathcal{O}_X(D) \otimes_{\mathcal{O}_X} \mathcal{O}_X(D')^{\otimes -1}
$$
for some effective Cartier divisors $D, D'$ in $X$.
\end{lemma}

\begin{proof}
Let $\mathcal{L}$ be an invertible $\mathcal{O}_X$-module.
Choose a dense affine open $U \subset X$ such that
$\mathcal{L}|_U$ is trivial. This is possible because
$X$ has a dense open subspace which is a scheme, see
Properties of Spaces, Proposition
\ref{spaces-properties-proposition-locally-quasi-separated-open-dense-scheme}.
Denote $s : \mathcal{O}_U \to \mathcal{L}|_U$ the trivialization.
The complement of $U$ is an effective Cartier divisor
$D$. We claim that for some $n > 0$ the map $s$ extends uniquely to a map
$$
s : \mathcal{O}_X(-nD) \longrightarrow \mathcal{L}
$$
The claim implies the lemma because it shows that
$\mathcal{L} \otimes_{\mathcal{O}_X} \mathcal{O}_X(nD)$
has a regular global section hence is isomorphic to
$\mathcal{O}_X(D')$ for some effective Cartier divisor $D'$
by Lemma \ref{lemma-characterize-OD}.
To prove the claim we may work \'etale locally. Thus we may assume
$X$ is an affine Noetherian scheme. Since
$\mathcal{O}_X(-nD) = \mathcal{I}^n$ where $\mathcal{I} = \mathcal{O}_X(-D)$
is the ideal sheaf of $D$ in $X$, this case follows from
Cohomology of Schemes, Lemma \ref{coherent-lemma-homs-over-open}.
\end{proof}

\noindent
The following lemma really belongs to a different section.

\begin{lemma}
\label{lemma-smooth-over-valuation-ring-effective-Cartier}
Let $R$ be a valuation ring with fraction field $K$.
Let $X$ be an algebraic space over $R$ such that $X \to \Spec(R)$
is smooth. For every effective Cartier divisor $D \subset X_K$
there exists an effective Cartier divisor $D' \subset X$
with $D'_K = D$.
\end{lemma}

\begin{proof}
Let $D' \subset X$ be the scheme theoretic image of $D \to X_K \to X$.
Since this morphism is quasi-compact, formation of $D'$
commutes with flat base change, see
Morphisms of Spaces, Lemma
\ref{spaces-morphisms-lemma-flat-base-change-scheme-theoretic-image}.
In particular we find that $D'_K = D$. Hence,
we may assume $X$ is affine. Say $X = \Spec(A)$.
Then $X_K = \Spec(A \otimes_R K)$ and $D$ corresponds to
an ideal $I \subset A \otimes_R K$. We have to show that
$J = I \cap A$ cuts out an effective Cartier divisor in $X$.
First, observe that $A/J$ is flat over $R$ (as a torsion
free $R$-module, see More on Algebra, Lemma
\ref{more-algebra-lemma-valuation-ring-torsion-free-flat}),
hence $J$ is finitely generated by
More on Algebra, Lemma
\ref{more-algebra-lemma-flat-finite-type-valuation-ring-finite-presentation}
and
Algebra, Lemma \ref{algebra-lemma-extension}.
Thus it suffices to show that $J_\mathfrak q \subset A_\mathfrak q$
is generated by a single element for each prime $\mathfrak q \subset A$.
Let $\mathfrak p = R \cap \mathfrak q$. Then
$R_\mathfrak p$ is a valuation ring
(Algebra, Lemma \ref{algebra-lemma-make-valuation-rings}).
Observe further that $A_\mathfrak q/\mathfrak p A_\mathfrak q$
is a regular ring by Algebra, Lemma
\ref{algebra-lemma-characterize-smooth-over-field}.
Thus we may apply More on Algebra, Lemma
\ref{more-algebra-lemma-picard-group-generic-fibre-regular}
to see that $I(A_\mathfrak q \otimes_R K)$ is generated by
a single element $f \in A_\mathfrak p \otimes_R K$.
After clearing denominators we may assume $f \in A_\mathfrak q$.
Let $\mathfrak c \subset R_\mathfrak p$ be the content ideal of $f$
(see More on Algebra, Definition \ref{more-algebra-definition-content-ideal}
and More on Flatness, Lemma
\ref{flat-lemma-flat-finite-type-local-valuation-ring-has-content}).
Since $R_\mathfrak p$ is a valuation ring and
since $\mathfrak c$ is finitely generated
(More on Algebra, Lemma \ref{more-algebra-lemma-content-finitely-generated})
we see $\mathfrak c = (\pi)$ for some $\pi \in R_\mathfrak p$
(Algebra, Lemma \ref{algebra-lemma-characterize-valuation-ring}).
After relacing $f$ by $\pi^{-1}f$ we see that $f \in A_\mathfrak q$
and $f \not \in \mathfrak pA_\mathfrak q$.
Claim: $I_\mathfrak q = (f)$ which finishes the proof.
To see the claim, observe that $f \in I_\mathfrak q$.
Hence we have a surjection $A_\mathfrak q/(f) \to A_\mathfrak q/I_\mathfrak q$
which is an isomorphism after tensoring over $R$ with $K$.
Thus we are done if
$A_\mathfrak q/(f)$ is $R_\mathfrak p$-flat.
This follows from
Algebra, Lemma \ref{algebra-lemma-grothendieck-general}
and our choice of $f$.
\end{proof}









\section{Relative Proj}
\label{section-relative-proj}

\noindent
This section revisits the construction of the relative proj
in the setting of algebraic spaces. The material in this section
corresponds to the material in Constructions, Section
\ref{constructions-section-relative-proj}
and Divisors, Section \ref{divisors-section-relative-proj}
in the case of schemes.

\begin{situation}
\label{situation-relative-proj}
Here $S$ is a scheme, $X$ is an algebraic space over $S$, and
$\mathcal{A}$ is a quasi-coherent graded $\mathcal{O}_X$-algebra.
\end{situation}

\noindent
In Situation \ref{situation-relative-proj} we are going to define
a functor $F : (\Sch/S)_{fppf}^{opp} \to \textit{Sets}$ which will
turn out to be an algebraic space. We will follow (mutatis mutandis)
the procedure of
Constructions, Section \ref{constructions-section-relative-proj}.
First, given a scheme $T$ over $S$ we define a
{\it quadruple over $T$} to be a system
$(d, f : T \to X, \mathcal{L}, \psi)$
\begin{enumerate}
\item $d \geq 1$ is an integer,
\item $f : T \to X$ is a morphism over $S$,
\item $\mathcal{L}$ is an invertible $\mathcal{O}_T$-module, and
\item
$\psi : f^*\mathcal{A}^{(d)} \to \bigoplus_{n \geq 0}\mathcal{L}^{\otimes n}$
is a homomorphism of graded $\mathcal{O}_T$-algebras
such that $f^*\mathcal{A}_d \to \mathcal{L}$ is surjective.
\end{enumerate}
We say two quadruples $(d, f, \mathcal{L}, \psi)$ and
$(d', f', \mathcal{L}', \psi')$ are {\it equivalent}\footnote{This
definition is motivated by
Constructions, Lemma \ref{constructions-lemma-equivalent-relative}.
The advantage of choosing this one is that it clearly defines
an equivalence relation.}
if and only if
we have $f = f'$ and for some positive integer $m = ad = a'd'$
there exists an isomorphism
$\beta : \mathcal{L}^{\otimes a} \to (\mathcal{L}')^{\otimes a'}$
with the property that $\beta \circ \psi|_{f^*\mathcal{A}^{(m)}}$
and $\psi'|_{f^*\mathcal{A}^{(m)}}$ agree
as graded ring maps
$f^*\mathcal{A}^{(m)} \to \bigoplus_{n \geq 0} (\mathcal{L}')^{\otimes mn}$.
Given a quadruple $(d, f, \mathcal{L}, \psi)$
and a morphism $h : T' \to T$ we have the pullback
$(d, f \circ h, h^*\mathcal{L}, h^*\psi)$. Pullback preserves 
the equivalence relation. Finally, for a {\it quasi-compact} scheme $T$
over $S$ we set
$$
F(T) = \text{the set of equivalence classes of quadruples over }T
$$
and for an arbitrary scheme $T$ over $S$ we set
$$
F(T)
=
\lim_{V \subset T\text{ quasi-compact open}} F(V).
$$
In other words, an element $\xi$ of $F(T)$ corresponds to a compatible
system of choices of elements $\xi_V \in F(V)$ where $V$ ranges over the
quasi-compact opens of $T$. Thus we have defined our functor
\begin{equation}
\label{equation-proj}
F : \Sch^{opp} \longrightarrow \textit{Sets}
\end{equation}
There is a morphism $F \to X$ of functors sending the quadruple
$(d, f, \mathcal{L}, \psi)$ to $f$.

\begin{lemma}
\label{lemma-relative-proj}
In Situation \ref{situation-relative-proj}. The functor $F$ above is an
algebraic space. For any morphism $g : Z \to X$ where $Z$ is a scheme
there is a canonical isomorphism
$\underline{\text{Proj}}_Z(g^*\mathcal{A}) = Z \times_X F$
compatible with further base change.
\end{lemma}

\begin{proof}
It suffices to prove the second assertion, see
Spaces, Lemma \ref{spaces-lemma-representable-over-space}.
Let $g : Z \to X$ be a morphism where $Z$ is a scheme.
Let $F'$ be the functor of quadruples associated
to the graded quasi-coherent $\mathcal{O}_Z$-algebra $g^*\mathcal{A}$.
Then there is a canonical isomorphism $F' = Z \times_X F$, sending
a quadruple $(d, f : T \to Z, \mathcal{L}, \psi)$ for $F'$
to $(d, g \circ f, \mathcal{L}, \psi)$ (details omitted, see proof of
Constructions, Lemma \ref{constructions-lemma-proj-base-change}).
By Constructions, Lemmas
\ref{constructions-lemma-equivalent-relative},
\ref{constructions-lemma-relative-proj}, and
\ref{constructions-lemma-glueing-gives-functor-proj} and
Definition \ref{constructions-definition-relative-proj}
we see that $F'$ is representable by
$\underline{\text{Proj}}_Z(g^*\mathcal{A})$.
\end{proof}

\noindent
The lemma above tells us the following definition makes sense.

\begin{definition}
\label{definition-relative-proj}
Let $S$ be a scheme. Let $X$ be an algebraic space over $S$.
Let $\mathcal{A}$ be a quasi-coherent sheaf of
graded $\mathcal{O}_X$-algebras. The
{\it relative homogeneous spectrum of $\mathcal{A}$ over $X$},
or the {\it homogeneous spectrum of $\mathcal{A}$ over $X$}, or the
{\it relative Proj of $\mathcal{A}$ over $X$} is the algebraic space
$F$ over $X$ of Lemma \ref{lemma-relative-proj}.
We denote it $\pi : \underline{\text{Proj}}_X(\mathcal{A}) \to X$.
\end{definition}

\noindent
In particular the structure morphism of the relative Proj is representable
by construction. We can also think about the relative Proj via glueing. Let
$\varphi : U \to X$ be a surjective \'etale morphism, where $U$ is a scheme.
Set $R = U \times_X U$ with projection morphisms $s, t : R  \to U$.
By Lemma \ref{lemma-relative-proj} there exists a canonical isomorphism
$$
\gamma : 
\underline{\text{Proj}}_U(\varphi^*\mathcal{A})
\longrightarrow
\underline{\text{Proj}}_X(\mathcal{A}) \times_X U
$$
over $U$. Let $\alpha : t^*\varphi^*\mathcal{A} \to s^*\varphi^*\mathcal{A}$
be the canonical isomorphism of
Properties of Spaces, Proposition
\ref{spaces-properties-proposition-quasi-coherent}.
Then the diagram
$$
\xymatrix{
&
\underline{\text{Proj}}_U(\varphi^*\mathcal{A}) \times_{U, s} R
\ar@{=}[r] &
\underline{\text{Proj}}_R(s^*\varphi^*\mathcal{A})
\ar[dd]_{\text{induced by }\alpha} \\
\underline{\text{Proj}}_X(\mathcal{A}) \times_X R
\ar[ru]_{s^*\gamma} \ar[rd]^{t^*\gamma} \\
&
\underline{\text{Proj}}_U(\varphi^*\mathcal{A}) \times_{U, t} R
\ar@{=}[r] &
\underline{\text{Proj}}_R(t^*\varphi^*\mathcal{A})
}
$$
is commutative (the equal signs come from
Constructions, Lemma \ref{constructions-lemma-relative-proj-base-change}).
Thus, if we denote $\mathcal{A}_U$, $\mathcal{A}_R$
the pullback of $\mathcal{A}$ to $U$, $R$, then
$P = \underline{\text{Proj}}_X(\mathcal{A})$ has an \'etale covering
by the scheme $P_U = \underline{\text{Proj}}_U(\mathcal{A}_U)$ and
$P_U \times_P P_U$ is equal to
$P_R = \underline{\text{Proj}}_R(\mathcal{A}_R)$.
Using these remarks we can argue in the usual fashion using \'etale
localization to transfer results on the relative proj from the case
of schemes to the case of algebraic spaces.

\begin{lemma}
\label{lemma-twists-of-structure-sheaf}
In Situation \ref{situation-relative-proj}. The relative Proj comes
equipped with a quasi-coherent sheaf of $\mathbf{Z}$-graded algebras
$\bigoplus_{n \in \mathbf{Z}}
\mathcal{O}_{\underline{\text{Proj}}_X(\mathcal{A})}(n)$
and a canonical homomorphism of graded algebras
$$
\psi :
\pi^*\mathcal{A}
\longrightarrow
\bigoplus\nolimits_{n \geq 0}
\mathcal{O}_{\underline{\text{Proj}}_X(\mathcal{A})}(n)
$$
whose base change to any scheme over $X$ agrees with
Constructions, Lemma \ref{constructions-lemma-glue-relative-proj-twists}.
\end{lemma}

\begin{proof}
As in the discussion following Definition \ref{definition-relative-proj}
choose a scheme $U$ and a surjective \'etale morphism
$U \to X$, set $R = U \times_X U$ with projections $s, t : R \to U$,
$\mathcal{A}_U = \mathcal{A}|_U$, $\mathcal{A}_R = \mathcal{A}|_R$,
and $\pi : P = \underline{\text{Proj}}_X(\mathcal{A}) \to X$,
$\pi_U : P_U = \underline{\text{Proj}}_U(\mathcal{A}_U)$ and
$\pi_R : P_R = \underline{\text{Proj}}_U(\mathcal{A}_R)$.
By the
Constructions, Lemma \ref{constructions-lemma-glue-relative-proj-twists}
we have a quasi-coherent sheaf of $\mathbf{Z}$-graded
$\mathcal{O}_{P_U}$-algebras
$\bigoplus_{n \in \mathbf{Z}} \mathcal{O}_{P_U}(n)$
and a canonical map
$\psi_U : \pi_U^*\mathcal{A}_U \to \bigoplus_{n \geq 0} \mathcal{O}_{P_U}(n)$
and similarly for $P_R$. By
Constructions, Lemma \ref{constructions-lemma-relative-proj-base-change}
the pullback of $\mathcal{O}_{P_U}(n)$ and $\psi_U$ by either projection
$P_R \to P_U$ is equal to $\mathcal{O}_{P_R}(n)$ and $\psi_R$.
By Properties of Spaces, Proposition
\ref{spaces-properties-proposition-quasi-coherent}
we obtain $\mathcal{O}_{P}(n)$ and $\psi$.
We omit the verification of compatibility with pullback to
arbitrary schemes over $X$.
\end{proof}

\noindent
Having constructed the relative Proj we turn to some basic
properties.

\begin{lemma}
\label{lemma-relative-proj-base-change}
Let $S$ be a scheme. Let $g : X' \to X$ be a morphism of algebraic spaces
over $S$ and let $\mathcal{A}$ be a quasi-coherent sheaf
of graded $\mathcal{O}_X$-algebras. Then there is a canonical isomorphism
$$
r :
\underline{\text{Proj}}_{X'}(g^*\mathcal{A})
\longrightarrow
X' \times_X \underline{\text{Proj}}_X(\mathcal{A})
$$
as well as a corresponding isomorphism
$$
\theta :
r^*\text{pr}_2^*\left(\bigoplus\nolimits_{d \in \mathbf{Z}}
\mathcal{O}_{\underline{\text{Proj}}_X(\mathcal{A})}(d)\right)
\longrightarrow
\bigoplus\nolimits_{d \in \mathbf{Z}}
\mathcal{O}_{\underline{\text{Proj}}_{X'}(g^*\mathcal{A})}(d)
$$
of $\mathbf{Z}$-graded
$\mathcal{O}_{\underline{\text{Proj}}_{X'}(g^*\mathcal{A})}$-algebras.
\end{lemma}

\begin{proof}
Let $F$ be the functor (\ref{equation-proj}) and let $F'$ be the
corresponding functor defined using $g^*\mathcal{A}$ on $X'$.
We claim there is a canonical isomorphism $r : F' \to X' \times_X F$
of functors (and of course $r$ is the isomorphism of the lemma).
It suffices to construct the bijection
$r : F'(T) \to X'(T) \times_{X(T)} F(T)$ for quasi-compact schemes $T$
over $S$. First, if $\xi = (d', f', \mathcal{L}', \psi')$ is a
quadruple over $T$ for $F'$, then we can set
$r(\xi) = (f', (d', g \circ f', \mathcal{L}', \psi'))$. This makes sense
as $(g \circ f')^*\mathcal{A}^{(d)} = (f')^*(g^*\mathcal{A})^{(d)}$.
The inverse map sends the pair $(f', (d, f, \mathcal{L}, \psi))$
to the quadruple $(d, f', \mathcal{L}, \psi)$. We omit the proof
of the final assertion (hint: reduce to the case of schemes by \'etale
localization and apply Constructions, Lemma
\ref{constructions-lemma-relative-proj-base-change}).
\end{proof}

\begin{lemma}
\label{lemma-relative-proj-separated}
In Situation \ref{situation-relative-proj} the morphism
$\pi : \underline{\text{Proj}}_X(\mathcal{A}) \to X$
is separated.
\end{lemma}

\begin{proof}
By Morphisms of Spaces, Lemma \ref{spaces-morphisms-lemma-separated-local}
and the construction of the relative Proj this follows from the
case of schemes which is
Constructions, Lemma \ref{constructions-lemma-relative-proj-separated}.
\end{proof}

\begin{lemma}
\label{lemma-relative-proj-quasi-compact}
In Situation \ref{situation-relative-proj}. If one of the following holds
\begin{enumerate}
\item $\mathcal{A}$ is of finite type as a sheaf of
$\mathcal{A}_0$-algebras,
\item $\mathcal{A}$ is generated by $\mathcal{A}_1$ as an
$\mathcal{A}_0$-algebra and $\mathcal{A}_1$ is a finite type
$\mathcal{A}_0$-module,
\item there exists a finite type quasi-coherent $\mathcal{A}_0$-submodule
$\mathcal{F} \subset \mathcal{A}_{+}$ such that
$\mathcal{A}_{+}/\mathcal{F}\mathcal{A}$ is a locally nilpotent
sheaf of ideals of $\mathcal{A}/\mathcal{F}\mathcal{A}$,
\end{enumerate}
then $\pi : \underline{\text{Proj}}_X(\mathcal{A}) \to X$ is quasi-compact.
\end{lemma}

\begin{proof}
By Morphisms of Spaces, Lemma \ref{spaces-morphisms-lemma-quasi-compact-local}
and the construction of the relative Proj this follows from the
case of schemes which is
Divisors, Lemma \ref{divisors-lemma-relative-proj-quasi-compact}.
\end{proof}

\begin{lemma}
\label{lemma-relative-proj-finite-type}
In Situation \ref{situation-relative-proj}.
If $\mathcal{A}$ is of finite type as a sheaf of
$\mathcal{O}_X$-algebras, then
$\pi : \underline{\text{Proj}}_X(\mathcal{A}) \to X$ is of finite type.
\end{lemma}

\begin{proof}
By Morphisms of Spaces, Lemma \ref{spaces-morphisms-lemma-finite-type-local}
and the construction of the relative Proj this follows from the
case of schemes which is
Divisors, Lemma \ref{divisors-lemma-relative-proj-finite-type}.
\end{proof}

\begin{lemma}
\label{lemma-relative-proj-universally-closed}
In Situation \ref{situation-relative-proj}. If
$\mathcal{O}_X \to \mathcal{A}_0$
is an integral algebra map\footnote{In other words, the integral
closure of $\mathcal{O}_X$ in $\mathcal{A}_0$, see
Morphisms of Spaces, Definition
\ref{spaces-morphisms-definition-integral-closure}, equals
$\mathcal{A}_0$.} and $\mathcal{A}$ is of finite type as an
$\mathcal{A}_0$-algebra, then
$\pi : \underline{\text{Proj}}_X(\mathcal{A}) \to X$ is universally closed.
\end{lemma}

\begin{proof}
By Morphisms of Spaces, Lemma
\ref{spaces-morphisms-lemma-universally-closed-local}
and the construction of the relative Proj this follows from the
case of schemes which is
Divisors, Lemma \ref{divisors-lemma-relative-proj-universally-closed}.
\end{proof}

\begin{lemma}
\label{lemma-relative-proj-proper}
In Situation \ref{situation-relative-proj}.
The following conditions are equivalent
\begin{enumerate}
\item $\mathcal{A}_0$ is a finite type $\mathcal{O}_X$-module
and $\mathcal{A}$ is of finite type as an $\mathcal{A}_0$-algebra,
\item $\mathcal{A}_0$ is a finite type $\mathcal{O}_X$-module 
and $\mathcal{A}$ is of finite type as an $\mathcal{O}_X$-algebra.
\end{enumerate}
If these conditions hold, then
$\pi : \underline{\text{Proj}}_X(\mathcal{A}) \to X$
is proper.
\end{lemma}

\begin{proof}
By Morphisms of Spaces, Lemma
\ref{spaces-morphisms-lemma-proper-local}
and the construction of the relative Proj this follows from the
case of schemes which is
Divisors, Lemma \ref{divisors-lemma-relative-proj-universally-closed}.
\end{proof}

\begin{lemma}
\label{lemma-relative-proj-generated-in-degree-1}
Let $S$ be a scheme. Let $X$ be an algebraic space over $S$.
Let $\mathcal{A}$ be a quasi-coherent sheaf of graded $\mathcal{O}_X$-modules
generated as an $\mathcal{A}_0$-algebra by $\mathcal{A}_1$.
With $P = \underline{\text{Proj}}_X(\mathcal{A})$ we have
\begin{enumerate}
\item $P$ represents the functor $F_1$ which associates to
$T$ over $S$ the set of isomorphism classes of
triples $(f, \mathcal{L}, \psi)$, where $f : T \to X$ is a morphism
over $S$, $\mathcal{L}$ is an invertible $\mathcal{O}_T$-module, and
$\psi : f^*\mathcal{A} \to \bigoplus_{n \geq 0} \mathcal{L}^{\otimes n}$
is a map of graded $\mathcal{O}_T$-algebras inducing a surjection
$f^*\mathcal{A}_1 \to \mathcal{L}$,
\item the canonical map $\pi^*\mathcal{A}_1 \to \mathcal{O}_P(1)$ is
surjective, and
\item each $\mathcal{O}_P(n)$ is invertible
and the multiplication maps induce isomorphisms
$\mathcal{O}_P(n) \otimes_{\mathcal{O}_P} \mathcal{O}_P(m) =
\mathcal{O}_P(n + m)$.
\end{enumerate}
\end{lemma}

\begin{proof}
Omitted.
See Constructions, Lemma \ref{constructions-lemma-apply-relative}
for the case of schemes.
\end{proof}







\section{Functoriality of relative proj}
\label{section-functoriality-relative-proj}

\noindent
This section is the analogue of
Constructions, Section \ref{constructions-section-functoriality-relative-proj}.

\begin{lemma}
\label{lemma-morphism-relative-proj}
Let $S$ be a scheme. Let $X$ be an algebraic space over $S$.
Let $\psi : \mathcal{A} \to \mathcal{B}$ be a map of
quasi-coherent graded $\mathcal{O}_X$-algebras. Set
$P = \underline{\text{Proj}}_X(\mathcal{A}) \to X$ and
$Q = \underline{\text{Proj}}_X(\mathcal{B}) \to X$.
There is a canonical open subspace
$U(\psi) \subset Q$ and a canonical morphism of
algebraic spaces
$$
r_\psi :
U(\psi)
\longrightarrow
P
$$
over $X$ and a map of $\mathbf{Z}$-graded $\mathcal{O}_{U(\psi)}$-algebras
$$
\theta = \theta_\psi :
r_\psi^*\left(
\bigoplus\nolimits_{d \in \mathbf{Z}} \mathcal{O}_P(d)
\right)
\longrightarrow
\bigoplus\nolimits_{d \in \mathbf{Z}} \mathcal{O}_{U(\psi)}(d).
$$
The triple $(U(\psi), r_\psi, \theta)$ is characterized by the property
that for any scheme $W$ \'etale over $X$ the triple
$$
(U(\psi) \times_X W,\quad
r_\psi|_{U(\psi) \times_X W} : U(\psi) \times_X W \to  P \times_X W,\quad
\theta|_{U(\psi) \times_X W})
$$
is equal to the triple associated to $\psi : \mathcal{A}|_W \to \mathcal{B}|_W$
of Constructions, Lemma \ref{constructions-lemma-morphism-relative-proj}.
\end{lemma}

\begin{proof}
This lemma follows from \'etale localization and the case of schemes, see
discussion following
Definition \ref{definition-relative-proj}. Details omitted.
\end{proof}

\begin{lemma}
\label{lemma-morphism-relative-proj-transitive}
Let $S$ be a scheme. Let $X$ be an algebraic space over $S$.
Let $\mathcal{A}$, $\mathcal{B}$, and $\mathcal{C}$ be
quasi-coherent graded $\mathcal{O}_X$-algebras.
Set $P = \underline{\text{Proj}}_X(\mathcal{A})$,
$Q = \underline{\text{Proj}}_X(\mathcal{B})$ and
$R = \underline{\text{Proj}}_X(\mathcal{C})$.
Let $\varphi : \mathcal{A} \to \mathcal{B}$,
$\psi : \mathcal{B} \to \mathcal{C}$ be graded $\mathcal{O}_X$-algebra maps.
Then we have
$$
U(\psi \circ \varphi) = r_\varphi^{-1}(U(\psi))
\quad
\text{and}
\quad
r_{\psi \circ \varphi}
=
r_\varphi \circ r_\psi|_{U(\psi \circ \varphi)}.
$$
In addition we have
$$
\theta_\psi \circ r_\psi^*\theta_\varphi
=
\theta_{\psi \circ \varphi}
$$
with obvious notation.
\end{lemma}

\begin{proof}
Omitted.
\end{proof}

\begin{lemma}
\label{lemma-surjective-graded-rings-map-relative-proj}
With hypotheses and notation as in Lemma \ref{lemma-morphism-relative-proj}
above. Assume $\mathcal{A}_d \to \mathcal{B}_d$ is surjective for
$d \gg 0$. Then
\begin{enumerate}
\item $U(\psi) = Q$,
\item $r_\psi : Q \to R$ is a closed immersion, and
\item the maps $\theta : r_\psi^*\mathcal{O}_P(n) \to \mathcal{O}_Q(n)$
are surjective but not isomorphisms in general (even if
$\mathcal{A} \to \mathcal{B}$ is surjective).
\end{enumerate}
\end{lemma}

\begin{proof}
Follows from the case of schemes
(Constructions, Lemma
\ref{constructions-lemma-surjective-graded-rings-map-relative-proj})
by \'etale localization.
\end{proof}

\begin{lemma}
\label{lemma-eventual-iso-graded-rings-map-relative-proj}
With hypotheses and notation as in Lemma \ref{lemma-morphism-relative-proj}
above. Assume $\mathcal{A}_d \to \mathcal{B}_d$ is an isomorphism for all
$d \gg 0$. Then
\begin{enumerate}
\item $U(\psi) = Q$,
\item $r_\psi : Q \to P$ is an isomorphism, and
\item the maps $\theta : r_\psi^*\mathcal{O}_P(n) \to \mathcal{O}_Q(n)$
are isomorphisms.
\end{enumerate}
\end{lemma}

\begin{proof}
Follows from the case of schemes
(Constructions, Lemma
\ref{constructions-lemma-eventual-iso-graded-rings-map-relative-proj})
by \'etale localization.
\end{proof}

\begin{lemma}
\label{lemma-surjective-generated-degree-1-map-relative-proj}
With hypotheses and notation as in Lemma \ref{lemma-morphism-relative-proj}
above. Assume $\mathcal{A}_d \to \mathcal{B}_d$ is surjective for $d \gg 0$
and that $\mathcal{A}$ is generated by $\mathcal{A}_1$ over $\mathcal{A}_0$.
Then
\begin{enumerate}
\item $U(\psi) = Q$,
\item $r_\psi : Q \to P$ is a closed immersion, and
\item the maps $\theta : r_\psi^*\mathcal{O}_P(n) \to \mathcal{O}_Q(n)$
are isomorphisms.
\end{enumerate}
\end{lemma}

\begin{proof}
Follows from the case of schemes
(Constructions, Lemma
\ref{constructions-lemma-surjective-generated-degree-1-map-relative-proj})
by \'etale localization.
\end{proof}










\section{Invertible sheaves and morphisms into relative Proj}
\label{section-invertible-relative-proj}

\noindent
It seems that we may need the following lemma somewhere.
The situation is the following:
\begin{enumerate}
\item Let $S$ be a scheme and $Y$ an algebraic space over $S$.
\item Let $\mathcal{A}$ be a quasi-coherent graded $\mathcal{O}_Y$-algebra.
\item Denote $\pi : \underline{\text{Proj}}_Y(\mathcal{A}) \to Y$ the relative
Proj of $\mathcal{A}$ over $Y$.
\item Let $f : X \to Y$ be a morphism of algebraic spaces over $S$.
\item Let $\mathcal{L}$ be an invertible $\mathcal{O}_Y$-module.
\item Let
$\psi : f^*\mathcal{A} \to \bigoplus_{d \geq 0} \mathcal{L}^{\otimes d}$
be a homomorphism of graded $\mathcal{O}_X$-algebras.
\end{enumerate}
Given this data let $U(\psi) \subset X$ be the open subspace with
$$
|U(\psi)| = \bigcup\nolimits_{d \geq 1}
\{\text{locus where }f^*\mathcal{A}_d \to \mathcal{L}^{\otimes d}
\text{ is surjective}\}
$$
Formation of $U(\psi) \subset X$ commutes with
pullback by any morphism $X' \to X$.

\begin{lemma}
\label{lemma-invertible-map-into-relative-proj}
With assumptions and notation as above. The morphism
$\psi$ induces a canonical morphism of algebraic spaces over $Y$
$$
r_{\mathcal{L}, \psi} :
U(\psi) \longrightarrow \underline{\text{Proj}}_Y(\mathcal{A})
$$
together with a map of graded $\mathcal{O}_{U(\psi)}$-algebras
$$
\theta :
r_{\mathcal{L}, \psi}^*\left(
\bigoplus\nolimits_{d \geq 0}
\mathcal{O}_{\underline{\text{Proj}}_Y(\mathcal{A})}(d)
\right)
\longrightarrow
\bigoplus\nolimits_{d \geq 0} \mathcal{L}^{\otimes d}|_{U(\psi)}
$$
characterized by the following properties:
\begin{enumerate}
\item For $V \to Y$ \'etale and $d \geq 0$ the diagram
$$
\xymatrix{
\mathcal{A}_d(V) \ar[d]_{\psi} \ar[r]_{\psi} &
\Gamma(V \times_Y X, \mathcal{L}^{\otimes d}) \ar[d]^{restrict} \\
\Gamma(V \times_Y \underline{\text{Proj}}_Y(\mathcal{A}),
\mathcal{O}_{\underline{\text{Proj}}_Y(\mathcal{A})}(d)) \ar[r]^-\theta &
\Gamma(V \times_Y U(\psi), \mathcal{L}^{\otimes d})
}
$$
is commutative.
\item For any $d \geq 1$ and any morphism $W \to X$ where $W$ is a scheme
such that $\psi|_W : f^*\mathcal{A}_d|_W \to \mathcal{L}^{\otimes d}|_W$
is surjective we have (a) $W \to X$ factors through $U(\psi)$ and
(b) composition of $W \to U(\psi)$ with $r_{\mathcal{L}, \psi}$
agrees with the morphism $W \to \underline{\text{Proj}}_Y(\mathcal{A})$
which exists by the construction of $\underline{\text{Proj}}_Y(\mathcal{A})$,
see Definition \ref{definition-relative-proj}.
\item Consider a commutative diagram
$$
\xymatrix{
X' \ar[r]_{g'} \ar[d]_{f'} & X \ar[d]^f \\
Y' \ar[r]^g & Y
}
$$
where $X'$ and $Y'$ are schemes, set $\mathcal{A}' = g^*\mathcal{A}$
and $\mathcal{L}' = (g')^*\mathcal{L}$ and denote
$\psi' : (f')^*\mathcal{A} \to \bigoplus_{d \geq 0} (\mathcal{L}')^{\otimes d}$
the pullback of $\psi$. Let $U(\psi')$, $r_{\psi', \mathcal{L}'}$,
and $\theta'$ be the open, morphism, and homomorphism constructed
in Constructions, Lemma \ref{lemma-invertible-map-into-relative-proj}.
Then $U(\psi') = (g')^{-1}(U(\psi))$
and $r_{\psi', \mathcal{L}'}$ agrees with the base change
of $r_{\psi, \mathcal{L}}$ via the isomorphism
$\underline{\text{Proj}}_{Y'}(\mathcal{A}') =
Y' \times_Y \underline{\text{Proj}}_Y(\mathcal{A})$
of Lemma \ref{lemma-relative-proj-base-change}.
Moreover, $\theta'$ is the pullback of $\theta$.
\end{enumerate}
\end{lemma}

\begin{proof}
Omitted. Hints:
First we observe that for a quasi-compact scheme
$W$ over $X$ the following are equivalent
\begin{enumerate}
\item $W \to X$ factors through $U(\psi)$, and
\item there exists a $d$ such that
$\psi|_W : f^*\mathcal{A}_d|_W \to \mathcal{L}^{\otimes d}|_W$
is surjective.
\end{enumerate}
This gives a description of $U(\psi)$ as a subfunctor of $X$
on our base category $(\Sch/S)_{fppf}$. For such a $W$ and $d$
we consider the quadruple
$(d, W \to Y, \mathcal{L}|_W, \psi^{(d)}|_W)$.
By definition of $\underline{\text{Proj}}_Y(\mathcal{A})$
we obtain a morphism $W \to \underline{\text{Proj}}_Y(\mathcal{A})$.
By our notion of equivalence of quadruples one sees that
this morphism is independent of the choice of $d$.
This clearly defines a transformation of functors
$r_{\psi, \mathcal{L}} : U(\psi) \to \underline{\text{Proj}}_Y(\mathcal{A})$,
i.e., a morphism of algebraic spaces.
By construction this morphism satisfies (2).
Since the morphism constructed in
Constructions, Lemma \ref{constructions-lemma-invertible-map-into-relative-proj}
satisfies the same property, we see that (3) is true.

\medskip\noindent
To construct $\theta$ and check the compatibility (1) of the
lemma, work \'etale locally on $Y$ and $X$, arguing as
in the discussion following
Definition \ref{definition-relative-proj}.
\end{proof}








\section{Relatively ample sheaves}
\label{section-relatively-ample}

\noindent
This section is the analogue of
Morphisms, Section \ref{morphisms-section-relatively-ample}
for algebraic spaces.
Our definition of a relatively ample invertible sheaf is as
follows.

\begin{definition}
\label{definition-relatively-ample}
Let $S$ be a scheme.
Let $f : X \to Y$ be a morphism of algebraic spaces over $S$.
Let $\mathcal{L}$ be an invertible $\mathcal{O}_X$-module.
We say $\mathcal{L}$ is {\it relatively ample}, or {\it $f$-relatively ample},
or {\it ample on $X/Y$}, or {\it $f$-ample} if $f : X \to Y$
is representable and for every morphism $Z \to Y$
where $Z$ is a scheme, the pullback $\mathcal{L}_Z$ of $\mathcal{L}$
to $X_Z = Z \times_Y X$ is ample on $X_Z/Z$ as in
Morphisms, Definition \ref{morphisms-definition-relatively-ample}.
\end{definition}

\noindent
We will almost always reduce questions about relatively ample invertible
sheaves to the case of schemes. Thus in this section we have
mainly sanity checks.

\begin{lemma}
\label{lemma-relatively-ample-sanity-check}
Let $S$ be a scheme.
Let $f : X \to Y$ be a morphism of algebraic spaces over $S$.
Let $\mathcal{L}$ be an invertible $\mathcal{O}_X$-module.
Assume $Y$ is a scheme. The following are equivalent
\begin{enumerate}
\item $\mathcal{L}$ is ample on $X/Y$ in the sense of
Definition \ref{definition-relatively-ample}, and
\item $X$ is a scheme and $\mathcal{L}$ is ample on $X/Y$
in the sense of
Morphisms, Definition \ref{morphisms-definition-relatively-ample}.
\end{enumerate}
\end{lemma}

\begin{proof}
This follows from the definitions and
Morphisms, Lemma \ref{morphisms-lemma-ample-base-change}
(which says that being relatively ample for schemes
is preserved under base change).
\end{proof}

\begin{lemma}
\label{lemma-ample-base-change}
Let $S$ be a scheme.
Let $f : X \to Y$ be a morphism of algebraic spaces over $S$.
Let $\mathcal{L}$ be an invertible $\mathcal{O}_X$-module.
Let $Y' \to Y$ be a morphism of algebraic spaces over $S$.
Let $f' : X' \to Y'$ be the base change of $f$ and denote
$\mathcal{L}'$ the pullback of $\mathcal{L}$ to $X'$.
If $\mathcal{L}$ is $f$-ample, then $\mathcal{L}'$ is $f'$-ample.
\end{lemma}

\begin{proof}
This follows immediately from the definition!
(Hint: transitivity of base change.)
\end{proof}

\begin{lemma}
\label{lemma-relatively-ample-properties}
Let $S$ be a scheme.
Let $f : X \to Y$ be a morphism of algebraic spaces over $S$.
If there exists an $f$-ample invertible sheaf, then
$f$ is representable, quasi-compact, and separated.
\end{lemma}

\begin{proof}
This is clear from the definitions and
Morphisms, Lemma \ref{morphisms-lemma-relatively-ample-separated}.
(If in doubt, take a look at the principle of
Algebraic Spaces, Lemma
\ref{spaces-lemma-representable-transformations-property-implication}.)
\end{proof}

\begin{lemma}
\label{lemma-descend-relatively-ample}
Let $V \to U$ be a surjective \'etale morphism of affine schemes.
Let $X$ be an algebraic space over $U$.
Let $\mathcal{L}$ be an invertible $\mathcal{O}_X$-module.
Let $Y = V \times_U X$ and let $\mathcal{N}$
be the pullback of $\mathcal{L}$ to $Y$.
The following are equivalent
\begin{enumerate}
\item $\mathcal{L}$ is ample on $X/U$, and
\item $\mathcal{N}$ is ample on $Y/V$.
\end{enumerate}
\end{lemma}

\begin{proof}
The implication (1) $\Rightarrow$ (2) follows from
Lemma \ref{lemma-ample-base-change}.
Assume (2). This implies that $Y \to V$ is
quasi-compact and separated (Lemma \ref{lemma-relatively-ample-properties})
and $Y$ is a scheme. Then we conclude that $X \to U$ is
quasi-compact and separated
(Morphisms of Spaces, Lemmas
\ref{spaces-morphisms-lemma-quasi-compact-local} and
\ref{spaces-morphisms-lemma-separated-local}).
Set $\mathcal{A} = \bigoplus_{d \geq 0} f_*\mathcal{L}^{\otimes d}$.
Thus is a quasi-coherent sheaf of graded $\mathcal{O}_U$-algebras
(Morphisms of Spaces, Lemma \ref{spaces-morphisms-lemma-pushforward}).
By adjunction we have a map
$\psi : f^*\mathcal{A} \to \bigoplus_{d \geq 0} \mathcal{L}^{\otimes d}$.
Applying Lemma \ref{lemma-invertible-map-into-relative-proj}
we obtain an open subspace $U(\psi) \subset X$ and a morphism
$$
r_{\mathcal{L}, \psi} : U(\psi) \to \underline{\text{Proj}}_U(\mathcal{A})
$$
Since $h : V \to U$ is \'etale we have
$\mathcal{A}|_V = (Y \to V)_*(\bigoplus_{d \geq 0} \mathcal{N}^{\otimes d})$,
see Properties of Spaces, Lemma
\ref{spaces-properties-lemma-pushforward-etale-base-change-modules}.
It follows that the pullback $\psi'$ of $\psi$ to
$Y$ is the adjunction map for the situation $(Y \to V, \mathcal{N})$ as in
Morphisms, Lemma \ref{morphisms-lemma-characterize-relatively-ample} part (5).
Since $\mathcal{N}$ is ample on $Y/V$ we conclude from the lemma just
cited that $U(\psi') = Y$ and that $r_{\mathcal{N}, \psi'}$
is an open immersion.
Since Lemma \ref{lemma-invertible-map-into-relative-proj}
tells us that the formation of $r_{\mathcal{L}, \psi}$
commutes with base change, we conclude that
$U(\psi) = X$ and that we have a commutative diagram
$$
\xymatrix{
Y \ar[r]_-{r'} \ar[d] &
\underline{\text{Proj}}_V(\mathcal{A}|_V) \ar[d] \ar[r] &
V \ar[d] \\
X \ar[r]^-r &
\underline{\text{Proj}}_U(\mathcal{A}) \ar[r] &
U
}
$$
whose squares are fibre products. We conclude that $r$ is an
open immersion by
Morphisms of Spaces, Lemma \ref{spaces-morphisms-lemma-closed-immersion-local}.
Thus $X$ is a scheme. Then we can apply
Morphisms, Lemma \ref{morphisms-lemma-characterize-relatively-ample} part (5)
to conclude that $\mathcal{L}$ is ample on $X/U$.
\end{proof}

\begin{lemma}
\label{lemma-relatively-ample-local}
Let $S$ be a scheme. Let $f : X \to Y$ be a morphism of algebraic spaces
over $S$. Let $\mathcal{L}$ be an invertible $\mathcal{O}_X$-module.
The following are equivalent
\begin{enumerate}
\item $\mathcal{L}$ is ample on $X/Y$,
\item for every scheme $Z$ and every morphism $Z \to Y$
the algebraic space $X_Z = Z \times_Y X$ is a scheme
and the pullback $\mathcal{L}_Z$ is ample on $X_Z/Z$,
\item for every affine scheme $Z$ and every morphism $Z \to Y$
the algebraic space $X_Z = Z \times_Y X$ is a scheme
and the pullback $\mathcal{L}_Z$ is ample on $X_Z/Z$,
\item there exists a scheme $V$ and a surjective \'etale morphism
$V \to Y$ such that the algebraic space $X_V = V \times_Y X$ is a scheme
and the pullback $\mathcal{L}_V$ is ample on $X_V/V$.
\end{enumerate}
\end{lemma}

\begin{proof}
Parts (1) and (2) are equivalent by definition.
The implication (2) $\Rightarrow$ (3) is immediate.
If (3) holds and $Z \to Y$ is as in (2), then we see
that $X_Z \to Z$ is affine locally on $Z$ representable.
Hence $X_Z$ is a scheme for example by
Properties of Spaces, Lemma \ref{spaces-properties-lemma-subscheme}.
Then it follows that $\mathcal{L}_Z$ is ample on $X_Z/Z$ because
it holds locally on $Z$ and we can use
Morphisms, Lemma \ref{morphisms-lemma-characterize-relatively-ample}.
Thus (1), (2), and (3) are equivalent. Clearly these conditions
imply (4).

\medskip\noindent
Assume (4). Let $Z \to Y$ be a morphism with $Z$ affine.
Then $U = V \times_Y Z \to Z$ is a surjective \'etale morphism
such that the pullback of $\mathcal{L}_Z$ by $X_U \to X_Z$
is relatively ample on $X_U/U$.
Of course we may replace $U$ by an affine open.
It follows that $\mathcal{L}_Z$ is ample on $X_Z/Z$ by
Lemma \ref{lemma-descend-relatively-ample}.
Thus (4) $\Rightarrow$ (3) and the proof is complete.
\end{proof}





\section{Relative ampleness and cohomology}
\label{section-ample-and-proper}

\noindent
This section contains some results related to the results
in Cohomology of Schemes, Sections
\ref{coherent-section-applications-formal-functions} and
\ref{coherent-section-ample-cohomology}.

\medskip\noindent
The following lemma is just an example of what we can do.

\begin{lemma}
\label{lemma-vanshing-gives-ample}
Let $R$ be a Noetherian ring. Let $X$ be an algebraic space over $R$
such that the structure morphism $f : X \to \Spec(R)$ is proper.
Let $\mathcal{L}$ be an invertible $\mathcal{O}_X$-module.
The following are equivalent
\begin{enumerate}
\item $\mathcal{L}$ is ample on $X/R$
(Definition \ref{definition-relatively-ample}),
\item for every coherent $\mathcal{O}_X$-module $\mathcal{F}$
there exists an $n_0 \geq 0$ such that
$H^p(X, \mathcal{F} \otimes \mathcal{L}^{\otimes n}) = 0$
for all $n \geq n_0$ and $p > 0$.
\end{enumerate}
\end{lemma}

\begin{proof}
The implication (1) $\Rightarrow$ (2) follows from
Cohomology of Schemes, Lemma \ref{coherent-lemma-coherent-proper-ample}
because assumption (1) implies that $X$ is a scheme.
The implication (2) $\Rightarrow$ (1) is
Cohomology of Spaces, Lemma
\ref{spaces-cohomology-lemma-Noetherian-h1-zero-invertible}.
\end{proof}

\begin{lemma}
\label{lemma-ample-on-fibre}
Let $Y$ be a Noetherian scheme. Let $X$ be an algebraic space over $Y$
such that the structure morphism $f : X \to Y$ is proper.
Let $\mathcal{L}$ be an invertible $\mathcal{O}_X$-module.
Let $\mathcal{F}$ be a coherent $\mathcal{O}_X$-module.
Let $y \in Y$ be a point such that $X_y$ is a scheme and
$\mathcal{L}_y$ is ample on $X_y$.
Then there exists a $d_0$ such that for all $d \geq d_0$ we have
$$
R^pf_*(\mathcal{F} \otimes_{\mathcal{O}_X} \mathcal{L}^{\otimes d})_y = 0
\text{ for }p > 0
$$
and the map
$$
f_*(\mathcal{F} \otimes_{\mathcal{O}_X} \mathcal{L}^{\otimes d})_y
\longrightarrow
H^0(X_y, \mathcal{F}_y \otimes_{\mathcal{O}_{X_y}} \mathcal{L}_y^{\otimes d})
$$
is surjective.
\end{lemma}

\begin{proof}
Note that $\mathcal{O}_{Y, y}$ is a Noetherian local ring.
Consider the canonical morphism
$c : \Spec(\mathcal{O}_{Y, y}) \to Y$, see
Schemes, Equation (\ref{schemes-equation-canonical-morphism}).
This is a flat morphism as it identifies local rings.
Denote momentarily $f' : X' \to \Spec(\mathcal{O}_{Y, y})$
the base change of $f$ to this local ring. We see that
$c^*R^pf_*\mathcal{F} = R^pf'_*\mathcal{F}'$ by
Cohomology of Spaces, Lemma
\ref{spaces-cohomology-lemma-flat-base-change-cohomology}.
Moreover, the fibres $X_y$ and $X'_y$ are identified.
Hence we may assume that $Y = \Spec(A)$ is the spectrum of
a Noetherian local ring $(A, \mathfrak m, \kappa)$ and $y \in Y$
corresponds to $\mathfrak m$. In this case
$R^pf_*(\mathcal{F} \otimes_{\mathcal{O}_X} \mathcal{L}^{\otimes d})_y =
H^p(X, \mathcal{F} \otimes_{\mathcal{O}_X} \mathcal{L}^{\otimes d})$
for all $p \geq 0$. Denote $f_y : X_y \to \Spec(\kappa)$ the projection.

\medskip\noindent
Let $B = \text{Gr}_\mathfrak m(A) =
\bigoplus_{n \geq 0} \mathfrak m^n/\mathfrak m^{n + 1}$.
Consider the sheaf $\mathcal{B} = f_y^*\widetilde{B}$
of quasi-coherent graded $\mathcal{O}_{X_y}$-algebras.
We will use notation as in Cohomology of Spaces, Section
\ref{spaces-cohomology-section-theorem-formal-functions}
with $I$ replaced by $\mathfrak m$.
Since $X_y$ is the closed subspace of $X$ cut out by
$\mathfrak m\mathcal{O}_X$ we may think of
$\mathfrak m^n\mathcal{F}/\mathfrak m^{n + 1}\mathcal{F}$
as a coherent $\mathcal{O}_{X_y}$-module, see
Cohomology of Spaces, Lemma \ref{spaces-cohomology-lemma-i-star-equivalence}.
Then
$\bigoplus_{n \geq 0} \mathfrak m^n\mathcal{F}/\mathfrak m^{n + 1}\mathcal{F}$
is a quasi-coherent graded $\mathcal{B}$-module of finite type
because it is generated in degree zero over $\mathcal{B}$
abd because the degree zero part is
$\mathcal{F}_y = \mathcal{F}/\mathfrak m \mathcal{F}$
which is a coherent $\mathcal{O}_{X_y}$-module.
Hence by Cohomology of Schemes, Lemma
\ref{coherent-lemma-graded-finiteness} part (2)
there exists a $d_0$ such that
$$
H^p(X_y, \mathfrak m^n \mathcal{F}/ \mathfrak m^{n + 1}\mathcal{F}
\otimes_{\mathcal{O}_{X_y}} \mathcal{L}_y^{\otimes d}) = 0
$$
for all $p > 0$, $d \geq d_0$, and $n \geq 0$. By
Cohomology of Spaces, Lemma
\ref{spaces-cohomology-lemma-relative-affine-cohomology}
this is the same as the statement that
$
H^p(X, \mathfrak m^n \mathcal{F}/ \mathfrak m^{n + 1}\mathcal{F}
\otimes_{\mathcal{O}_X} \mathcal{L}^{\otimes d}) = 0
$
for all $p > 0$, $d \geq d_0$, and $n \geq 0$.

\medskip\noindent
Consider the short exact sequences
$$
0 \to \mathfrak m^n\mathcal{F}/\mathfrak m^{n + 1} \mathcal{F}
\to \mathcal{F}/\mathfrak m^{n + 1} \mathcal{F}
\to \mathcal{F}/\mathfrak m^n \mathcal{F} \to 0
$$
of coherent $\mathcal{O}_X$-modules. Tensoring with $\mathcal{L}^{\otimes d}$
is an exact functor and we obtain short exact sequences
$$
0 \to
\mathfrak m^n\mathcal{F}/\mathfrak m^{n + 1} \mathcal{F}
\otimes_{\mathcal{O}_X} \mathcal{L}^{\otimes d}
\to \mathcal{F}/\mathfrak m^{n + 1} \mathcal{F}
\otimes_{\mathcal{O}_X} \mathcal{L}^{\otimes d}
\to \mathcal{F}/\mathfrak m^n \mathcal{F}
\otimes_{\mathcal{O}_X} \mathcal{L}^{\otimes d} \to 0
$$
Using the long exact cohomology sequence and the vanishing above
we conclude (using induction) that
\begin{enumerate}
\item $H^p(X, \mathcal{F}/\mathfrak m^n \mathcal{F}
\otimes_{\mathcal{O}_X} \mathcal{L}^{\otimes d}) = 0$
for all $p > 0$, $d \geq d_0$, and $n \geq 0$, and
\item $H^0(X, \mathcal{F}/\mathfrak m^n \mathcal{F}
\otimes_{\mathcal{O}_X} \mathcal{L}^{\otimes d}) \to
H^0(X_y, \mathcal{F}_y \otimes_{\mathcal{O}_{X_y}} \mathcal{L}_y^{\otimes d})$
is surjective for all $d \geq d_0$ and $n \geq 1$.
\end{enumerate}
By the theorem on formal functions
(Cohomology of Spaces, Theorem
\ref{spaces-cohomology-theorem-formal-functions})
we find that the $\mathfrak m$-adic completion of
$H^p(X, \mathcal{F} \otimes_{\mathcal{O}_X} \mathcal{L}^{\otimes d})$
is zero for all $d \geq d_0$ and $p > 0$.
Since $H^p(X, \mathcal{F} \otimes_{\mathcal{O}_X} \mathcal{L}^{\otimes d})$
is a finite $A$-module by
Cohomology of Spaces, Lemma
\ref{spaces-cohomology-lemma-proper-over-affine-cohomology-finite}
it follows from Nakayama's lemma (Algebra, Lemma \ref{algebra-lemma-NAK})
that $H^p(X, \mathcal{F} \otimes_{\mathcal{O}_X} \mathcal{L}^{\otimes d})$
is zero for all $d \geq d_0$ and $p > 0$.
For $p = 0$ we deduce from
Cohomology of Spaces, Lemma
\ref{spaces-cohomology-lemma-ML-cohomology-powers-ideal} part (3)
that $H^0(X, \mathcal{F} \otimes_{\mathcal{O}_X} \mathcal{L}^{\otimes d}) \to
H^0(X_y, \mathcal{F}_y \otimes_{\mathcal{O}_{X_y}} \mathcal{L}_y^{\otimes d})$
is surjective, which gives the final statement of the lemma.
\end{proof}

\begin{lemma}
\label{lemma-ample-in-neighbourhood}
(For a more general version see
Descent on Spaces, Lemma \ref{spaces-descent-lemma-ample-in-neighbourhood}).
Let $Y$ be a Noetherian scheme. Let $X$ be an algebraic space over $Y$
such that the structure morphism $f : X \to Y$ is proper.
Let $\mathcal{L}$ be an invertible $\mathcal{O}_X$-module.
Let $y \in Y$ be a point such that $X_y$ is a scheme and
$\mathcal{L}_y$ is ample on $X_y$.
Then there is an open neighbourhood $V \subset Y$
of $y$ such that $\mathcal{L}|_{f^{-1}(V)}$ is ample on $f^{-1}(V)/V$
(as in Definition \ref{definition-relatively-ample}).
\end{lemma}

\begin{proof}
Pick $d_0$ as in Lemma \ref{lemma-ample-on-fibre} for
$\mathcal{F} = \mathcal{O}_X$. Pick $d \geq d_0$
so that we can find $r \geq 0$ and sections
$s_{y, 0}, \ldots, s_{y, r} \in H^0(X_y, \mathcal{L}_y^{\otimes d})$
which define a closed immersion
$$
\varphi_y =
\varphi_{\mathcal{L}_y^{\otimes d}, (s_{y, 0}, \ldots, s_{y, r})} :
X_y \to \mathbf{P}^r_{\kappa(y)}.
$$
This is possible by Morphisms, Lemma
\ref{morphisms-lemma-finite-type-over-affine-ample-very-ample}
but we also use
Morphisms, Lemma \ref{morphisms-lemma-image-proper-scheme-closed}
to see that $\varphi_y$ is a closed immersion and
Constructions, Section \ref{constructions-section-projective-space}
for the description of morphisms into projective
space in terms of invertible sheaves and sections.
By our choice of $d_0$, after replacing $Y$ by an open neighbourhood
of $y$, we can choose
$s_0, \ldots, s_r \in H^0(X, \mathcal{L}^{\otimes d})$
mapping to $s_{y, 0}, \ldots, s_{y, r}$.
Let $X_{s_i} \subset X$ be the open subspace where $s_i$
is a generator of $\mathcal{L}^{\otimes d}$. Since
the $s_{y, i}$ generate $\mathcal{L}_y^{\otimes d}$ we see that
$|X_y| \subset U = \bigcup |X_{s_i}|$. Since $X \to Y$ is closed,
we see that there is an open neighbourhood $y \in V \subset Y$
such that $|f|^{-1}(V) \subset U$. After replacing $Y$ by $V$ we may
assume that the $s_i$ generate $\mathcal{L}^{\otimes d}$. Thus we
obtain a morphism
$$
\varphi = \varphi_{\mathcal{L}^{\otimes d}, (s_0, \ldots, s_r)} :
X \longrightarrow \mathbf{P}^r_Y
$$
with $\mathcal{L}^{\otimes d} \cong \varphi^*\mathcal{O}_{\mathbf{P}^r_Y}(1)$
whose base change to $y$ gives $\varphi_y$ (strictly speaking we need
to write out a proof that the construction of morphisms into projective
space given in
Constructions, Section \ref{constructions-section-projective-space}
also works to describe morphisms of algebraic spaces into projective
space; we omit the details).

\medskip\noindent
We will finish the proof by a sleight of hand; the ``correct'' proof
proceeds by directly showing that $\varphi$ is a closed
immersion after base changing to an open neighbourhood of $y$.
Namely, by
Cohomology of Spaces, Lemma
\ref{spaces-cohomology-lemma-proper-finite-fibre-finite-in-neighbourhood}
we see that $\varphi$ is a finite over an open neighbourhood
of the fibre $\mathbf{P}^r_{\kappa(y)}$ of $\mathbf{P}^r_Y \to Y$
above $y$. Using that $\mathbf{P}^r_Y \to Y$ is closed, after
shrinking $Y$ we may assume that $\varphi$ is finite.
In particular $X$ is a scheme.
Then $\mathcal{L}^{\otimes d} \cong \varphi^*\mathcal{O}_{\mathbf{P}^r_Y}(1)$
is ample by the very general
Morphisms, Lemma \ref{morphisms-lemma-pullback-ample-tensor-relatively-ample}.
\end{proof}








\section{Closed subspaces of relative proj}
\label{section-closed-in-relative-proj}

\noindent
Some auxiliary lemmas about closed subspaces of relative proj.
This section is the analogue of
Divisors, Section \ref{divisors-section-closed-in-relative-proj}.

\begin{lemma}
\label{lemma-closed-subscheme-proj}
Let $S$ be a scheme. Let $X$ be an algebraic space over $S$.
Let $\mathcal{A}$ be a quasi-coherent graded $\mathcal{O}_X$-algebra. Let
$\pi : P = \underline{\text{Proj}}_X(\mathcal{A}) \to X$ be the relative
Proj of $\mathcal{A}$. Let $i : Z \to P$ be a closed subspace. Denote
$\mathcal{I} \subset \mathcal{A}$ the kernel of the canonical map
$$
\mathcal{A}
\longrightarrow
\bigoplus\nolimits_{d \geq 0} \pi_*\left((i_*\mathcal{O}_Z)(d)\right)
$$
If $\pi$ is quasi-compact, then there is an isomorphism
$Z = \underline{\text{Proj}}_X(\mathcal{A}/\mathcal{I})$.
\end{lemma}

\begin{proof}
The morphism $\pi$ is separated by
Lemma \ref{lemma-relative-proj-separated}.
As $\pi$ is quasi-compact, $\pi_*$ transforms quasi-coherent modules
into quasi-coherent modules, see
Morphisms of Spaces, Lemma \ref{spaces-morphisms-lemma-pushforward}.
Hence $\mathcal{I}$ is a quasi-coherent $\mathcal{O}_X$-module.
In particular, $\mathcal{B} = \mathcal{A}/\mathcal{I}$ is a
quasi-coherent graded $\mathcal{O}_X$-algebra. The functoriality
morphism $Z' = \underline{\text{Proj}}_X(\mathcal{B}) \to
\underline{\text{Proj}}_X(\mathcal{A})$ is everywhere defined and
a closed immersion, see Lemma
\ref{lemma-surjective-graded-rings-map-relative-proj}.
Hence it suffices to prove $Z = Z'$ as closed subspaces of $P$.

\medskip\noindent
Having said this, the question is \'etale local on the base and we
reduce to the case of schemes
(Divisors, Lemma \ref{divisors-lemma-closed-subscheme-proj})
by \'etale localization.
\end{proof}

\noindent
In case the closed subspace is locally cut out by finitely many
equations we can define it by a finite type ideal sheaf of
$\mathcal{A}$.

\begin{lemma}
\label{lemma-closed-subscheme-proj-finite}
Let $S$ be a scheme. Let $X$ be a quasi-compact and quasi-separated
algebraic space over $S$.
Let $\mathcal{A}$ be a quasi-coherent graded $\mathcal{O}_X$-algebra. Let
$\pi : P = \underline{\text{Proj}}_X(\mathcal{A}) \to X$ be the relative
Proj of $\mathcal{A}$. Let $i : Z \to P$ be a closed subscheme.
If $\pi$ is quasi-compact and $i$ of finite presentation, then there exists
a $d > 0$ and a quasi-coherent finite type $\mathcal{O}_X$-submodule
$\mathcal{F} \subset \mathcal{A}_d$ such that
$Z = \underline{\text{Proj}}_X(\mathcal{A}/\mathcal{F}\mathcal{A})$.
\end{lemma}

\begin{proof}
The reader can redo the arguments used in the case of schemes. However, we
will show the lemma follows from the case of schemes by a trick.
Let $\mathcal{I} \subset \mathcal{A}$ be the quasi-coherent graded
ideal cutting out $Z$ of Lemma \ref{lemma-closed-subscheme-proj}.
Choose an affine scheme $U$ and a surjective \'etale morphism
$U \to X$, see Properties of Spaces, Lemma
\ref{spaces-properties-lemma-quasi-compact-affine-cover}.
By the case of schemes
(Divisors, Lemma \ref{divisors-lemma-closed-subscheme-proj-finite})
there exists a $d > 0$ and a quasi-coherent finite type
$\mathcal{O}_U$-submodule
$\mathcal{F}' \subset \mathcal{I}_d|_U \subset \mathcal{A}_d|_U$
such that $Z \times_X U$ is equal to
$\underline{\text{Proj}}_U(\mathcal{A}|_U/\mathcal{F}'\mathcal{A}|_U)$.
By Limits of Spaces, Lemma
\ref{spaces-limits-lemma-directed-colimit-finite-type}
we can find a finite type quasi-coherent submodule
$\mathcal{F} \subset \mathcal{I}_d$ such that
$\mathcal{F}' \subset \mathcal{F}|_U$. Let
$Z' = \underline{\text{Proj}}_X(\mathcal{A}/\mathcal{F}\mathcal{A})$.
Then $Z' \to P$ is a closed immersion
(Lemma \ref{lemma-surjective-generated-degree-1-map-relative-proj})
and $Z \subset Z'$ as $\mathcal{F}\mathcal{A} \subset \mathcal{I}$.
On the other hand, $Z' \times_X U \subset Z \times_X U$ by our
choice of $\mathcal{F}$. Thus $Z = Z'$ as desired.
\end{proof}

\begin{lemma}
\label{lemma-closed-subscheme-proj-finite-type}
Let $S$ be a scheme. Let $X$ be a quasi-compact and quasi-separated
algebraic space over $S$.
Let $\mathcal{A}$ be a quasi-coherent graded $\mathcal{O}_X$-algebra.
Let $\pi : P = \underline{\text{Proj}}_X(\mathcal{A}) \to X$ be the relative
Proj of $\mathcal{A}$. Let $i : Z \to X$ be a closed subspace.
Let $U \subset X$ be an open. Assume that
\begin{enumerate}
\item $\pi$ is quasi-compact,
\item $i$ of finite presentation,
\item $|U| \cap |\pi|(|i|(|Z|)) = \emptyset$,
\item $U$ is quasi-compact,
\item $\mathcal{A}_n$ is a finite type $\mathcal{O}_X$-module for all $n$.
\end{enumerate}
Then there exists a $d > 0$ and a quasi-coherent finite type
$\mathcal{O}_X$-submodule $\mathcal{F} \subset \mathcal{A}_d$ with (a)
$Z = \underline{\text{Proj}}_X(\mathcal{A}/\mathcal{F}\mathcal{A})$
and (b) the support of $\mathcal{A}_d/\mathcal{F}$ is disjoint from $U$.
\end{lemma}

\begin{proof}
We use the same trick as in the proof of
Lemma \ref{lemma-closed-subscheme-proj-finite}
to reduce to the case of schemes.
Let $\mathcal{I} \subset \mathcal{A}$ be the quasi-coherent graded
ideal cutting out $Z$ of Lemma \ref{lemma-closed-subscheme-proj}.
Choose an affine scheme $W$ and a surjective \'etale morphism
$W \to X$, see Properties of Spaces, Lemma
\ref{spaces-properties-lemma-quasi-compact-affine-cover}.
By the case of schemes
(Divisors, Lemma \ref{divisors-lemma-closed-subscheme-proj-finite-type})
there exists a $d > 0$ and a quasi-coherent finite type
$\mathcal{O}_W$-submodule
$\mathcal{F}' \subset \mathcal{I}_d|_W \subset \mathcal{A}_d|_W$
such that (a) $Z \times_X W$ is equal to
$\underline{\text{Proj}}_W(\mathcal{A}|_W/\mathcal{F}'\mathcal{A}|_W)$
and (b) the support of $\mathcal{A}_d|_W/\mathcal{F}'$ is disjoint from
$U \times_X W$. By Limits of Spaces, Lemma
\ref{spaces-limits-lemma-directed-colimit-finite-type}
we can find a finite type quasi-coherent submodule
$\mathcal{F} \subset \mathcal{I}_d$ such that
$\mathcal{F}' \subset \mathcal{F}|_W$. Let
$Z' = \underline{\text{Proj}}_X(\mathcal{A}/\mathcal{F}\mathcal{A})$.
Then $Z' \to P$ is a closed immersion
(Lemma \ref{lemma-surjective-generated-degree-1-map-relative-proj})
and $Z \subset Z'$ as $\mathcal{F}\mathcal{A} \subset \mathcal{I}$.
On the other hand, $Z' \times_X W \subset Z \times_X W$ by our
choice of $\mathcal{F}$. Thus $Z = Z'$.
Finally, we see that $\mathcal{A}_d/\mathcal{F}$ is supported on
$X \setminus U$ as $\mathcal{A}_d|_W/\mathcal{F}|_W$ is a quotient
of $\mathcal{A}_d|_W/\mathcal{F}'$ which is supported on
$W \setminus U \times_X W$. Thus the lemma follows.
\end{proof}

\begin{lemma}
\label{lemma-conormal-sheaf-section-projective-bundle}
Let $S$ be a scheme and let $X$ be an algebraic space over $S$.
Let $\mathcal{E}$ be a quasi-coherent $\mathcal{O}_X$-module.
There is a bijection
$$
\left\{
\begin{matrix}
\text{sections }\sigma\text{ of the } \\
\text{morphism } \mathbf{P}(\mathcal{E}) \to X
\end{matrix}
\right\}
\leftrightarrow
\left\{
\begin{matrix}
\text{surjections }\mathcal{E} \to \mathcal{L}\text{ where} \\
\mathcal{L}\text{ is an invertible }\mathcal{O}_X\text{-module}
\end{matrix}
\right\}
$$
In this case $\sigma$ is a closed immersion and there is a canonical
isomorphism
$$
\Ker(\mathcal{E} \to \mathcal{L})
\otimes_{\mathcal{O}_X} \mathcal{L}^{\otimes -1}
\longrightarrow
\mathcal{C}_{\sigma(X)/\mathbf{P}(\mathcal{E})}
$$
Both the bijection and isomorphism are compatible with base change.
\end{lemma}

\begin{proof}
Because the constructions are compatible with base change, it suffices to
check the statement \'etale locally on $X$. Thus we may assume $X$ is
a scheme and the result is
Divisors, Lemma \ref{divisors-lemma-conormal-sheaf-section-projective-bundle}.
\end{proof}






\section{Blowing up}
\label{section-blowing-up}

\noindent
Blowing up is an important tool in algebraic geometry.

\begin{definition}
\label{definition-blow-up}
Let $S$ be a scheme. Let $X$ be an algebraic space over $S$.
Let $\mathcal{I} \subset \mathcal{O}_X$ be a quasi-coherent sheaf
of ideals, and let $Z \subset X$ be the closed subspace corresponding
to $\mathcal{I}$
(Morphisms of Spaces, Lemma
\ref{spaces-morphisms-lemma-closed-immersion-ideals}).
The {\it blowing up of $X$ along $Z$}, or the
{\it blowing up of $X$ in the ideal sheaf $\mathcal{I}$} is
the morphism
$$
b :
\underline{\text{Proj}}_X
\left(\bigoplus\nolimits_{n \geq 0} \mathcal{I}^n\right)
\longrightarrow
X
$$
The {\it exceptional divisor} of the blowup is the inverse image
$b^{-1}(Z)$. Sometimes $Z$ is called the {\it center} of the blowup.
\end{definition}

\noindent
We will see later that the exceptional divisor is an effective Cartier
divisor. Moreover, the blowing up is characterized as the ``smallest''
algebraic space over $X$ such that the inverse image of $Z$ is an
effective Cartier divisor.

\medskip\noindent
If $b : X' \to X$ is the blowup of $X$ in $Z$, then we often denote
$\mathcal{O}_{X'}(n)$ the twists of the structure sheaf. Note that these
are invertible $\mathcal{O}_{X'}$-modules and that
$\mathcal{O}_{X'}(n) = \mathcal{O}_{X'}(1)^{\otimes n}$
because $X'$ is the relative Proj of a quasi-coherent graded
$\mathcal{O}_X$-algebra which is generated in degree $1$, see
Lemma \ref{lemma-relative-proj-generated-in-degree-1}.

\begin{lemma}
\label{lemma-blowing-up-affine}
Let $S$ be a scheme. Let $X$ be an algebraic space over $S$.
Let $\mathcal{I} \subset \mathcal{O}_X$ be a
quasi-coherent sheaf of ideals. Let $U = \Spec(A)$ be an affine scheme
\'etale over $X$ and let $I \subset A$ be the ideal corresponding to
$\mathcal{I}|_U$. If $X' \to X$ is the blowup of $X$ in $\mathcal{I}$,
then there is a canonical isomorphism
$$
U \times_X X' = \text{Proj}(\bigoplus\nolimits_{d \geq 0} I^d)
$$
of schemes over $U$, where the right hand side is
the homogeneous spectrum of the Rees algebra of $I$ in $A$.
Moreover, $U \times_X X'$ has an affine open covering by
spectra of the affine blowup algebras $A[\frac{I}{a}]$.
\end{lemma}

\begin{proof}
Note that the restriction $\mathcal{I}|_U$ is equal to the pullback
of $\mathcal{I}$ via the morphism $U \to X$, see
Properties of Spaces, Section \ref{spaces-properties-section-modules}.
Thus the lemma follows on combining Lemma \ref{lemma-relative-proj} with
Divisors, Lemma \ref{divisors-lemma-blowing-up-affine}.
\end{proof}

\begin{lemma}
\label{lemma-flat-base-change-blowing-up}
Let $S$ be a scheme.
Let $X_1 \to X_2$ be a flat morphism of algebraic spaces over $S$.
Let $Z_2 \subset X_2$ be a closed subspace.
Let $Z_1$ be the inverse image of $Z_2$ in $X_1$.
Let $X'_i$ be the blowup of $Z_i$ in $X_i$. Then there exists a cartesian
diagram
$$
\xymatrix{
X_1' \ar[r] \ar[d] & X_2' \ar[d] \\
X_1 \ar[r] & X_2
}
$$
of algebraic spaces over $S$.
\end{lemma}

\begin{proof}
Let $\mathcal{I}_2$ be the ideal sheaf of $Z_2$ in $X_2$.
Denote $g : X_1 \to X_2$ the given morphism. Then the ideal sheaf
$\mathcal{I}_1$ of $Z_1$ is the image of
$g^*\mathcal{I}_2 \to \mathcal{O}_{X_1}$
(see Morphisms of Spaces, Definition
\ref{spaces-morphisms-definition-inverse-image-closed-subspace}
and discussion following the definition).
By Lemma \ref{lemma-relative-proj-base-change}
we see that $X_1 \times_{X_2} X_2'$ is the relative Proj of
$\bigoplus_{n \geq 0} g^*\mathcal{I}_2^n$. Because $g$ is flat the map
$g^*\mathcal{I}_2^n \to \mathcal{O}_{X_1}$ is injective with image
$\mathcal{I}_1^n$. Thus we see that $X_1 \times_{X_2} X_2' = X_1'$.
\end{proof}

\begin{lemma}
\label{lemma-blowing-up-gives-effective-Cartier-divisor}
Let $S$ be a scheme. Let $X$ be an algebraic space over $S$.
Let $Z \subset X$ be a closed subspace.
The blowing up $b : X' \to X$ of $Z$ in $X$
has the following properties:
\begin{enumerate}
\item $b|_{b^{-1}(X \setminus Z)} : b^{-1}(X \setminus Z) \to X \setminus Z$
is an isomorphism,
\item the exceptional divisor $E = b^{-1}(Z)$ is an effective Cartier divisor
on $X'$,
\item there is a canonical isomorphism
$\mathcal{O}_{X'}(-1) = \mathcal{O}_{X'}(E)$
\end{enumerate}
\end{lemma}

\begin{proof}
Let $U$ be a scheme and let $U \to X$ be a surjective \'etale morphism.
As blowing up commutes with flat base change
(Lemma \ref{lemma-flat-base-change-blowing-up})
we can prove each of these statements after base change to $U$.
This reduces us to the case of schemes.
In this case the result is
Divisors, Lemma
\ref{divisors-lemma-blowing-up-gives-effective-Cartier-divisor}.
\end{proof}

\begin{lemma}[Universal property blowing up]
\label{lemma-universal-property-blowing-up}
\begin{slogan}
Blow up a closed subset to make it Cartier.
\end{slogan}
Let $S$ be a scheme.
Let $X$ be an algebraic space over $S$.
Let $Z \subset X$ be a closed subspace.
Let $\mathcal{C}$ be the full subcategory of $(\textit{Spaces}/X)$ consisting
of $Y \to X$ such that the inverse image of $Z$ is an effective
Cartier divisor on $Y$. Then the blowing up $b : X' \to X$ of $Z$ in $X$
is a final object of $\mathcal{C}$.
\end{lemma}

\begin{proof}
We see that $b : X' \to X$ is an object of $\mathcal{C}$ according to
Lemma \ref{lemma-blowing-up-gives-effective-Cartier-divisor}.
Let $f : Y \to X$ be an object of $\mathcal{C}$. We have to show there exists
a unique morphism $Y \to X'$ over $X$. Let $D = f^{-1}(Z)$.
Let $\mathcal{I} \subset \mathcal{O}_X$ be the ideal sheaf of $Z$
and let $\mathcal{I}_D$ be the ideal sheaf of $D$. Then
$f^*\mathcal{I} \to \mathcal{I}_D$ is a surjection
to an invertible $\mathcal{O}_Y$-module. This extends to a map
$\psi : \bigoplus f^*\mathcal{I}^d \to \bigoplus \mathcal{I}_D^d$
of graded $\mathcal{O}_Y$-algebras. (We observe that
$\mathcal{I}_D^d = \mathcal{I}_D^{\otimes d}$ as $D$ is an
effective Cartier divisor.) By
Lemma \ref{lemma-relative-proj-generated-in-degree-1}.
the triple $(f : Y \to X, \mathcal{I}_D, \psi)$ defines a
morphism $Y \to X'$ over $X$. The restriction
$$
Y \setminus D \longrightarrow X' \setminus b^{-1}(Z) = X \setminus Z
$$
is unique. The open $Y \setminus D$ is scheme theoretically dense in $Y$
according to Lemma \ref{lemma-complement-effective-Cartier-divisor}. 
Thus the morphism $Y \to X'$ is unique by
Morphisms of Spaces, Lemma \ref{spaces-morphisms-lemma-equality-of-morphisms}
(also $b$ is separated by Lemma
\ref{lemma-relative-proj-separated}).
\end{proof}

\begin{lemma}
\label{lemma-blow-up-effective-Cartier-divisor}
Let $S$ be a scheme. Let $X$ be an algebraic space over $S$.
Let $Z \subset X$ be an effective Cartier divisor.
The blowup of $X$ in $Z$ is the identity morphism of $X$.
\end{lemma}

\begin{proof}
Immediate from the universal property of blowups
(Lemma \ref{lemma-universal-property-blowing-up}).
\end{proof}

\begin{lemma}
\label{lemma-blow-up-reduced-space}
Let $S$ be a scheme. Let $X$ be an algebraic space over $S$.
Let $\mathcal{I} \subset \mathcal{O}_X$ be a
quasi-coherent sheaf of ideals. If $X$ is reduced, then the
blowup $X'$ of $X$ in $\mathcal{I}$ is reduced.
\end{lemma}

\begin{proof}
Let $U$ be a scheme and let $U \to X$ be a surjective \'etale morphism.
As blowing up commutes with flat base change
(Lemma \ref{lemma-flat-base-change-blowing-up})
we can prove each of these statements after base change to $U$.
This reduces us to the case of schemes.
In this case the result is
Divisors, Lemma \ref{divisors-lemma-blow-up-reduced-scheme}.
\end{proof}

\begin{lemma}
\label{lemma-blowup-finite-nr-irreducibles}
Let $S$ be a scheme. Let $X$ be an algebraic space over $S$. Let
$b : X' \to X$ be the blowup of $X$ is a closed subspace. If
$X$ satisfies the equivalent conditions of
Morphisms of Spaces, Lemma \ref{spaces-morphisms-lemma-prepare-normalization}
then so does $X'$.
\end{lemma}

\begin{proof}
Follows immediately from the lemma cited in the statement,
the \'etale local description of blowing ups in
Lemma \ref{lemma-blowing-up-affine}, and
Divisors, Lemma \ref{divisors-lemma-blow-up-and-irreducible-components}.
\end{proof}

\begin{lemma}
\label{lemma-blow-up-pullback-effective-Cartier}
Let $S$ be a scheme. Let $X$ be an algebraic space over $S$.
Let $b : X' \to X$ be a blowup of $X$ in a closed subspace.
For any effective Cartier divisor $D$ on $X$ the pullback
$b^{-1}D$ is defined (see Definition
\ref{definition-pullback-effective-Cartier-divisor}).
\end{lemma}

\begin{proof}
By Lemmas \ref{lemma-blowing-up-affine} and
\ref{lemma-characterize-effective-Cartier-divisor}
this reduces to the following algebra fact:
Let $A$ be a ring, $I \subset A$ an ideal, $a \in I$, and $x \in A$
a nonzerodivisor. Then the image of $x$ in $A[\frac{I}{a}]$ is a
nonzerodivisor. Namely, suppose that $x (y/a^n) = 0$ in $A[\frac{I}{a}]$.
Then $a^mxy = 0$ in $A$ for some $m$. Hence $a^my = 0$ as $x$ is a
nonzerodivisor. Whence $y/a^n$ is zero in $A[\frac{I}{a}]$ as desired.
\end{proof}

\begin{lemma}
\label{lemma-blowing-up-two-ideals}
Let $S$ be a scheme. Let $X$ be an algebraic space over $S$.
Let $\mathcal{I} \subset \mathcal{O}_X$ and $\mathcal{J}$ be
quasi-coherent sheaves of ideals. Let $b : X' \to X$ be the blowing up
of $X$ in $\mathcal{I}$. Let $b' : X'' \to X'$ be the blowing up of
$X'$ in $b^{-1}\mathcal{J} \mathcal{O}_{X'}$. Then $X'' \to X$
is canonically isomorphic to the blowing up of $X$ in $\mathcal{I}\mathcal{J}$.
\end{lemma}

\begin{proof}
Let $E \subset X'$ be the exceptional divisor of $b$ which is an effective
Cartier divisor by
Lemma \ref{lemma-blowing-up-gives-effective-Cartier-divisor}.
Then $(b')^{-1}E$ is an effective Cartier divisor on $X''$ by
Lemma \ref{lemma-blow-up-pullback-effective-Cartier}.
Let $E' \subset X''$ be the exceptional divisor of $b'$ (also an effective
Cartier divisor). Consider the effective Cartier divisor
$E'' = E' + (b')^{-1}E$. By construction the ideal of $E''$ is
$(b \circ b')^{-1}\mathcal{I} (b \circ b')^{-1}\mathcal{J} \mathcal{O}_{X''}$.
Hence according to Lemma \ref{lemma-universal-property-blowing-up}
there is a canonical morphism from $X''$ to the blowup $c : Y \to X$
of $X$ in $\mathcal{I}\mathcal{J}$. Conversely, as $\mathcal{I}\mathcal{J}$
pulls back to an invertible ideal we see that
$c^{-1}\mathcal{I}\mathcal{O}_Y$ defines
an effective Cartier divisor, see
Lemma \ref{lemma-sum-closed-subschemes-effective-Cartier}.
Thus a morphism $c' : Y \to X'$ over $X$ by
Lemma \ref{lemma-universal-property-blowing-up}.
Then $(c')^{-1}b^{-1}\mathcal{J}\mathcal{O}_Y = c^{-1}\mathcal{J}\mathcal{O}_Y$
which also defines an effective Cartier divisor. Thus a morphism
$c'' : Y \to X''$ over $X'$. We omit the verification that this
morphism is inverse to the morphism $X'' \to Y$ constructed earlier.
\end{proof}

\begin{lemma}
\label{lemma-blowing-up-projective}
Let $S$ be a scheme. Let $X$ be an algebraic space over $S$.
Let $\mathcal{I} \subset \mathcal{O}_X$ be a quasi-coherent
sheaf of ideals. Let $b : X' \to X$ be the blowing up of $X$
in the ideal sheaf $\mathcal{I}$. If $\mathcal{I}$ is of finite type, then
$b : X' \to X$ is a proper morphism.
\end{lemma}

\begin{proof}
Let $U$ be a scheme and let $U \to X$ be a surjective \'etale morphism.
As blowing up commutes with flat base change
(Lemma \ref{lemma-flat-base-change-blowing-up})
we can prove each of these statements after base change to $U$
(see Morphisms of Spaces, Lemma
\ref{spaces-morphisms-lemma-proper-local}).
This reduces us to the case of schemes.
In this case the morphism $b$ is projective by
Divisors, Lemma \ref{divisors-lemma-blowing-up-projective}
hence proper by
Morphisms, Lemma \ref{morphisms-lemma-locally-projective-proper}.
\end{proof}

\begin{lemma}
\label{lemma-composition-finite-type-blowups}
Let $S$ be a scheme and let $X$ be an algebraic space over $S$.
Assume $X$ is quasi-compact and quasi-separated.
Let $Z \subset X$ be a closed subspace of finite presentation.
Let $b : X' \to X$ be the blowing up with center $Z$.
Let $Z' \subset X'$ be a closed subspace of finite presentation.
Let $X'' \to X'$ be the blowing up with center $Z'$.
There exists a closed subspace $Y \subset X$ of finite presentation,
such that
\begin{enumerate}
\item $|Y| = |Z| \cup |b|(|Z'|)$, and
\item the composition $X'' \to X$ is isomorphic to the blowing up
of $X$ in $Y$.
\end{enumerate}
\end{lemma}

\begin{proof}
The condition that $Z \to X$ is of finite presentation means that
$Z$ is cut out by a finite type quasi-coherent sheaf of ideals
$\mathcal{I} \subset \mathcal{O}_X$, see
Morphisms of Spaces, Lemma
\ref{spaces-morphisms-lemma-closed-immersion-finite-presentation}.
Write $\mathcal{A} = \bigoplus_{n \geq 0} \mathcal{I}^n$ so that
$X' = \underline{\text{Proj}}(\mathcal{A})$.
Note that $X \setminus Z$ is a quasi-compact open subspace of $X$ by
Limits of Spaces, Lemma
\ref{spaces-limits-lemma-quasi-coherent-finite-type-ideals}.
Since $b^{-1}(X \setminus Z) \to X \setminus Z$ is an isomorphism
(Lemma \ref{lemma-blowing-up-gives-effective-Cartier-divisor}) the same
result shows that
$b^{-1}(X \setminus Z) \setminus Z'$ is quasi-compact open subspace in $X'$.
Hence $U = X \setminus (Z \cup b(Z'))$ is quasi-compact open subspace in $X$.
By Lemma \ref{lemma-closed-subscheme-proj-finite-type}
there exist a $d > 0$ and a finite type
$\mathcal{O}_X$-submodule $\mathcal{F} \subset \mathcal{I}^d$ such
that $Z' = \underline{\text{Proj}}(\mathcal{A}/\mathcal{F}\mathcal{A})$
and such that the support of $\mathcal{I}^d/\mathcal{F}$ is contained
in $X \setminus U$.

\medskip\noindent
Since $\mathcal{F} \subset \mathcal{I}^d$ is an $\mathcal{O}_X$-submodule
we may think of $\mathcal{F} \subset \mathcal{I}^d \subset \mathcal{O}_X$
as a finite type quasi-coherent sheaf of ideals on $X$. Let's denote this
$\mathcal{J} \subset \mathcal{O}_X$ to prevent confusion. Since
$\mathcal{I}^d / \mathcal{J}$ and $\mathcal{O}/\mathcal{I}^d$ are
supported on $|X| \setminus |U|$ we see that $|V(\mathcal{J})|$ is contained
in $|X| \setminus |U|$. Conversely, as $\mathcal{J} \subset \mathcal{I}^d$
we see that $|Z| \subset |V(\mathcal{J})|$. Over
$X \setminus Z \cong X' \setminus b^{-1}(Z)$ the sheaf of ideals
$\mathcal{J}$ cuts out $Z'$ (see displayed formula below). Hence
$|V(\mathcal{J})|$ equals $|Z| \cup |b|(|Z'|)$. It follows that also
$|V(\mathcal{I}\mathcal{J})| = |Z| \cup |b|(|Z'|)$. Moreover,
$\mathcal{I}\mathcal{J}$ is an ideal of finite type as a product of two such.
We claim that $X'' \to X$ is isomorphic to the blowing up of $X$ in
$\mathcal{I}\mathcal{J}$ which finishes the proof of the lemma by setting
$Y = V(\mathcal{I}\mathcal{J})$.

\medskip\noindent
First, recall that the blowup of $X$ in $\mathcal{I}\mathcal{J}$
is the same as the blowup of $X'$ in $b^{-1}\mathcal{J} \mathcal{O}_{X'}$,
see Lemma \ref{lemma-blowing-up-two-ideals}.
Hence it suffices to show that the blowup of $X'$ in
$b^{-1}\mathcal{J} \mathcal{O}_{X'}$ agrees with the blowup of $X'$
in $Z'$. We will show that
$$
b^{-1}\mathcal{J} \mathcal{O}_{X'} = \mathcal{I}_E^d \mathcal{I}_{Z'}
$$
as ideal sheaves on $X''$. This will prove what we want as
$\mathcal{I}_E^d$ cuts out the effective Cartier divisor $dE$
and we can use Lemmas \ref{lemma-blow-up-effective-Cartier-divisor} and
\ref{lemma-blowing-up-two-ideals}.

\medskip\noindent
To see the displayed equality of the ideals we may work locally.
With notation $A$, $I$, $a \in I$ as in Lemma \ref{lemma-blowing-up-affine}
we see that $\mathcal{F}$ corresponds to an $R$-submodule $M \subset I^d$
mapping isomorphically to an ideal $J \subset R$. The condition
$Z' = \underline{\text{Proj}}(\mathcal{A}/\mathcal{F}\mathcal{A})$
means that $Z' \cap \Spec(A[\frac{I}{a}])$ is cut out by the ideal
generated by the elements $m/a^d$, $m \in M$. Say the element $m \in M$
corresponds to the function $f \in J$. Then in the affine blowup algebra
$A' = A[\frac{I}{a}]$ we see that $f = (a^dm)/a^d = a^d (m/a^d)$.
Thus the equality holds.
\end{proof}









\section{Strict transform}
\label{section-strict-transform}

\noindent
This section is the analogue of
Divisors, Section \ref{divisors-section-strict-transform}.
Let $S$ be a scheme, let $B$ be an algebraic space over $S$, and
let $Z \subset B$ be a closed subspace.
Let $b : B' \to B$ be the blowing up of $B$ in $Z$ and denote $E \subset B'$
the exceptional divisor $E = b^{-1}Z$. In the following we will often
consider an algebraic space $X$ over $B$ and form the cartesian diagram
$$
\xymatrix{
\text{pr}_{B'}^{-1}E \ar[r] \ar[d] &
X \times_B B' \ar[r]_-{\text{pr}_X} \ar[d]_{\text{pr}_{B'}} &
X \ar[d]^f \\
E \ar[r] & B' \ar[r] & B
}
$$
Since $E$ is an effective Cartier divisor
(Lemma \ref{lemma-blowing-up-gives-effective-Cartier-divisor})
we see that $\text{pr}_{B'}^{-1}E \subset X \times_B B'$
is locally principal
(Lemma \ref{lemma-pullback-locally-principal}).
Thus the inclusion morphism of the complement of
$\text{pr}_{B'}^{-1}E$ in $X \times_B B'$
is affine and in particular quasi-compact
(Lemma \ref{lemma-complement-locally-principal-closed-subscheme}).
Consequently, for a quasi-coherent $\mathcal{O}_{X \times_B B'}$-module
$\mathcal{G}$ the subsheaf of sections supported on $|\text{pr}_{B'}^{-1}E|$
is a quasi-coherent submodule, see
Limits of Spaces, Definition
\ref{spaces-limits-definition-subsheaf-sections-supported-on-closed}.
If $\mathcal{G}$ is a quasi-coherent sheaf of algebras, e.g.,
$\mathcal{G} = \mathcal{O}_{X \times_B B'}$, then this subsheaf is an ideal
of $\mathcal{G}$.

\begin{definition}
\label{definition-strict-transform}
With $Z \subset B$ and $f : X \to B$ as above.
\begin{enumerate}
\item Given a quasi-coherent $\mathcal{O}_X$-module $\mathcal{F}$
the {\it strict transform} of $\mathcal{F}$ with respect to the blowup
of $B$ in $Z$ is the quotient $\mathcal{F}'$ of $\text{pr}_X^*\mathcal{F}$
by the submodule of sections supported on $|\text{pr}_{B'}^{-1}E|$.
\item The {\it strict transform} of $X$ is the closed subspace
$X' \subset X \times_B B'$ cut out by the quasi-coherent ideal of
sections of $\mathcal{O}_{X \times_B B'}$ supported on
$|\text{pr}_{B'}^{-1}E|$.
\end{enumerate}
\end{definition}

\noindent
Note that taking the strict transform along a blowup depends on the
closed subspace used for the blowup
(and not just on the morphism $B' \to B$).

\begin{lemma}[\'Etale localization and strict transform]
\label{lemma-strict-transform-local}
In the situation of Definition \ref{definition-strict-transform}.
Let
$$
\xymatrix{
U \ar[r] \ar[d] & X \ar[d] \\
V \ar[r] & B
}
$$
be a commutative diagram of morphisms with $U$ and $V$ schemes and
\'etale horizontal arrows. Let $V' \to V$ be the blowup of $V$
in $Z \times_B V$. Then
\begin{enumerate}
\item $V' = V \times_B B'$ and the maps
$V' \to B'$ and $U \times_V V' \to X \times_B B'$ are \'etale,
\item the strict transform $U'$ of $U$ relative to $V' \to V$
is equal to $X' \times_X U$ where $X'$ is the strict transform of $X$
relative to $B' \to B$, and
\item for a quasi-coherent $\mathcal{O}_X$-module $\mathcal{F}$ the
restriction of the strict transform $\mathcal{F}'$ to
$U \times_V V'$ is the strict transform of $\mathcal{F}|_U$ relative
to $V' \to V$.
\end{enumerate}
\end{lemma}

\begin{proof}
Part (1) follows from the fact that blowup commutes with flat base
change (Lemma \ref{lemma-flat-base-change-blowing-up}), the fact that
\'etale morphisms are flat, and that the base change of an \'etale
morphism is \'etale. Part (3) then follows from the fact that taking
the sheaf of sections supported on a closed commutes with pullback
by \'etale morphisms, see Limits of Spaces, Lemma
\ref{spaces-limits-lemma-sections-supported-on-closed-subset}.
Part (2) follows from (3) applied to $\mathcal{F} = \mathcal{O}_X$.
\end{proof}

\begin{lemma}
\label{lemma-strict-transform}
In the situation of Definition \ref{definition-strict-transform}.
\begin{enumerate}
\item The strict transform $X'$ of $X$ is the blowup of $X$ in the closed
subspace $f^{-1}Z$ of $X$.
\item For a quasi-coherent $\mathcal{O}_X$-module $\mathcal{F}$ the
strict transform $\mathcal{F}'$ is canonically isomorphic to
the pushforward along $X' \to X \times_B B'$ of the strict transform of
$\mathcal{F}$ relative to the blowing up $X' \to X$.
\end{enumerate}
\end{lemma}

\begin{proof}
Let $X'' \to X$ be the blowup of $X$ in $f^{-1}Z$. By the universal
property of blowing up (Lemma \ref{lemma-universal-property-blowing-up})
there exists a commutative diagram
$$
\xymatrix{
X'' \ar[r] \ar[d] & X \ar[d] \\
B' \ar[r] & B
}
$$
whence a morphism $i : X'' \to X \times_B B'$. The first assertion
of the lemma is that $i$ is a closed immersion with image $X'$.
The second assertion of the lemma is that $\mathcal{F}' = i_*\mathcal{F}''$
where $\mathcal{F}''$ is the strict transform of $\mathcal{F}$ with
respect to the blowing up $X'' \to X$. We can check these assertions
\'etale locally on $X$, hence we reduce to the case of schemes
(Divisors, Lemma \ref{divisors-lemma-strict-transform}).
Some details omitted.
\end{proof}

\begin{lemma}
\label{lemma-strict-transform-flat}
In the situation of Definition \ref{definition-strict-transform}.
\begin{enumerate}
\item If $X$ is flat over $B$ at all points lying over $Z$, then
the strict transform of $X$ is equal to the base change $X \times_B B'$.
\item Let $\mathcal{F}$ be a quasi-coherent $\mathcal{O}_X$-module.
If $\mathcal{F}$ is flat over $B$ at all points lying over $Z$, then
the strict transform $\mathcal{F}'$ of $\mathcal{F}$ is equal to the
pullback $\text{pr}_X^*\mathcal{F}$.
\end{enumerate}
\end{lemma}

\begin{proof}
Omitted. Hint: Follows from the case of schemes
(Divisors, Lemma \ref{divisors-lemma-strict-transform-flat})
by \'etale localization
(Lemma \ref{lemma-strict-transform-local}).
\end{proof}

\begin{lemma}
\label{lemma-strict-transform-affine}
Let $S$ be a scheme. Let $B$ be an algebraic space over $S$.
Let $Z \subset B$ be a closed subspace.
Let $b : B' \to B$ be the blowing up of $Z$ in $B$. Let
$g : X \to Y$ be an affine morphism of spaces over $B$.
Let $\mathcal{F}$ be a quasi-coherent sheaf on $X$.
Let $g' : X \times_B B' \to Y \times_B B'$ be the base change
of $g$. Let $\mathcal{F}'$ be the strict transform of $\mathcal{F}$
relative to $b$. Then $g'_*\mathcal{F}'$ is the strict transform
of $g_*\mathcal{F}$.
\end{lemma}

\begin{proof}
Omitted. Hint: Follows from the case of schemes
(Divisors, Lemma \ref{divisors-lemma-strict-transform-affine})
by \'etale localization (Lemma \ref{lemma-strict-transform-local}).
\end{proof}

\begin{lemma}
\label{lemma-strict-transform-different-centers}
Let $S$ be a scheme. Let $B$ be an algebraic space over $S$.
Let $Z \subset B$ be a closed subspace.
Let $D \subset B$ be an effective Cartier divisor.
Let $Z' \subset B$ be the closed subspace cut out by the product
of the ideal sheaves of $Z$ and $D$.
Let $B' \to B$ be the blowup of $B$ in $Z$.
\begin{enumerate}
\item The blowup of $B$ in $Z'$ is isomorphic to $B' \to B$.
\item Let $f : X \to B$ be a morphism of algebraic spaces and let $\mathcal{F}$
be a quasi-coherent $\mathcal{O}_X$-module. If the subsheaf of $\mathcal{F}$ of
sections supported on $|f^{-1}D|$ is zero, then the
strict transform of $\mathcal{F}$ relative to the blowing up
in $Z$ agrees with the strict transform of $\mathcal{F}$ relative
to the blowing up of $B$ in $Z'$.
\end{enumerate}
\end{lemma}

\begin{proof}
Omitted. Hint: Follows from the case of schemes
(Divisors, Lemma \ref{divisors-lemma-strict-transform-different-centers})
by \'etale localization (Lemma \ref{lemma-strict-transform-local}).
\end{proof}

\begin{lemma}
\label{lemma-strict-transform-composition-blowups}
Let $S$ be a scheme. Let $B$ be an algebraic space over $S$.
Let $Z \subset B$ be a closed subspace.
Let $b : B' \to B$ be the blowing up with center $Z$.
Let $Z' \subset B'$ be a closed subspace.
Let $B'' \to B'$ be the blowing up with center $Z'$.
Let $Y \subset B$ be a closed subscheme such that
$|Y| = |Z| \cup |b|(|Z'|)$ and the composition $B'' \to B$
is isomorphic to the blowing up of $B$ in $Y$.
In this situation, given any scheme $X$ over $B$ and
$\mathcal{F} \in \QCoh(\mathcal{O}_X)$ we have
\begin{enumerate}
\item the strict transform of $\mathcal{F}$ with respect to the blowing
up of $B$ in $Y$ is equal to the strict transform with respect to the
blowup $B'' \to B'$ in $Z'$ of the strict transform of $\mathcal{F}$
with respect to the blowup $B' \to B$ of $B$ in $Z$, and
\item the strict transform of $X$ with respect to the blowing
up of $B$ in $Y$ is equal to the strict transform with respect to the
blowup $B'' \to B'$ in $Z'$ of the strict transform of $X$
with respect to the blowup $B' \to B$ of $B$ in $Z$.
\end{enumerate}
\end{lemma}

\begin{proof}
Omitted. Hint: Follows from the case of schemes
(Divisors, Lemma \ref{divisors-lemma-strict-transform-composition-blowups})
by \'etale localization (Lemma \ref{lemma-strict-transform-local}).
\end{proof}

\begin{lemma}
\label{lemma-strict-transform-universally-injective}
In the situation of Definition \ref{definition-strict-transform}.
Suppose that
$$
0 \to \mathcal{F}_1 \to \mathcal{F}_2 \to \mathcal{F}_3 \to 0
$$
is an exact sequence of quasi-coherent sheaves on $X$ which remains
exact after any base change $T \to B$. Then the strict transforms of
$\mathcal{F}_i'$ relative to any blowup $B' \to B$
form a short exact sequence
$0 \to \mathcal{F}'_1 \to \mathcal{F}'_2 \to \mathcal{F}'_3 \to 0$ too.
\end{lemma}

\begin{proof}
Omitted. Hint: Follows from the case of schemes
(Divisors, Lemma \ref{divisors-lemma-strict-transform-universally-injective})
by \'etale localization (Lemma \ref{lemma-strict-transform-local}).
\end{proof}

\begin{lemma}
\label{lemma-strict-transform-blowup-fitting-ideal}
Let $S$ be a scheme. Let $B$ be an algebraic space over $S$.
Let $\mathcal{F}$ be a finite type quasi-coherent $\mathcal{O}_B$-module.
Let $Z_k \subset S$ be the closed subscheme cut out by
$\text{Fit}_k(\mathcal{F})$, see Section \ref{section-fitting-ideals}.
Let $B' \to B$ be the blowup of $B$ in $Z_k$ and let
$\mathcal{F}'$ be the strict transform of $\mathcal{F}$.
Then $\mathcal{F}'$ can locally be generated by $\leq k$
sections.
\end{lemma}

\begin{proof}
Omitted. Follows from the case of schemes
(Divisors, Lemma \ref{divisors-lemma-strict-transform-blowup-fitting-ideal})
by \'etale localization (Lemma \ref{lemma-strict-transform-local}).
\end{proof}

\begin{lemma}
\label{lemma-strict-transform-blowup-fitting-ideal-locally-free}
Let $S$ be a scheme. Let $B$ be an algebraic space over $S$.
Let $\mathcal{F}$ be a finite type quasi-coherent $\mathcal{O}_B$-module.
Let $Z_k \subset S$ be the closed subscheme cut out by
$\text{Fit}_k(\mathcal{F})$, see Section \ref{section-fitting-ideals}.
Assume that $\mathcal{F}$ is locally free of rank $k$ on $B \setminus Z_k$.
Let $B' \to B$ be the blowup of $B$ in $Z_k$ and let
$\mathcal{F}'$ be the strict transform of $\mathcal{F}$.
Then $\mathcal{F}'$ is locally free of rank $k$.
\end{lemma}

\begin{proof}
Omitted. Follows from the case of schemes
(Divisors, Lemma
\ref{divisors-lemma-strict-transform-blowup-fitting-ideal-locally-free})
by \'etale localization (Lemma \ref{lemma-strict-transform-local}).
\end{proof}












\section{Admissible blowups}
\label{section-admissible-blowups}

\noindent
To have a bit more control over our blowups we introduce the following
standard terminology.

\begin{definition}
\label{definition-admissible-blowup}
Let $S$ be a scheme. Let $X$ be an algebraic space over $S$.
Let $U \subset X$ be an open subspace. A morphism
$X' \to X$ is called a {\it $U$-admissible blowup} if there exists a
closed immersion $Z \to X$ of finite presentation with $Z$ disjoint from
$U$ such that $X'$ is isomorphic to the blowup of $X$ in $Z$.
\end{definition}

\noindent
We recall that $Z \to X$ is of finite presentation if and only if the
ideal sheaf $\mathcal{I}_Z \subset \mathcal{O}_X$ is of finite type, see
Morphisms of Spaces, Lemma
\ref{spaces-morphisms-lemma-closed-immersion-finite-presentation}.
In particular, a $U$-admissible blowup is a proper morphism, see
Lemma \ref{lemma-blowing-up-projective}.
Note that there can be multiple centers which give rise to the same morphism.
Hence the requirement is just the existence of some center disjoint from
$U$ which produces $X'$.
Finally, as the morphism $b : X' \to X$ is an isomorphism over $U$ (see
Lemma \ref{lemma-blowing-up-gives-effective-Cartier-divisor}) we will often
abuse notation and think of $U$ as an open subspace of $X'$ as well.

\begin{lemma}
\label{lemma-composition-admissible-blowups}
Let $S$ be a scheme.
Let $X$ be a quasi-compact and quasi-separated algebraic space over $S$.
Let $U \subset X$ be a quasi-compact open subspace.
Let $b : X' \to X$ be a $U$-admissible blowup.
Let $X'' \to X'$ be a $U$-admissible blowup.
Then the composition $X'' \to X$ is a $U$-admissible blowup.
\end{lemma}

\begin{proof}
Immediate from the more precise
Lemma \ref{lemma-composition-finite-type-blowups}.
\end{proof}

\begin{lemma}
\label{lemma-extend-admissible-blowups}
Let $S$ be a scheme.
Let $X$ be a quasi-compact and quasi-separated algebraic space.
Let $U, V \subset X$ be quasi-compact open subspaces.
Let $b : V' \to V$ be a $U \cap V$-admissible blowup.
Then there exists a $U$-admissible blowup $X' \to X$
whose restriction to $V$ is $V'$.
\end{lemma}

\begin{proof}
Let $\mathcal{I} \subset \mathcal{O}_V$ be the finite type
quasi-coherent sheaf of ideals such that $V(\mathcal{I})$ is
disjoint from $U \cap V$ and such that $V'$ is isomorphic to the
blowup of $V$ in $\mathcal{I}$. Let
$\mathcal{I}' \subset \mathcal{O}_{U \cup V}$ be the quasi-coherent
sheaf of ideals whose restriction to $U$ is $\mathcal{O}_U$ and
whose restriction to $V$ is $\mathcal{I}$.
By Limits of Spaces, Lemma \ref{spaces-limits-lemma-extend}
there exists a finite type quasi-coherent sheaf of ideals
$\mathcal{J} \subset \mathcal{O}_X$ whose restriction to $U \cup V$ is
$\mathcal{I}'$. The lemma follows.
\end{proof}

\begin{lemma}
\label{lemma-dominate-admissible-blowups}
Let $S$ be a scheme.
Let $X$ be a quasi-compact and quasi-separated algebraic space over $S$.
Let $U \subset X$ be a quasi-compact open subspace.
Let $b_i : X_i \to X$, $i = 1, \ldots, n$ be $U$-admissible blowups.
There exists a $U$-admissible blowup $b : X' \to X$ such that
(a) $b$ factors as $X' \to X_i \to X$ for $i = 1, \ldots, n$ and
(b) each of the morphisms $X' \to X_i$ is a $U$-admissible blowup.
\end{lemma}

\begin{proof}
Let $\mathcal{I}_i \subset \mathcal{O}_X$ be the finite type
quasi-coherent sheaf of ideals such that $V(\mathcal{I}_i)$ is
disjoint from $U$ and such that $X_i$ is isomorphic to the
blowup of $X$ in $\mathcal{I}_i$. Set
$\mathcal{I} = \mathcal{I}_1 \cdot \ldots \cdot \mathcal{I}_n$
and let $X'$ be the blowup of $X$ in $\mathcal{I}$. Then
$X' \to X$ factors through $b_i$ by Lemma \ref{lemma-blowing-up-two-ideals}.
\end{proof}

\begin{lemma}
\label{lemma-separate-disjoint-opens-by-blowing-up}
Let $S$ be a scheme.
Let $X$ be a quasi-compact and quasi-separated algebraic space over $S$.
Let $U, V$ be quasi-compact disjoint open subspaces of $X$.
Then there exist a $U \cup V$-admissible blowup $b : X' \to X$
such that $X'$ is a disjoint union of open subspaces
$X' = X'_1 \amalg X'_2$ with $b^{-1}(U) \subset X'_1$ and
$b^{-1}(V) \subset X'_2$.
\end{lemma}

\begin{proof}
Choose a finite type quasi-coherent sheaf of ideals $\mathcal{I}$,
resp.\ $\mathcal{J}$ such that $X \setminus U = V(\mathcal{I})$,
resp.\ $X \setminus V = V(\mathcal{J})$, see
Limits of Spaces, Lemma
\ref{spaces-limits-lemma-quasi-coherent-finite-type-ideals}.
Then $|V(\mathcal{I}\mathcal{J})| = |X|$. Hence
$\mathcal{I}\mathcal{J}$ is a locally nilpotent sheaf of ideals.
Since $\mathcal{I}$ and $\mathcal{J}$ are of finite type and $X$
is quasi-compact there exists an $n > 0$ such that
$\mathcal{I}^n \mathcal{J}^n = 0$. We may and do replace $\mathcal{I}$
by $\mathcal{I}^n$ and $\mathcal{J}$ by $\mathcal{J}^n$. Whence
$\mathcal{I} \mathcal{J} = 0$. Let $b : X' \to X$ be the blowing
up in $\mathcal{I} + \mathcal{J}$. This is $U \cup V$-admissible
as $|V(\mathcal{I} + \mathcal{J})| = |X| \setminus |U| \cup |V|$.
We will show that $X'$ is a disjoint union of open subspaces
$X' = X'_1 \amalg X'_2$ as in the statement of the lemma.

\medskip\noindent
Since $|V(\mathcal{I} + \mathcal{J})|$ is the complement of
$|U \cup V|$ we conclude that $V \cup U$ is scheme theoretically
dense in $X'$, see
Lemmas \ref{lemma-blowing-up-gives-effective-Cartier-divisor} and
\ref{lemma-complement-effective-Cartier-divisor}.
Thus if such a decomposition $X' = X'_1 \amalg X'_2$
into open and closed subspaces exists, then $X'_1$ is the
scheme theoretic closure of $U$ in $X'$ and similarly $X'_2$ is
the scheme theoretic closure of $V$ in $X'$. Since $U \to X'$
and $V \to X'$ are quasi-compact taking scheme theoretic
closures commutes with \'etale localization (Morphisms of Spaces,
Lemma \ref{spaces-morphisms-lemma-quasi-compact-scheme-theoretic-image}).
Hence to verify the existence of $X'_1$ and $X'_2$ we may work \'etale
locally on $X$. This reduces us to the case of schemes which is
treated in the proof of Divisors, Lemma
\ref{divisors-lemma-separate-disjoint-opens-by-blowing-up}.
\end{proof}




















\input{chapters}

\bibliography{my}
\bibliographystyle{amsalpha}

\end{document}
